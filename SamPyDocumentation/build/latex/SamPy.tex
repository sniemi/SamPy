% Generated by Sphinx.
\def\sphinxdocclass{report}
\documentclass[letterpaper,10pt,english]{sphinxmanual}
\usepackage[utf8]{inputenc}
\DeclareUnicodeCharacter{00A0}{\nobreakspace}
\usepackage[T1]{fontenc}
\usepackage{babel}
\usepackage{times}
\usepackage[Bjarne]{fncychap}
\usepackage{longtable}
\usepackage{sphinx}


\title{SamPy Documentation}
\date{April 27, 2011}
\release{0.1}
\author{Sami-Matias Niemi}
\newcommand{\sphinxlogo}{}
\renewcommand{\releasename}{Release}
\makeindex

\makeatletter
\def\PYG@reset{\let\PYG@it=\relax \let\PYG@bf=\relax%
    \let\PYG@ul=\relax \let\PYG@tc=\relax%
    \let\PYG@bc=\relax \let\PYG@ff=\relax}
\def\PYG@tok#1{\csname PYG@tok@#1\endcsname}
\def\PYG@toks#1+{\ifx\relax#1\empty\else%
    \PYG@tok{#1}\expandafter\PYG@toks\fi}
\def\PYG@do#1{\PYG@bc{\PYG@tc{\PYG@ul{%
    \PYG@it{\PYG@bf{\PYG@ff{#1}}}}}}}
\def\PYG#1#2{\PYG@reset\PYG@toks#1+\relax+\PYG@do{#2}}

\def\PYG@tok@gd{\def\PYG@tc##1{\textcolor[rgb]{0.63,0.00,0.00}{##1}}}
\def\PYG@tok@gu{\let\PYG@bf=\textbf\def\PYG@tc##1{\textcolor[rgb]{0.50,0.00,0.50}{##1}}}
\def\PYG@tok@gt{\def\PYG@tc##1{\textcolor[rgb]{0.00,0.25,0.82}{##1}}}
\def\PYG@tok@gs{\let\PYG@bf=\textbf}
\def\PYG@tok@gr{\def\PYG@tc##1{\textcolor[rgb]{1.00,0.00,0.00}{##1}}}
\def\PYG@tok@cm{\let\PYG@it=\textit\def\PYG@tc##1{\textcolor[rgb]{0.25,0.50,0.56}{##1}}}
\def\PYG@tok@vg{\def\PYG@tc##1{\textcolor[rgb]{0.73,0.38,0.84}{##1}}}
\def\PYG@tok@m{\def\PYG@tc##1{\textcolor[rgb]{0.13,0.50,0.31}{##1}}}
\def\PYG@tok@mh{\def\PYG@tc##1{\textcolor[rgb]{0.13,0.50,0.31}{##1}}}
\def\PYG@tok@cs{\def\PYG@tc##1{\textcolor[rgb]{0.25,0.50,0.56}{##1}}\def\PYG@bc##1{\colorbox[rgb]{1.00,0.94,0.94}{##1}}}
\def\PYG@tok@ge{\let\PYG@it=\textit}
\def\PYG@tok@vc{\def\PYG@tc##1{\textcolor[rgb]{0.73,0.38,0.84}{##1}}}
\def\PYG@tok@il{\def\PYG@tc##1{\textcolor[rgb]{0.13,0.50,0.31}{##1}}}
\def\PYG@tok@go{\def\PYG@tc##1{\textcolor[rgb]{0.19,0.19,0.19}{##1}}}
\def\PYG@tok@cp{\def\PYG@tc##1{\textcolor[rgb]{0.00,0.44,0.13}{##1}}}
\def\PYG@tok@gi{\def\PYG@tc##1{\textcolor[rgb]{0.00,0.63,0.00}{##1}}}
\def\PYG@tok@gh{\let\PYG@bf=\textbf\def\PYG@tc##1{\textcolor[rgb]{0.00,0.00,0.50}{##1}}}
\def\PYG@tok@ni{\let\PYG@bf=\textbf\def\PYG@tc##1{\textcolor[rgb]{0.84,0.33,0.22}{##1}}}
\def\PYG@tok@nl{\let\PYG@bf=\textbf\def\PYG@tc##1{\textcolor[rgb]{0.00,0.13,0.44}{##1}}}
\def\PYG@tok@nn{\let\PYG@bf=\textbf\def\PYG@tc##1{\textcolor[rgb]{0.05,0.52,0.71}{##1}}}
\def\PYG@tok@no{\def\PYG@tc##1{\textcolor[rgb]{0.38,0.68,0.84}{##1}}}
\def\PYG@tok@na{\def\PYG@tc##1{\textcolor[rgb]{0.25,0.44,0.63}{##1}}}
\def\PYG@tok@nb{\def\PYG@tc##1{\textcolor[rgb]{0.00,0.44,0.13}{##1}}}
\def\PYG@tok@nc{\let\PYG@bf=\textbf\def\PYG@tc##1{\textcolor[rgb]{0.05,0.52,0.71}{##1}}}
\def\PYG@tok@nd{\let\PYG@bf=\textbf\def\PYG@tc##1{\textcolor[rgb]{0.33,0.33,0.33}{##1}}}
\def\PYG@tok@ne{\def\PYG@tc##1{\textcolor[rgb]{0.00,0.44,0.13}{##1}}}
\def\PYG@tok@nf{\def\PYG@tc##1{\textcolor[rgb]{0.02,0.16,0.49}{##1}}}
\def\PYG@tok@si{\let\PYG@it=\textit\def\PYG@tc##1{\textcolor[rgb]{0.44,0.63,0.82}{##1}}}
\def\PYG@tok@s2{\def\PYG@tc##1{\textcolor[rgb]{0.25,0.44,0.63}{##1}}}
\def\PYG@tok@vi{\def\PYG@tc##1{\textcolor[rgb]{0.73,0.38,0.84}{##1}}}
\def\PYG@tok@nt{\let\PYG@bf=\textbf\def\PYG@tc##1{\textcolor[rgb]{0.02,0.16,0.45}{##1}}}
\def\PYG@tok@nv{\def\PYG@tc##1{\textcolor[rgb]{0.73,0.38,0.84}{##1}}}
\def\PYG@tok@s1{\def\PYG@tc##1{\textcolor[rgb]{0.25,0.44,0.63}{##1}}}
\def\PYG@tok@gp{\let\PYG@bf=\textbf\def\PYG@tc##1{\textcolor[rgb]{0.78,0.36,0.04}{##1}}}
\def\PYG@tok@sh{\def\PYG@tc##1{\textcolor[rgb]{0.25,0.44,0.63}{##1}}}
\def\PYG@tok@ow{\let\PYG@bf=\textbf\def\PYG@tc##1{\textcolor[rgb]{0.00,0.44,0.13}{##1}}}
\def\PYG@tok@sx{\def\PYG@tc##1{\textcolor[rgb]{0.78,0.36,0.04}{##1}}}
\def\PYG@tok@bp{\def\PYG@tc##1{\textcolor[rgb]{0.00,0.44,0.13}{##1}}}
\def\PYG@tok@c1{\let\PYG@it=\textit\def\PYG@tc##1{\textcolor[rgb]{0.25,0.50,0.56}{##1}}}
\def\PYG@tok@kc{\let\PYG@bf=\textbf\def\PYG@tc##1{\textcolor[rgb]{0.00,0.44,0.13}{##1}}}
\def\PYG@tok@c{\let\PYG@it=\textit\def\PYG@tc##1{\textcolor[rgb]{0.25,0.50,0.56}{##1}}}
\def\PYG@tok@mf{\def\PYG@tc##1{\textcolor[rgb]{0.13,0.50,0.31}{##1}}}
\def\PYG@tok@err{\def\PYG@bc##1{\fcolorbox[rgb]{1.00,0.00,0.00}{1,1,1}{##1}}}
\def\PYG@tok@kd{\let\PYG@bf=\textbf\def\PYG@tc##1{\textcolor[rgb]{0.00,0.44,0.13}{##1}}}
\def\PYG@tok@ss{\def\PYG@tc##1{\textcolor[rgb]{0.32,0.47,0.09}{##1}}}
\def\PYG@tok@sr{\def\PYG@tc##1{\textcolor[rgb]{0.14,0.33,0.53}{##1}}}
\def\PYG@tok@mo{\def\PYG@tc##1{\textcolor[rgb]{0.13,0.50,0.31}{##1}}}
\def\PYG@tok@mi{\def\PYG@tc##1{\textcolor[rgb]{0.13,0.50,0.31}{##1}}}
\def\PYG@tok@kn{\let\PYG@bf=\textbf\def\PYG@tc##1{\textcolor[rgb]{0.00,0.44,0.13}{##1}}}
\def\PYG@tok@o{\def\PYG@tc##1{\textcolor[rgb]{0.40,0.40,0.40}{##1}}}
\def\PYG@tok@kr{\let\PYG@bf=\textbf\def\PYG@tc##1{\textcolor[rgb]{0.00,0.44,0.13}{##1}}}
\def\PYG@tok@s{\def\PYG@tc##1{\textcolor[rgb]{0.25,0.44,0.63}{##1}}}
\def\PYG@tok@kp{\def\PYG@tc##1{\textcolor[rgb]{0.00,0.44,0.13}{##1}}}
\def\PYG@tok@w{\def\PYG@tc##1{\textcolor[rgb]{0.73,0.73,0.73}{##1}}}
\def\PYG@tok@kt{\def\PYG@tc##1{\textcolor[rgb]{0.56,0.13,0.00}{##1}}}
\def\PYG@tok@sc{\def\PYG@tc##1{\textcolor[rgb]{0.25,0.44,0.63}{##1}}}
\def\PYG@tok@sb{\def\PYG@tc##1{\textcolor[rgb]{0.25,0.44,0.63}{##1}}}
\def\PYG@tok@k{\let\PYG@bf=\textbf\def\PYG@tc##1{\textcolor[rgb]{0.00,0.44,0.13}{##1}}}
\def\PYG@tok@se{\let\PYG@bf=\textbf\def\PYG@tc##1{\textcolor[rgb]{0.25,0.44,0.63}{##1}}}
\def\PYG@tok@sd{\let\PYG@it=\textit\def\PYG@tc##1{\textcolor[rgb]{0.25,0.44,0.63}{##1}}}

\def\PYGZbs{\char`\\}
\def\PYGZus{\char`\_}
\def\PYGZob{\char`\{}
\def\PYGZcb{\char`\}}
\def\PYGZca{\char`\^}
\def\PYGZsh{\char`\#}
\def\PYGZpc{\char`\%}
\def\PYGZdl{\char`\$}
\def\PYGZti{\char`\~}
% for compatibility with earlier versions
\def\PYGZat{@}
\def\PYGZlb{[}
\def\PYGZrb{]}
\makeatother

\begin{document}

\maketitle
\tableofcontents
\phantomsection\label{index::doc}


Contents:


\chapter{Introduction}
\label{intro:introduction}\label{intro::doc}\label{intro:welcome-to-sampy-s-documentation}
SamPy contains random Python related modules that are mostly related to Astronomy or Cosmology.


\chapter{SamPy Package}
\label{SamPy::doc}\label{SamPy:sampy-package}

\section{Subpackages}
\label{SamPy:subpackages}

\subsection{astronomy Package}
\label{SamPy.astronomy:astronomy-package}\label{SamPy.astronomy::doc}

\subsubsection{\texttt{baryons} Module}
\label{SamPy.astronomy:module-SamPy.astronomy.baryons}\label{SamPy.astronomy:baryons-module}
\index{SamPy.astronomy.baryons (module)}
Helper functions related to baryonic constrains such as mass functions.
\begin{quote}\begin{description}
\item[{requires}] \leavevmode
NumPy

\item[{version}] \leavevmode
0.1

\item[{author}] \leavevmode
Sami-Matias Niemi

\item[{contact}] \leavevmode
\href{mailto:niemi@stsci.edu}{niemi@stsci.edu}

\end{description}\end{quote}

\index{BellBaryonicMassFunction() (in module SamPy.astronomy.baryons)}

\begin{fulllineitems}
\phantomsection\label{SamPy.astronomy:SamPy.astronomy.baryons.BellBaryonicMassFunction}\pysiglinewithargsret{\code{SamPy.astronomy.baryons.}\bfcode{BellBaryonicMassFunction}}{\emph{h=0.7}}{}
G-derived baryonic mass function using default gas
Schecter Function fit parameters
phi* M* Alpha j  (next line formal errors)
real errors are probably systematic: see Bell et al. 2003 for
guidance (the errors depend on passband and/or stellar mass)
\begin{quote}
\begin{quote}

0.0105038      10.7236     -1.22776  6.65221e+08
\end{quote}

0.000857033    0.0448084    0.0552834  1.45862e+07
\end{quote}

Then we present the V/V\_max data points; x   phi  phi-1sig  phi+1sig
\begin{quote}\begin{description}
\item[{Parameters}] \leavevmode
\textbf{h} (\emph{float}) -- Hubble parameter

\item[{Note }] \leavevmode
In Chabrier IMF.

\end{description}\end{quote}

\end{fulllineitems}



\subsubsection{\texttt{basics} Module}
\label{SamPy.astronomy:module-SamPy.astronomy.basics}\label{SamPy.astronomy:basics-module}
\index{SamPy.astronomy.basics (module)}
This module contains basic information of different
subjects often needed in astronomy.
\begin{quote}\begin{description}
\item[{author}] \leavevmode
Sami-Matias Niemi

\item[{version}] \leavevmode
0.01

\end{description}\end{quote}

\index{JohnsonToABmagnitudes() (in module SamPy.astronomy.basics)}

\begin{fulllineitems}
\phantomsection\label{SamPy.astronomy:SamPy.astronomy.basics.JohnsonToABmagnitudes}\pysiglinewithargsret{\code{SamPy.astronomy.basics.}\bfcode{JohnsonToABmagnitudes}}{}{}
These values can be used to convert
Johnson U, B, V, and K bands to Vega system.

\end{fulllineitems}



\subsubsection{\texttt{blackholes} Module}
\label{SamPy.astronomy:module-SamPy.astronomy.blackholes}\label{SamPy.astronomy:blackholes-module}
\index{SamPy.astronomy.blackholes (module)}
Observational constrains related to black hole
masses in galaxies.
\begin{quote}\begin{description}
\item[{requires}] \leavevmode
NumPy

\item[{version}] \leavevmode
0.1

\item[{author}] \leavevmode
Sami-Matias Niemi

\item[{contact}] \leavevmode
\href{mailto:niemi@stsci.edu}{niemi@stsci.edu}

\end{description}\end{quote}

\index{MarconiMassFunction() (in module SamPy.astronomy.blackholes)}

\begin{fulllineitems}
\phantomsection\label{SamPy.astronomy:SamPy.astronomy.blackholes.MarconiMassFunction}\pysiglinewithargsret{\code{SamPy.astronomy.blackholes.}\bfcode{MarconiMassFunction}}{\emph{const=0.362216}}{}
Marconi et al. BHMF -- columns:
(1)  Log\_\{10\}{[} M\_bh / M\_sun {]}
(2)  Log\_\{10\}{[}-1sigma phi(M\_bh){]}
(3)  Log\_\{10\}(phi(M\_bh)) == dN / {[}dV*dln(M\_bh){]}, i.e.
\begin{quote}
\begin{description}
\item[{n(M\_bh) in Mpc\textasciicircum{}\{-3\}*ln(M\_bh)\textasciicircum{}\{-1\} (number per interval in}] \leavevmode
natural log(M\_bh))

\item[{-- if you want what is more typically plotted, i.e. number}] \leavevmode
per log\_\{10\} interval in M\_bh, then you need to multiply
phi by ln{[}10{]}, or add Log\_\{10\}(ln{[}10{]}) = 0.362216 to the value of
Log\_\{10\}{[}Phi{]} given below

\end{description}
\end{quote}
\begin{enumerate}
\setcounter{enumi}{3}
\item {} 
log\_\{10\}{[}+1sigma phi(M\_bh){]}

\end{enumerate}
\begin{quote}\begin{description}
\item[{Parameters}] \leavevmode
\textbf{(float)} (\emph{constant}) -- 
?


\end{description}\end{quote}

\end{fulllineitems}


\index{MassesHaeringRix() (in module SamPy.astronomy.blackholes)}

\begin{fulllineitems}
\phantomsection\label{SamPy.astronomy:SamPy.astronomy.blackholes.MassesHaeringRix}\pysiglinewithargsret{\code{SamPy.astronomy.blackholes.}\bfcode{MassesHaeringRix}}{}{}
The full data file contains:
Galaxy Type M\_\{bh\} mbherrp mbherrm mbhexp Ref sigma
L\_\{bulge\} Upsilon M\_bulge Ref dist
:return: bulge mass, black hole mass, ellipticals, spirals, S0s

\end{fulllineitems}



\subsubsection{\texttt{conversions} Module}
\label{SamPy.astronomy:module-SamPy.astronomy.conversions}\label{SamPy.astronomy:conversions-module}
\index{SamPy.astronomy.conversions (module)}
Some functions for astronomy related unit conversions.
\begin{quote}\begin{description}
\item[{requires}] \leavevmode
NumPy

\item[{requires}] \leavevmode
cosmocalc (\href{http://cxc.harvard.edu/contrib/cosmocalc/}{http://cxc.harvard.edu/contrib/cosmocalc/})

\item[{version}] \leavevmode
0.11

\item[{author}] \leavevmode
Sami Niemi

\item[{contact}] \leavevmode
\href{mailto:niemi@stsci.edu}{niemi@stsci.edu}

\end{description}\end{quote}

\index{ABMagnitudeToJansky() (in module SamPy.astronomy.conversions)}

\begin{fulllineitems}
\phantomsection\label{SamPy.astronomy:SamPy.astronomy.conversions.ABMagnitudeToJansky}\pysiglinewithargsret{\code{SamPy.astronomy.conversions.}\bfcode{ABMagnitudeToJansky}}{\emph{ABmagnitude}}{}
Converts AB magnitudes to Janskys.
\begin{quote}\begin{description}
\item[{Note }] \leavevmode
Can be used with SQLite3 database.

\item[{Parameters}] \leavevmode
\textbf{ABmagnitude} -- can be either a number or a NumPy array

\item[{Returns}] \leavevmode
either a float or NumPy array

\end{description}\end{quote}

\end{fulllineitems}


\index{Luminosity() (in module SamPy.astronomy.conversions)}

\begin{fulllineitems}
\phantomsection\label{SamPy.astronomy:SamPy.astronomy.conversions.Luminosity}\pysiglinewithargsret{\code{SamPy.astronomy.conversions.}\bfcode{Luminosity}}{\emph{abs\_mag}}{}
Converts AB magnitudes to luminosities in L\_sun
\begin{quote}\begin{description}
\item[{Parameters}] \leavevmode
\textbf{abs\_mag} -- AB magnitude of the object

\item[{Returns}] \leavevmode
luminosity

\end{description}\end{quote}

\end{fulllineitems}


\index{RAandDECfromStandardCoordinates() (in module SamPy.astronomy.conversions)}

\begin{fulllineitems}
\phantomsection\label{SamPy.astronomy:SamPy.astronomy.conversions.RAandDECfromStandardCoordinates}\pysiglinewithargsret{\code{SamPy.astronomy.conversions.}\bfcode{RAandDECfromStandardCoordinates}}{\emph{data}}{}
Converts Standard Coordinates on tangent plane
to RA and DEC on the sky.
data dictionary must also contain the CD matrix.
Full equations:
xi  = cdmatrix(0,0) * (x-crpix(0)) + cdmatrix(0,1)* (y - crpix(1))
eta = cdmatrix(1,0) * (x-crpix(0)) + cdmatrix(1,1)* (y - crpix(1))
then
ra = atan2(xi, cos(dec0)-eta*sin(dec0)) + ra0
dec = atan2(eta*cos(dec0)+sin(dec0),
\begin{quote}

sqrt((cos(dec0)-eta*sin(dec0))**2 + xi**2))
\end{quote}
\begin{quote}\begin{description}
\item[{Parameters}] \leavevmode
\textbf{(dictionary)} (\emph{data}) -- should contain standard coordinates X, Y,

\end{description}\end{quote}

RA and DEC of the centre point, and the CD matrix.

\end{fulllineitems}


\index{angularDiameterDistance() (in module SamPy.astronomy.conversions)}

\begin{fulllineitems}
\phantomsection\label{SamPy.astronomy:SamPy.astronomy.conversions.angularDiameterDistance}\pysiglinewithargsret{\code{SamPy.astronomy.conversions.}\bfcode{angularDiameterDistance}}{\emph{z}, \emph{H0=70}, \emph{WM=0.28}}{}
The angular diameter distance DA is defined as the ratio of
an object's physical transverse size to its angular size
(in radians). It is used to convert angular separations in
telescope images into proper separations at the source. It
is famous for not increasing indefinitely as z to inf; it turns
over at z about 1 and thereafter more distant objects actually
appear larger in angular size.

\end{fulllineitems}


\index{arccot() (in module SamPy.astronomy.conversions)}

\begin{fulllineitems}
\phantomsection\label{SamPy.astronomy:SamPy.astronomy.conversions.arccot}\pysiglinewithargsret{\code{SamPy.astronomy.conversions.}\bfcode{arccot}}{\emph{x}}{}
\end{fulllineitems}


\index{arcminSquaredToSolidAnge() (in module SamPy.astronomy.conversions)}

\begin{fulllineitems}
\phantomsection\label{SamPy.astronomy:SamPy.astronomy.conversions.arcminSquaredToSolidAnge}\pysiglinewithargsret{\code{SamPy.astronomy.conversions.}\bfcode{arcminSquaredToSolidAnge}}{\emph{arcmin2}}{}
Converts arcmin**2 to solid angle.
Calls arcminSqauredToSteradians to
convert arcmin**2 to steradians and
then divides this with 4pi.
\begin{quote}\begin{description}
\item[{Parameters}] \leavevmode
\textbf{arcmin2} -- arcmin**2

\item[{Returns}] \leavevmode
solid angle

\end{description}\end{quote}

\end{fulllineitems}


\index{arcminSquaredToSteradians() (in module SamPy.astronomy.conversions)}

\begin{fulllineitems}
\phantomsection\label{SamPy.astronomy:SamPy.astronomy.conversions.arcminSquaredToSteradians}\pysiglinewithargsret{\code{SamPy.astronomy.conversions.}\bfcode{arcminSquaredToSteradians}}{\emph{arcmin2}}{}
Converts arcmin**2 to steradians.
\begin{quote}\begin{description}
\item[{Parameters}] \leavevmode
\textbf{arcmin2} -- arcmin**2

\item[{Returns}] \leavevmode
steradians

\end{description}\end{quote}

\end{fulllineitems}


\index{comovingVolume() (in module SamPy.astronomy.conversions)}

\begin{fulllineitems}
\phantomsection\label{SamPy.astronomy:SamPy.astronomy.conversions.comovingVolume}\pysiglinewithargsret{\code{SamPy.astronomy.conversions.}\bfcode{comovingVolume}}{\emph{arcmin2}, \emph{zmin}, \emph{zmax}, \emph{H0=70}, \emph{WM=0.28}}{}
Calculates the comoving volume between two redshifts when
the sky survey has covered arcmin**2 region.
\begin{quote}\begin{description}
\item[{Parameters}] \leavevmode\begin{itemize}
\item {} 
\textbf{arcmin2} -- area on the sky in arcmin**2

\item {} 
\textbf{zmin} -- redshift of the front part of the volume

\item {} 
\textbf{zmax} -- redshift of the back part of the volume

\item {} 
\textbf{H0} -- Value of the Hubble constant

\item {} 
\textbf{WM} -- Value of the mass density

\end{itemize}

\item[{Returns}] \leavevmode
comoving volume between zmin and zmax of arcmin2
solid angle in Mpc**3

\end{description}\end{quote}

\end{fulllineitems}


\index{convertSphericalToCartesian() (in module SamPy.astronomy.conversions)}

\begin{fulllineitems}
\phantomsection\label{SamPy.astronomy:SamPy.astronomy.conversions.convertSphericalToCartesian}\pysiglinewithargsret{\code{SamPy.astronomy.conversions.}\bfcode{convertSphericalToCartesian}}{\emph{r}, \emph{theta}, \emph{phi}}{}
Converts Spherical coordiantes to Cartesian.
Returns a dictionary.

\end{fulllineitems}


\index{cot() (in module SamPy.astronomy.conversions)}

\begin{fulllineitems}
\phantomsection\label{SamPy.astronomy:SamPy.astronomy.conversions.cot}\pysiglinewithargsret{\code{SamPy.astronomy.conversions.}\bfcode{cot}}{\emph{x}}{}
\end{fulllineitems}


\index{degTodms() (in module SamPy.astronomy.conversions)}

\begin{fulllineitems}
\phantomsection\label{SamPy.astronomy:SamPy.astronomy.conversions.degTodms}\pysiglinewithargsret{\code{SamPy.astronomy.conversions.}\bfcode{degTodms}}{\emph{ideg}}{}
\end{fulllineitems}


\index{degTohms() (in module SamPy.astronomy.conversions)}

\begin{fulllineitems}
\phantomsection\label{SamPy.astronomy:SamPy.astronomy.conversions.degTohms}\pysiglinewithargsret{\code{SamPy.astronomy.conversions.}\bfcode{degTohms}}{\emph{ideg}}{}
\end{fulllineitems}


\index{get\_flat\_flambda\_dmag() (in module SamPy.astronomy.conversions)}

\begin{fulllineitems}
\phantomsection\label{SamPy.astronomy:SamPy.astronomy.conversions.get_flat_flambda_dmag}\pysiglinewithargsret{\code{SamPy.astronomy.conversions.}\bfcode{get\_flat\_flambda\_dmag}}{\emph{plambda}, \emph{plambda\_ref}}{}
Compute the differential AB-mag for an object flat in f\_lambda

\end{fulllineitems}


\index{get\_magAB\_from\_flambda() (in module SamPy.astronomy.conversions)}

\begin{fulllineitems}
\phantomsection\label{SamPy.astronomy:SamPy.astronomy.conversions.get_magAB_from_flambda}\pysiglinewithargsret{\code{SamPy.astronomy.conversions.}\bfcode{get\_magAB\_from\_flambda}}{\emph{flambda}, \emph{wlength}}{}
Converts a mag\_AB value at a wavelength to f\_lambda
\begin{quote}\begin{description}
\item[{Parameters}] \leavevmode\begin{itemize}
\item {} 
\textbf{flambda} (\emph{float}) -- mag\_AB value

\item {} 
\textbf{wlength} (\emph{float}) -- wavelength value {[}nm{]}

\end{itemize}

\item[{Returns}] \leavevmode
the mag\_AB value

\item[{Return type}] \leavevmode
float

\end{description}\end{quote}

\end{fulllineitems}


\index{janskyToMagnitude() (in module SamPy.astronomy.conversions)}

\begin{fulllineitems}
\phantomsection\label{SamPy.astronomy:SamPy.astronomy.conversions.janskyToMagnitude}\pysiglinewithargsret{\code{SamPy.astronomy.conversions.}\bfcode{janskyToMagnitude}}{\emph{jansky}}{}
Converts Janskys to AB magnitudes.
\begin{quote}\begin{description}
\item[{Note }] \leavevmode
Can be used with SQLite3 database.

\item[{Parameters}] \leavevmode
\textbf{jansky} -- can either be a number or a NumPy array

\item[{Returns}] \leavevmode
either a float or NumPy array

\end{description}\end{quote}

\end{fulllineitems}


\index{redshiftFromScale() (in module SamPy.astronomy.conversions)}

\begin{fulllineitems}
\phantomsection\label{SamPy.astronomy:SamPy.astronomy.conversions.redshiftFromScale}\pysiglinewithargsret{\code{SamPy.astronomy.conversions.}\bfcode{redshiftFromScale}}{\emph{scale}}{}
Converts a scale factor to redshift.

\end{fulllineitems}


\index{scaleFromRedshift() (in module SamPy.astronomy.conversions)}

\begin{fulllineitems}
\phantomsection\label{SamPy.astronomy:SamPy.astronomy.conversions.scaleFromRedshift}\pysiglinewithargsret{\code{SamPy.astronomy.conversions.}\bfcode{scaleFromRedshift}}{\emph{redshift}}{}
Converts a redshift to a scale factor.

\end{fulllineitems}



\subsubsection{\texttt{datamanipulation} Module}
\label{SamPy.astronomy:datamanipulation-module}\label{SamPy.astronomy:module-SamPy.astronomy.datamanipulation}
\index{SamPy.astronomy.datamanipulation (module)}
A random collection of functions for data manipulation.
\begin{quote}\begin{description}
\item[{requires}] \leavevmode
NumPy

\item[{requires}] \leavevmode
SciPy

\item[{version}] \leavevmode
0.1

\item[{author}] \leavevmode
Sami-Matias Niemi

\item[{contact}] \leavevmode
\href{mailto:niemi@stsci.edu}{niemi@stsci.edu}

\end{description}\end{quote}

\index{average\_bins() (in module SamPy.astronomy.datamanipulation)}

\begin{fulllineitems}
\phantomsection\label{SamPy.astronomy:SamPy.astronomy.datamanipulation.average_bins}\pysiglinewithargsret{\code{SamPy.astronomy.datamanipulation.}\bfcode{average\_bins}}{\emph{xdata}, \emph{ydata}, \emph{xmin}, \emph{xmax}, \emph{nxbins=15}}{}
Computes mean and 16 and 84 percentiles of y-data in bins in x
\begin{quote}\begin{description}
\item[{Parameters}] \leavevmode\begin{itemize}
\item {} 
\textbf{xdata} -- numpy array of xdata

\item {} 
\textbf{ydata} -- numpy arrya of ydata

\item {} 
\textbf{xmax} -- maximumx value of x that data are binned to

\item {} 
\textbf{xmin} -- minimum value of x that data are binned to

\item {} 
\textbf{nxbins} -- number of bins in x

\end{itemize}

\item[{Returns}] \leavevmode
mid points of the bins, mean, 16 per cent percentile, and

\end{description}\end{quote}

84 per cent percentile.

\end{fulllineitems}


\index{binAndReturnFractions() (in module SamPy.astronomy.datamanipulation)}

\begin{fulllineitems}
\phantomsection\label{SamPy.astronomy:SamPy.astronomy.datamanipulation.binAndReturnFractions}\pysiglinewithargsret{\code{SamPy.astronomy.datamanipulation.}\bfcode{binAndReturnFractions}}{\emph{x}, \emph{y1}, \emph{y2}, \emph{xmin=9}, \emph{xmax=11.5}, \emph{xbins=10}, \emph{logscale=False}}{}
\end{fulllineitems}


\index{binAndReturnMergerFractions() (in module SamPy.astronomy.datamanipulation)}

\begin{fulllineitems}
\phantomsection\label{SamPy.astronomy:SamPy.astronomy.datamanipulation.binAndReturnMergerFractions}\pysiglinewithargsret{\code{SamPy.astronomy.datamanipulation.}\bfcode{binAndReturnMergerFractions}}{\emph{mstar}, \emph{nomerge}, \emph{mergers}, \emph{majors}, \emph{mstarmin=9}, \emph{mstarmax=11.5}, \emph{mbins=10}, \emph{logscale=False}}{}
\end{fulllineitems}


\index{binAndReturnMergerFractions2() (in module SamPy.astronomy.datamanipulation)}

\begin{fulllineitems}
\phantomsection\label{SamPy.astronomy:SamPy.astronomy.datamanipulation.binAndReturnMergerFractions2}\pysiglinewithargsret{\code{SamPy.astronomy.datamanipulation.}\bfcode{binAndReturnMergerFractions2}}{\emph{mstar}, \emph{nomerge}, \emph{mergers}, \emph{majors}, \emph{mergers2}, \emph{majors2}, \emph{mstarmin=9}, \emph{mstarmax=11.5}, \emph{mbins=10}, \emph{logscale=False}}{}
\end{fulllineitems}


\index{binned\_average() (in module SamPy.astronomy.datamanipulation)}

\begin{fulllineitems}
\phantomsection\label{SamPy.astronomy:SamPy.astronomy.datamanipulation.binned_average}\pysiglinewithargsret{\code{SamPy.astronomy.datamanipulation.}\bfcode{binned\_average}}{\emph{xdata}, \emph{ydata}, \emph{xbins}, \emph{step}}{}
\end{fulllineitems}


\index{movingAverage() (in module SamPy.astronomy.datamanipulation)}

\begin{fulllineitems}
\phantomsection\label{SamPy.astronomy:SamPy.astronomy.datamanipulation.movingAverage}\pysiglinewithargsret{\code{SamPy.astronomy.datamanipulation.}\bfcode{movingAverage}}{\emph{x}, \emph{n}, \emph{type='simple'}}{}
compute an n period moving average.
type is `simple' \textbar{} `exponential'

\end{fulllineitems}


\index{percentile\_bins() (in module SamPy.astronomy.datamanipulation)}

\begin{fulllineitems}
\phantomsection\label{SamPy.astronomy:SamPy.astronomy.datamanipulation.percentile_bins}\pysiglinewithargsret{\code{SamPy.astronomy.datamanipulation.}\bfcode{percentile\_bins}}{\emph{xdata}, \emph{ydata}, \emph{xmin}, \emph{xmax}, \emph{nxbins=15}, \emph{log=False}, \emph{limit=6}}{}
Computes median and 16 and 84 percentiles of y-data in bins in x
\begin{quote}\begin{description}
\item[{Parameters}] \leavevmode\begin{itemize}
\item {} 
\textbf{xdata} -- numpy array of xdata

\item {} 
\textbf{ydata} -- numpy arrya of ydata

\item {} 
\textbf{xmax} -- maximumx value of x that data are binned to

\item {} 
\textbf{xmin} -- minimum value of x that data are binned to

\item {} 
\textbf{nxbins} -- number of bins in x

\item {} 
\textbf{log} -- if True, xbins are logarithmically spaced, else linearly

\item {} 
\textbf{limit} -- the minimum number of values in a bin for which the
median and percentiles are returned for.

\end{itemize}

\item[{Returns}] \leavevmode
mid points of the bins, median, 16 per cent percentile, and

\end{description}\end{quote}

84 per cent percentile.

\end{fulllineitems}


\index{rollingAverage() (in module SamPy.astronomy.datamanipulation)}

\begin{fulllineitems}
\phantomsection\label{SamPy.astronomy:SamPy.astronomy.datamanipulation.rollingAverage}\pysiglinewithargsret{\code{SamPy.astronomy.datamanipulation.}\bfcode{rollingAverage}}{\emph{x}}{}
Returns the average between the cells of a list.
\begin{quote}\begin{description}
\item[{Parameters}] \leavevmode
\textbf{x} -- a Python list of values

\item[{Returns}] \leavevmode
a NumPy array of averages

\end{description}\end{quote}

\end{fulllineitems}



\subsubsection{\texttt{differentialfunctions} Module}
\label{SamPy.astronomy:module-SamPy.astronomy.differentialfunctions}\label{SamPy.astronomy:differentialfunctions-module}
\index{SamPy.astronomy.differentialfunctions (module)}
Some functions to calculate differential functions. These
can be used to calculate luminosity functions, stellar and
dark matter halo functions etc.
\begin{quote}\begin{description}
\item[{requires}] \leavevmode
NumPy

\item[{version}] \leavevmode
0.15

\item[{author}] \leavevmode
Sami Niemi

\item[{contact}] \leavevmode
\href{mailto:niemi@stsci.edu}{niemi@stsci.edu}

\end{description}\end{quote}

\index{diffFunctionLogBinning() (in module SamPy.astronomy.differentialfunctions)}

\begin{fulllineitems}
\phantomsection\label{SamPy.astronomy:SamPy.astronomy.differentialfunctions.diffFunctionLogBinning}\pysiglinewithargsret{\code{SamPy.astronomy.differentialfunctions.}\bfcode{diffFunctionLogBinning}}{\emph{data}, \emph{column=0}, \emph{log=False}, \emph{wgth=None}, \emph{mmax=15.5}, \emph{mmin=9.0}, \emph{nbins=35}, \emph{h=0.7}, \emph{volume=250}, \emph{nvols=1}, \emph{physical\_units=False}, \emph{verbose=False}}{}
Calculates a differential function from data.
Uses NumPy to calculate a histogram and then divides.
each bin value with the length of the bin.
The log binning refers to 10 based log, i.e. log10(data)

\end{fulllineitems}


\index{diff\_function\_log\_binning() (in module SamPy.astronomy.differentialfunctions)}

\begin{fulllineitems}
\phantomsection\label{SamPy.astronomy:SamPy.astronomy.differentialfunctions.diff_function_log_binning}\pysiglinewithargsret{\code{SamPy.astronomy.differentialfunctions.}\bfcode{diff\_function\_log\_binning}}{\emph{data}, \emph{column=0}, \emph{log=False}, \emph{wgth=None}, \emph{mmax=15.5}, \emph{mmin=9.0}, \emph{nbins=35}, \emph{h=0.7}, \emph{volume=250}, \emph{nvols=1}, \emph{physical\_units=False}, \emph{verbose=False}}{}
Calculates a differential function from data.
:warning: One should not use this, unless the number of
systems for each bin is used. One should use diffFunctionLogBinning
instead, which is probably faster as it uses NumPy.histogram
rathrer than my own algorithm.

\end{fulllineitems}


\index{mass\_function() (in module SamPy.astronomy.differentialfunctions)}

\begin{fulllineitems}
\phantomsection\label{SamPy.astronomy:SamPy.astronomy.differentialfunctions.mass_function}\pysiglinewithargsret{\code{SamPy.astronomy.differentialfunctions.}\bfcode{mass\_function}}{\emph{data}, \emph{column=0}, \emph{log=False}, \emph{wght=None}, \emph{mmin=9.0}, \emph{mmax=15.0}, \emph{nbins=35}, \emph{h=0.7}, \emph{volume=250}, \emph{nvols=1}, \emph{verbose=False}}{}
Calculates a mass function from data.
Returns differential mass function and bins:
dN / dlnM
:TODO: add calculating the cumulative mass function.

\end{fulllineitems}


\index{stellarMassFunction() (in module SamPy.astronomy.differentialfunctions)}

\begin{fulllineitems}
\phantomsection\label{SamPy.astronomy:SamPy.astronomy.differentialfunctions.stellarMassFunction}\pysiglinewithargsret{\code{SamPy.astronomy.differentialfunctions.}\bfcode{stellarMassFunction}}{\emph{data}, \emph{wght=None}, \emph{mmin=None}, \emph{mmax=None}, \emph{nbins=40}, \emph{h=0.7}, \emph{volume=50}, \emph{nvols=8}, \emph{verbose=False}, \emph{early\_type\_galaxies=0.4}, \emph{central\_galaxy\_id=1}}{}
Calculates a stellar mass function from data. Calculates
stellar mass functions for all galaxies, early- and late-types
and central galaxies seprately.
\begin{quote}\begin{description}
\item[{Params data (dictionary)}] \leavevmode
data should be in format:

\end{description}\end{quote}

data{[}'stellar\_mass'{]} = {[}{]}
data{[}'bulge\_mass'{]} = {[}{]}
data{[}'galaxy\_id'{]} = {[}{]}
\begin{quote}\begin{description}
\item[{Return output (dictionary)}] \leavevmode
\end{description}\end{quote}

\end{fulllineitems}



\subsubsection{\texttt{footprintfinder} Module}
\label{SamPy.astronomy:footprintfinder-module}\label{SamPy.astronomy:module-SamPy.astronomy.footprintfinder}
\index{SamPy.astronomy.footprintfinder (module)}
\index{alert() (in module SamPy.astronomy.footprintfinder)}

\begin{fulllineitems}
\phantomsection\label{SamPy.astronomy:SamPy.astronomy.footprintfinder.alert}\pysiglinewithargsret{\code{SamPy.astronomy.footprintfinder.}\bfcode{alert}}{\emph{string}}{}
prints the string in blue.

\end{fulllineitems}


\index{directborder() (in module SamPy.astronomy.footprintfinder)}

\begin{fulllineitems}
\phantomsection\label{SamPy.astronomy:SamPy.astronomy.footprintfinder.directborder}\pysiglinewithargsret{\code{SamPy.astronomy.footprintfinder.}\bfcode{directborder}}{\emph{filename}, \emph{zerovalue}, \emph{extension}, \emph{epsilon=1e-08}, \emph{verbose=False}}{}
This function reads the data in slices (for low memory usage), cleans it an then
identifies the borderpixels. Pyfits only allows this kind of access for non-scaled data!

The crval, crpix amd cdmatrix are read from the file, too.

\end{fulllineitems}


\index{error() (in module SamPy.astronomy.footprintfinder)}

\begin{fulllineitems}
\phantomsection\label{SamPy.astronomy:SamPy.astronomy.footprintfinder.error}\pysiglinewithargsret{\code{SamPy.astronomy.footprintfinder.}\bfcode{error}}{\emph{string}}{}
prints the string in red.

\end{fulllineitems}


\index{getborderdictionary() (in module SamPy.astronomy.footprintfinder)}

\begin{fulllineitems}
\phantomsection\label{SamPy.astronomy:SamPy.astronomy.footprintfinder.getborderdictionary}\pysiglinewithargsret{\code{SamPy.astronomy.footprintfinder.}\bfcode{getborderdictionary}}{\emph{borderpixels}, \emph{verbose=False}}{}
Gets a dictionary of the borderpixels.

\end{fulllineitems}


\index{getcentroid() (in module SamPy.astronomy.footprintfinder)}

\begin{fulllineitems}
\phantomsection\label{SamPy.astronomy:SamPy.astronomy.footprintfinder.getcentroid}\pysiglinewithargsret{\code{SamPy.astronomy.footprintfinder.}\bfcode{getcentroid}}{\emph{simplefootprintlist}, \emph{borderpixels}, \emph{heirarchy}, \emph{verbose}}{}
computes the centroid (=barycenter) and area for each polygon as well as the barycenter for the entire set.
Holes have negative areas, computations are done in pixel space. It is assumed that the scalefactor is 1, i.e.
that pixels are square.

\end{fulllineitems}


\index{getcloseparticles() (in module SamPy.astronomy.footprintfinder)}

\begin{fulllineitems}
\phantomsection\label{SamPy.astronomy:SamPy.astronomy.footprintfinder.getcloseparticles}\pysiglinewithargsret{\code{SamPy.astronomy.footprintfinder.}\bfcode{getcloseparticles}}{\emph{i}, \emph{borderpixels}, \emph{borderdictionary}}{}
returns a list of borderpixel ids that are close to the given pixel.

\end{fulllineitems}


\index{getdatetime() (in module SamPy.astronomy.footprintfinder)}

\begin{fulllineitems}
\phantomsection\label{SamPy.astronomy:SamPy.astronomy.footprintfinder.getdatetime}\pysiglinewithargsret{\code{SamPy.astronomy.footprintfinder.}\bfcode{getdatetime}}{\emph{time=None}}{}
returns a SYBSE accepted datetime string. If no datetime object is given, it returns
the (local) time.

\end{fulllineitems}


\index{getgroups() (in module SamPy.astronomy.footprintfinder)}

\begin{fulllineitems}
\phantomsection\label{SamPy.astronomy:SamPy.astronomy.footprintfinder.getgroups}\pysiglinewithargsret{\code{SamPy.astronomy.footprintfinder.}\bfcode{getgroups}}{\emph{borderpixels}, \emph{borderdictionary}, \emph{verbose=False}}{}
This function finds groups of borderpixels using the bordertree. It is a classical groupfinder.

\end{fulllineitems}


\index{getheirarchy() (in module SamPy.astronomy.footprintfinder)}

\begin{fulllineitems}
\phantomsection\label{SamPy.astronomy:SamPy.astronomy.footprintfinder.getheirarchy}\pysiglinewithargsret{\code{SamPy.astronomy.footprintfinder.}\bfcode{getheirarchy}}{\emph{footprintlist}, \emph{borderpixels}}{}
This function checks whether one footprint is contained in an other one.

\end{fulllineitems}


\index{getpixelorder() (in module SamPy.astronomy.footprintfinder)}

\begin{fulllineitems}
\phantomsection\label{SamPy.astronomy:SamPy.astronomy.footprintfinder.getpixelorder}\pysiglinewithargsret{\code{SamPy.astronomy.footprintfinder.}\bfcode{getpixelorder}}{\emph{chip}, \emph{borderpixels}, \emph{borderdictionary}, \emph{groupid}, \emph{grouplen}, \emph{firstid}, \emph{nextid}, \emph{verbose}}{}
This function finds the order of the pixels going round the chip. The array of borderpixelids is returned.
The orientation is clockwise.

\end{fulllineitems}


\index{getreturnstring() (in module SamPy.astronomy.footprintfinder)}

\begin{fulllineitems}
\phantomsection\label{SamPy.astronomy:SamPy.astronomy.footprintfinder.getreturnstring}\pysiglinewithargsret{\code{SamPy.astronomy.footprintfinder.}\bfcode{getreturnstring}}{\emph{simplefootprintlist}, \emph{borderpixels}, \emph{heirarchy}, \emph{centroid}, \emph{wcs}, \emph{verbose}}{}
returns a simplified version of the footprint information that can be stored in the database

x, y, ra, dec
x0,y0,x1,y1,x2,y2,...; x0,y0,x1,y1,x2,y2,...;...
The points are ordered anti-clickwise (chip) or clockwise (hole)

\end{fulllineitems}


\index{main() (in module SamPy.astronomy.footprintfinder)}

\begin{fulllineitems}
\phantomsection\label{SamPy.astronomy:SamPy.astronomy.footprintfinder.main}\pysiglinewithargsret{\code{SamPy.astronomy.footprintfinder.}\bfcode{main}}{\emph{arguments}}{}
This is the main function doing the full loop.

Note that pyfits reads y,x which is why all the plotting routines are using inverted coordinates.
For the description of directions in the comments, we use this inverted notation (i.e. as plotted
on the screen)

\end{fulllineitems}


\index{plotfootprints() (in module SamPy.astronomy.footprintfinder)}

\begin{fulllineitems}
\phantomsection\label{SamPy.astronomy:SamPy.astronomy.footprintfinder.plotfootprints}\pysiglinewithargsret{\code{SamPy.astronomy.footprintfinder.}\bfcode{plotfootprints}}{\emph{simplefootprintlist}, \emph{borderpixels}, \emph{dimshape}, \emph{chips}, \emph{groupid}, \emph{firstid}, \emph{nextid}, \emph{grouplen}, \emph{heirarchy}, \emph{centroid}, \emph{colorlist}, \emph{verbose}}{}
This function plots the footprints using matplotlib

\end{fulllineitems}


\index{pointinpolygon() (in module SamPy.astronomy.footprintfinder)}

\begin{fulllineitems}
\phantomsection\label{SamPy.astronomy:SamPy.astronomy.footprintfinder.pointinpolygon}\pysiglinewithargsret{\code{SamPy.astronomy.footprintfinder.}\bfcode{pointinpolygon}}{\emph{x}, \emph{y}, \emph{footprint}, \emph{borderpixels}}{}
This function checks weather the point x,y is contained in the polygon given by footprint.
It is assumed that the first point and the last point are identical!

\end{fulllineitems}


\index{rd2xy() (in module SamPy.astronomy.footprintfinder)}

\begin{fulllineitems}
\phantomsection\label{SamPy.astronomy:SamPy.astronomy.footprintfinder.rd2xy}\pysiglinewithargsret{\code{SamPy.astronomy.footprintfinder.}\bfcode{rd2xy}}{\emph{wcs}, \emph{ra}, \emph{dec}}{}
Code as in stsdas.imgtools.xy2rd
Can not treat distorted images.

ra dec must be in degrees

\end{fulllineitems}


\index{runtimestring() (in module SamPy.astronomy.footprintfinder)}

\begin{fulllineitems}
\phantomsection\label{SamPy.astronomy:SamPy.astronomy.footprintfinder.runtimestring}\pysiglinewithargsret{\code{SamPy.astronomy.footprintfinder.}\bfcode{runtimestring}}{\emph{date=None}}{}
returns a string of the current time and if a date is given, adds the running time from that
date.

\end{fulllineitems}


\index{showprogress() (in module SamPy.astronomy.footprintfinder)}

\begin{fulllineitems}
\phantomsection\label{SamPy.astronomy:SamPy.astronomy.footprintfinder.showprogress}\pysiglinewithargsret{\code{SamPy.astronomy.footprintfinder.}\bfcode{showprogress}}{\emph{i}, \emph{n}, \emph{lastprogress}}{}
This function shows the progress of an iteration. i=current number, n=total number.

\end{fulllineitems}


\index{simplify() (in module SamPy.astronomy.footprintfinder)}

\begin{fulllineitems}
\phantomsection\label{SamPy.astronomy:SamPy.astronomy.footprintfinder.simplify}\pysiglinewithargsret{\code{SamPy.astronomy.footprintfinder.}\bfcode{simplify}}{\emph{order}, \emph{borderpixels}, \emph{a}, \emph{b}, \emph{tolerance}, \emph{iscorner}}{}
\end{fulllineitems}


\index{simplifyDP() (in module SamPy.astronomy.footprintfinder)}

\begin{fulllineitems}
\phantomsection\label{SamPy.astronomy:SamPy.astronomy.footprintfinder.simplifyDP}\pysiglinewithargsret{\code{SamPy.astronomy.footprintfinder.}\bfcode{simplifyDP}}{\emph{order}, \emph{borderpixels}, \emph{tolerance}}{}
\end{fulllineitems}


\index{usage() (in module SamPy.astronomy.footprintfinder)}

\begin{fulllineitems}
\phantomsection\label{SamPy.astronomy:SamPy.astronomy.footprintfinder.usage}\pysiglinewithargsret{\code{SamPy.astronomy.footprintfinder.}\bfcode{usage}}{}{}
Help.

\end{fulllineitems}


\index{warning() (in module SamPy.astronomy.footprintfinder)}

\begin{fulllineitems}
\phantomsection\label{SamPy.astronomy:SamPy.astronomy.footprintfinder.warning}\pysiglinewithargsret{\code{SamPy.astronomy.footprintfinder.}\bfcode{warning}}{\emph{string}}{}
prints the string in orange.

\end{fulllineitems}


\index{whocalls() (in module SamPy.astronomy.footprintfinder)}

\begin{fulllineitems}
\phantomsection\label{SamPy.astronomy:SamPy.astronomy.footprintfinder.whocalls}\pysiglinewithargsret{\code{SamPy.astronomy.footprintfinder.}\bfcode{whocalls}}{\emph{level=1}, \emph{verbose=False}}{}
This function returns the caller, the function and the line number.

\end{fulllineitems}


\index{writeds9output() (in module SamPy.astronomy.footprintfinder)}

\begin{fulllineitems}
\phantomsection\label{SamPy.astronomy:SamPy.astronomy.footprintfinder.writeds9output}\pysiglinewithargsret{\code{SamPy.astronomy.footprintfinder.}\bfcode{writeds9output}}{\emph{filename}, \emph{simplefootprintlist}, \emph{borderpixels}, \emph{heirarchy}, \emph{wcs}, \emph{centroid}, \emph{colorlist}, \emph{verbose}}{}
writing ds9 region file(s). One file with the image coordinates, and if possible (i.e. of the projection
is RA-TAN/DEC-TAN) also a ds9 WCS file.

\end{fulllineitems}


\index{writeoutput() (in module SamPy.astronomy.footprintfinder)}

\begin{fulllineitems}
\phantomsection\label{SamPy.astronomy:SamPy.astronomy.footprintfinder.writeoutput}\pysiglinewithargsret{\code{SamPy.astronomy.footprintfinder.}\bfcode{writeoutput}}{\emph{filename}, \emph{simplefootprintlist}, \emph{borderpixels}, \emph{heirarchy}, \emph{wcs}, \emph{centroid}, \emph{arguments}, \emph{writepoints}, \emph{verbose}}{}
write the footprints to the output file. This file contains all information (including heirarchy) and is 
always produced.

\end{fulllineitems}


\index{xy2rd() (in module SamPy.astronomy.footprintfinder)}

\begin{fulllineitems}
\phantomsection\label{SamPy.astronomy:SamPy.astronomy.footprintfinder.xy2rd}\pysiglinewithargsret{\code{SamPy.astronomy.footprintfinder.}\bfcode{xy2rd}}{\emph{wcs}, \emph{x}, \emph{y}}{}
Code as in stsdas.imgtools.xy2rd
Can not treat distorted images.

\end{fulllineitems}



\subsubsection{\texttt{gasmasses} Module}
\label{SamPy.astronomy:gasmasses-module}\label{SamPy.astronomy:module-SamPy.astronomy.gasmasses}
\index{SamPy.astronomy.gasmasses (module)}
Observational constrains related to gas masses in galaxies.
\begin{quote}\begin{description}
\item[{requires}] \leavevmode
NumPy

\item[{version}] \leavevmode
0.1

\item[{author}] \leavevmode
Sami-Matias Niemi

\item[{contact}] \leavevmode
\href{mailto:niemi@stsci.edu}{niemi@stsci.edu}

\end{description}\end{quote}

\index{H2MassFunctionBell() (in module SamPy.astronomy.gasmasses)}

\begin{fulllineitems}
\phantomsection\label{SamPy.astronomy:SamPy.astronomy.gasmasses.H2MassFunctionBell}\pysiglinewithargsret{\code{SamPy.astronomy.gasmasses.}\bfcode{H2MassFunctionBell}}{\emph{h=0.7}}{}
H2 mass function from Bell et al.

\end{fulllineitems}


\index{HIMassFunctionBell() (in module SamPy.astronomy.gasmasses)}

\begin{fulllineitems}
\phantomsection\label{SamPy.astronomy:SamPy.astronomy.gasmasses.HIMassFunctionBell}\pysiglinewithargsret{\code{SamPy.astronomy.gasmasses.}\bfcode{HIMassFunctionBell}}{\emph{h=0.7}}{}
HI mass function (predicted using default technique)
Schecter Function fit parameters
phi* M* Alpha j  (next line formal errors)
real errors are probably systematic: see Bell et al.2003 for
guidance (the errors depend on passband and/or stellar mass)
0.0142750      9.67205     -1.42495  1.03937e+08
0.00612444     0.154853     0.205068  1.64674e+07
Then we present the V/V\_max data points; x   phi  phi-1sig  phi+1sig

\end{fulllineitems}


\index{HIMassFunctionZwaan() (in module SamPy.astronomy.gasmasses)}

\begin{fulllineitems}
\phantomsection\label{SamPy.astronomy:SamPy.astronomy.gasmasses.HIMassFunctionZwaan}\pysiglinewithargsret{\code{SamPy.astronomy.gasmasses.}\bfcode{HIMassFunctionZwaan}}{\emph{mbin}, \emph{H0=70.0}}{}
Observed HI mass function.
Zwaan et al.
:param mbin: bins in log10(masses)

\end{fulllineitems}


\index{gasFractionKannappan() (in module SamPy.astronomy.gasmasses)}

\begin{fulllineitems}
\phantomsection\label{SamPy.astronomy:SamPy.astronomy.gasmasses.gasFractionKannappan}\pysiglinewithargsret{\code{SamPy.astronomy.gasmasses.}\bfcode{gasFractionKannappan}}{\emph{chabrier=True}}{}
Gas fraction fitting function from Kannappan et al.
using either the Chabrier (default) or diet Salpeter IMF.

\end{fulllineitems}



\subsubsection{\texttt{hess\_plot} Module}
\label{SamPy.astronomy:module-SamPy.astronomy.hess_plot}\label{SamPy.astronomy:hess-plot-module}
\index{SamPy.astronomy.hess\_plot (module)}
This file contains a function that can be used to 
generate a Hess plot.

\index{hess\_plot() (in module SamPy.astronomy.hess\_plot)}

\begin{fulllineitems}
\phantomsection\label{SamPy.astronomy:SamPy.astronomy.hess_plot.hess_plot}\pysiglinewithargsret{\code{SamPy.astronomy.hess\_plot.}\bfcode{hess\_plot}}{\emph{xdata}, \emph{ydata}, \emph{weight}, \emph{xmin}, \emph{xmax}, \emph{nxbins}, \emph{ymin}, \emph{ymax}, \emph{nybins}, \emph{pmax=1.0}, \emph{pmin=0.1}}{}
This function can be used to calculate a hess plot
i.e. the conditional probability 2D histogram.
\begin{quote}\begin{description}
\item[{Parameters}] \leavevmode\begin{itemize}
\item {} 
\textbf{xdata} -- 1D NumPy array containing data in x dim.

\item {} 
\textbf{ydata} -- 1D NumPy array containing data in y dim.

\item {} 
\textbf{weight} -- 1D NumPy array containing weight for
each data point

\item {} 
\textbf{xmin} -- a minimum x value to be considered

\item {} 
\textbf{xmax} -- a maximum x value to be considered

\item {} 
\textbf{nxbins} -- the number of bins in x dim.

\item {} 
\textbf{nybins} -- the number of bins in y dim.

\item {} 
\textbf{pmax} -- 1.0

\item {} 
\textbf{pmin} -- 0.1

\end{itemize}

\item[{Returns}] \leavevmode
2D array, min, max

\end{description}\end{quote}

\end{fulllineitems}


\index{hess\_plot\_old() (in module SamPy.astronomy.hess\_plot)}

\begin{fulllineitems}
\phantomsection\label{SamPy.astronomy:SamPy.astronomy.hess_plot.hess_plot_old}\pysiglinewithargsret{\code{SamPy.astronomy.hess\_plot.}\bfcode{hess\_plot\_old}}{\emph{xdata}, \emph{ydata}, \emph{weight}, \emph{xmin}, \emph{xmax}, \emph{nxbins}, \emph{ymin}, \emph{ymax}, \emph{nybins}, \emph{pmax=1.0}, \emph{pmin=0.1}}{}~\begin{quote}\begin{description}
\item[{Note }] \leavevmode
obsolete, do not use this version!

\end{description}\end{quote}

\end{fulllineitems}



\subsubsection{\texttt{luminosityFunctionss} Module}
\label{SamPy.astronomy:luminosityfunctionss-module}\label{SamPy.astronomy:module-SamPy.astronomy.luminosityFunctionss}
\index{SamPy.astronomy.luminosityFunctionss (module)}
Different luminosity functions.
\begin{quote}\begin{description}
\item[{requires}] \leavevmode
NumPy

\item[{version}] \leavevmode
0.1

\item[{author}] \leavevmode
Sami-Matias Niemi

\item[{contact}] \leavevmode
\href{mailto:niemi@stsci.edu}{niemi@stsci.edu}

\item[{warning}] \leavevmode
This has never been tested!

\item[{todo}] \leavevmode
add more LFs. There are for example several more from the Bell.

\end{description}\end{quote}

\index{bellG() (in module SamPy.astronomy.luminosityFunctionss)}

\begin{fulllineitems}
\phantomsection\label{SamPy.astronomy:SamPy.astronomy.luminosityFunctionss.bellG}\pysiglinewithargsret{\code{SamPy.astronomy.luminosityFunctionss.}\bfcode{bellG}}{}{}
G-band LF for all galaxies
Schecter Function fit parameters
phi* M* Alpha j  (next line formal errors)
real errors are probably systematic: see Bell et al.2003 for
guidance (the errors depend on passband and/or stellar mass)
0.0172471     -19.7320     -1.02577  1.57122e+08
0.000552186    0.0264740    0.0249926  2.29536e+06
Then we present the V/V\_max data points; x   phi  phi-1sig  phi+1sig
\begin{quote}\begin{description}
\item[{Returns}] \leavevmode
absolute magnitude, phi, phi\_lo, phi\_high

\end{description}\end{quote}

\end{fulllineitems}



\subsubsection{\texttt{metals} Module}
\label{SamPy.astronomy:module-SamPy.astronomy.metals}\label{SamPy.astronomy:metals-module}
\index{SamPy.astronomy.metals (module)}
Observational constrains related to metallicities in galaxies.
\begin{quote}\begin{description}
\item[{requires}] \leavevmode
NumPy

\item[{version}] \leavevmode
0.1

\item[{author}] \leavevmode
Sami-Matias Niemi

\item[{contact}] \leavevmode
\href{mailto:niemi@stsci.edu}{niemi@stsci.edu}

\end{description}\end{quote}

\index{gallazzi() (in module SamPy.astronomy.metals)}

\begin{fulllineitems}
\phantomsection\label{SamPy.astronomy:SamPy.astronomy.metals.gallazzi}\pysiglinewithargsret{\code{SamPy.astronomy.metals.}\bfcode{gallazzi}}{\emph{h=0.7}}{}
Scale stellar masses with h.
n.b. masses in file are for H0=70 so no scaling is required
if using Hubble constant 70.
\begin{quote}\begin{description}
\item[{Parameters}] \leavevmode
\textbf{h} -- Hubble parameter to scale

\end{description}\end{quote}

return:

\end{fulllineitems}



\subsubsection{\texttt{randomizers} Module}
\label{SamPy.astronomy:module-SamPy.astronomy.randomizers}\label{SamPy.astronomy:randomizers-module}
\index{SamPy.astronomy.randomizers (module)}
\index{randomUnitSphere() (in module SamPy.astronomy.randomizers)}

\begin{fulllineitems}
\phantomsection\label{SamPy.astronomy:SamPy.astronomy.randomizers.randomUnitSphere}\pysiglinewithargsret{\code{SamPy.astronomy.randomizers.}\bfcode{randomUnitSphere}}{\emph{points=1}}{}
This function returns random positions
on a unit sphere. The number of random
points returned can be controlled with
the optional points keyword argument.
\begin{quote}\begin{description}
\item[{Parameters}] \leavevmode
\textbf{(int)} (\emph{points}) -- the number of points drawn

\end{description}\end{quote}

\end{fulllineitems}



\subsubsection{\texttt{stellarMFs} Module}
\label{SamPy.astronomy:module-SamPy.astronomy.stellarMFs}\label{SamPy.astronomy:stellarmfs-module}
\index{SamPy.astronomy.stellarMFs (module)}
Different stellar mass functions. Some are based on
observational data while others are fitting functions.
\begin{quote}\begin{description}
\item[{requires}] \leavevmode
NumPy

\item[{version}] \leavevmode
0.2

\item[{author}] \leavevmode
Sami-Matias Niemi

\end{description}\end{quote}

:contact : \href{mailto:niemi@stsci.edu}{niemi@stsci.edu}

\index{bellG() (in module SamPy.astronomy.stellarMFs)}

\begin{fulllineitems}
\phantomsection\label{SamPy.astronomy:SamPy.astronomy.stellarMFs.bellG}\pysiglinewithargsret{\code{SamPy.astronomy.stellarMFs.}\bfcode{bellG}}{\emph{h=0.7}, \emph{chabrier=True}}{}
G-band derived stellar mass function for all galaxies
Schecter Function fit parameters
phi* M* Alpha j  (next line formal errors)
real errors are probably systematic: see Bell et al.2003 for
guidance (the errors depend on passband and/or stellar mass)
0.0101742      10.7003     -1.10350  5.46759e+08
0.000512733    0.0154157    0.0239414  1.06609e+07
Then we present the V/V\_max data points; x   phi  phi-1sig  phi+1sig
Convert to h (default = 0.7) and Chabrier IMF.
\begin{quote}\begin{description}
\item[{Parameters}] \leavevmode\begin{itemize}
\item {} 
\textbf{h} -- the Hubble parameter (default h = 0.7)

\item {} 
\textbf{chabrier} -- whether or not to convert to Chabrier IMF (default = True)

\end{itemize}

\item[{Returns}] \leavevmode
log10(stellar mass), log10(phi), log10(phi\_low), log10(phi\_high)

\end{description}\end{quote}

\end{fulllineitems}


\index{bellK() (in module SamPy.astronomy.stellarMFs)}

\begin{fulllineitems}
\phantomsection\label{SamPy.astronomy:SamPy.astronomy.stellarMFs.bellK}\pysiglinewithargsret{\code{SamPy.astronomy.stellarMFs.}\bfcode{bellK}}{\emph{h=0.7}, \emph{chabrier=True}}{}
K-band derived stellar mass function for all galaxies
Schecter Function fit parameters
phi* M* Alpha j  (next line formal errors)
real errors are probably systematic: see Bell et al.2003 for
guidance (the errors depend on passband and/or stellar mass)
0.0132891      10.6269    -0.856790  5.26440e+08
0.000586424    0.0144582    0.0422601  1.18873e+07
Then we present the V/V\_max data points; x   phi  phi-1sig  phi+1sig
Convert to h (default = 0.7) and Chabrier IMF.
\begin{quote}\begin{description}
\item[{Parameters}] \leavevmode\begin{itemize}
\item {} 
\textbf{h} -- the Hubble parameter (default h = 0.7)

\item {} 
\textbf{chabrier} -- whether or not to convert to Chabrier IMF (default = True)

\end{itemize}

\item[{Returns}] \leavevmode
log10(stellar mass), log10(phi), log10(phi\_low), log10(phi\_high)

\end{description}\end{quote}

\end{fulllineitems}


\index{fstarBehroozi() (in module SamPy.astronomy.stellarMFs)}

\begin{fulllineitems}
\phantomsection\label{SamPy.astronomy:SamPy.astronomy.stellarMFs.fstarBehroozi}\pysiglinewithargsret{\code{SamPy.astronomy.stellarMFs.}\bfcode{fstarBehroozi}}{}{}
Stellar mass to halo mass ratio as a function of halo mass.
Data from Behroozi et al. ???

\end{fulllineitems}


\index{fstarBen() (in module SamPy.astronomy.stellarMFs)}

\begin{fulllineitems}
\phantomsection\label{SamPy.astronomy:SamPy.astronomy.stellarMFs.fstarBen}\pysiglinewithargsret{\code{SamPy.astronomy.stellarMFs.}\bfcode{fstarBen}}{\emph{mh}, \emph{m1}, \emph{f0}, \emph{beta}, \emph{gamma}}{}
Stellar mass to halo mass ratio as a function of halo mass.
Fitting function from Moster et al.

\end{fulllineitems}


\index{highRedshiftMFs() (in module SamPy.astronomy.stellarMFs)}

\begin{fulllineitems}
\phantomsection\label{SamPy.astronomy:SamPy.astronomy.stellarMFs.highRedshiftMFs}\pysiglinewithargsret{\code{SamPy.astronomy.stellarMFs.}\bfcode{highRedshiftMFs}}{}{}
Stellar mass functions from Valentino et al. arXiv:1008.3901v2
These values probably use the Salpeter IMF.
If so then subtract 0.25 dex from log(m*) to get to Chabrier.
Table 1
\begin{quote}

log10 (dN/dlog10 (M/M\_sun)/Mpc3 )
\end{quote}

log10(M/M\_sun)     \#z = 3.8    \#5.0    \#5.9    \#6.8

\end{fulllineitems}


\index{mstar\_bell() (in module SamPy.astronomy.stellarMFs)}

\begin{fulllineitems}
\phantomsection\label{SamPy.astronomy:SamPy.astronomy.stellarMFs.mstar_bell}\pysiglinewithargsret{\code{SamPy.astronomy.stellarMFs.}\bfcode{mstar\_bell}}{\emph{h=0.7}, \emph{chabrier=True}}{}~\begin{quote}\begin{description}
\item[{Warning }] \leavevmode
Data are missing??

\end{description}\end{quote}

Stellar mass function from bell et al. (H0=100).
Convert to Chabrier IMF.

\end{fulllineitems}


\index{mstar\_lin() (in module SamPy.astronomy.stellarMFs)}

\begin{fulllineitems}
\phantomsection\label{SamPy.astronomy:SamPy.astronomy.stellarMFs.mstar_lin}\pysiglinewithargsret{\code{SamPy.astronomy.stellarMFs.}\bfcode{mstar\_lin}}{\emph{h=0.7}}{}~\begin{quote}\begin{description}
\item[{Warning }] \leavevmode
The observation file is missing. Do not use!

\end{description}\end{quote}

\end{fulllineitems}


\index{panter() (in module SamPy.astronomy.stellarMFs)}

\begin{fulllineitems}
\phantomsection\label{SamPy.astronomy:SamPy.astronomy.stellarMFs.panter}\pysiglinewithargsret{\code{SamPy.astronomy.stellarMFs.}\bfcode{panter}}{}{}
This function returns Benjamin Panter's stellar mass function from:
\href{http://www.blackwell-synergy.com/doi/pdf/10.1111/j.1365-2966.2007.11909.x}{http://www.blackwell-synergy.com/doi/pdf/10.1111/j.1365-2966.2007.11909.x}
It uses the DR3 data with the BC03 models and a Chabrier IMF.
:return:

\end{fulllineitems}


\index{stellarMfs() (in module SamPy.astronomy.stellarMFs)}

\begin{fulllineitems}
\phantomsection\label{SamPy.astronomy:SamPy.astronomy.stellarMFs.stellarMfs}\pysiglinewithargsret{\code{SamPy.astronomy.stellarMFs.}\bfcode{stellarMfs}}{}{}~\begin{quote}\begin{description}
\item[{Requires }] \leavevmode
io.sexutils

\item[{Returns}] \leavevmode
sexutils instance to MFs data file.

\end{description}\end{quote}

\end{fulllineitems}



\subsection{bolshoi Package}
\label{SamPy.bolshoi:bolshoi-package}\label{SamPy.bolshoi::doc}

\subsubsection{\texttt{collecthalomasses} Module}
\label{SamPy.bolshoi:collecthalomasses-module}\label{SamPy.bolshoi:module-SamPy.bolshoi.collecthalomasses}
\index{SamPy.bolshoi.collecthalomasses (module)}
This one uses parallel python:
\href{http://www.parallelpython.com}{http://www.parallelpython.com}

\index{findDMhaloes() (in module SamPy.bolshoi.collecthalomasses)}

\begin{fulllineitems}
\phantomsection\label{SamPy.bolshoi:SamPy.bolshoi.collecthalomasses.findDMhaloes}\pysiglinewithargsret{\code{SamPy.bolshoi.collecthalomasses.}\bfcode{findDMhaloes}}{\emph{file}, \emph{times}, \emph{columns}}{}
Find all dark matter haloes from a given file
that are in self.times. Capture the data
that are specified in self.columns.
Returns a dictionary where each key is a single
time. Each line of data is a string so it's easy
to write out to a file.

\end{fulllineitems}


\index{writeOutput() (in module SamPy.bolshoi.collecthalomasses)}

\begin{fulllineitems}
\phantomsection\label{SamPy.bolshoi:SamPy.bolshoi.collecthalomasses.writeOutput}\pysiglinewithargsret{\code{SamPy.bolshoi.collecthalomasses.}\bfcode{writeOutput}}{\emph{data}, \emph{file}, \emph{times}}{}
Writes the output data to ascii files.
Each time step is recorded to a single file.
The filename will contain the output \emph{redshift}

\end{fulllineitems}



\subsubsection{\texttt{collecthalomassesThreading} Module}
\label{SamPy.bolshoi:module-SamPy.bolshoi.collecthalomassesThreading}\label{SamPy.bolshoi:collecthalomassesthreading-module}
\index{SamPy.bolshoi.collecthalomassesThreading (module)}
For some reason this script does not work.
I am not sure what's wrong in it, but the finding
part does not work, but randomly dies. It might
be something to do with threads getting confused
which file they are supposed to open or something...

\index{combineFiles() (in module SamPy.bolshoi.collecthalomassesThreading)}

\begin{fulllineitems}
\phantomsection\label{SamPy.bolshoi:SamPy.bolshoi.collecthalomassesThreading.combineFiles}\pysiglinewithargsret{\code{SamPy.bolshoi.collecthalomassesThreading.}\bfcode{combineFiles}}{\emph{files}, \emph{outputfile}}{}
Combines the content of all files that are listed
in the files list to a single file named outputfile.
Iterates over the input files line-by-line to save
memory.

\end{fulllineitems}


\index{findDMDriver() (in module SamPy.bolshoi.collecthalomassesThreading)}

\begin{fulllineitems}
\phantomsection\label{SamPy.bolshoi:SamPy.bolshoi.collecthalomassesThreading.findDMDriver}\pysiglinewithargsret{\code{SamPy.bolshoi.collecthalomassesThreading.}\bfcode{findDMDriver}}{\emph{input\_files}, \emph{times}, \emph{cores=6}}{}
Main driver function of the wrapper.

\end{fulllineitems}


\index{findDMhaloes (class in SamPy.bolshoi.collecthalomassesThreading)}

\begin{fulllineitems}
\phantomsection\label{SamPy.bolshoi:SamPy.bolshoi.collecthalomassesThreading.findDMhaloes}\pysiglinewithargsret{\strong{class }\code{SamPy.bolshoi.collecthalomassesThreading.}\bfcode{findDMhaloes}}{\emph{queue}, \emph{times}}{}
Bases: \code{threading.Thread}

Threaded way of finding dark matter haloes from
Bolshoi isotree files.

\index{find() (SamPy.bolshoi.collecthalomassesThreading.findDMhaloes method)}

\begin{fulllineitems}
\phantomsection\label{SamPy.bolshoi:SamPy.bolshoi.collecthalomassesThreading.findDMhaloes.find}\pysiglinewithargsret{\bfcode{find}}{\emph{file}}{}
Find all dark matter haloes from a given file
that are in self.times. Capture the data
that are specified in self.columns.
Returns a dictionary where each key is a single
time. Each line of data is a string so it's easy
to write out to a file.

\end{fulllineitems}


\index{run() (SamPy.bolshoi.collecthalomassesThreading.findDMhaloes method)}

\begin{fulllineitems}
\phantomsection\label{SamPy.bolshoi:SamPy.bolshoi.collecthalomassesThreading.findDMhaloes.run}\pysiglinewithargsret{\bfcode{run}}{}{}
Method threading will call.

\end{fulllineitems}


\end{fulllineitems}


\index{writeOutput() (in module SamPy.bolshoi.collecthalomassesThreading)}

\begin{fulllineitems}
\phantomsection\label{SamPy.bolshoi:SamPy.bolshoi.collecthalomassesThreading.writeOutput}\pysiglinewithargsret{\code{SamPy.bolshoi.collecthalomassesThreading.}\bfcode{writeOutput}}{\emph{data}, \emph{file}, \emph{times}}{}
Writes the output data to ascii files.
Each time step is recorded to a single file.
The filename will contain the output \emph{redshift}

\end{fulllineitems}



\subsubsection{\texttt{findRedshiftsBolshoiTrees} Module}
\label{SamPy.bolshoi:findredshiftsbolshoitrees-module}\label{SamPy.bolshoi:module-SamPy.bolshoi.findRedshiftsBolshoiTrees}
\index{SamPy.bolshoi.findRedshiftsBolshoiTrees (module)}

\subsubsection{\texttt{generateTestPlots} Module}
\label{SamPy.bolshoi:generatetestplots-module}\label{SamPy.bolshoi:module-SamPy.bolshoi.generateTestPlots}
\index{SamPy.bolshoi.generateTestPlots (module)}
This script can be used to generate some basic plots
from Rachel's SAM. In these plots comparisons to
observational constrains are also performed.

:author : Sami-Matias Niemi
:contact : \href{mailto:niemi@stsci.edu}{niemi@stsci.edu}

\index{fix() (in module SamPy.bolshoi.generateTestPlots)}

\begin{fulllineitems}
\phantomsection\label{SamPy.bolshoi:SamPy.bolshoi.generateTestPlots.fix}\pysiglinewithargsret{\code{SamPy.bolshoi.generateTestPlots.}\bfcode{fix}}{\emph{x}}{}
IDL function.

\end{fulllineitems}


\index{fstar\_plot() (in module SamPy.bolshoi.generateTestPlots)}

\begin{fulllineitems}
\phantomsection\label{SamPy.bolshoi:SamPy.bolshoi.generateTestPlots.fstar_plot}\pysiglinewithargsret{\code{SamPy.bolshoi.generateTestPlots.}\bfcode{fstar\_plot}}{\emph{behroozi}, \emph{g}, \emph{prop}, \emph{h}, \emph{output}}{}
Stellar mass to halo mass ratio as a function of halo mass.
This plot differs slightly from the one in the Somerville 2008:
the plot in the paper is the fraction of baryons in stars.

\end{fulllineitems}


\index{gasfrac\_hess() (in module SamPy.bolshoi.generateTestPlots)}

\begin{fulllineitems}
\phantomsection\label{SamPy.bolshoi:SamPy.bolshoi.generateTestPlots.gasfrac_hess}\pysiglinewithargsret{\code{SamPy.bolshoi.generateTestPlots.}\bfcode{gasfrac\_hess}}{\emph{mstar}, \emph{fgas}, \emph{weight}, \emph{label}, \emph{output}, \emph{mmin=8.5}, \emph{mmax=11.7}, \emph{nmbins=16}, \emph{fgasmin=0.0}, \emph{fgasmax=1.0}, \emph{nfbins=10}, \emph{pmax=1.0}, \emph{pmin=0.01}}{}
Plot \_conditional\_ distribution of gas fraction vs. stellar mass.
Ported from an IDL code, should be cleaned.
Most likely N.zeros are not needed

\end{fulllineitems}


\index{gasfrac\_sam\_cent() (in module SamPy.bolshoi.generateTestPlots)}

\begin{fulllineitems}
\phantomsection\label{SamPy.bolshoi:SamPy.bolshoi.generateTestPlots.gasfrac_sam_cent}\pysiglinewithargsret{\code{SamPy.bolshoi.generateTestPlots.}\bfcode{gasfrac\_sam\_cent}}{\emph{halos}, \emph{prop}, \emph{label='Bolshoi'}, \emph{output='fgascentral'}, \emph{tpy='.pdf'}}{}
\end{fulllineitems}


\index{main() (in module SamPy.bolshoi.generateTestPlots)}

\begin{fulllineitems}
\phantomsection\label{SamPy.bolshoi:SamPy.bolshoi.generateTestPlots.main}\pysiglinewithargsret{\code{SamPy.bolshoi.generateTestPlots.}\bfcode{main}}{\emph{path}, \emph{label}}{}
Driver function, call this with a path to the data,
and label you wish to use for the files.

\end{fulllineitems}


\index{massfunc\_plot() (in module SamPy.bolshoi.generateTestPlots)}

\begin{fulllineitems}
\phantomsection\label{SamPy.bolshoi:SamPy.bolshoi.generateTestPlots.massfunc_plot}\pysiglinewithargsret{\code{SamPy.bolshoi.generateTestPlots.}\bfcode{massfunc\_plot}}{\emph{halos}, \emph{props}, \emph{obs}, \emph{mmax=12.5}, \emph{mmin=5.0}, \emph{nbins=30}, \emph{output='SMN'}, \emph{nvolumes=20}}{}
Plots stellar mass functions, cold gas mass functions, \& BH MF

\end{fulllineitems}


\index{massmet\_hess() (in module SamPy.bolshoi.generateTestPlots)}

\begin{fulllineitems}
\phantomsection\label{SamPy.bolshoi:SamPy.bolshoi.generateTestPlots.massmet_hess}\pysiglinewithargsret{\code{SamPy.bolshoi.generateTestPlots.}\bfcode{massmet\_hess}}{\emph{mstar}, \emph{met}, \emph{weight}, \emph{obs\_data}, \emph{output}, \emph{mmin=8.8}, \emph{mmax=12.4}, \emph{nmbins=18}, \emph{Zmin=-1.5}, \emph{Zmax=1.0}, \emph{nzbins=20}, \emph{pmax=1.0}, \emph{pmin=0.001}}{}
\end{fulllineitems}


\index{massmet\_star() (in module SamPy.bolshoi.generateTestPlots)}

\begin{fulllineitems}
\phantomsection\label{SamPy.bolshoi:SamPy.bolshoi.generateTestPlots.massmet_star}\pysiglinewithargsret{\code{SamPy.bolshoi.generateTestPlots.}\bfcode{massmet\_star}}{\emph{halos}, \emph{g}, \emph{p}, \emph{obs\_data}, \emph{label}, \emph{typ='.pdf'}}{}
\end{fulllineitems}


\index{mbh\_hess() (in module SamPy.bolshoi.generateTestPlots)}

\begin{fulllineitems}
\phantomsection\label{SamPy.bolshoi:SamPy.bolshoi.generateTestPlots.mbh_hess}\pysiglinewithargsret{\code{SamPy.bolshoi.generateTestPlots.}\bfcode{mbh\_hess}}{\emph{mbulge}, \emph{mbh}, \emph{weight}, \emph{ymin}, \emph{ymax}, \emph{nybins}, \emph{mbin\_fit}, \emph{mbh\_fit}, \emph{output}, \emph{obs}, \emph{mmin=8.0}, \emph{mmax=12.0}, \emph{nmbins=40}, \emph{pmax=1.0}, \emph{pmin=0.001}}{}
\end{fulllineitems}


\index{mhb() (in module SamPy.bolshoi.generateTestPlots)}

\begin{fulllineitems}
\phantomsection\label{SamPy.bolshoi:SamPy.bolshoi.generateTestPlots.mhb}\pysiglinewithargsret{\code{SamPy.bolshoi.generateTestPlots.}\bfcode{mhb}}{\emph{h}, \emph{p}, \emph{obs}, \emph{label}}{}
\end{fulllineitems}


\index{perc\_bin() (in module SamPy.bolshoi.generateTestPlots)}

\begin{fulllineitems}
\phantomsection\label{SamPy.bolshoi:SamPy.bolshoi.generateTestPlots.perc_bin}\pysiglinewithargsret{\code{SamPy.bolshoi.generateTestPlots.}\bfcode{perc\_bin}}{\emph{xbin}, \emph{xdata}, \emph{ydata}}{}
Compute median and 16 and 8 percentiles of y-data in bins in x

\end{fulllineitems}


\index{ssfr\_plot() (in module SamPy.bolshoi.generateTestPlots)}

\begin{fulllineitems}
\phantomsection\label{SamPy.bolshoi:SamPy.bolshoi.generateTestPlots.ssfr_plot}\pysiglinewithargsret{\code{SamPy.bolshoi.generateTestPlots.}\bfcode{ssfr\_plot}}{\emph{halos}, \emph{prop}, \emph{mask}, \emph{output}, \emph{typ='.pdf'}}{}
Specific starformation rate plots.

\end{fulllineitems}


\index{ssfr\_wrapper() (in module SamPy.bolshoi.generateTestPlots)}

\begin{fulllineitems}
\phantomsection\label{SamPy.bolshoi:SamPy.bolshoi.generateTestPlots.ssfr_wrapper}\pysiglinewithargsret{\code{SamPy.bolshoi.generateTestPlots.}\bfcode{ssfr\_wrapper}}{\emph{halos}, \emph{prop}, \emph{label}}{}
\end{fulllineitems}


\index{ssfrhess\_sam() (in module SamPy.bolshoi.generateTestPlots)}

\begin{fulllineitems}
\phantomsection\label{SamPy.bolshoi:SamPy.bolshoi.generateTestPlots.ssfrhess_sam}\pysiglinewithargsret{\code{SamPy.bolshoi.generateTestPlots.}\bfcode{ssfrhess\_sam}}{\emph{mstar}, \emph{ssfr}, \emph{weight}, \emph{output}, \emph{mmin=9.027}, \emph{dm=0.309}, \emph{nmbins=11}, \emph{ssfrmin=-13.0}, \emph{ssfrmax=-8.5}, \emph{nssfrbins=24}, \emph{ssfr\_cut=-11.1}, \emph{pmax=0.5}, \emph{pmin=1e-05}}{}
log ssfr in yr

\end{fulllineitems}



\subsubsection{\texttt{plotMassRatios} Module}
\label{SamPy.bolshoi:plotmassratios-module}\label{SamPy.bolshoi:module-SamPy.bolshoi.plotMassRatios}
\index{SamPy.bolshoi.plotMassRatios (module)}
\index{main() (in module SamPy.bolshoi.plotMassRatios)}

\begin{fulllineitems}
\phantomsection\label{SamPy.bolshoi:SamPy.bolshoi.plotMassRatios.main}\pysiglinewithargsret{\code{SamPy.bolshoi.plotMassRatios.}\bfcode{main}}{\emph{redshifts}, \emph{path}, \emph{database}, \emph{output\_folder}, \emph{outfile}}{}
Driver function, call this with a path to the data,
and label you wish to use for the files.

\end{fulllineitems}


\index{stellarHaloMFRatio() (in module SamPy.bolshoi.plotMassRatios)}

\begin{fulllineitems}
\phantomsection\label{SamPy.bolshoi:SamPy.bolshoi.plotMassRatios.stellarHaloMFRatio}\pysiglinewithargsret{\code{SamPy.bolshoi.plotMassRatios.}\bfcode{stellarHaloMFRatio}}{\emph{path}, \emph{database}, \emph{redshifts}, \emph{output\_folder}, \emph{outfile}, \emph{xmin=10.0}, \emph{xmax=14.4}, \emph{ymin=-3}, \emph{ymax=-1}, \emph{nbins=15}, \emph{h=0.7}}{}
Plots stellar to dark matter halo mass ratios as a function of redshift.

\end{fulllineitems}


\index{stellarHaloMFRatioMultiPanel() (in module SamPy.bolshoi.plotMassRatios)}

\begin{fulllineitems}
\phantomsection\label{SamPy.bolshoi:SamPy.bolshoi.plotMassRatios.stellarHaloMFRatioMultiPanel}\pysiglinewithargsret{\code{SamPy.bolshoi.plotMassRatios.}\bfcode{stellarHaloMFRatioMultiPanel}}{\emph{path}, \emph{database}, \emph{redshifts}, \emph{output\_folder}, \emph{outfile}, \emph{xmin=10.0}, \emph{xmax=14.4}, \emph{ymin=-3}, \emph{ymax=-1}, \emph{nbins=15}}{}
Plots stellar to dark matter halo mass ratios as a function of redshift.

\end{fulllineitems}



\subsubsection{\texttt{plotdarkmattermf} Module}
\label{SamPy.bolshoi:module-SamPy.bolshoi.plotdarkmattermf}\label{SamPy.bolshoi:plotdarkmattermf-module}
\index{SamPy.bolshoi.plotdarkmattermf (module)}
Plots a dark matter halo mass function at different
redshifts. Input data are from the Bolshoi simulation.

@author: Sami-Matias Niemi

\index{compareGalpropzToBolshoiTrees() (in module SamPy.bolshoi.plotdarkmattermf)}

\begin{fulllineitems}
\phantomsection\label{SamPy.bolshoi:SamPy.bolshoi.plotdarkmattermf.compareGalpropzToBolshoiTrees}\pysiglinewithargsret{\code{SamPy.bolshoi.plotdarkmattermf.}\bfcode{compareGalpropzToBolshoiTrees}}{\emph{analyticalData}, \emph{BolshoiTrees}, \emph{redshifts}, \emph{h}, \emph{outputdir}, \emph{no\_phantoms=True}, \emph{galid=True}}{}
\end{fulllineitems}


\index{plotDMMFfromGalpropz() (in module SamPy.bolshoi.plotdarkmattermf)}

\begin{fulllineitems}
\phantomsection\label{SamPy.bolshoi:SamPy.bolshoi.plotdarkmattermf.plotDMMFfromGalpropz}\pysiglinewithargsret{\code{SamPy.bolshoi.plotdarkmattermf.}\bfcode{plotDMMFfromGalpropz}}{\emph{redshift}, \emph{h}, \emph{*data}}{}
\end{fulllineitems}


\index{plotDMMFfromGalpropzAnalytical2() (in module SamPy.bolshoi.plotdarkmattermf)}

\begin{fulllineitems}
\phantomsection\label{SamPy.bolshoi:SamPy.bolshoi.plotdarkmattermf.plotDMMFfromGalpropzAnalytical2}\pysiglinewithargsret{\code{SamPy.bolshoi.plotdarkmattermf.}\bfcode{plotDMMFfromGalpropzAnalytical2}}{\emph{redshift}, \emph{h}, \emph{*data}}{}
\end{fulllineitems}


\index{plot\_mass\_function() (in module SamPy.bolshoi.plotdarkmattermf)}

\begin{fulllineitems}
\phantomsection\label{SamPy.bolshoi:SamPy.bolshoi.plotdarkmattermf.plot_mass_function}\pysiglinewithargsret{\code{SamPy.bolshoi.plotdarkmattermf.}\bfcode{plot\_mass\_function}}{\emph{redshift}, \emph{h}, \emph{no\_phantoms}, \emph{*data}}{}
\end{fulllineitems}


\index{plot\_mass\_functionAnalytical2() (in module SamPy.bolshoi.plotdarkmattermf)}

\begin{fulllineitems}
\phantomsection\label{SamPy.bolshoi:SamPy.bolshoi.plotdarkmattermf.plot_mass_functionAnalytical2}\pysiglinewithargsret{\code{SamPy.bolshoi.plotdarkmattermf.}\bfcode{plot\_mass\_functionAnalytical2}}{\emph{redshift}, \emph{h}, \emph{no\_phantoms}, \emph{*data}}{}
\end{fulllineitems}



\subsubsection{\texttt{plotstellarmf} Module}
\label{SamPy.bolshoi:module-SamPy.bolshoi.plotstellarmf}\label{SamPy.bolshoi:plotstellarmf-module}
\index{SamPy.bolshoi.plotstellarmf (module)}
\index{main() (in module SamPy.bolshoi.plotstellarmf)}

\begin{fulllineitems}
\phantomsection\label{SamPy.bolshoi:SamPy.bolshoi.plotstellarmf.main}\pysiglinewithargsret{\code{SamPy.bolshoi.plotstellarmf.}\bfcode{main}}{\emph{redshifts}, \emph{path}, \emph{database}, \emph{output\_folder}, \emph{outfile}}{}
Driver function, call this with a path to the data,
and label you wish to use for the files.

\end{fulllineitems}


\index{stellarmassfunc\_plot() (in module SamPy.bolshoi.plotstellarmf)}

\begin{fulllineitems}
\phantomsection\label{SamPy.bolshoi:SamPy.bolshoi.plotstellarmf.stellarmassfunc_plot}\pysiglinewithargsret{\code{SamPy.bolshoi.plotstellarmf.}\bfcode{stellarmassfunc\_plot}}{\emph{path}, \emph{database}, \emph{redshifts}, \emph{output\_folder}, \emph{outfile}, \emph{mmax=12.5}, \emph{mmin=8.0}, \emph{nbins=30}, \emph{nvolumes=26}, \emph{single\_volume=50.0}, \emph{h=0.7}, \emph{lowlim=-4.9}}{}
Plots stellar mass functions as a function of redshift.
Compares to observations.

\end{fulllineitems}



\subsubsection{\texttt{randomizeSubhaloPosition} Module}
\label{SamPy.bolshoi:module-SamPy.bolshoi.randomizeSubhaloPosition}\label{SamPy.bolshoi:randomizesubhaloposition-module}
\index{SamPy.bolshoi.randomizeSubhaloPosition (module)}
\index{plotRandomization() (in module SamPy.bolshoi.randomizeSubhaloPosition)}

\begin{fulllineitems}
\phantomsection\label{SamPy.bolshoi:SamPy.bolshoi.randomizeSubhaloPosition.plotRandomization}\pysiglinewithargsret{\code{SamPy.bolshoi.randomizeSubhaloPosition.}\bfcode{plotRandomization}}{}{}
\end{fulllineitems}


\index{plotTestRandmizer() (in module SamPy.bolshoi.randomizeSubhaloPosition)}

\begin{fulllineitems}
\phantomsection\label{SamPy.bolshoi:SamPy.bolshoi.randomizeSubhaloPosition.plotTestRandmizer}\pysiglinewithargsret{\code{SamPy.bolshoi.randomizeSubhaloPosition.}\bfcode{plotTestRandmizer}}{\emph{size}, \emph{rds}, \emph{fudge=1.1}}{}
\end{fulllineitems}


\index{testRandomizer() (in module SamPy.bolshoi.randomizeSubhaloPosition)}

\begin{fulllineitems}
\phantomsection\label{SamPy.bolshoi:SamPy.bolshoi.randomizeSubhaloPosition.testRandomizer}\pysiglinewithargsret{\code{SamPy.bolshoi.randomizeSubhaloPosition.}\bfcode{testRandomizer}}{}{}
\end{fulllineitems}



\subsection{candels Package}
\label{SamPy.candels:candels-package}\label{SamPy.candels::doc}

\subsubsection{\texttt{plotFilters} Module}
\label{SamPy.candels:module-SamPy.candels.plotFilters}\label{SamPy.candels:plotfilters-module}
\index{SamPy.candels.plotFilters (module)}
\index{InsetPosition (class in SamPy.candels.plotFilters)}

\begin{fulllineitems}
\phantomsection\label{SamPy.candels:SamPy.candels.plotFilters.InsetPosition}\pysiglinewithargsret{\strong{class }\code{SamPy.candels.plotFilters.}\bfcode{InsetPosition}}{\emph{parent}, \emph{lbwh}}{}
Bases: \code{object}

\end{fulllineitems}


\index{main() (in module SamPy.candels.plotFilters)}

\begin{fulllineitems}
\phantomsection\label{SamPy.candels:SamPy.candels.plotFilters.main}\pysiglinewithargsret{\code{SamPy.candels.plotFilters.}\bfcode{main}}{\emph{color}}{}
\end{fulllineitems}


\index{main2() (in module SamPy.candels.plotFilters)}

\begin{fulllineitems}
\phantomsection\label{SamPy.candels:SamPy.candels.plotFilters.main2}\pysiglinewithargsret{\code{SamPy.candels.plotFilters.}\bfcode{main2}}{\emph{color}, \emph{size='x-large'}}{}
\end{fulllineitems}


\index{read\_data() (in module SamPy.candels.plotFilters)}

\begin{fulllineitems}
\phantomsection\label{SamPy.candels:SamPy.candels.plotFilters.read_data}\pysiglinewithargsret{\code{SamPy.candels.plotFilters.}\bfcode{read\_data}}{\emph{filename}}{}
\end{fulllineitems}



\subsection{cos Package}
\label{SamPy.cos::doc}\label{SamPy.cos:cos-package}

\subsubsection{\texttt{plot\_GET} Module}
\label{SamPy.cos:plot-get-module}\label{SamPy.cos:module-SamPy.cos.plot_GET}
\index{SamPy.cos.plot\_GET (module)}
\index{fromJulian() (in module SamPy.cos.plot\_GET)}

\begin{fulllineitems}
\phantomsection\label{SamPy.cos:SamPy.cos.plot_GET.fromJulian}\pysiglinewithargsret{\code{SamPy.cos.plot\_GET.}\bfcode{fromJulian}}{\emph{j}}{}
Converts Julian Date to human readable format
@return: human readable date and time

\end{fulllineitems}



\subsubsection{\texttt{plot\_all\_GET} Module}
\label{SamPy.cos:plot-all-get-module}\label{SamPy.cos:module-SamPy.cos.plot_all_GET}
\index{SamPy.cos.plot\_all\_GET (module)}
\index{findDifferentRatios() (in module SamPy.cos.plot\_all\_GET)}

\begin{fulllineitems}
\phantomsection\label{SamPy.cos:SamPy.cos.plot_all_GET.findDifferentRatios}\pysiglinewithargsret{\code{SamPy.cos.plot\_all\_GET.}\bfcode{findDifferentRatios}}{\emph{dict}}{}
\end{fulllineitems}


\index{fromJulian() (in module SamPy.cos.plot\_all\_GET)}

\begin{fulllineitems}
\phantomsection\label{SamPy.cos:SamPy.cos.plot_all_GET.fromJulian}\pysiglinewithargsret{\code{SamPy.cos.plot\_all\_GET.}\bfcode{fromJulian}}{\emph{j}}{}
Converts Julian Date to human readable format
@return: human readable date and time

\end{fulllineitems}


\index{mylinearregression() (in module SamPy.cos.plot\_all\_GET)}

\begin{fulllineitems}
\phantomsection\label{SamPy.cos:SamPy.cos.plot_all_GET.mylinearregression}\pysiglinewithargsret{\code{SamPy.cos.plot\_all\_GET.}\bfcode{mylinearregression}}{\emph{x}, \emph{y}, \emph{confidence\_level=95}, \emph{show\_plots=False}, \emph{report=False}}{}
function to calculate simple linear regression
inputs: x,y-data pairs and a confidence level (in percent)
outputs: slope = b1, intercept = b0 its confidence intervals
and R (+squared)

\end{fulllineitems}



\subsection{cosmology Package}
\label{SamPy.cosmology:cosmology-package}\label{SamPy.cosmology::doc}

\subsubsection{\texttt{cc} Module}
\label{SamPy.cosmology:cc-module}\label{SamPy.cosmology:module-SamPy.cosmology.cc}
\index{SamPy.cosmology.cc (module)}

\subsubsection{\texttt{distances} Module}
\label{SamPy.cosmology:distances-module}\label{SamPy.cosmology:module-SamPy.cosmology.distances}
\index{SamPy.cosmology.distances (module)}
Plot some distance measures versus redshift and omega\_M.
\begin{quote}\begin{description}
\item[{requires}] \leavevmode
CosmoloPy

\end{description}\end{quote}

\index{plot\_DA() (in module SamPy.cosmology.distances)}

\begin{fulllineitems}
\phantomsection\label{SamPy.cosmology:SamPy.cosmology.distances.plot_DA}\pysiglinewithargsret{\code{SamPy.cosmology.distances.}\bfcode{plot\_DA}}{\emph{filename}}{}
The dimensionless angular diameter distance DA/DH.

\end{fulllineitems}


\index{plot\_DM() (in module SamPy.cosmology.distances)}

\begin{fulllineitems}
\phantomsection\label{SamPy.cosmology:SamPy.cosmology.distances.plot_DM}\pysiglinewithargsret{\code{SamPy.cosmology.distances.}\bfcode{plot\_DM}}{\emph{filename}}{}
The dimensionless proper motion distance DM/DH.

\end{fulllineitems}


\index{plot\_dist() (in module SamPy.cosmology.distances)}

\begin{fulllineitems}
\phantomsection\label{SamPy.cosmology:SamPy.cosmology.distances.plot_dist}\pysiglinewithargsret{\code{SamPy.cosmology.distances.}\bfcode{plot\_dist}}{\emph{z}, \emph{dz}, \emph{om}, \emph{dom}, \emph{dist}, \emph{dh}, \emph{name}, \emph{mathname}, \emph{filename=None}}{}
Make a 2-D plot of a distance versus redshift (x) and matter density (y).

\end{fulllineitems}


\index{plot\_dist\_ony() (in module SamPy.cosmology.distances)}

\begin{fulllineitems}
\phantomsection\label{SamPy.cosmology:SamPy.cosmology.distances.plot_dist_ony}\pysiglinewithargsret{\code{SamPy.cosmology.distances.}\bfcode{plot\_dist\_ony}}{\emph{z}, \emph{dz}, \emph{om}, \emph{dom}, \emph{dist}, \emph{dh}, \emph{name}, \emph{mathname}, \emph{filename=None}}{}
Make a 2-D plot of matter density versus redshift (x) and distance (y)

\end{fulllineitems}



\subsection{cuda Package}
\label{SamPy.cuda:cuda-package}\label{SamPy.cuda::doc}

\subsubsection{\texttt{first\_example} Module}
\label{SamPy.cuda:first-example-module}

\subsection{dates Package}
\label{SamPy.dates::doc}\label{SamPy.dates:dates-package}

\subsubsection{\texttt{NumberDate} Module}
\label{SamPy.dates:numberdate-module}\label{SamPy.dates:module-SamPy.dates.NumberDate}
\index{SamPy.dates.NumberDate (module)}\begin{description}
\item[{ABOUT:}] \leavevmode
This script converts the day number to a human readable date.

\item[{USAGE:   }] \leavevmode
NumberDate day\_number

E.g NumberDate 250

\item[{DEPENDS:}] \leavevmode
Python 2.6 (no 3.0 compatible)

\item[{EXITSTA:  }] \leavevmode\begin{quote}

0: No errors
\end{quote}

-8: Invalid day\_number
-9: No day\_number given

\item[{HISTORY:}] \leavevmode
Sep 9 2009: Initial Version 0.1

\end{description}

@author: Sami-Matias Niemi (\href{mailto:niemi@stsci.edu}{niemi@stsci.edu}) for STScI

\index{print\_day\_number() (in module SamPy.dates.NumberDate)}

\begin{fulllineitems}
\phantomsection\label{SamPy.dates:SamPy.dates.NumberDate.print_day_number}\pysiglinewithargsret{\code{SamPy.dates.NumberDate.}\bfcode{print\_day\_number}}{\emph{day}}{}
\end{fulllineitems}



\subsubsection{\texttt{julians} Module}
\label{SamPy.dates:julians-module}\label{SamPy.dates:module-SamPy.dates.julians}
\index{SamPy.dates.julians (module)}
\index{HSTdayToRealDate() (in module SamPy.dates.julians)}

\begin{fulllineitems}
\phantomsection\label{SamPy.dates:SamPy.dates.julians.HSTdayToRealDate}\pysiglinewithargsret{\code{SamPy.dates.julians.}\bfcode{HSTdayToRealDate}}{\emph{hstday}}{}
\end{fulllineitems}


\index{fromHSTDeployment() (in module SamPy.dates.julians)}

\begin{fulllineitems}
\phantomsection\label{SamPy.dates:SamPy.dates.julians.fromHSTDeployment}\pysiglinewithargsret{\code{SamPy.dates.julians.}\bfcode{fromHSTDeployment}}{\emph{julian}}{}
@return: number of days since HST was deployed (24 Apr 1990)

\end{fulllineitems}


\index{fromJulian() (in module SamPy.dates.julians)}

\begin{fulllineitems}
\phantomsection\label{SamPy.dates:SamPy.dates.julians.fromJulian}\pysiglinewithargsret{\code{SamPy.dates.julians.}\bfcode{fromJulian}}{\emph{j}}{}
Converts Modified Julian days to human readable format
@return: human readable date and time

\end{fulllineitems}


\index{toJulian() (in module SamPy.dates.julians)}

\begin{fulllineitems}
\phantomsection\label{SamPy.dates:SamPy.dates.julians.toJulian}\pysiglinewithargsret{\code{SamPy.dates.julians.}\bfcode{toJulian}}{\emph{year}, \emph{month}, \emph{day}, \emph{hour}, \emph{minute}, \emph{timezone='UTC'}}{}
@param: 
year, month, day, hour, minute, and timezone 
Uses time functions.
@return: Julian Date

\end{fulllineitems}


\index{toJulian2() (in module SamPy.dates.julians)}

\begin{fulllineitems}
\phantomsection\label{SamPy.dates:SamPy.dates.julians.toJulian2}\pysiglinewithargsret{\code{SamPy.dates.julians.}\bfcode{toJulian2}}{\emph{date}}{}
Converts date and time to Modified Julian Date.
Uses time functions. Note that date has to be in Python time format.

\end{fulllineitems}



\subsection{db Package}
\label{SamPy.db:db-package}\label{SamPy.db::doc}

\subsubsection{\texttt{DB} Module}
\label{SamPy.db:module-SamPy.db.DB}\label{SamPy.db:db-module}
\index{SamPy.db.DB (module)}
This file contains a wrapper class to fetch data from a Sybase or MySQL database.

@author: Sami-Matias Niemi (\href{mailto:niemi@stsci.edu}{niemi@stsci.edu}) for STScI.

\index{DBSMN (class in SamPy.db.DB)}

\begin{fulllineitems}
\phantomsection\label{SamPy.db:SamPy.db.DB.DBSMN}\pysiglinewithargsret{\strong{class }\code{SamPy.db.DB.}\bfcode{DBSMN}}{\emph{sql}, \emph{user}, \emph{password}, \emph{database}, \emph{address}}{}
Wrapper class to fetch data from a database. Separate functions
for Sybase and MySQL database. A new class instance must be
initialised for every server/database combination.

\index{fetchMySQLData() (SamPy.db.DB.DBSMN method)}

\begin{fulllineitems}
\phantomsection\label{SamPy.db:SamPy.db.DB.DBSMN.fetchMySQLData}\pysiglinewithargsret{\bfcode{fetchMySQLData}}{}{}
Function to fetch data from MySQL database.
Fetches all data from the database given in the class constructor.
Returns the fetched data as an array.

\end{fulllineitems}


\index{fetchSybaseData() (SamPy.db.DB.DBSMN method)}

\begin{fulllineitems}
\phantomsection\label{SamPy.db:SamPy.db.DB.DBSMN.fetchSybaseData}\pysiglinewithargsret{\bfcode{fetchSybaseData}}{}{}
Function to fetch data from Sybase database.
Fetches all data from the database given in the class constructor.
Returns the fetched data as an array.

\end{fulllineitems}


\end{fulllineitems}



\subsubsection{\texttt{MSdata} Module}
\label{SamPy.db:msdata-module}\label{SamPy.db:module-SamPy.db.MSdata}
\index{SamPy.db.MSdata (module)}
\index{MillenniumData (class in SamPy.db.MSdata)}

\begin{fulllineitems}
\phantomsection\label{SamPy.db:SamPy.db.MSdata.MillenniumData}\pysiglinewithargsret{\strong{class }\code{SamPy.db.MSdata.}\bfcode{MillenniumData}}{\emph{sql}}{}
This class was designed to fetch data from the Millennium Simulation database.
Input for the constructor is SQL query to be performed. Syntax must be valid
SQL.

\index{dataonly() (SamPy.db.MSdata.MillenniumData method)}

\begin{fulllineitems}
\phantomsection\label{SamPy.db:SamPy.db.MSdata.MillenniumData.dataonly}\pysiglinewithargsret{\bfcode{dataonly}}{\emph{data}, \emph{splitvalue}}{}
\end{fulllineitems}


\index{fetchdata() (SamPy.db.MSdata.MillenniumData method)}

\begin{fulllineitems}
\phantomsection\label{SamPy.db:SamPy.db.MSdata.MillenniumData.fetchdata}\pysiglinewithargsret{\bfcode{fetchdata}}{}{}
\end{fulllineitems}


\index{iterateGalaxies() (SamPy.db.MSdata.MillenniumData method)}

\begin{fulllineitems}
\phantomsection\label{SamPy.db:SamPy.db.MSdata.MillenniumData.iterateGalaxies}\pysiglinewithargsret{\bfcode{iterateGalaxies}}{\emph{galaxies}}{}
\end{fulllineitems}


\index{savetofile() (SamPy.db.MSdata.MillenniumData method)}

\begin{fulllineitems}
\phantomsection\label{SamPy.db:SamPy.db.MSdata.MillenniumData.savetofile}\pysiglinewithargsret{\bfcode{savetofile}}{\emph{data}, \emph{filename}}{}
\end{fulllineitems}


\index{savetofileseparated() (SamPy.db.MSdata.MillenniumData method)}

\begin{fulllineitems}
\phantomsection\label{SamPy.db:SamPy.db.MSdata.MillenniumData.savetofileseparated}\pysiglinewithargsret{\bfcode{savetofileseparated}}{\emph{data}, \emph{filename}, \emph{separator}}{}
\end{fulllineitems}


\end{fulllineitems}



\subsubsection{\texttt{MySQLdbSMN} Module}
\label{SamPy.db:module-SamPy.db.MySQLdbSMN}\label{SamPy.db:mysqldbsmn-module}
\index{SamPy.db.MySQLdbSMN (module)}
\index{MySQLdbSMN (class in SamPy.db.MySQLdbSMN)}

\begin{fulllineitems}
\phantomsection\label{SamPy.db:SamPy.db.MySQLdbSMN.MySQLdbSMN}\pysiglinewithargsret{\strong{class }\code{SamPy.db.MySQLdbSMN.}\bfcode{MySQLdbSMN}}{\emph{sql}, \emph{database}}{}~
\index{fetchdata() (SamPy.db.MySQLdbSMN.MySQLdbSMN method)}

\begin{fulllineitems}
\phantomsection\label{SamPy.db:SamPy.db.MySQLdbSMN.MySQLdbSMN.fetchdata}\pysiglinewithargsret{\bfcode{fetchdata}}{}{}
\end{fulllineitems}


\end{fulllineitems}



\subsubsection{\texttt{insertSAMTablesToMySQL} Module}
\label{SamPy.db:module-SamPy.db.insertSAMTablesToMySQL}\label{SamPy.db:insertsamtablestomysql-module}
\index{SamPy.db.insertSAMTablesToMySQL (module)}
\index{addToMySQLDBfromSAMTables() (in module SamPy.db.insertSAMTablesToMySQL)}

\begin{fulllineitems}
\phantomsection\label{SamPy.db:SamPy.db.insertSAMTablesToMySQL.addToMySQLDBfromSAMTables}\pysiglinewithargsret{\code{SamPy.db.insertSAMTablesToMySQL.}\bfcode{addToMySQLDBfromSAMTables}}{\emph{user='sammy'}, \emph{passwd='Asd1Zxc8'}, \emph{host='localhost'}, \emph{database='SAM100GOODS'}, \emph{fileidentifier='*.dat'}}{}
This little function can be used to add a table to a
mySQL database from Rachel's GF output.

The script will make a table out from each ascii 
output file. The halo\_id and gal\_id columns of
each table are indexed for faster table joining.
Each index is names as table\_id, albeit there
should be no need to know the name of the index.
\begin{quote}\begin{description}
\item[{Parameters}] \leavevmode\begin{itemize}
\item {} 
\textbf{user} -- user name to be used

\item {} 
\textbf{passwd} -- user password

\item {} 
\textbf{host} -- address of the database host

\item {} 
\textbf{database} -- name of the database to which insert

\item {} 
\textbf{fileidentifier} -- string how to identify input data

\end{itemize}

\end{description}\end{quote}

\end{fulllineitems}



\subsubsection{\texttt{insertSAMTablesToSQLite} Module}
\label{SamPy.db:module-SamPy.db.insertSAMTablesToSQLite}\label{SamPy.db:insertsamtablestosqlite-module}
\index{SamPy.db.insertSAMTablesToSQLite (module)}
This module contains a function that can be used to
generate an SQLite3 database from ascii files.
\begin{quote}\begin{description}
\item[{author}] \leavevmode
Sami-Matias Niemi

\item[{contact}] \leavevmode
\href{mailto:niemi@stsci.edu}{niemi@stsci.edu}

\end{description}\end{quote}

\index{generateSQLiteDBfromSAMTables() (in module SamPy.db.insertSAMTablesToSQLite)}

\begin{fulllineitems}
\phantomsection\label{SamPy.db:SamPy.db.insertSAMTablesToSQLite.generateSQLiteDBfromSAMTables}\pysiglinewithargsret{\code{SamPy.db.insertSAMTablesToSQLite.}\bfcode{generateSQLiteDBfromSAMTables}}{\emph{output='sams.db'}, \emph{fileidentifier='*.dat'}}{}
This little function can be used to generate an
SQLite3 database from Rachel's GF output.

The script will make a table out from each ascii 
output file. The halo\_id and gal\_id columns of
each table are indexed for faster table joining.
Each index is names as table\_id, albeit there
should be no need to know the name of the index.
\begin{quote}\begin{description}
\item[{Parameters}] \leavevmode\begin{itemize}
\item {} 
\textbf{output} (\emph{string}) -- name of the output file

\item {} 
\textbf{fileidentifier} (\emph{string}) -- string how to identify input data

\end{itemize}

\end{description}\end{quote}

\end{fulllineitems}



\subsubsection{\texttt{sqlite} Module}
\label{SamPy.db:sqlite-module}\label{SamPy.db:module-SamPy.db.sqlite}
\index{SamPy.db.sqlite (module)}
This file contains SQLite3 related functions.

@requires: NumPy

@author: Sami-Matias Niemi
@version: 0.1

\index{SSFR() (in module SamPy.db.sqlite)}

\begin{fulllineitems}
\phantomsection\label{SamPy.db:SamPy.db.sqlite.SSFR}\pysiglinewithargsret{\code{SamPy.db.sqlite.}\bfcode{SSFR}}{\emph{mstardot}, \emph{mstar}}{}
Log\_10(value1 / 10**value2)
@note: This function can be passed on to slite3 connection
@param mstardot: star formation rate
@param mstar: stellar mass in log10(M\_solar)
@return: specific star formation rate in Gyr**-1

\end{fulllineitems}


\index{generateSQLString() (in module SamPy.db.sqlite)}

\begin{fulllineitems}
\phantomsection\label{SamPy.db:SamPy.db.sqlite.generateSQLString}\pysiglinewithargsret{\code{SamPy.db.sqlite.}\bfcode{generateSQLString}}{\emph{columns}, \emph{format}, \emph{start}}{}
Generates an SQL string from two vectors that
describe the name of the column and the format.
Can be used to assist when generating a string
to create a new table.

\end{fulllineitems}


\index{get\_data\_sqlite() (in module SamPy.db.sqlite)}

\begin{fulllineitems}
\phantomsection\label{SamPy.db:SamPy.db.sqlite.get_data_sqlite}\pysiglinewithargsret{\code{SamPy.db.sqlite.}\bfcode{get\_data\_sqlite}}{\emph{path}, \emph{db}, \emph{query}}{}
This function can be used to pull out data
from an slite3 database. Output is given as
a numpy array for ease of further processing.
@param path: path to the db file, should end with /
@param db: name of the database file
@param query: query to be performed
@return: numpy array of data

\end{fulllineitems}


\index{get\_data\_sqlitePowerTen() (in module SamPy.db.sqlite)}

\begin{fulllineitems}
\phantomsection\label{SamPy.db:SamPy.db.sqlite.get_data_sqlitePowerTen}\pysiglinewithargsret{\code{SamPy.db.sqlite.}\bfcode{get\_data\_sqlitePowerTen}}{\emph{path}, \emph{db}, \emph{query}}{}
Run an SQL query to a database with a custom
made function ``toPowerTen''.
@param path: path to the SQLite3 database
@param db: name of the SQLite3 database
@param query: valid SQL query
@return: all data in a NumPy array

\end{fulllineitems}


\index{get\_data\_sqliteSMNfunctions() (in module SamPy.db.sqlite)}

\begin{fulllineitems}
\phantomsection\label{SamPy.db:SamPy.db.sqlite.get_data_sqliteSMNfunctions}\pysiglinewithargsret{\code{SamPy.db.sqlite.}\bfcode{get\_data\_sqliteSMNfunctions}}{\emph{path}, \emph{db}, \emph{query}}{}
Run an SQL query to a database with custom
made functions ``toPowerTen'' and ``janskyToMagnitude''.
@param path: path to the SQLite3 database
@param db: name of the SQLite3 database
@param query: valid SQL query
@return: all data in a NumPy array

\end{fulllineitems}


\index{parseColumnNamesSAMTables() (in module SamPy.db.sqlite)}

\begin{fulllineitems}
\phantomsection\label{SamPy.db:SamPy.db.sqlite.parseColumnNamesSAMTables}\pysiglinewithargsret{\code{SamPy.db.sqlite.}\bfcode{parseColumnNamesSAMTables}}{\emph{filename}, \emph{commentchar='\#'}, \emph{colnumber=2}}{}
Parse column names from a text file that follows
SExtractor format, i.e., columns are specified in
the beginning of the file. Each column are specified
in a single line that starts with a comment charachter.
The line assumed to follow the following format:
\# number name
For example:
\# 1 the\_first\_column.
In the case of the example, the function would return
a list {[}'the\_first\_column',{]}.

\end{fulllineitems}


\index{toLogTen() (in module SamPy.db.sqlite)}

\begin{fulllineitems}
\phantomsection\label{SamPy.db:SamPy.db.sqlite.toLogTen}\pysiglinewithargsret{\code{SamPy.db.sqlite.}\bfcode{toLogTen}}{\emph{value}}{}
@note: This function can be passed on to slite3 connection
@param value: can either be a number or a NumPy array 
@return: Log\_10(value)

\end{fulllineitems}


\index{toPowerTen() (in module SamPy.db.sqlite)}

\begin{fulllineitems}
\phantomsection\label{SamPy.db:SamPy.db.sqlite.toPowerTen}\pysiglinewithargsret{\code{SamPy.db.sqlite.}\bfcode{toPowerTen}}{\emph{value}}{}
@note: This function can be passed on to slite3 connection
@param value: can either be a number or a NumPy array
@return: 10**value

\end{fulllineitems}



\subsection{finance Package}
\label{SamPy.finance:finance-package}\label{SamPy.finance::doc}

\subsubsection{\texttt{finance} Module}
\label{SamPy.finance:finance-module}\label{SamPy.finance:module-SamPy.finance.finance}
\index{SamPy.finance.finance (module)}
\index{MyLocator (class in SamPy.finance.finance)}

\begin{fulllineitems}
\phantomsection\label{SamPy.finance:SamPy.finance.finance.MyLocator}\pysiglinewithargsret{\strong{class }\code{SamPy.finance.finance.}\bfcode{MyLocator}}{\emph{*args}, \emph{**kwargs}}{}
Bases: \code{matplotlib.ticker.MaxNLocator}

\end{fulllineitems}


\index{moving\_average() (in module SamPy.finance.finance)}

\begin{fulllineitems}
\phantomsection\label{SamPy.finance:SamPy.finance.finance.moving_average}\pysiglinewithargsret{\code{SamPy.finance.finance.}\bfcode{moving\_average}}{\emph{x}, \emph{n}, \emph{type='simple'}}{}
compute an n period moving average.
type is `simple' \textbar{} `exponential'

\end{fulllineitems}


\index{moving\_average\_convergence() (in module SamPy.finance.finance)}

\begin{fulllineitems}
\phantomsection\label{SamPy.finance:SamPy.finance.finance.moving_average_convergence}\pysiglinewithargsret{\code{SamPy.finance.finance.}\bfcode{moving\_average\_convergence}}{\emph{x}, \emph{nslow=26}, \emph{nfast=12}}{}~\begin{description}
\item[{compute the MACD (Moving Average Convergence/Divergence) using a fast}] \leavevmode
and slow exponential moving avg'

\end{description}

return value is emaslow, emafast, macd which are len(x) arrays

\end{fulllineitems}


\index{relative\_strength() (in module SamPy.finance.finance)}

\begin{fulllineitems}
\phantomsection\label{SamPy.finance:SamPy.finance.finance.relative_strength}\pysiglinewithargsret{\code{SamPy.finance.finance.}\bfcode{relative\_strength}}{\emph{prices}, \emph{n=14}}{}
compute the n period relative strength indicator
\href{http://stockcharts.com/school/doku.php?id=chart\_school:glossary\_r\#relativestrengthindex}{http://stockcharts.com/school/doku.php?id=chart\_school:glossary\_r\#relativestrengthindex}
\href{http://www.investopedia.com/terms/r/rsi.asp}{http://www.investopedia.com/terms/r/rsi.asp}

\end{fulllineitems}



\subsubsection{\texttt{simple\_example} Module}
\label{SamPy.finance:simple-example-module}\label{SamPy.finance:module-SamPy.finance.simple_example}
\index{SamPy.finance.simple\_example (module)}
Created on Fri Jun 25 09:26:40 2010

@author: -


\subsection{fits Package}
\label{SamPy.fits::doc}\label{SamPy.fits:fits-package}

\subsubsection{\texttt{ShowHeader} Module}
\label{SamPy.fits:module-SamPy.fits.ShowHeader}\label{SamPy.fits:showheader-module}
\index{SamPy.fits.ShowHeader (module)}
Extremely simple script that can be used to print FITS headers to stdout.

Accepts wildcard in the name, but then the filename must be given inside quote marks
i.e. ``{\color{red}\bfseries{}*}.fits''
\begin{quote}\begin{description}
\item[{date}] \leavevmode
Mar 27, 2009

\item[{author}] \leavevmode
Sami-Matias Niemi

\item[{contact}] \leavevmode
\href{mailto:niemi@stsci.edu}{niemi@stsci.edu}

\end{description}\end{quote}

\index{containsAll() (in module SamPy.fits.ShowHeader)}

\begin{fulllineitems}
\phantomsection\label{SamPy.fits:SamPy.fits.ShowHeader.containsAll}\pysiglinewithargsret{\code{SamPy.fits.ShowHeader.}\bfcode{containsAll}}{\emph{str}, \emph{set}}{}
Checks if a given string contains all characters in a given set.
\begin{quote}\begin{description}
\item[{Parameters}] \leavevmode\begin{itemize}
\item {} 
\textbf{str} (\emph{string}) -- input string

\item {} 
\textbf{set} (\emph{string}) -- set if characters

\end{itemize}

\item[{Return type}] \leavevmode
boolean

\end{description}\end{quote}

\end{fulllineitems}


\index{containsAny() (in module SamPy.fits.ShowHeader)}

\begin{fulllineitems}
\phantomsection\label{SamPy.fits:SamPy.fits.ShowHeader.containsAny}\pysiglinewithargsret{\code{SamPy.fits.ShowHeader.}\bfcode{containsAny}}{\emph{str}, \emph{set}}{}
Checks if a given string contains any of the characters in a given set.
\begin{quote}\begin{description}
\item[{Parameters}] \leavevmode\begin{itemize}
\item {} 
\textbf{str} (\emph{string}) -- input string

\item {} 
\textbf{set} (\emph{string}) -- set if characters

\end{itemize}

\item[{Return type}] \leavevmode
boolean

\end{description}\end{quote}

\end{fulllineitems}


\index{showHeader() (in module SamPy.fits.ShowHeader)}

\begin{fulllineitems}
\phantomsection\label{SamPy.fits:SamPy.fits.ShowHeader.showHeader}\pysiglinewithargsret{\code{SamPy.fits.ShowHeader.}\bfcode{showHeader}}{\emph{filename}, \emph{extension}}{}
Shows the FITS header of a given file.
\begin{quote}\begin{description}
\item[{Parameters}] \leavevmode\begin{itemize}
\item {} 
\textbf{filename} (\emph{string}) -- name of the file

\item {} 
\textbf{extension} (\emph{integer}) -- number of the FITS extension

\end{itemize}

\end{description}\end{quote}

\end{fulllineitems}



\subsubsection{\texttt{combine} Module}
\label{SamPy.fits:combine-module}\label{SamPy.fits:module-SamPy.fits.combine}
\index{SamPy.fits.combine (module)}\begin{quote}\begin{description}
\item[{about}] \leavevmode
This script is a quick way of combining fits files

\item[{usage}] \leavevmode
reduce.py {[}-n{]} {[}-m{]} {[}-a{]} {[}-f{]} \textless{}file list or name\textgreater{} {[}-o{]} \textless{}output file\textgreater{}
where:
\begin{quote}

{[}-m{]} median combine
{[}-a{]} average combine
{[}-f{]} combines given files
{[}-o{]} name of the output file
{[}-n{]} uses NumPy and PyFits rather than Pyraf
\end{quote}

\item[{date}] \leavevmode
26/11/2008 Initial Release

\item[{author}] \leavevmode
Sami-Matias Niemi

\item[{contact}] \leavevmode
\href{mailto:niemi@stsci.edu}{niemi@stsci.edu}

\end{description}\end{quote}

\index{combine() (in module SamPy.fits.combine)}

\begin{fulllineitems}
\phantomsection\label{SamPy.fits:SamPy.fits.combine.combine}\pysiglinewithargsret{\code{SamPy.fits.combine.}\bfcode{combine}}{\emph{filelist}, \emph{median=False}, \emph{verbose=False}}{}
Combines given images.
\begin{quote}\begin{description}
\item[{Parameters}] \leavevmode\begin{itemize}
\item {} 
\textbf{filelist} (\emph{list}) -- list of files to be combined.

\item {} 
\textbf{median} (\emph{boolean}) -- Performs median combining, if False uses average values

\item {} 
\textbf{verbose} (\emph{boolean}) -- verbose mode.

\end{itemize}

\item[{Returns}] \leavevmode
The combined image (2d-array with pixel values)

\end{description}\end{quote}

\end{fulllineitems}


\index{combinePyraf() (in module SamPy.fits.combine)}

\begin{fulllineitems}
\phantomsection\label{SamPy.fits:SamPy.fits.combine.combinePyraf}\pysiglinewithargsret{\code{SamPy.fits.combine.}\bfcode{combinePyraf}}{\emph{input}, \emph{output}, \emph{median}, \emph{scale}}{}
Combines given images using IRAF's imcombine.

\end{fulllineitems}


\index{process\_args() (in module SamPy.fits.combine)}

\begin{fulllineitems}
\phantomsection\label{SamPy.fits:SamPy.fits.combine.process_args}\pysiglinewithargsret{\code{SamPy.fits.combine.}\bfcode{process\_args}}{\emph{just\_print\_help=False}}{}
Processes the command line arguments o FITSCombine.py data reduction script.
\begin{quote}\begin{description}
\item[{Parameters}] \leavevmode
\textbf{just\_print\_help} (\emph{boolean}) -- will print help

\item[{Returns}] \leavevmode
parsed commmand line options

\end{description}\end{quote}

\end{fulllineitems}



\subsection{fitting Package}
\label{SamPy.fitting::doc}\label{SamPy.fitting:fitting-package}

\subsubsection{\texttt{SplineArtist} Module}
\label{SamPy.fitting:splineartist-module}\label{SamPy.fitting:module-SamPy.fitting.SplineArtist}
\index{SamPy.fitting.SplineArtist (module)}
\index{SplineArtist (class in SamPy.fitting.SplineArtist)}

\begin{fulllineitems}
\phantomsection\label{SamPy.fitting:SamPy.fitting.SplineArtist.SplineArtist}\pysiglinewithargsret{\strong{class }\code{SamPy.fitting.SplineArtist.}\bfcode{SplineArtist}}{\emph{xdata}, \emph{ydata}, \emph{xrange}, \emph{ylog=False}, \emph{order=3}}{}~
\index{draw\_spline() (SamPy.fitting.SplineArtist.SplineArtist method)}

\begin{fulllineitems}
\phantomsection\label{SamPy.fitting:SamPy.fitting.SplineArtist.SplineArtist.draw_spline}\pysiglinewithargsret{\bfcode{draw\_spline}}{}{}
Draws a spline function through the nods that have been set.

\end{fulllineitems}


\index{get\_spline() (SamPy.fitting.SplineArtist.SplineArtist method)}

\begin{fulllineitems}
\phantomsection\label{SamPy.fitting:SamPy.fitting.SplineArtist.SplineArtist.get_spline}\pysiglinewithargsret{\bfcode{get\_spline}}{}{}
\end{fulllineitems}


\index{onclick() (SamPy.fitting.SplineArtist.SplineArtist method)}

\begin{fulllineitems}
\phantomsection\label{SamPy.fitting:SamPy.fitting.SplineArtist.SplineArtist.onclick}\pysiglinewithargsret{\bfcode{onclick}}{\emph{event}}{}
Define a mouse click event:

\end{fulllineitems}


\index{onpress() (SamPy.fitting.SplineArtist.SplineArtist method)}

\begin{fulllineitems}
\phantomsection\label{SamPy.fitting:SamPy.fitting.SplineArtist.SplineArtist.onpress}\pysiglinewithargsret{\bfcode{onpress}}{\emph{event}}{}
Define key press events:
q/Q quits the whole script!

\end{fulllineitems}


\index{plot\_diff() (SamPy.fitting.SplineArtist.SplineArtist method)}

\begin{fulllineitems}
\phantomsection\label{SamPy.fitting:SamPy.fitting.SplineArtist.SplineArtist.plot_diff}\pysiglinewithargsret{\bfcode{plot\_diff}}{}{}
\end{fulllineitems}


\index{run() (SamPy.fitting.SplineArtist.SplineArtist method)}

\begin{fulllineitems}
\phantomsection\label{SamPy.fitting:SamPy.fitting.SplineArtist.SplineArtist.run}\pysiglinewithargsret{\bfcode{run}}{}{}
\end{fulllineitems}


\end{fulllineitems}



\subsubsection{\texttt{SplineFitting} Module}
\label{SamPy.fitting:module-SamPy.fitting.SplineFitting}\label{SamPy.fitting:splinefitting-module}
\index{SamPy.fitting.SplineFitting (module)}
ABOUT:
Example how to spline B-spline to fake data.

DEPENDS:
Python 2.5 or later (no 3.x compatible)
NumPy
SciPy

TESTED:
Python 2.5.1
NumPy: 1.4.0.dev7576
SciPy: 0.7.1
matplotlib 1.0.svn

HISTORY:
Created on November 26, 2009

VERSION:
0.1: test release (SMN)

@author: Sami-Matias Niemi (\href{mailto:niemi@stsci.edu}{niemi@stsci.edu})

\index{SplineFitting (class in SamPy.fitting.SplineFitting)}

\begin{fulllineitems}
\phantomsection\label{SamPy.fitting:SamPy.fitting.SplineFitting.SplineFitting}\pysiglinewithargsret{\strong{class }\code{SamPy.fitting.SplineFitting.}\bfcode{SplineFitting}}{\emph{xnodes}, \emph{spline\_order=3}}{}
Fits a B-spline representation of 1-D curve.
Uses Levenberg-Marquardt algorithm for minimizing 
the sum of squares.

\index{doFit() (SamPy.fitting.SplineFitting.SplineFitting method)}

\begin{fulllineitems}
\phantomsection\label{SamPy.fitting:SamPy.fitting.SplineFitting.SplineFitting.doFit}\pysiglinewithargsret{\bfcode{doFit}}{\emph{ynodes}, \emph{x}, \emph{y}}{}
Return the point which minimizes the sum of squares of M (non-linear)
equations in N unknowns given a starting estimate, x0, using a
modification of the Levenberg-Marquardt algorithm.
@return: fitted parameters, error/success message

\end{fulllineitems}


\index{errfunc() (SamPy.fitting.SplineFitting.SplineFitting method)}

\begin{fulllineitems}
\phantomsection\label{SamPy.fitting:SamPy.fitting.SplineFitting.SplineFitting.errfunc}\pysiglinewithargsret{\bfcode{errfunc}}{\emph{ynodes}, \emph{x}, \emph{y}}{}
Error function.
@return: fit - ydata

\end{fulllineitems}


\index{fitfunc() (SamPy.fitting.SplineFitting.SplineFitting method)}

\begin{fulllineitems}
\phantomsection\label{SamPy.fitting:SamPy.fitting.SplineFitting.SplineFitting.fitfunc}\pysiglinewithargsret{\bfcode{fitfunc}}{\emph{x}, \emph{ynodes}}{}
Function that is fitted.
This can be changed to whatever function.
Note that ynodes can then be a list of parameters.
@return: 1-D B-spline value at each x.

\end{fulllineitems}


\end{fulllineitems}



\subsubsection{\texttt{SplineFitting2D} Module}
\label{SamPy.fitting:splinefitting2d-module}\label{SamPy.fitting:module-SamPy.fitting.SplineFitting2D}
\index{SamPy.fitting.SplineFitting2D (module)}
Created on Thu Nov 26 22:00:20 2009

Author: josef-pktd and scipy mailinglist example

\index{makeLSQspline() (in module SamPy.fitting.SplineFitting2D)}

\begin{fulllineitems}
\phantomsection\label{SamPy.fitting:SamPy.fitting.SplineFitting2D.makeLSQspline}\pysiglinewithargsret{\code{SamPy.fitting.SplineFitting2D.}\bfcode{makeLSQspline}}{\emph{xl}, \emph{yl}, \emph{xr}, \emph{yr}}{}
docstring for makespline

\end{fulllineitems}



\subsubsection{\texttt{fits} Module}
\label{SamPy.fitting:module-SamPy.fitting.fits}\label{SamPy.fitting:fits-module}
\index{SamPy.fitting.fits (module)}
\index{FindZeroDoubleExp() (in module SamPy.fitting.fits)}

\begin{fulllineitems}
\phantomsection\label{SamPy.fitting:SamPy.fitting.fits.FindZeroDoubleExp}\pysiglinewithargsret{\code{SamPy.fitting.fits.}\bfcode{FindZeroDoubleExp}}{\emph{p}, \emph{x0}, \emph{yshift=0.0}}{}
\end{fulllineitems}


\index{FindZeroSingleExp() (in module SamPy.fitting.fits)}

\begin{fulllineitems}
\phantomsection\label{SamPy.fitting:SamPy.fitting.fits.FindZeroSingleExp}\pysiglinewithargsret{\code{SamPy.fitting.fits.}\bfcode{FindZeroSingleExp}}{\emph{p}, \emph{x0}, \emph{yshift=0.0}}{}
\end{fulllineitems}


\index{FitDoubleExponent() (in module SamPy.fitting.fits)}

\begin{fulllineitems}
\phantomsection\label{SamPy.fitting:SamPy.fitting.fits.FitDoubleExponent}\pysiglinewithargsret{\code{SamPy.fitting.fits.}\bfcode{FitDoubleExponent}}{\emph{xcorr}, \emph{ycorr}, \emph{initials}}{}
Fits a double exponential to data.
@return: corrfit

\end{fulllineitems}


\index{FitExponent() (in module SamPy.fitting.fits)}

\begin{fulllineitems}
\phantomsection\label{SamPy.fitting:SamPy.fitting.fits.FitExponent}\pysiglinewithargsret{\code{SamPy.fitting.fits.}\bfcode{FitExponent}}{\emph{xcorr}, \emph{ycorr}, \emph{initials}}{}
Fits an exponential to data.

\end{fulllineitems}


\index{PolyFit() (in module SamPy.fitting.fits)}

\begin{fulllineitems}
\phantomsection\label{SamPy.fitting:SamPy.fitting.fits.PolyFit}\pysiglinewithargsret{\code{SamPy.fitting.fits.}\bfcode{PolyFit}}{\emph{x}, \emph{y}, \emph{order=1}}{}
Fits a polynomial to the data.
@return: 
fitted y values, error of the fit

\end{fulllineitems}


\index{doubleExponent() (in module SamPy.fitting.fits)}

\begin{fulllineitems}
\phantomsection\label{SamPy.fitting:SamPy.fitting.fits.doubleExponent}\pysiglinewithargsret{\code{SamPy.fitting.fits.}\bfcode{doubleExponent}}{\emph{x}, \emph{p}, \emph{yshift}}{}
\end{fulllineitems}


\index{linearregression() (in module SamPy.fitting.fits)}

\begin{fulllineitems}
\phantomsection\label{SamPy.fitting:SamPy.fitting.fits.linearregression}\pysiglinewithargsret{\code{SamPy.fitting.fits.}\bfcode{linearregression}}{\emph{x}, \emph{y}, \emph{confidence\_level=95}, \emph{show\_plots=False}, \emph{report=False}}{}
A function to calculate simple linear regression
inputs: x,y-data pairs and a confidence level (in percent)
outputs: slope = b1, intercept = b0 its confidence intervals
and R (+squared)

\end{fulllineitems}


\index{singleExponent() (in module SamPy.fitting.fits)}

\begin{fulllineitems}
\phantomsection\label{SamPy.fitting:SamPy.fitting.fits.singleExponent}\pysiglinewithargsret{\code{SamPy.fitting.fits.}\bfcode{singleExponent}}{\emph{x}, \emph{p}, \emph{yshift}}{}
\end{fulllineitems}



\subsubsection{\texttt{weigtedFittingExample} Module}
\label{SamPy.fitting:weigtedfittingexample-module}\label{SamPy.fitting:module-SamPy.fitting.weigtedFittingExample}
\index{SamPy.fitting.weigtedFittingExample (module)}
\index{first\_order() (in module SamPy.fitting.weigtedFittingExample)}

\begin{fulllineitems}
\phantomsection\label{SamPy.fitting:SamPy.fitting.weigtedFittingExample.first_order}\pysiglinewithargsret{\code{SamPy.fitting.weigtedFittingExample.}\bfcode{first\_order}}{\emph{x}, \emph{a0}, \emph{a1}}{}
\end{fulllineitems}


\index{rand() (in module SamPy.fitting.weigtedFittingExample)}

\begin{fulllineitems}
\phantomsection\label{SamPy.fitting:SamPy.fitting.weigtedFittingExample.rand}\pysiglinewithargsret{\code{SamPy.fitting.weigtedFittingExample.}\bfcode{rand}}{}{}
rand(d0, d1, ..., dn)

Random values in a given shape.

Create an array of the given shape and propagate it with
random samples from a uniform distribution
over \code{{[}0, 1)}.
\begin{description}
\item[{d0, d1, ..., dn}] \leavevmode{[}int{]}
Shape of the output.

\end{description}
\begin{description}
\item[{out}] \leavevmode{[}ndarray, shape \code{(d0, d1, ..., dn)}{]}
Random values.

\end{description}

random

This is a convenience function. If you want an interface that
takes a shape-tuple as the first argument, refer to
\emph{random}.

\begin{Verbatim}[commandchars=\\\{\}]
\PYG{g+gp}{\textgreater{}\textgreater{}\textgreater{} }\PYG{n}{np}\PYG{o}{.}\PYG{n}{random}\PYG{o}{.}\PYG{n}{rand}\PYG{p}{(}\PYG{l+m+mi}{3}\PYG{p}{,}\PYG{l+m+mi}{2}\PYG{p}{)}
\PYG{g+go}{array([[ 0.14022471,  0.96360618],  \PYGZsh{}random}
\PYG{g+go}{       [ 0.37601032,  0.25528411],  \PYGZsh{}random}
\PYG{g+go}{       [ 0.49313049,  0.94909878]]) \PYGZsh{}random}
\end{Verbatim}

\end{fulllineitems}


\index{second\_order() (in module SamPy.fitting.weigtedFittingExample)}

\begin{fulllineitems}
\phantomsection\label{SamPy.fitting:SamPy.fitting.weigtedFittingExample.second_order}\pysiglinewithargsret{\code{SamPy.fitting.weigtedFittingExample.}\bfcode{second\_order}}{\emph{x}, \emph{a0}, \emph{a1}, \emph{a2}}{}
\end{fulllineitems}


\index{third\_order() (in module SamPy.fitting.weigtedFittingExample)}

\begin{fulllineitems}
\phantomsection\label{SamPy.fitting:SamPy.fitting.weigtedFittingExample.third_order}\pysiglinewithargsret{\code{SamPy.fitting.weigtedFittingExample.}\bfcode{third\_order}}{\emph{x}, \emph{a0}, \emph{a1}, \emph{a2}, \emph{a3}}{}
\end{fulllineitems}



\subsection{focus Package}
\label{SamPy.focus:focus-package}\label{SamPy.focus::doc}

\subsubsection{\texttt{ACSImageExtensionerTinyTim} Module}
\label{SamPy.focus:acsimageextensionertinytim-module}\label{SamPy.focus:module-SamPy.focus.ACSImageExtensionerTinyTim}
\index{SamPy.focus.ACSImageExtensionerTinyTim (module)}
Created on March 7, 2011
\begin{quote}\begin{description}
\item[{author}] \leavevmode
Sami-Matias Niemi

\item[{contact}] \leavevmode
\href{mailto:niemi@stsci.edu}{niemi@stsci.edu}

\item[{version}] \leavevmode
0.1

\end{description}\end{quote}

\index{parse\_parameterfile() (in module SamPy.focus.ACSImageExtensionerTinyTim)}

\begin{fulllineitems}
\phantomsection\label{SamPy.focus:SamPy.focus.ACSImageExtensionerTinyTim.parse_parameterfile}\pysiglinewithargsret{\code{SamPy.focus.ACSImageExtensionerTinyTim.}\bfcode{parse\_parameterfile}}{\emph{file='./complete\_results/parameters.txt'}}{}
\end{fulllineitems}



\subsubsection{\texttt{FocusModel} Module}
\label{SamPy.focus:focusmodel-module}\label{SamPy.focus:module-SamPy.focus.FocusModel}
\index{SamPy.focus.FocusModel (module)}
User Interface for Focus Monitor

\index{FocusMenu (class in SamPy.focus.FocusModel)}

\begin{fulllineitems}
\phantomsection\label{SamPy.focus:SamPy.focus.FocusModel.FocusMenu}\pysigline{\strong{class }\code{SamPy.focus.FocusModel.}\bfcode{FocusMenu}}{}
Bases: \code{object}

\index{ChooseDate() (SamPy.focus.FocusModel.FocusMenu method)}

\begin{fulllineitems}
\phantomsection\label{SamPy.focus:SamPy.focus.FocusModel.FocusMenu.ChooseDate}\pysiglinewithargsret{\bfcode{ChooseDate}}{\emph{event}}{}
\end{fulllineitems}


\index{CreateForm() (SamPy.focus.FocusModel.FocusMenu method)}

\begin{fulllineitems}
\phantomsection\label{SamPy.focus:SamPy.focus.FocusModel.FocusMenu.CreateForm}\pysiglinewithargsret{\bfcode{CreateForm}}{}{}
\end{fulllineitems}


\index{Finish() (SamPy.focus.FocusModel.FocusMenu method)}

\begin{fulllineitems}
\phantomsection\label{SamPy.focus:SamPy.focus.FocusModel.FocusMenu.Finish}\pysiglinewithargsret{\bfcode{Finish}}{}{}
\end{fulllineitems}


\index{GetDate() (SamPy.focus.FocusModel.FocusMenu method)}

\begin{fulllineitems}
\phantomsection\label{SamPy.focus:SamPy.focus.FocusModel.FocusMenu.GetDate}\pysiglinewithargsret{\bfcode{GetDate}}{}{}
\end{fulllineitems}


\index{GetTimes() (SamPy.focus.FocusModel.FocusMenu method)}

\begin{fulllineitems}
\phantomsection\label{SamPy.focus:SamPy.focus.FocusModel.FocusMenu.GetTimes}\pysiglinewithargsret{\bfcode{GetTimes}}{}{}
\end{fulllineitems}


\index{Show() (SamPy.focus.FocusModel.FocusMenu method)}

\begin{fulllineitems}
\phantomsection\label{SamPy.focus:SamPy.focus.FocusModel.FocusMenu.Show}\pysiglinewithargsret{\bfcode{Show}}{}{}
\end{fulllineitems}


\index{StartGraph() (SamPy.focus.FocusModel.FocusMenu method)}

\begin{fulllineitems}
\phantomsection\label{SamPy.focus:SamPy.focus.FocusModel.FocusMenu.StartGraph}\pysiglinewithargsret{\bfcode{StartGraph}}{}{}
\end{fulllineitems}


\end{fulllineitems}


\index{comparison() (in module SamPy.focus.FocusModel)}

\begin{fulllineitems}
\phantomsection\label{SamPy.focus:SamPy.focus.FocusModel.comparison}\pysiglinewithargsret{\code{SamPy.focus.FocusModel.}\bfcode{comparison}}{\emph{camera}, \emph{date}}{}
\end{fulllineitems}


\index{fromJulian() (in module SamPy.focus.FocusModel)}

\begin{fulllineitems}
\phantomsection\label{SamPy.focus:SamPy.focus.FocusModel.fromJulian}\pysiglinewithargsret{\code{SamPy.focus.FocusModel.}\bfcode{fromJulian}}{\emph{j}}{}
\end{fulllineitems}


\index{graph() (in module SamPy.focus.FocusModel)}

\begin{fulllineitems}
\phantomsection\label{SamPy.focus:SamPy.focus.FocusModel.graph}\pysiglinewithargsret{\code{SamPy.focus.FocusModel.}\bfcode{graph}}{}{}
\end{fulllineitems}


\index{measured() (in module SamPy.focus.FocusModel)}

\begin{fulllineitems}
\phantomsection\label{SamPy.focus:SamPy.focus.FocusModel.measured}\pysiglinewithargsret{\code{SamPy.focus.FocusModel.}\bfcode{measured}}{\emph{camera}, \emph{testDate}}{}
Extract focus measurements

\end{fulllineitems}


\index{modelled() (in module SamPy.focus.FocusModel)}

\begin{fulllineitems}
\phantomsection\label{SamPy.focus:SamPy.focus.FocusModel.modelled}\pysiglinewithargsret{\code{SamPy.focus.FocusModel.}\bfcode{modelled}}{\emph{camera}, \emph{date}, \emph{startTime}, \emph{stopTime}}{}
Date in form of a string: month day  year. Times as 15:33 - 24 hour clock

\end{fulllineitems}


\index{toJulian() (in module SamPy.focus.FocusModel)}

\begin{fulllineitems}
\phantomsection\label{SamPy.focus:SamPy.focus.FocusModel.toJulian}\pysiglinewithargsret{\code{SamPy.focus.FocusModel.}\bfcode{toJulian}}{\emph{year}, \emph{month}, \emph{day}}{}
Use time functions

\end{fulllineitems}



\subsubsection{\texttt{FocusPlots} Module}
\label{SamPy.focus:module-SamPy.focus.FocusPlots}\label{SamPy.focus:focusplots-module}
\index{SamPy.focus.FocusPlots (module)}
DESCRIPTION:
Creates few different plots from the focus data.

USAGE:
python focusPlots.py

HISTORY:
Created on Sep 10, 2009
Added to the repository on Dec 3, 2010
\begin{quote}\begin{description}
\item[{author}] \leavevmode
Sami-Matias Niemi

\item[{todo}] \leavevmode
\end{description}\end{quote}
\begin{enumerate}
\item {} 
change focus trend since mirror move to two x axis mode (one with date)

\item {} 
Create a new plot: all focus data since last mirror move, fit functions

\end{enumerate}

\index{FocusTrend() (in module SamPy.focus.FocusPlots)}

\begin{fulllineitems}
\phantomsection\label{SamPy.focus:SamPy.focus.FocusPlots.FocusTrend}\pysiglinewithargsret{\code{SamPy.focus.FocusPlots.}\bfcode{FocusTrend}}{\emph{xmin}, \emph{xmax}, \emph{title}, \emph{type}, \emph{input\_folder}, \emph{output\_folder}, \emph{output='FocusTrend'}}{}
Plots Focus trend since given minimum J-L date (mxin).
xmax is used to limit the fit.

\end{fulllineitems}


\index{FocusTrendNoBreathing() (in module SamPy.focus.FocusPlots)}

\begin{fulllineitems}
\phantomsection\label{SamPy.focus:SamPy.focus.FocusPlots.FocusTrendNoBreathing}\pysiglinewithargsret{\code{SamPy.focus.FocusPlots.}\bfcode{FocusTrendNoBreathing}}{\emph{xmin}, \emph{xmax}, \emph{title}, \emph{type}, \emph{input\_folder}, \emph{output\_folder}, \emph{output='FocusTrend'}}{}
Plots Focus trend since given minimum J-L date (mxin).
Uses data that has not been breathing corrected.
xmax is used to limit the fit.

\end{fulllineitems}


\index{FocusTrendRemoveLatestMovement() (in module SamPy.focus.FocusPlots)}

\begin{fulllineitems}
\phantomsection\label{SamPy.focus:SamPy.focus.FocusPlots.FocusTrendRemoveLatestMovement}\pysiglinewithargsret{\code{SamPy.focus.FocusPlots.}\bfcode{FocusTrendRemoveLatestMovement}}{\emph{xmin}, \emph{xmax}, \emph{title}, \emph{type}, \emph{input\_folder}, \emph{output\_folder}, \emph{output='FocusTrendUptoDate'}}{}
@param xmin: minimum Modified Julian Date to be plotted
@param xmax: maximum Modified Julian Date to be used for the fits
@param title: title of the plot
@param output: name of the output file

Plots Focus trend since xmin while taking into account the last mirror move.
The latest mirror move is subtracted from all the previous data.
Straight line and an exponential are fitted to the all data since xmin.

\end{fulllineitems}


\index{FocusTrendRemoveLatestMovementNoBreathing() (in module SamPy.focus.FocusPlots)}

\begin{fulllineitems}
\phantomsection\label{SamPy.focus:SamPy.focus.FocusPlots.FocusTrendRemoveLatestMovementNoBreathing}\pysiglinewithargsret{\code{SamPy.focus.FocusPlots.}\bfcode{FocusTrendRemoveLatestMovementNoBreathing}}{\emph{xmin}, \emph{xmax}, \emph{title}, \emph{type}, \emph{input\_folder}, \emph{output\_folder}, \emph{output='FocusTrendUptoDateNoBreathing'}}{}
@param xmin: minimum Modified Julian Date to be plotted
@param xmax: maximum Modified Julian Date to be used for the fits
@param title: title of the plot
@param output: name of the output file

Plots Focus trend since xmin while taking into account the last mirror move.
The latest mirror move is subtracted from all the previous data.
Straight line and an exponential are fitted to the all data since xmin.

\end{fulllineitems}


\index{FocusTrendRemoveLatestMovementNoBreathingOffset() (in module SamPy.focus.FocusPlots)}

\begin{fulllineitems}
\phantomsection\label{SamPy.focus:SamPy.focus.FocusPlots.FocusTrendRemoveLatestMovementNoBreathingOffset}\pysiglinewithargsret{\code{SamPy.focus.FocusPlots.}\bfcode{FocusTrendRemoveLatestMovementNoBreathingOffset}}{\emph{xmin}, \emph{xmax}, \emph{title}, \emph{type}, \emph{input\_folder}, \emph{output\_folder}, \emph{output='FocusTrendUptoDateNoBreathingOffset'}}{}
@param xmin: minimum Modified Julian Date to be plotted
@param xmax: maximum Modified Julian Date to be used for the fits
@param title: title of the plot
@param output: name of the output file

Plots Focus trend since xmin while taking into account the last mirror move.
The latest mirror move is subtracted from all the previous data.
Straight line and an exponential are fitted to the all data since xmin.

\end{fulllineitems}


\index{FocusTrendRemoveLatestMovementOffset() (in module SamPy.focus.FocusPlots)}

\begin{fulllineitems}
\phantomsection\label{SamPy.focus:SamPy.focus.FocusPlots.FocusTrendRemoveLatestMovementOffset}\pysiglinewithargsret{\code{SamPy.focus.FocusPlots.}\bfcode{FocusTrendRemoveLatestMovementOffset}}{\emph{xmin}, \emph{xmax}, \emph{title}, \emph{type}, \emph{input\_folder}, \emph{output\_folder}, \emph{output='FocusTrendUptoDateOffset'}, \emph{WFC3offset=0.5}}{}
@param xmin: minimum Modified Julian Date to be plotted
@param xmax: maximum Modified Julian Date to be used for the fits
@param title: title of the plot
@param output: name of the output file

Plots Focus trend since xmin while taking into account the last mirror move.
The latest mirror move is subtracted from all the previous data.
Straight line and an exponential are fitted to the all data since xmin.

\end{fulllineitems}


\index{FocusTrendSinceDayZero() (in module SamPy.focus.FocusPlots)}

\begin{fulllineitems}
\phantomsection\label{SamPy.focus:SamPy.focus.FocusPlots.FocusTrendSinceDayZero}\pysiglinewithargsret{\code{SamPy.focus.FocusPlots.}\bfcode{FocusTrendSinceDayZero}}{\emph{title}, \emph{output}, \emph{input\_folder}, \emph{output\_folder}, \emph{stepFunction=False}, \emph{filename='AllData.txt'}, \emph{endday=8100}}{}
\end{fulllineitems}


\index{FocusTrendSinceDayZeroDates() (in module SamPy.focus.FocusPlots)}

\begin{fulllineitems}
\phantomsection\label{SamPy.focus:SamPy.focus.FocusPlots.FocusTrendSinceDayZeroDates}\pysiglinewithargsret{\code{SamPy.focus.FocusPlots.}\bfcode{FocusTrendSinceDayZeroDates}}{\emph{title}, \emph{output}, \emph{input\_folder}, \emph{output\_folder}, \emph{stepFunction=False}, \emph{filename='AllData.txt'}}{}
\end{fulllineitems}


\index{FocusTrendSinceDayZeroDates2() (in module SamPy.focus.FocusPlots)}

\begin{fulllineitems}
\phantomsection\label{SamPy.focus:SamPy.focus.FocusPlots.FocusTrendSinceDayZeroDates2}\pysiglinewithargsret{\code{SamPy.focus.FocusPlots.}\bfcode{FocusTrendSinceDayZeroDates2}}{\emph{output}, \emph{input\_folder}, \emph{output\_folder}, \emph{filename='AllData.txt'}}{}
Plots the overall focus trend since the HST launch. Will not plot errors as they
are smaller or similar size to the markers.

\end{fulllineitems}


\index{FocusTrendSinceDayZeroOLD() (in module SamPy.focus.FocusPlots)}

\begin{fulllineitems}
\phantomsection\label{SamPy.focus:SamPy.focus.FocusPlots.FocusTrendSinceDayZeroOLD}\pysiglinewithargsret{\code{SamPy.focus.FocusPlots.}\bfcode{FocusTrendSinceDayZeroOLD}}{\emph{title}, \emph{output}, \emph{input\_folder}, \emph{output\_folder}, \emph{filename='comp2009allfocus.txt'}}{}
@deprecated: This function is no longer used. Please see the other two functions.

\end{fulllineitems}


\index{confocality() (in module SamPy.focus.FocusPlots)}

\begin{fulllineitems}
\phantomsection\label{SamPy.focus:SamPy.focus.FocusPlots.confocality}\pysiglinewithargsret{\code{SamPy.focus.FocusPlots.}\bfcode{confocality}}{\emph{type}, \emph{input\_folder}, \emph{output\_folder}}{}
Creates a plot where WFC3 UVIS focus is compared to ACS WFC.

\end{fulllineitems}


\index{findMaxAndPair() (in module SamPy.focus.FocusPlots)}

\begin{fulllineitems}
\phantomsection\label{SamPy.focus:SamPy.focus.FocusPlots.findMaxAndPair}\pysiglinewithargsret{\code{SamPy.focus.FocusPlots.}\bfcode{findMaxAndPair}}{\emph{data}}{}
\end{fulllineitems}



\subsubsection{\texttt{HSTfocus} Module}
\label{SamPy.focus:module-SamPy.focus.HSTfocus}\label{SamPy.focus:hstfocus-module}
\index{SamPy.focus.HSTfocus (module)}
\index{MirrorMoves() (in module SamPy.focus.HSTfocus)}

\begin{fulllineitems}
\phantomsection\label{SamPy.focus:SamPy.focus.HSTfocus.MirrorMoves}\pysiglinewithargsret{\code{SamPy.focus.HSTfocus.}\bfcode{MirrorMoves}}{}{}
HST Secondary mirror moves and amounts.

\end{fulllineitems}


\index{MirrorMovesInHSTTime() (in module SamPy.focus.HSTfocus)}

\begin{fulllineitems}
\phantomsection\label{SamPy.focus:SamPy.focus.HSTfocus.MirrorMovesInHSTTime}\pysiglinewithargsret{\code{SamPy.focus.HSTfocus.}\bfcode{MirrorMovesInHSTTime}}{}{}
\end{fulllineitems}



\subsubsection{\texttt{PhaseretrievalresultsTinyTim} Module}
\label{SamPy.focus:phaseretrievalresultstinytim-module}\label{SamPy.focus:module-SamPy.focus.PhaseretrievalresultsTinyTim}
\index{SamPy.focus.PhaseretrievalresultsTinyTim (module)}
HISTORY:
Created on March 3, 2011
\begin{quote}\begin{description}
\item[{author}] \leavevmode
Sami-Matias Niemi

\item[{contact}] \leavevmode
\href{mailto:niemi@stsci.edu}{niemi@stsci.edu}

\item[{version}] \leavevmode
0.1

\end{description}\end{quote}

\index{TinyTimResults (class in SamPy.focus.PhaseretrievalresultsTinyTim)}

\begin{fulllineitems}
\phantomsection\label{SamPy.focus:SamPy.focus.PhaseretrievalresultsTinyTim.TinyTimResults}\pysiglinewithargsret{\strong{class }\code{SamPy.focus.PhaseretrievalresultsTinyTim.}\bfcode{TinyTimResults}}{\emph{cameras}, \emph{str}}{}
Bases: \code{focus.phaseretrievalresults.PhaseRetResults}

Expansion to PhaseRetResults class

\index{findRealFocus() (SamPy.focus.PhaseretrievalresultsTinyTim.TinyTimResults method)}

\begin{fulllineitems}
\phantomsection\label{SamPy.focus:SamPy.focus.PhaseretrievalresultsTinyTim.TinyTimResults.findRealFocus}\pysiglinewithargsret{\bfcode{findRealFocus}}{\emph{file}}{}
\end{fulllineitems}


\index{plotFocusDifference() (SamPy.focus.PhaseretrievalresultsTinyTim.TinyTimResults method)}

\begin{fulllineitems}
\phantomsection\label{SamPy.focus:SamPy.focus.PhaseretrievalresultsTinyTim.TinyTimResults.plotFocusDifference}\pysiglinewithargsret{\bfcode{plotFocusDifference}}{\emph{input}, \emph{title}, \emph{abs=False}}{}
\end{fulllineitems}


\index{plotFocusFieldPosition() (SamPy.focus.PhaseretrievalresultsTinyTim.TinyTimResults method)}

\begin{fulllineitems}
\phantomsection\label{SamPy.focus:SamPy.focus.PhaseretrievalresultsTinyTim.TinyTimResults.plotFocusFieldPosition}\pysiglinewithargsret{\bfcode{plotFocusFieldPosition}}{\emph{input}, \emph{title}}{}
\end{fulllineitems}


\end{fulllineitems}



\subsubsection{\texttt{PhaseretrievalresultsTinyTimACS} Module}
\label{SamPy.focus:phaseretrievalresultstinytimacs-module}\label{SamPy.focus:module-SamPy.focus.PhaseretrievalresultsTinyTimACS}
\index{SamPy.focus.PhaseretrievalresultsTinyTimACS (module)}
HISTORY:
Created on March 3, 2011
\begin{quote}\begin{description}
\item[{author}] \leavevmode
Sami-Matias Niemi

\item[{contact}] \leavevmode
\href{mailto:niemi@stsci.edu}{niemi@stsci.edu}

\item[{version}] \leavevmode
0.1

\end{description}\end{quote}

\index{TinyTimResults (class in SamPy.focus.PhaseretrievalresultsTinyTimACS)}

\begin{fulllineitems}
\phantomsection\label{SamPy.focus:SamPy.focus.PhaseretrievalresultsTinyTimACS.TinyTimResults}\pysiglinewithargsret{\strong{class }\code{SamPy.focus.PhaseretrievalresultsTinyTimACS.}\bfcode{TinyTimResults}}{\emph{cameras}, \emph{str}}{}
Bases: {\hyperref[SamPy.focus:SamPy.focus.phaseretrievalresults.PhaseRetResults]{\code{SamPy.focus.phaseretrievalresults.PhaseRetResults}}}

Expansion to PhaseRetResults class

\index{findRealFocus() (SamPy.focus.PhaseretrievalresultsTinyTimACS.TinyTimResults method)}

\begin{fulllineitems}
\phantomsection\label{SamPy.focus:SamPy.focus.PhaseretrievalresultsTinyTimACS.TinyTimResults.findRealFocus}\pysiglinewithargsret{\bfcode{findRealFocus}}{\emph{file}}{}
\end{fulllineitems}


\index{plotFocusDifference() (SamPy.focus.PhaseretrievalresultsTinyTimACS.TinyTimResults method)}

\begin{fulllineitems}
\phantomsection\label{SamPy.focus:SamPy.focus.PhaseretrievalresultsTinyTimACS.TinyTimResults.plotFocusDifference}\pysiglinewithargsret{\bfcode{plotFocusDifference}}{\emph{input}, \emph{title}}{}
\end{fulllineitems}


\index{plotFocusFieldPosition() (SamPy.focus.PhaseretrievalresultsTinyTimACS.TinyTimResults method)}

\begin{fulllineitems}
\phantomsection\label{SamPy.focus:SamPy.focus.PhaseretrievalresultsTinyTimACS.TinyTimResults.plotFocusFieldPosition}\pysiglinewithargsret{\bfcode{plotFocusFieldPosition}}{\emph{input}, \emph{title}}{}
\end{fulllineitems}


\end{fulllineitems}



\subsubsection{\texttt{WFC3ImageExtensioner} Module}
\label{SamPy.focus:module-SamPy.focus.WFC3ImageExtensioner}\label{SamPy.focus:wfc3imageextensioner-module}
\index{SamPy.focus.WFC3ImageExtensioner (module)}
Created on Apr 15, 2010
\begin{quote}\begin{description}
\item[{author}] \leavevmode
Sami-Matias Niemi

\item[{contact}] \leavevmode
\href{mailto:niemi@stsci.edu}{niemi@stsci.edu}

\item[{version}] \leavevmode
0.1

\end{description}\end{quote}


\subsubsection{\texttt{WFC3ImageExtensionerTinyTim} Module}
\label{SamPy.focus:wfc3imageextensionertinytim-module}\label{SamPy.focus:module-SamPy.focus.WFC3ImageExtensionerTinyTim}
\index{SamPy.focus.WFC3ImageExtensionerTinyTim (module)}
Created on March 3, 2011
\begin{quote}\begin{description}
\item[{author}] \leavevmode
Sami-Matias Niemi

\item[{contact}] \leavevmode
\href{mailto:niemi@stsci.edu}{niemi@stsci.edu}

\item[{version}] \leavevmode
0.1

\end{description}\end{quote}

\index{parse\_parameterfile() (in module SamPy.focus.WFC3ImageExtensionerTinyTim)}

\begin{fulllineitems}
\phantomsection\label{SamPy.focus:SamPy.focus.WFC3ImageExtensionerTinyTim.parse_parameterfile}\pysiglinewithargsret{\code{SamPy.focus.WFC3ImageExtensionerTinyTim.}\bfcode{parse\_parameterfile}}{\emph{file='./complete\_results/parameters.txt'}}{}
\end{fulllineitems}



\subsubsection{\texttt{collect\_focus} Module}
\label{SamPy.focus:module-SamPy.focus.collect_focus}\label{SamPy.focus:collect-focus-module}
\index{SamPy.focus.collect\_focus (module)}
Collects focus measurement results and outputs them
to files, one for each instrument and chip.

This is so horribly written that it should be redone 
completely from scratch. There are Python tuples, lists
and NumPy arrays, and they are all badly mixed.
\begin{quote}\begin{description}
\item[{note}] \leavevmode
Uses hardcoded path to python 2.7 of IRAF env.

\item[{author}] \leavevmode
Sami-Matias Niemi

\item[{contact}] \leavevmode
\href{mailto:niemi@stsci.edu}{niemi@stsci.edu}

\item[{organization}] \leavevmode
Space Telescope Science Insitute

\item[{todo}] \leavevmode
improve documentation

\end{description}\end{quote}

\index{append\_to\_FocusModel() (in module SamPy.focus.collect\_focus)}

\begin{fulllineitems}
\phantomsection\label{SamPy.focus:SamPy.focus.collect_focus.append_to_FocusModel}\pysiglinewithargsret{\code{SamPy.focus.collect\_focus.}\bfcode{append\_to\_FocusModel}}{\emph{inputfile}, \emph{outputfile}, \emph{path}}{}
This function appends new measurements to existing data
files that are read by e.g. Colin's focustool.
@param inputfile: Name of the input data file that
\begin{quote}

potentially contains new data.
\end{quote}
\begin{description}
\item[{@param outputfile: Name of the output dat file to which}] \leavevmode
new data are appended to.

\end{description}

@param path: Path from where the outputfile can be found.
@return: Boolean indicating if new data were identified.

\end{fulllineitems}


\index{findZernikes() (in module SamPy.focus.collect\_focus)}

\begin{fulllineitems}
\phantomsection\label{SamPy.focus:SamPy.focus.collect_focus.findZernikes}\pysiglinewithargsret{\code{SamPy.focus.collect\_focus.}\bfcode{findZernikes}}{\emph{files}, \emph{instrument='WFC3'}, \emph{file\_instrument='i'}}{}
\end{fulllineitems}


\index{main() (in module SamPy.focus.collect\_focus)}

\begin{fulllineitems}
\phantomsection\label{SamPy.focus:SamPy.focus.collect_focus.main}\pysiglinewithargsret{\code{SamPy.focus.collect\_focus.}\bfcode{main}}{}{}
Driver of collect\_focus.py

\end{fulllineitems}


\index{modList() (in module SamPy.focus.collect\_focus)}

\begin{fulllineitems}
\phantomsection\label{SamPy.focus:SamPy.focus.collect_focus.modList}\pysiglinewithargsret{\code{SamPy.focus.collect\_focus.}\bfcode{modList}}{\emph{data}, \emph{dtype=\{`names': (`file'}, \emph{`cam'}, \emph{`date'}, \emph{`MJD'}, \emph{`Focus')}, \emph{`formats': (`\textbar{}S12'}, \emph{`\textbar{}S8'}, \emph{`\textbar{}S12'}, \emph{`\textless{}f8'}, \emph{`\textless{}f8')\}}, \emph{filecol='file'}}{}
\end{fulllineitems}


\index{writeOutput() (in module SamPy.focus.collect\_focus)}

\begin{fulllineitems}
\phantomsection\label{SamPy.focus:SamPy.focus.collect_focus.writeOutput}\pysiglinewithargsret{\code{SamPy.focus.collect\_focus.}\bfcode{writeOutput}}{\emph{data}, \emph{outputfile}, \emph{chip}, \emph{header='\#file camera date mjd focus error\textbackslash{}n'}}{}
\end{fulllineitems}



\subsubsection{\texttt{phaseretrievalresults} Module}
\label{SamPy.focus:phaseretrievalresults-module}\label{SamPy.focus:module-SamPy.focus.phaseretrievalresults}
\index{SamPy.focus.phaseretrievalresults (module)}
DESCRIPTION:
Combines the Phase Retrieval software results, produces plots
and calculates focus with and without breathing correction.

USAGE:
python PhaseRetResults.py

HISTORY:
Created on Dec 17, 2009
\begin{quote}\begin{description}
\item[{author}] \leavevmode
Sami-Matias Niemi

\item[{contact}] \leavevmode
\href{mailto:niemi@stsci.edu}{niemi@stsci.edu}

\item[{version}] \leavevmode
0.91

\item[{todo}] \leavevmode
maybe introduce sigma clipping to the means?

\item[{todo}] \leavevmode
how the legend has been implemented is a dirty,
it should be done better.

\end{description}\end{quote}

\index{PhaseRetResults (class in SamPy.focus.phaseretrievalresults)}

\begin{fulllineitems}
\phantomsection\label{SamPy.focus:SamPy.focus.phaseretrievalresults.PhaseRetResults}\pysiglinewithargsret{\strong{class }\code{SamPy.focus.phaseretrievalresults.}\bfcode{PhaseRetResults}}{\emph{cameras}, \emph{str}}{}~
\index{findFiles() (SamPy.focus.phaseretrievalresults.PhaseRetResults method)}

\begin{fulllineitems}
\phantomsection\label{SamPy.focus:SamPy.focus.phaseretrievalresults.PhaseRetResults.findFiles}\pysiglinewithargsret{\bfcode{findFiles}}{\emph{data}}{}
\end{fulllineitems}


\index{plotFocus() (SamPy.focus.phaseretrievalresults.PhaseRetResults method)}

\begin{fulllineitems}
\phantomsection\label{SamPy.focus:SamPy.focus.phaseretrievalresults.PhaseRetResults.plotFocus}\pysiglinewithargsret{\bfcode{plotFocus}}{\emph{chip1}, \emph{chip2}, \emph{brdata=}\optional{}, \emph{noBreathing=False}}{}
\end{fulllineitems}


\index{plotStars() (SamPy.focus.phaseretrievalresults.PhaseRetResults method)}

\begin{fulllineitems}
\phantomsection\label{SamPy.focus:SamPy.focus.phaseretrievalresults.PhaseRetResults.plotStars}\pysiglinewithargsret{\bfcode{plotStars}}{\emph{file}, \emph{ext}, \emph{xpos}, \emph{ypos}, \emph{rad=25}}{}
\end{fulllineitems}


\index{readBreathing() (SamPy.focus.phaseretrievalresults.PhaseRetResults method)}

\begin{fulllineitems}
\phantomsection\label{SamPy.focus:SamPy.focus.phaseretrievalresults.PhaseRetResults.readBreathing}\pysiglinewithargsret{\bfcode{readBreathing}}{\emph{file}}{}
\end{fulllineitems}


\index{readResults() (SamPy.focus.phaseretrievalresults.PhaseRetResults method)}

\begin{fulllineitems}
\phantomsection\label{SamPy.focus:SamPy.focus.phaseretrievalresults.PhaseRetResults.readResults}\pysiglinewithargsret{\bfcode{readResults}}{\emph{file}}{}
\end{fulllineitems}


\end{fulllineitems}



\subsubsection{\texttt{plotextractionboxdata} Module}
\label{SamPy.focus:module-SamPy.focus.plotextractionboxdata}\label{SamPy.focus:plotextractionboxdata-module}
\index{SamPy.focus.plotextractionboxdata (module)}\begin{quote}\begin{description}
\item[{author}] \leavevmode
Sami-Matias Niemi

\end{description}\end{quote}

\index{parse\_data() (in module SamPy.focus.plotextractionboxdata)}

\begin{fulllineitems}
\phantomsection\label{SamPy.focus:SamPy.focus.plotextractionboxdata.parse_data}\pysiglinewithargsret{\code{SamPy.focus.plotextractionboxdata.}\bfcode{parse\_data}}{\emph{file}, \emph{mjd=2}, \emph{focus=6}}{}
\end{fulllineitems}


\index{plot\_extractionbox\_comparison() (in module SamPy.focus.plotextractionboxdata)}

\begin{fulllineitems}
\phantomsection\label{SamPy.focus:SamPy.focus.plotextractionboxdata.plot_extractionbox_comparison}\pysiglinewithargsret{\code{SamPy.focus.plotextractionboxdata.}\bfcode{plot\_extractionbox\_comparison}}{\emph{data}, \emph{breathing}, \emph{outdir}}{}
\end{fulllineitems}


\index{read\_breathing() (in module SamPy.focus.plotextractionboxdata)}

\begin{fulllineitems}
\phantomsection\label{SamPy.focus:SamPy.focus.plotextractionboxdata.read_breathing}\pysiglinewithargsret{\code{SamPy.focus.plotextractionboxdata.}\bfcode{read\_breathing}}{\emph{file}}{}
\end{fulllineitems}



\subsection{grisms Package}
\label{SamPy.grisms:grisms-package}\label{SamPy.grisms::doc}

\subsubsection{\texttt{extract\_splitless} Module}
\label{SamPy.grisms:extract-splitless-module}

\subsubsection{\texttt{generateAssociationDB} Module}
\label{SamPy.grisms:module-SamPy.grisms.generateAssociationDB}\label{SamPy.grisms:generateassociationdb-module}
\index{SamPy.grisms.generateAssociationDB (module)}
This little script can be used to generate a cPickled
association database that is in the format that the PHLAG
grism pipeline developed by ECF will understand.

The script looks all FITS files in the current working
directory and tries to generate the association database
by matching direct images to grism images. Note that the
matching algorithm is extremely limited and works only if
the direct images and grism images have the same RA or DEC.
Thus, the script is only suitable for the test that will
be carried out in April 2011, as for this test all the
datasets have been hand picked.

This script should not be used unless its limitations
have been understood!

\index{createAssociationDB() (in module SamPy.grisms.generateAssociationDB)}

\begin{fulllineitems}
\phantomsection\label{SamPy.grisms:SamPy.grisms.generateAssociationDB.createAssociationDB}\pysiglinewithargsret{\code{SamPy.grisms.generateAssociationDB.}\bfcode{createAssociationDB}}{\emph{grism}, \emph{direct}, \emph{match}, \emph{output='ACSTEST.db'}}{}
Email from Martin Kuemmel concerning the format:
\begin{itemize}
\item {} 
Each association (see listing below contains two main entries, the
first entry (a list)  containing information on the grism images and
the second a dictionary) containing direct imaging information.

\item {} 
The grism image information is a list, which each entry consisting of
a tuple with four items. The first item is the name of the grism
image, the second item is the exposure time of the grism image,
the third item is the name of the associated direct image and the
fourth item a real number measuring the `distance' between the grism
image and the direct image. Very close pairs of grism image - direct
image have a distance of 0.0. I think there is a very complicated
formula that measures this distance, however the most important input
is the pointing on the sky. If a grism and a direct image have the
same pointing on the sky, they are very close. Also the time
difference between the direct image and the grism image has some
influence.

\item {} 
The direct image information is a dictionary that contains the filter
names as keys. For each filter there is a list of tuples with three
items. The first item is the direct image name, the second item is
the exposure time and the third item is the overlap with the area covered
by the grism images of that association. For some reason (I have forgotten),
this overlap can be more than 1.

\end{itemize}

So the two `dir\_min\_overlap' that you quote are two different values,
the first one measuring the distance between the grism image and the
closest associated direct image (from all possible associated direct
images, the closest one is chosen), and the second one measuring the
area overlap with all grism images.

We did have algorithms and formulas to compute these numbers, but in
the end it was always some sort of magic with Alberto Micol and Marco
Lombardi the magicians. I asked Alberto, but he was rather involved in
NICMOS, and Marco is really far away now, his account at ESO does no
longer exists and his old computer (standing in front of me) needed a
new disk some weeks ago.

So I guess that we can not offer much support on this. In any case,
WFC3 is a different instrument with 2 complementary grisms and
different prescriptions for taking associated direct imaging and hence
a different basis of how data is taken.

Hope this helped,
Martin

({[}(`J6FL8YKUQ', 764.0, `J6FL8YKRQ', 0.0),
(`J6FL8YKWQ', 764.0, `J6FL8YKRQ', 0.0),
(`J6FL8YKYQ', 764.0, `J6FL8YKRQ', 0.0),
(`J6FL8YL0Q', 764.0, `J6FL8YKRQ', 0.0),
(`J6FL8YL2Q', 764.0, `J6FL8YKRQ', 0.0),
(`J6FL8YL4Q', 764.0, `J6FL8YKRQ', 0.0),
(`J6FL8YL6Q', 764.0, `J6FL8YKRQ', 0.0),
(`J6FL8YL8Q', 764.0, `J6FL8YKRQ', 0.0){]},
\{`F850LP':
{[}(`J8WQ91E8Q', 500.0, 0.89063763040941479),
(`J8WQ91EAQ', 500.0, 0.89063763040941479),
(`J8WQ91ECQ', 500.0, 0.89063763040941479),
(`J6FL7YIDQ', 500.0, 0.97062910394576651),
(`J6FL8YKRQ', 500.0, 1.011511942107943),
(`J8WQ91E7Q', 500.0, 0.89063763040941479){]}\})

\end{fulllineitems}


\index{findGrismImages() (in module SamPy.grisms.generateAssociationDB)}

\begin{fulllineitems}
\phantomsection\label{SamPy.grisms:SamPy.grisms.generateAssociationDB.findGrismImages}\pysiglinewithargsret{\code{SamPy.grisms.generateAssociationDB.}\bfcode{findGrismImages}}{\emph{filelist}}{}
Finds all direct and grism images from a
given filelist. Will pull out some information
such as exposure time and filter from the header.
Will also record RA and DEC, this could potentially
be used to calculate the overlap and is currently
used to match images later.

\end{fulllineitems}


\index{makeDirectImageDictionary() (in module SamPy.grisms.generateAssociationDB)}

\begin{fulllineitems}
\phantomsection\label{SamPy.grisms:SamPy.grisms.generateAssociationDB.makeDirectImageDictionary}\pysiglinewithargsret{\code{SamPy.grisms.generateAssociationDB.}\bfcode{makeDirectImageDictionary}}{\emph{direct}, \emph{match}}{}
Generates a dictionary from the direct image information.
This dictionary is in the format that ECF used to make
their association db, so it is more or less ready to be
pickled. For details, see the comment in createAssociationDB.

\end{fulllineitems}


\index{makeGrismImageList() (in module SamPy.grisms.generateAssociationDB)}

\begin{fulllineitems}
\phantomsection\label{SamPy.grisms:SamPy.grisms.generateAssociationDB.makeGrismImageList}\pysiglinewithargsret{\code{SamPy.grisms.generateAssociationDB.}\bfcode{makeGrismImageList}}{\emph{grism}, \emph{match}}{}
Generates a list of grism images. This list is in the
format that ECF used to make their association db, so it
is more or less ready to be pickled. For details,
see the comment in createAssociationDB.

\end{fulllineitems}


\index{matchDirectImageToGrismImage() (in module SamPy.grisms.generateAssociationDB)}

\begin{fulllineitems}
\phantomsection\label{SamPy.grisms:SamPy.grisms.generateAssociationDB.matchDirectImageToGrismImage}\pysiglinewithargsret{\code{SamPy.grisms.generateAssociationDB.}\bfcode{matchDirectImageToGrismImage}}{\emph{direct}, \emph{grism}}{}
Matches direct images to grism images in an extremely
dummy way. Will just check that either RA or DEC are
equal. This can potentially lead to crazy results as
it does not guarantee that there is overlap. Moreover,
if one has applied a tiny offset to the grism image to
for example account for the fact that the dispersion
is always towards the same direction, then this matching
would fail. One should probably use some header keyword
information such as POSTARGs together with the RA and DEC
to find the closest matches. However, for the testing
that will be performed in April 2011 this dummy way of
matching should be sufficient.

\end{fulllineitems}



\subsubsection{\texttt{overplot\_sextractor} Module}
\label{SamPy.grisms:module-SamPy.grisms.overplot_sextractor}\label{SamPy.grisms:overplot-sextractor-module}
\index{SamPy.grisms.overplot\_sextractor (module)}

\subsubsection{\texttt{prepare\_slitless} Module}
\label{SamPy.grisms:prepare-slitless-module}

\subsection{herschel Package}
\label{SamPy.herschel::doc}\label{SamPy.herschel:herschel-package}

\subsubsection{\texttt{calculateMergers} Module}
\label{SamPy.herschel:calculatemergers-module}\label{SamPy.herschel:module-SamPy.herschel.calculateMergers}
\index{SamPy.herschel.calculateMergers (module)}

\subsubsection{\texttt{plotFluxDistribution} Module}
\label{SamPy.herschel:plotfluxdistribution-module}\label{SamPy.herschel:module-SamPy.herschel.plotFluxDistribution}
\index{SamPy.herschel.plotFluxDistribution (module)}
\index{plot\_flux\_dist() (in module SamPy.herschel.plotFluxDistribution)}

\begin{fulllineitems}
\phantomsection\label{SamPy.herschel:SamPy.herschel.plotFluxDistribution.plot_flux_dist}\pysiglinewithargsret{\code{SamPy.herschel.plotFluxDistribution.}\bfcode{plot\_flux\_dist}}{\emph{table}, \emph{zmin}, \emph{zmax}, \emph{depths}, \emph{colname}, \emph{path}, \emph{database}, \emph{out\_folder}, \emph{solid\_angle=1600.0}, \emph{fluxbins=22}, \emph{ymin=1e-07}, \emph{ymax=0.1}, \emph{bins=8}, \emph{H0=70.0}, \emph{WM=0.28}}{}
\end{fulllineitems}



\subsubsection{\texttt{plotFluxRedshiftDistribution} Module}
\label{SamPy.herschel:plotfluxredshiftdistribution-module}\label{SamPy.herschel:module-SamPy.herschel.plotFluxRedshiftDistribution}
\index{SamPy.herschel.plotFluxRedshiftDistribution (module)}
\index{plotFluxRedshiftDistribution() (in module SamPy.herschel.plotFluxRedshiftDistribution)}

\begin{fulllineitems}
\phantomsection\label{SamPy.herschel:SamPy.herschel.plotFluxRedshiftDistribution.plotFluxRedshiftDistribution}\pysiglinewithargsret{\code{SamPy.herschel.plotFluxRedshiftDistribution.}\bfcode{plotFluxRedshiftDistribution}}{\emph{path}, \emph{database}, \emph{out\_folder}}{}
\end{fulllineitems}


\index{scatterHistograms() (in module SamPy.herschel.plotFluxRedshiftDistribution)}

\begin{fulllineitems}
\phantomsection\label{SamPy.herschel:SamPy.herschel.plotFluxRedshiftDistribution.scatterHistograms}\pysiglinewithargsret{\code{SamPy.herschel.plotFluxRedshiftDistribution.}\bfcode{scatterHistograms}}{\emph{xdata}, \emph{ydata}, \emph{xlabel}, \emph{ylabel}, \emph{output}}{}
This functions generates a scatter plot and
projected histograms to both axes.

\end{fulllineitems}



\subsubsection{\texttt{plotage} Module}
\label{SamPy.herschel:module-SamPy.herschel.plotage}\label{SamPy.herschel:plotage-module}
\index{SamPy.herschel.plotage (module)}
\index{plot\_ages() (in module SamPy.herschel.plotage)}

\begin{fulllineitems}
\phantomsection\label{SamPy.herschel:SamPy.herschel.plotage.plot_ages}\pysiglinewithargsret{\code{SamPy.herschel.plotage.}\bfcode{plot\_ages}}{\emph{query}, \emph{xlabel}, \emph{ylabel}, \emph{output}, \emph{out\_folder}, \emph{flux=5}, \emph{band=250}, \emph{title='\$2.0 \textless{} z \textless{} 4.0\$'}, \emph{pmin=0.05}, \emph{pmax=1.0}, \emph{xbin=10}, \emph{ybin=15}, \emph{xmin=7.9}, \emph{xmax=11.7}, \emph{ymin=-1.5}, \emph{ymax=1.8}, \emph{scatters=False}, \emph{mean=False}}{}
Plots age versus a given quantity.

\end{fulllineitems}



\subsubsection{\texttt{plotcolourproperties} Module}
\label{SamPy.herschel:plotcolourproperties-module}\label{SamPy.herschel:module-SamPy.herschel.plotcolourproperties}
\index{SamPy.herschel.plotcolourproperties (module)}
\index{plotColourFlux() (in module SamPy.herschel.plotcolourproperties)}

\begin{fulllineitems}
\phantomsection\label{SamPy.herschel:SamPy.herschel.plotcolourproperties.plotColourFlux}\pysiglinewithargsret{\code{SamPy.herschel.plotcolourproperties.}\bfcode{plotColourFlux}}{\emph{query}, \emph{xlabel}, \emph{ylabel}, \emph{output}, \emph{out\_folder}, \emph{ymin=-0.5}, \emph{ymax=2}, \emph{xmin=0.0}, \emph{xmax=60}, \emph{title='\$S\_\{250\} \textgreater{} 5 \textbackslash{}\textbackslash{} \textbackslash{}\textbackslash{}mathrm\{mJy\}\$'}, \emph{clabel='\$\textbackslash{}\textbackslash{}mathrm\{Redshift\}\$'}}{}
\end{fulllineitems}


\index{plotColourProperties() (in module SamPy.herschel.plotcolourproperties)}

\begin{fulllineitems}
\phantomsection\label{SamPy.herschel:SamPy.herschel.plotcolourproperties.plotColourProperties}\pysiglinewithargsret{\code{SamPy.herschel.plotcolourproperties.}\bfcode{plotColourProperties}}{\emph{query}, \emph{xlabel}, \emph{ylabel}, \emph{output}, \emph{out\_folder}, \emph{ymin=-12}, \emph{ymax=2}, \emph{xmin=-6}, \emph{xmax=3.2}, \emph{title='`}, \emph{clabel='\$\textbackslash{}\textbackslash{}mathrm\{Redshift\}\$'}}{}
\end{fulllineitems}


\index{plotColourProperties2() (in module SamPy.herschel.plotcolourproperties)}

\begin{fulllineitems}
\phantomsection\label{SamPy.herschel:SamPy.herschel.plotcolourproperties.plotColourProperties2}\pysiglinewithargsret{\code{SamPy.herschel.plotcolourproperties.}\bfcode{plotColourProperties2}}{\emph{query}, \emph{xlabel}, \emph{ylabel}, \emph{output}, \emph{out\_folder}, \emph{ymin=-13}, \emph{ymax=2}, \emph{xmin=-8}, \emph{xmax=4}, \emph{title='`}, \emph{ylog=False}}{}
\end{fulllineitems}


\index{plotColourProperties3() (in module SamPy.herschel.plotcolourproperties)}

\begin{fulllineitems}
\phantomsection\label{SamPy.herschel:SamPy.herschel.plotcolourproperties.plotColourProperties3}\pysiglinewithargsret{\code{SamPy.herschel.plotcolourproperties.}\bfcode{plotColourProperties3}}{\emph{query}, \emph{xlabel}, \emph{ylabel}, \emph{output}, \emph{out\_folder}, \emph{ymin=-12}, \emph{ymax=2}, \emph{xmin=-6}, \emph{xmax=3.2}, \emph{title='`}}{}
\end{fulllineitems}


\index{plotColourProperties4() (in module SamPy.herschel.plotcolourproperties)}

\begin{fulllineitems}
\phantomsection\label{SamPy.herschel:SamPy.herschel.plotcolourproperties.plotColourProperties4}\pysiglinewithargsret{\code{SamPy.herschel.plotcolourproperties.}\bfcode{plotColourProperties4}}{\emph{query}, \emph{output}, \emph{out\_folder}}{}
\end{fulllineitems}


\index{plotProperties() (in module SamPy.herschel.plotcolourproperties)}

\begin{fulllineitems}
\phantomsection\label{SamPy.herschel:SamPy.herschel.plotcolourproperties.plotProperties}\pysiglinewithargsret{\code{SamPy.herschel.plotcolourproperties.}\bfcode{plotProperties}}{\emph{query}, \emph{xlabel}, \emph{ylabel}, \emph{output}, \emph{out\_folder}, \emph{ymin=-12}, \emph{ymax=2}, \emph{xmin=-6}, \emph{xmax=3.2}, \emph{title='`}, \emph{clabel='\$\textbackslash{}\textbackslash{}mathrm\{Redshift\}\$'}}{}
\end{fulllineitems}



\subsubsection{\texttt{plotcorrelations} Module}
\label{SamPy.herschel:plotcorrelations-module}\label{SamPy.herschel:module-SamPy.herschel.plotcorrelations}
\index{SamPy.herschel.plotcorrelations (module)}
\index{plot\_correlation() (in module SamPy.herschel.plotcorrelations)}

\begin{fulllineitems}
\phantomsection\label{SamPy.herschel:SamPy.herschel.plotcorrelations.plot_correlation}\pysiglinewithargsret{\code{SamPy.herschel.plotcorrelations.}\bfcode{plot\_correlation}}{\emph{query}, \emph{xlabel}, \emph{ylabel}, \emph{zmin}, \emph{zmax}, \emph{out\_folder}}{}
\end{fulllineitems}



\subsubsection{\texttt{plotluminosityfunctions} Module}
\label{SamPy.herschel:plotluminosityfunctions-module}\label{SamPy.herschel:module-SamPy.herschel.plotluminosityfunctions}
\index{SamPy.herschel.plotluminosityfunctions (module)}
Plots luminosity functions at different redshifts.
Pulls data from an sqlite3 database.

@author: Sami Niemi

\index{plot\_luminosityfunction() (in module SamPy.herschel.plotluminosityfunctions)}

\begin{fulllineitems}
\phantomsection\label{SamPy.herschel:SamPy.herschel.plotluminosityfunctions.plot_luminosityfunction}\pysiglinewithargsret{\code{SamPy.herschel.plotluminosityfunctions.}\bfcode{plot\_luminosityfunction}}{\emph{path}, \emph{database}, \emph{redshifts}, \emph{band}, \emph{out\_folder}, \emph{obs\_data}, \emph{solid\_angle=1600.0}, \emph{ymin=1000}, \emph{ymax=2000000}, \emph{xmin=0.5}, \emph{xmax=100}, \emph{nbins=15}, \emph{sigma=5.0}, \emph{H0=70.0}, \emph{WM=0.28}, \emph{zmax=6.0}}{}~\begin{description}
\item[{@param solid\_angle: area of the sky survey in arcmin**2}] \leavevmode
GOODS = 160, hence 10*160

\end{description}

@param sigma: sigma level of the errors to be plotted
@param nbins: number of bins (for simulated data)

\end{fulllineitems}


\index{plot\_luminosityfunction2() (in module SamPy.herschel.plotluminosityfunctions)}

\begin{fulllineitems}
\phantomsection\label{SamPy.herschel:SamPy.herschel.plotluminosityfunctions.plot_luminosityfunction2}\pysiglinewithargsret{\code{SamPy.herschel.plotluminosityfunctions.}\bfcode{plot\_luminosityfunction2}}{\emph{path}, \emph{database}, \emph{redshifts}, \emph{band}, \emph{out\_folder}, \emph{obs\_data}, \emph{solid\_angle=1600.0}, \emph{ymin=1000}, \emph{ymax=2000000}, \emph{xmin=0.5}, \emph{xmax=100}, \emph{nbins=15}, \emph{sigma=5.0}, \emph{H0=70.0}, \emph{WM=0.28}, \emph{zmax=6.0}}{}~\begin{description}
\item[{@param solid\_angle: area of the sky survey in arcmin**2}] \leavevmode
GOODS = 160, hence 10*160

\end{description}

@param sigma: sigma level of the errors to be plotted
@param nbins: number of bins (for simulated data)

\end{fulllineitems}


\index{plot\_luminosityfunctionKDE() (in module SamPy.herschel.plotluminosityfunctions)}

\begin{fulllineitems}
\phantomsection\label{SamPy.herschel:SamPy.herschel.plotluminosityfunctions.plot_luminosityfunctionKDE}\pysiglinewithargsret{\code{SamPy.herschel.plotluminosityfunctions.}\bfcode{plot\_luminosityfunctionKDE}}{\emph{path}, \emph{database}, \emph{redshifts}, \emph{band}, \emph{out\_folder}, \emph{obs\_data}, \emph{solid\_angle=1600.0}, \emph{ymin=1000}, \emph{ymax=2000000}, \emph{xmin=0.5}, \emph{xmax=100}, \emph{nbins=15}, \emph{H0=70.0}, \emph{WM=0.28}, \emph{zmax=6.0}}{}~\begin{description}
\item[{@param solid\_angle: area of the sky survey in arcmin**2}] \leavevmode
GOODS = 160, hence 10*160

\end{description}

@param sigma: sigma level of the errors to be plotted
@param nbins: number of bins (for simulated data)

\end{fulllineitems}


\index{plot\_luminosityfunctionPaper() (in module SamPy.herschel.plotluminosityfunctions)}

\begin{fulllineitems}
\phantomsection\label{SamPy.herschel:SamPy.herschel.plotluminosityfunctions.plot_luminosityfunctionPaper}\pysiglinewithargsret{\code{SamPy.herschel.plotluminosityfunctions.}\bfcode{plot\_luminosityfunctionPaper}}{\emph{path}, \emph{database}, \emph{redshifts}, \emph{bands}, \emph{out\_folder}, \emph{solid\_angle=16000.0}, \emph{ymin=1e-05}, \emph{ymax=0.05}, \emph{xmin=8.0}, \emph{xmax=12.3}, \emph{H0=70.0}, \emph{WM=0.28}, \emph{zmax=6.0}}{}~\begin{description}
\item[{@param solid\_angle: area of the sky survey in arcmin**2}] \leavevmode
GOODS = 160, hence 100*160

\end{description}

\end{fulllineitems}



\subsubsection{\texttt{plotmergers} Module}
\label{SamPy.herschel:module-SamPy.herschel.plotmergers}\label{SamPy.herschel:plotmergers-module}
\index{SamPy.herschel.plotmergers (module)}
\index{plot\_tmerge() (in module SamPy.herschel.plotmergers)}

\begin{fulllineitems}
\phantomsection\label{SamPy.herschel:SamPy.herschel.plotmergers.plot_tmerge}\pysiglinewithargsret{\code{SamPy.herschel.plotmergers.}\bfcode{plot\_tmerge}}{\emph{query1, query2, xlabel, ylabel, output, out\_folder, pmin=0.05, pmax=1.0, xbin1=15, ybin1=15, xbin2=15, ybin2=15, y1ticks={[}0, 0.4, 0.8, 1.2, 1.6, 2.0, 2.5, 3{]}, y2ticks={[}0.4, 0.8, 1.2, 1.6, 2.0, 2.5{]}, xmin1=7.9, xmax1=11.7, xmin2=7.9, xmax2=11.7, ymin=0.0, ymax=3.0, scatters=False, mean=False}}{}
\end{fulllineitems}


\index{plot\_tmerge\_bluest() (in module SamPy.herschel.plotmergers)}

\begin{fulllineitems}
\phantomsection\label{SamPy.herschel:SamPy.herschel.plotmergers.plot_tmerge_bluest}\pysiglinewithargsret{\code{SamPy.herschel.plotmergers.}\bfcode{plot\_tmerge\_bluest}}{\emph{query1, query2, xlabel, ylabel, output, out\_folder, pmin=0.05, pmax=1.0, xbin1=15, ybin1=15, xbin2=15, ybin2=15, y1ticks={[}0, 0.4, 0.8, 1.2, 1.6, 2.0{]}, y2ticks={[}0.02, 0.04, 0.06, 0.08{]}, xmin1=7.9, xmax1=11.7, xmin2=7.9, xmax2=11.7, ymin1=0.0, ymax1=2.0, ymin2=0.0, ymax2=0.1, scatters=False, mean=False}}{}
\end{fulllineitems}



\subsubsection{\texttt{plotmergers2} Module}
\label{SamPy.herschel:plotmergers2-module}\label{SamPy.herschel:module-SamPy.herschel.plotmergers2}
\index{SamPy.herschel.plotmergers2 (module)}
\index{plotMergerFractions() (in module SamPy.herschel.plotmergers2)}

\begin{fulllineitems}
\phantomsection\label{SamPy.herschel:SamPy.herschel.plotmergers2.plotMergerFractions}\pysiglinewithargsret{\code{SamPy.herschel.plotmergers2.}\bfcode{plotMergerFractions}}{\emph{query}, \emph{xlabel}, \emph{ylabel}, \emph{output}, \emph{out\_folder}, \emph{mergetimelimit=0.25}, \emph{mstarmin=8.0}, \emph{mstarmax=11.5}, \emph{mbins=15}, \emph{ymin=-0.01}, \emph{ymax=1.01}, \emph{logscale=False}}{}
\end{fulllineitems}


\index{plotMergerFractions2() (in module SamPy.herschel.plotmergers2)}

\begin{fulllineitems}
\phantomsection\label{SamPy.herschel:SamPy.herschel.plotmergers2.plotMergerFractions2}\pysiglinewithargsret{\code{SamPy.herschel.plotmergers2.}\bfcode{plotMergerFractions2}}{\emph{query}, \emph{xlabel}, \emph{ylabel}, \emph{output}, \emph{out\_folder}, \emph{mergetimelimit=0.25}, \emph{mstarmin=8.0}, \emph{mstarmax=11.5}, \emph{mbins=15}, \emph{ymin=0.0}, \emph{ymax=1.0}, \emph{logscale=False}}{}
\end{fulllineitems}


\index{plotMergerFractions3() (in module SamPy.herschel.plotmergers2)}

\begin{fulllineitems}
\phantomsection\label{SamPy.herschel:SamPy.herschel.plotmergers2.plotMergerFractions3}\pysiglinewithargsret{\code{SamPy.herschel.plotmergers2.}\bfcode{plotMergerFractions3}}{\emph{query}, \emph{xlabel}, \emph{ylabel}, \emph{output}, \emph{out\_folder}, \emph{mergetimelimit=0.25}, \emph{mstarmin=8.0}, \emph{mstarmax=11.5}, \emph{mbins=15}, \emph{ymin=0.0}, \emph{ymax=1.0}, \emph{logscale=False}}{}
\end{fulllineitems}



\subsubsection{\texttt{plotmergers3} Module}
\label{SamPy.herschel:plotmergers3-module}\label{SamPy.herschel:module-SamPy.herschel.plotmergers3}
\index{SamPy.herschel.plotmergers3 (module)}
\index{plotMergerFractions() (in module SamPy.herschel.plotmergers3)}

\begin{fulllineitems}
\phantomsection\label{SamPy.herschel:SamPy.herschel.plotmergers3.plotMergerFractions}\pysiglinewithargsret{\code{SamPy.herschel.plotmergers3.}\bfcode{plotMergerFractions}}{\emph{query}, \emph{xlabel}, \emph{ylabel}, \emph{output}, \emph{out\_folder}, \emph{mergetimelimit=0.25}, \emph{mstarmin=8.0}, \emph{mstarmax=11.5}, \emph{mbins=15}, \emph{ymin=-0.001}, \emph{ymax=1.01}, \emph{logscale=False}}{}
\end{fulllineitems}


\index{plotMergerFractions2() (in module SamPy.herschel.plotmergers3)}

\begin{fulllineitems}
\phantomsection\label{SamPy.herschel:SamPy.herschel.plotmergers3.plotMergerFractions2}\pysiglinewithargsret{\code{SamPy.herschel.plotmergers3.}\bfcode{plotMergerFractions2}}{\emph{query}, \emph{xlabel}, \emph{ylabel}, \emph{output}, \emph{out\_folder}, \emph{mergetimelimit=0.25}, \emph{mstarmin=8.0}, \emph{mstarmax=11.5}, \emph{mbins=15}, \emph{ymin=0.0}, \emph{ymax=1.0}, \emph{logscale=False}}{}
\end{fulllineitems}


\index{plotMergerFractions3() (in module SamPy.herschel.plotmergers3)}

\begin{fulllineitems}
\phantomsection\label{SamPy.herschel:SamPy.herschel.plotmergers3.plotMergerFractions3}\pysiglinewithargsret{\code{SamPy.herschel.plotmergers3.}\bfcode{plotMergerFractions3}}{\emph{query}, \emph{xlabel}, \emph{ylabel}, \emph{output}, \emph{out\_folder}, \emph{mergetimelimit=0.4}, \emph{mstarmin=8.0}, \emph{mstarmax=11.5}, \emph{mbins=15}, \emph{ymin=0.0}, \emph{ymax=1.0}, \emph{logscale=False}}{}
\end{fulllineitems}



\subsubsection{\texttt{plotmergers4} Module}
\label{SamPy.herschel:module-SamPy.herschel.plotmergers4}\label{SamPy.herschel:plotmergers4-module}
\index{SamPy.herschel.plotmergers4 (module)}
\index{plotMergerFractions() (in module SamPy.herschel.plotmergers4)}

\begin{fulllineitems}
\phantomsection\label{SamPy.herschel:SamPy.herschel.plotmergers4.plotMergerFractions}\pysiglinewithargsret{\code{SamPy.herschel.plotmergers4.}\bfcode{plotMergerFractions}}{\emph{query}, \emph{xlabel}, \emph{ylabel}, \emph{output}, \emph{out\_folder}, \emph{mergetimelimit=0.25}, \emph{ymin=-0.2}, \emph{ymax=1.0}, \emph{xmin=-8}, \emph{xmax=4.1}, \emph{xbins=70}, \emph{ybins=70}, \emph{title='All galaxies in \$2 \textbackslash{}\textbackslash{}leq z \textless{} 4\$'}}{}
\end{fulllineitems}


\index{plotMergerFractionsMultiplot() (in module SamPy.herschel.plotmergers4)}

\begin{fulllineitems}
\phantomsection\label{SamPy.herschel:SamPy.herschel.plotmergers4.plotMergerFractionsMultiplot}\pysiglinewithargsret{\code{SamPy.herschel.plotmergers4.}\bfcode{plotMergerFractionsMultiplot}}{\emph{query}, \emph{xlabel}, \emph{ylabel}, \emph{output}, \emph{out\_folder}, \emph{mergetimelimit=0.25}, \emph{ymin=-0.2}, \emph{ymax=0.8}, \emph{xmin=-9}, \emph{xmax=4.1}, \emph{xbins=50}, \emph{ybins=50}, \emph{title='`}}{}
\end{fulllineitems}



\subsubsection{\texttt{plotmergersPaper} Module}
\label{SamPy.herschel:plotmergerspaper-module}\label{SamPy.herschel:module-SamPy.herschel.plotmergersPaper}
\index{SamPy.herschel.plotmergersPaper (module)}
\index{plotMergerFractionsMultiplot() (in module SamPy.herschel.plotmergersPaper)}

\begin{fulllineitems}
\phantomsection\label{SamPy.herschel:SamPy.herschel.plotmergersPaper.plotMergerFractionsMultiplot}\pysiglinewithargsret{\code{SamPy.herschel.plotmergersPaper.}\bfcode{plotMergerFractionsMultiplot}}{\emph{query}, \emph{xlabel}, \emph{ylabel}, \emph{output}, \emph{out\_folder}, \emph{obs}, \emph{mergetimelimit=0.25}, \emph{ymin=-0.2}, \emph{ymax=1.5}, \emph{xmin=-1.9}, \emph{xmax=4.15}, \emph{xbins=80}, \emph{ybins=80}, \emph{title='Simulated Galaxies'}, \emph{size=4.5}, \emph{alpha=0.2}, \emph{ch=1}}{}
\end{fulllineitems}


\index{ujy\_to\_abmag() (in module SamPy.herschel.plotmergersPaper)}

\begin{fulllineitems}
\phantomsection\label{SamPy.herschel:SamPy.herschel.plotmergersPaper.ujy_to_abmag}\pysiglinewithargsret{\code{SamPy.herschel.plotmergersPaper.}\bfcode{ujy\_to\_abmag}}{\emph{flux}}{}
\end{fulllineitems}



\subsubsection{\texttt{plotnumbercounts} Module}
\label{SamPy.herschel:plotnumbercounts-module}\label{SamPy.herschel:module-SamPy.herschel.plotnumbercounts}
\index{SamPy.herschel.plotnumbercounts (module)}
\index{diff\_function() (in module SamPy.herschel.plotnumbercounts)}

\begin{fulllineitems}
\phantomsection\label{SamPy.herschel:SamPy.herschel.plotnumbercounts.diff_function}\pysiglinewithargsret{\code{SamPy.herschel.plotnumbercounts.}\bfcode{diff\_function}}{\emph{data}, \emph{column=0}, \emph{log=False}, \emph{wgth=None}, \emph{mmax=15.5}, \emph{mmin=9.0}, \emph{nbins=35}, \emph{h=0.7}, \emph{volume=250}, \emph{nvols=1}, \emph{physical\_units=False}, \emph{verbose=False}}{}
Calculates a differential function from data.

\end{fulllineitems}


\index{plotTemplateComparison() (in module SamPy.herschel.plotnumbercounts)}

\begin{fulllineitems}
\phantomsection\label{SamPy.herschel:SamPy.herschel.plotnumbercounts.plotTemplateComparison}\pysiglinewithargsret{\code{SamPy.herschel.plotnumbercounts.}\bfcode{plotTemplateComparison}}{\emph{database}, \emph{band}, \emph{redshifts}, \emph{ymin=1000}, \emph{ymax=2000000}, \emph{xmin=0.5}, \emph{xmax=100}, \emph{nbins=15}, \emph{sigma=3.0}}{}
\end{fulllineitems}


\index{plot\_number\_counts() (in module SamPy.herschel.plotnumbercounts)}

\begin{fulllineitems}
\phantomsection\label{SamPy.herschel:SamPy.herschel.plotnumbercounts.plot_number_counts}\pysiglinewithargsret{\code{SamPy.herschel.plotnumbercounts.}\bfcode{plot\_number\_counts}}{\emph{path}, \emph{database}, \emph{band}, \emph{redshifts}, \emph{out\_folder}, \emph{obs\_data}, \emph{area=2.25}, \emph{ymin=1000}, \emph{ymax=2000000}, \emph{xmin=0.5}, \emph{xmax=100}, \emph{nbins=15}, \emph{sigma=3.0}, \emph{write\_out=False}}{}
160 (arcminutes squared) = 0.0444444444 square degrees
Simulation was 10 times the GOODS realization, so
area = 0.44444444, thus, the weighting is 1/0.44444444
i.e. 2.25.
@param sigma: sigma level of the errors to be plotted
@param nbins: number of bins (for simulated data)
@param area: actually 1 / area, used to weight galaxies

\end{fulllineitems}


\index{plot\_number\_counts2() (in module SamPy.herschel.plotnumbercounts)}

\begin{fulllineitems}
\phantomsection\label{SamPy.herschel:SamPy.herschel.plotnumbercounts.plot_number_counts2}\pysiglinewithargsret{\code{SamPy.herschel.plotnumbercounts.}\bfcode{plot\_number\_counts2}}{\emph{path}, \emph{database}, \emph{band}, \emph{redshifts}, \emph{out\_folder}, \emph{obs\_data}, \emph{goods}, \emph{area=2.25}, \emph{ymin=1000}, \emph{ymax=2000000}, \emph{xmin=0.5}, \emph{xmax=100}, \emph{nbins=15}, \emph{sigma=3.0}, \emph{write\_out=False}}{}
160 (arcminutes squared) = 0.0444444444 square degrees
Simulation was 10 times the GOODS realization, so
area = 0.44444444, thus, the weighting is 1/0.44444444
i.e. 2.25.
@param sigma: sigma level of the errors to be plotted
@param nbins: number of bins (for simulated data)
@param area: actually 1 / area, used to weight galaxies

\end{fulllineitems}


\index{plot\_number\_counts3() (in module SamPy.herschel.plotnumbercounts)}

\begin{fulllineitems}
\phantomsection\label{SamPy.herschel:SamPy.herschel.plotnumbercounts.plot_number_counts3}\pysiglinewithargsret{\code{SamPy.herschel.plotnumbercounts.}\bfcode{plot\_number\_counts3}}{\emph{path}, \emph{database}, \emph{band}, \emph{redshifts}, \emph{out\_folder}, \emph{obs\_data}, \emph{goods}, \emph{area=2.25}, \emph{ymin=1000}, \emph{ymax=2000000}, \emph{xmin=0.5}, \emph{xmax=100}, \emph{nbins=15}, \emph{sigma=3.0}, \emph{write\_out=True}}{}
160 (arcminutes squared) = 0.0444444444 square degrees
Simulation was 10 times the GOODS realization, so
area = 0.44444444, thus, the weighting is 1/0.44444444
i.e. 2.25.
@param sigma: sigma level of the errors to be plotted
@param nbins: number of bins (for simulated data)
@param area: actually 1 / area, used to weight galaxies

\end{fulllineitems}



\subsubsection{\texttt{plotpredictions} Module}
\label{SamPy.herschel:plotpredictions-module}\label{SamPy.herschel:module-SamPy.herschel.plotpredictions}
\index{SamPy.herschel.plotpredictions (module)}
\index{plot\_Age() (in module SamPy.herschel.plotpredictions)}

\begin{fulllineitems}
\phantomsection\label{SamPy.herschel:SamPy.herschel.plotpredictions.plot_Age}\pysiglinewithargsret{\code{SamPy.herschel.plotpredictions.}\bfcode{plot\_Age}}{\emph{path}, \emph{db}, \emph{reshifts}, \emph{out\_folder}, \emph{xmin=0.0}, \emph{xmax=2.1}, \emph{fluxlimit=5}}{}
Plots

\end{fulllineitems}


\index{plot\_BHmass() (in module SamPy.herschel.plotpredictions)}

\begin{fulllineitems}
\phantomsection\label{SamPy.herschel:SamPy.herschel.plotpredictions.plot_BHmass}\pysiglinewithargsret{\code{SamPy.herschel.plotpredictions.}\bfcode{plot\_BHmass}}{\emph{path}, \emph{db}, \emph{reshifts}, \emph{out\_folder}, \emph{xmin=0.0}, \emph{xmax=2.0}, \emph{fluxlimit=5}}{}
Plots

\end{fulllineitems}


\index{plot\_DMmass() (in module SamPy.herschel.plotpredictions)}

\begin{fulllineitems}
\phantomsection\label{SamPy.herschel:SamPy.herschel.plotpredictions.plot_DMmass}\pysiglinewithargsret{\code{SamPy.herschel.plotpredictions.}\bfcode{plot\_DMmass}}{\emph{path}, \emph{db}, \emph{reshifts}, \emph{out\_folder}, \emph{xmin=0.0}, \emph{xmax=2.0}, \emph{fluxlimit=5}}{}
Plots

\end{fulllineitems}


\index{plot\_Lbol() (in module SamPy.herschel.plotpredictions)}

\begin{fulllineitems}
\phantomsection\label{SamPy.herschel:SamPy.herschel.plotpredictions.plot_Lbol}\pysiglinewithargsret{\code{SamPy.herschel.plotpredictions.}\bfcode{plot\_Lbol}}{\emph{path}, \emph{db}, \emph{redshifts}, \emph{out\_folder}, \emph{xmin=0.0}, \emph{xmax=2.3}, \emph{fluxlimit=5}}{}
Plots SPIRE 250 flux versus bolometric luminosity

\end{fulllineitems}


\index{plot\_Ldust() (in module SamPy.herschel.plotpredictions)}

\begin{fulllineitems}
\phantomsection\label{SamPy.herschel:SamPy.herschel.plotpredictions.plot_Ldust}\pysiglinewithargsret{\code{SamPy.herschel.plotpredictions.}\bfcode{plot\_Ldust}}{\emph{path}, \emph{db}, \emph{redshifts}, \emph{out\_folder}, \emph{xmin=0.0}, \emph{xmax=2.3}, \emph{fluxlimit=5}}{}
Plots SPIRE 250 flux versus dust luminosity

\end{fulllineitems}


\index{plot\_burstmass() (in module SamPy.herschel.plotpredictions)}

\begin{fulllineitems}
\phantomsection\label{SamPy.herschel:SamPy.herschel.plotpredictions.plot_burstmass}\pysiglinewithargsret{\code{SamPy.herschel.plotpredictions.}\bfcode{plot\_burstmass}}{\emph{path}, \emph{db}, \emph{reshifts}, \emph{out\_folder}, \emph{xmin=0.0}, \emph{xmax=2.0}, \emph{fluxlimit=5}}{}
Plots

\end{fulllineitems}


\index{plot\_coldgas() (in module SamPy.herschel.plotpredictions)}

\begin{fulllineitems}
\phantomsection\label{SamPy.herschel:SamPy.herschel.plotpredictions.plot_coldgas}\pysiglinewithargsret{\code{SamPy.herschel.plotpredictions.}\bfcode{plot\_coldgas}}{\emph{path}, \emph{db}, \emph{reshifts}, \emph{out\_folder}, \emph{xmin=0.0}, \emph{xmax=2.3}, \emph{fluxlimit=5}}{}
Plots

\end{fulllineitems}


\index{plot\_massratios() (in module SamPy.herschel.plotpredictions)}

\begin{fulllineitems}
\phantomsection\label{SamPy.herschel:SamPy.herschel.plotpredictions.plot_massratios}\pysiglinewithargsret{\code{SamPy.herschel.plotpredictions.}\bfcode{plot\_massratios}}{\emph{path}, \emph{db}, \emph{reshifts}, \emph{out\_folder}, \emph{xmin=0.0}, \emph{xmax=2.0}, \emph{fluxlimit=5}}{}
Plots

\end{fulllineitems}


\index{plot\_mergerfraction() (in module SamPy.herschel.plotpredictions)}

\begin{fulllineitems}
\phantomsection\label{SamPy.herschel:SamPy.herschel.plotpredictions.plot_mergerfraction}\pysiglinewithargsret{\code{SamPy.herschel.plotpredictions.}\bfcode{plot\_mergerfraction}}{\emph{path, db, reshifts, out\_folder, outname, xmin=-0.01, xmax=2.3, fluxlimit=5, png=True, mergetimelimit=0.25, xbin={[}10, 8, 9, 7, 7, 5{]}, neverMerged=False, obs=True}}{}
Plots

\end{fulllineitems}


\index{plot\_mergerfraction2() (in module SamPy.herschel.plotpredictions)}

\begin{fulllineitems}
\phantomsection\label{SamPy.herschel:SamPy.herschel.plotpredictions.plot_mergerfraction2}\pysiglinewithargsret{\code{SamPy.herschel.plotpredictions.}\bfcode{plot\_mergerfraction2}}{\emph{path, db, reshifts, out\_folder, outname, xmin=-0.01, xmax=2.3, fluxlimit=5, png=True, mergetimelimit=0.25, xbin={[}10, 8, 9, 7, 7, 5{]}, neverMerged=False, obs=True}}{}
Plots

\end{fulllineitems}


\index{plot\_metallicity() (in module SamPy.herschel.plotpredictions)}

\begin{fulllineitems}
\phantomsection\label{SamPy.herschel:SamPy.herschel.plotpredictions.plot_metallicity}\pysiglinewithargsret{\code{SamPy.herschel.plotpredictions.}\bfcode{plot\_metallicity}}{\emph{path}, \emph{db}, \emph{reshifts}, \emph{out\_folder}, \emph{xmin=0.0}, \emph{xmax=2.0}, \emph{fluxlimit=5}}{}
Plots

\end{fulllineitems}


\index{plot\_sfrs() (in module SamPy.herschel.plotpredictions)}

\begin{fulllineitems}
\phantomsection\label{SamPy.herschel:SamPy.herschel.plotpredictions.plot_sfrs}\pysiglinewithargsret{\code{SamPy.herschel.plotpredictions.}\bfcode{plot\_sfrs}}{\emph{path}, \emph{db}, \emph{redshifts}, \emph{out\_folder}, \emph{xmin=0.0}, \emph{xmax=2.3}, \emph{fluxlimit=5}, \emph{obs=True}}{}
Plots SFR

\end{fulllineitems}


\index{plot\_ssfr() (in module SamPy.herschel.plotpredictions)}

\begin{fulllineitems}
\phantomsection\label{SamPy.herschel:SamPy.herschel.plotpredictions.plot_ssfr}\pysiglinewithargsret{\code{SamPy.herschel.plotpredictions.}\bfcode{plot\_ssfr}}{\emph{path}, \emph{db}, \emph{redshifts}, \emph{out\_folder}, \emph{xmin=0.0}, \emph{xmax=2.3}, \emph{fluxlimit=5}}{}
Plots SSFR.

\end{fulllineitems}


\index{plot\_starburst() (in module SamPy.herschel.plotpredictions)}

\begin{fulllineitems}
\phantomsection\label{SamPy.herschel:SamPy.herschel.plotpredictions.plot_starburst}\pysiglinewithargsret{\code{SamPy.herschel.plotpredictions.}\bfcode{plot\_starburst}}{\emph{path}, \emph{db}, \emph{reshifts}, \emph{out\_folder}, \emph{xmin=0.0}, \emph{xmax=2.0}, \emph{fluxlimit=5}}{}
Plots

\end{fulllineitems}


\index{plot\_stellarmass() (in module SamPy.herschel.plotpredictions)}

\begin{fulllineitems}
\phantomsection\label{SamPy.herschel:SamPy.herschel.plotpredictions.plot_stellarmass}\pysiglinewithargsret{\code{SamPy.herschel.plotpredictions.}\bfcode{plot\_stellarmass}}{\emph{path}, \emph{db}, \emph{redshifts}, \emph{out\_folder}, \emph{xmin=0.0}, \emph{xmax=2.3}, \emph{fluxlimit=5}}{}
Plots

\end{fulllineitems}



\subsubsection{\texttt{plotproperties} Module}
\label{SamPy.herschel:plotproperties-module}\label{SamPy.herschel:module-SamPy.herschel.plotproperties}
\index{SamPy.herschel.plotproperties (module)}
\index{plot\_properties() (in module SamPy.herschel.plotproperties)}

\begin{fulllineitems}
\phantomsection\label{SamPy.herschel:SamPy.herschel.plotproperties.plot_properties}\pysiglinewithargsret{\code{SamPy.herschel.plotproperties.}\bfcode{plot\_properties}}{\emph{query}, \emph{xlabel}, \emph{ylabel}, \emph{output}, \emph{out\_folder}, \emph{flux=5}, \emph{band=250}, \emph{pmin=0.1}, \emph{pmax=1.0}, \emph{xbin=15}, \emph{ybin=11}, \emph{xmin=7.9}, \emph{xmax=11.7}, \emph{ymin=0}, \emph{ymax=50}}{}
Plots

\end{fulllineitems}



\subsubsection{\texttt{plotsfr} Module}
\label{SamPy.herschel:plotsfr-module}\label{SamPy.herschel:module-SamPy.herschel.plotsfr}
\index{SamPy.herschel.plotsfr (module)}
\index{plotFIRLbolvsSFR() (in module SamPy.herschel.plotsfr)}

\begin{fulllineitems}
\phantomsection\label{SamPy.herschel:SamPy.herschel.plotsfr.plotFIRLbolvsSFR}\pysiglinewithargsret{\code{SamPy.herschel.plotsfr.}\bfcode{plotFIRLbolvsSFR}}{\emph{query}, \emph{path}, \emph{database}, \emph{output}, \emph{out\_folder}, \emph{xbins=90}, \emph{ybins=90}}{}
Plots star formation rate as a function of total IR luminosity.
Uses KDE to plot contours using the probability density function.

\end{fulllineitems}


\index{plotSFRFractions() (in module SamPy.herschel.plotsfr)}

\begin{fulllineitems}
\phantomsection\label{SamPy.herschel:SamPy.herschel.plotsfr.plotSFRFractions}\pysiglinewithargsret{\code{SamPy.herschel.plotsfr.}\bfcode{plotSFRFractions}}{\emph{query}, \emph{xlabel}, \emph{ylabel}, \emph{output}, \emph{out\_folder}, \emph{xmin=8.0}, \emph{xmax=11.5}, \emph{xbins=15}, \emph{ymin=-0.01}, \emph{ymax=1.01}, \emph{logscale=False}}{}
\end{fulllineitems}



\subsubsection{\texttt{plotsize} Module}
\label{SamPy.herschel:module-SamPy.herschel.plotsize}\label{SamPy.herschel:plotsize-module}
\index{SamPy.herschel.plotsize (module)}
\index{plot\_size() (in module SamPy.herschel.plotsize)}

\begin{fulllineitems}
\phantomsection\label{SamPy.herschel:SamPy.herschel.plotsize.plot_size}\pysiglinewithargsret{\code{SamPy.herschel.plotsize.}\bfcode{plot\_size}}{\emph{query, xlabel, ylabel, output, out\_folder, bulge=0.4, title='\$2.0 \textless{} z \textless{} 4.0\$', pmin=0.05, pmax=1.0, xbin=15, ybin=15, y1ticks={[}1, 3, 5, 7, 10{]}, y2ticks={[}1, 3, 5, 7{]}, xmin=7.9, xmax=11.7, ymin=0.1, ymax=10, scatters=False, mean=False}}{}
Plots size versus a given quantity.

\end{fulllineitems}


\index{plot\_size2() (in module SamPy.herschel.plotsize)}

\begin{fulllineitems}
\phantomsection\label{SamPy.herschel:SamPy.herschel.plotsize.plot_size2}\pysiglinewithargsret{\code{SamPy.herschel.plotsize.}\bfcode{plot\_size2}}{\emph{query}, \emph{xlabel}, \emph{ylabel}, \emph{output}, \emph{out\_folder}, \emph{bulge=0.4}, \emph{title='\$2.0 \textless{} z \textless{} 4.0\$'}, \emph{pmin=0.05}, \emph{pmax=1.0}, \emph{xbin=15}, \emph{ybin=15}, \emph{xmin=7.9}, \emph{xmax=11.7}, \emph{ymin=0.1}, \emph{ymax=10}}{}
Plots size versus a given quantity.

\end{fulllineitems}


\index{plot\_size\_paper() (in module SamPy.herschel.plotsize)}

\begin{fulllineitems}
\phantomsection\label{SamPy.herschel:SamPy.herschel.plotsize.plot_size_paper}\pysiglinewithargsret{\code{SamPy.herschel.plotsize.}\bfcode{plot\_size\_paper}}{\emph{xd1, xd2, yd1, yd2, xlabel, ylabel, output, out\_folder, pmin=0.05, pmax=1.0, xbin1=15, ybin1=15, xbin2=15, ybin2=15, y1ticks={[}1, 3, 5, 7, 9, 12{]}, y2ticks={[}1, 3, 5, 7, 9{]}, xmin1=7.9, xmax1=11.7, xmin2=7.9, xmax2=11.7, ymin=0.1, ymax=12, scatters=False, mean=False}}{}
\end{fulllineitems}


\index{plot\_size\_paperKDE() (in module SamPy.herschel.plotsize)}

\begin{fulllineitems}
\phantomsection\label{SamPy.herschel:SamPy.herschel.plotsize.plot_size_paperKDE}\pysiglinewithargsret{\code{SamPy.herschel.plotsize.}\bfcode{plot\_size\_paperKDE}}{\emph{xd1, xd2, yd1, yd2, xlabel, ylabel, output, out\_folder, xbin1=15, ybin1=15, xbin2=15, ybin2=15, y1ticks={[}1, 3, 5, 7, 9, 12{]}, y2ticks={[}1, 3, 5, 7, 9{]}, xmin1=7.9, xmax1=11.7, xmin2=7.9, xmax2=11.7, ymin=0.1, ymax=12, scatters=False, mean=False, lvls=None}}{}
\end{fulllineitems}



\subsubsection{\texttt{plotssfr} Module}
\label{SamPy.herschel:module-SamPy.herschel.plotssfr}\label{SamPy.herschel:plotssfr-module}
\index{SamPy.herschel.plotssfr (module)}
\index{plot\_ssfr\_paper() (in module SamPy.herschel.plotssfr)}

\begin{fulllineitems}
\phantomsection\label{SamPy.herschel:SamPy.herschel.plotssfr.plot_ssfr_paper}\pysiglinewithargsret{\code{SamPy.herschel.plotssfr.}\bfcode{plot\_ssfr\_paper}}{\emph{query1}, \emph{query2}, \emph{out\_folder}, \emph{xmin1=7.7}, \emph{xmax1=11.8}, \emph{xbin1=16}, \emph{ymin1=-11}, \emph{ymax1=-7.5}, \emph{ybin1=16}, \emph{xmin2=9.8}, \emph{xmax2=11.8}, \emph{xbin2=15}, \emph{ymin2=-10}, \emph{ymax2=-7.5}, \emph{ybin2=15}, \emph{pmax=1.0}, \emph{pmin=0.01}}{}
\end{fulllineitems}



\subsubsection{\texttt{plotssfrredshift} Module}
\label{SamPy.herschel:plotssfrredshift-module}\label{SamPy.herschel:module-SamPy.herschel.plotssfrredshift}
\index{SamPy.herschel.plotssfrredshift (module)}
\index{plot() (in module SamPy.herschel.plotssfrredshift)}

\begin{fulllineitems}
\phantomsection\label{SamPy.herschel:SamPy.herschel.plotssfrredshift.plot}\pysiglinewithargsret{\code{SamPy.herschel.plotssfrredshift.}\bfcode{plot}}{\emph{query}, \emph{xlabel}, \emph{ylabel}, \emph{output}, \emph{out\_folder}, \emph{bulge=0.4}, \emph{xbin=10}, \emph{ybin=10}, \emph{xmin=-0.1}, \emph{xmax=6}}{}
\end{fulllineitems}



\subsubsection{\texttt{plotstellarmassfunctions} Module}
\label{SamPy.herschel:plotstellarmassfunctions-module}\label{SamPy.herschel:module-SamPy.herschel.plotstellarmassfunctions}
\index{SamPy.herschel.plotstellarmassfunctions (module)}
@todo: fix the comoving volume calculation.
Take a look the luminosity function plotting...

@author: Sami Niemi

\index{plot\_stellarmasses() (in module SamPy.herschel.plotstellarmassfunctions)}

\begin{fulllineitems}
\phantomsection\label{SamPy.herschel:SamPy.herschel.plotstellarmassfunctions.plot_stellarmasses}\pysiglinewithargsret{\code{SamPy.herschel.plotstellarmassfunctions.}\bfcode{plot\_stellarmasses}}{\emph{path}, \emph{database}, \emph{cut}, \emph{redshifts}, \emph{out\_folder}, \emph{obs\_data}, \emph{solid\_angle=1.077e-05}, \emph{ymin=1000}, \emph{ymax=2000000}, \emph{xmin=0.5}, \emph{xmax=100}, \emph{nbins=15}, \emph{sigma=3.0}, \emph{H0=70.0}, \emph{WM=0.28}, \emph{zmax=6.0}, \emph{write\_out=False}}{}
160 square arcminutes in steradians =
1.354*10**-5 sr (steradians)
Simulation was 10 times the GOODS realization, so
the solid angle is 1.354*10**-4

@param sigma: sigma level of the errors to be plotted
@param nbins: number of bins (for simulated data)
@param area: actually 1 / area, used to weight galaxies

\end{fulllineitems}



\subsection{image Package}
\label{SamPy.image::doc}\label{SamPy.image:image-package}

\subsubsection{\texttt{ImageConvolution} Module}
\label{SamPy.image:module-SamPy.image.ImageConvolution}\label{SamPy.image:imageconvolution-module}
\index{SamPy.image.ImageConvolution (module)}
Created on Jan 20, 2010

@author: Sami-Matias Niemi

\index{Convolve() (in module SamPy.image.ImageConvolution)}

\begin{fulllineitems}
\phantomsection\label{SamPy.image:SamPy.image.ImageConvolution.Convolve}\pysiglinewithargsret{\code{SamPy.image.ImageConvolution.}\bfcode{Convolve}}{\emph{image1}, \emph{image2}, \emph{MinPad=True}, \emph{pad=True}}{}
Convolves image1 with image2.

\end{fulllineitems}



\subsection{log Package}
\label{SamPy.log:log-package}\label{SamPy.log::doc}

\subsubsection{\texttt{Logger} Module}
\label{SamPy.log:module-SamPy.log.Logger}\label{SamPy.log:logger-module}
\index{SamPy.log.Logger (module)}
\index{SimpleLogger (class in SamPy.log.Logger)}

\begin{fulllineitems}
\phantomsection\label{SamPy.log:SamPy.log.Logger.SimpleLogger}\pysiglinewithargsret{\strong{class }\code{SamPy.log.Logger.}\bfcode{SimpleLogger}}{\emph{filename}, \emph{verbose=False}}{}
Bases: \code{object}

A simple class to create a log file or print the information on screen.

\index{write() (SamPy.log.Logger.SimpleLogger method)}

\begin{fulllineitems}
\phantomsection\label{SamPy.log:SamPy.log.Logger.SimpleLogger.write}\pysiglinewithargsret{\bfcode{write}}{\emph{text}}{}
Writes text either to file or screen.

\end{fulllineitems}


\end{fulllineitems}


\index{setUpLogger() (in module SamPy.log.Logger)}

\begin{fulllineitems}
\phantomsection\label{SamPy.log:SamPy.log.Logger.setUpLogger}\pysiglinewithargsret{\code{SamPy.log.Logger.}\bfcode{setUpLogger}}{\emph{log\_filename}, \emph{loggername='logger'}}{}
Sets up a logger.
@param log\_filename: name of the file to save the log. 
@param loggername: name of the logger 
@return: logger instance

\end{fulllineitems}



\subsection{parsing Package}
\label{SamPy.parsing:parsing-package}\label{SamPy.parsing::doc}

\subsubsection{\texttt{BeautifulSoup} Module}
\label{SamPy.parsing:module-SamPy.parsing.BeautifulSoup}\label{SamPy.parsing:beautifulsoup-module}
\index{SamPy.parsing.BeautifulSoup (module)}
Beautiful Soup
Elixir and Tonic
``The Screen-Scraper's Friend''
\href{http://www.crummy.com/software/BeautifulSoup/}{http://www.crummy.com/software/BeautifulSoup/}

Beautiful Soup parses a (possibly invalid) XML or HTML document into a
tree representation. It provides methods and Pythonic idioms that make
it easy to navigate, search, and modify the tree.

A well-formed XML/HTML document yields a well-formed data
structure. An ill-formed XML/HTML document yields a correspondingly
ill-formed data structure. If your document is only locally
well-formed, you can use this library to find and process the
well-formed part of it.

Beautiful Soup works with Python 2.2 and up. It has no external
dependencies, but you'll have more success at converting data to UTF-8
if you also install these three packages:
\begin{itemize}
\item {} 
chardet, for auto-detecting character encodings
\href{http://chardet.feedparser.org/}{http://chardet.feedparser.org/}

\item {} 
cjkcodecs and iconv\_codec, which add more encodings to the ones supported
by stock Python.
\href{http://cjkpython.i18n.org/}{http://cjkpython.i18n.org/}

\end{itemize}

Beautiful Soup defines classes for two main parsing strategies:
\begin{itemize}
\item {} 
BeautifulStoneSoup, for parsing XML, SGML, or your domain-specific
language that kind of looks like XML.

\item {} 
BeautifulSoup, for parsing run-of-the-mill HTML code, be it valid
or invalid. This class has web browser-like heuristics for
obtaining a sensible parse tree in the face of common HTML errors.

\end{itemize}

Beautiful Soup also defines a class (UnicodeDammit) for autodetecting
the encoding of an HTML or XML document, and converting it to
Unicode. Much of this code is taken from Mark Pilgrim's Universal Feed Parser.

For more than you ever wanted to know about Beautiful Soup, see the
documentation:
\href{http://www.crummy.com/software/BeautifulSoup/documentation.html}{http://www.crummy.com/software/BeautifulSoup/documentation.html}

Here, have some legalese:

Copyright (c) 2004-2009, Leonard Richardson

All rights reserved.

Redistribution and use in source and binary forms, with or without
modification, are permitted provided that the following conditions are
met:
\begin{itemize}
\item {} 
Redistributions of source code must retain the above copyright
notice, this list of conditions and the following disclaimer.

\item {} 
Redistributions in binary form must reproduce the above
copyright notice, this list of conditions and the following
disclaimer in the documentation and/or other materials provided
with the distribution.

\item {} 
Neither the name of the the Beautiful Soup Consortium and All
Night Kosher Bakery nor the names of its contributors may be
used to endorse or promote products derived from this software
without specific prior written permission.

\end{itemize}

THIS SOFTWARE IS PROVIDED BY THE COPYRIGHT HOLDERS AND CONTRIBUTORS
``AS IS'' AND ANY EXPRESS OR IMPLIED WARRANTIES, INCLUDING, BUT NOT
LIMITED TO, THE IMPLIED WARRANTIES OF MERCHANTABILITY AND FITNESS FOR
A PARTICULAR PURPOSE ARE DISCLAIMED. IN NO EVENT SHALL THE COPYRIGHT OWNER OR
CONTRIBUTORS BE LIABLE FOR ANY DIRECT, INDIRECT, INCIDENTAL, SPECIAL,
EXEMPLARY, OR CONSEQUENTIAL DAMAGES (INCLUDING, BUT NOT LIMITED TO,
PROCUREMENT OF SUBSTITUTE GOODS OR SERVICES; LOSS OF USE, DATA, OR
PROFITS; OR BUSINESS INTERRUPTION) HOWEVER CAUSED AND ON ANY THEORY OF
LIABILITY, WHETHER IN CONTRACT, STRICT LIABILITY, OR TORT (INCLUDING
NEGLIGENCE OR OTHERWISE) ARISING IN ANY WAY OUT OF THE USE OF THIS
SOFTWARE, EVEN IF ADVISED OF THE POSSIBILITY OF SUCH DAMAGE, DAMMIT.

\index{BeautifulSOAP (class in SamPy.parsing.BeautifulSoup)}

\begin{fulllineitems}
\phantomsection\label{SamPy.parsing:SamPy.parsing.BeautifulSoup.BeautifulSOAP}\pysiglinewithargsret{\strong{class }\code{SamPy.parsing.BeautifulSoup.}\bfcode{BeautifulSOAP}}{\emph{markup='`}, \emph{parseOnlyThese=None}, \emph{fromEncoding=None}, \emph{markupMassage=True}, \emph{smartQuotesTo='xml'}, \emph{convertEntities=None}, \emph{selfClosingTags=None}, \emph{isHTML=False}, \emph{builder=\textless{}class SamPy.parsing.BeautifulSoup.HTMLParserBuilder at 0x110956738\textgreater{}}}{}
Bases: {\hyperref[SamPy.parsing:SamPy.parsing.BeautifulSoup.BeautifulStoneSoup]{\code{SamPy.parsing.BeautifulSoup.BeautifulStoneSoup}}}

This class will push a tag with only a single string child into
the tag's parent as an attribute. The attribute's name is the tag
name, and the value is the string child. An example should give
the flavor of the change:
\begin{description}
\item[{\textless{}foo\textgreater{}\textless{}bar\textgreater{}baz\textless{}/bar\textgreater{}\textless{}/foo\textgreater{}}] \leavevmode
=\textgreater{}

\end{description}

\textless{}foo bar=''baz''\textgreater{}\textless{}bar\textgreater{}baz\textless{}/bar\textgreater{}\textless{}/foo\textgreater{}

You can then access fooTag{[}'bar'{]} instead of fooTag.barTag.string.

This is, of course, useful for scraping structures that tend to
use subelements instead of attributes, such as SOAP messages. Note
that it modifies its input, so don't print the modified version
out.

I'm not sure how many people really want to use this class; let me
know if you do. Mainly I like the name.

\index{popTag() (SamPy.parsing.BeautifulSoup.BeautifulSOAP method)}

\begin{fulllineitems}
\phantomsection\label{SamPy.parsing:SamPy.parsing.BeautifulSoup.BeautifulSOAP.popTag}\pysiglinewithargsret{\bfcode{popTag}}{}{}
\end{fulllineitems}


\end{fulllineitems}


\index{BeautifulSoup (class in SamPy.parsing.BeautifulSoup)}

\begin{fulllineitems}
\phantomsection\label{SamPy.parsing:SamPy.parsing.BeautifulSoup.BeautifulSoup}\pysiglinewithargsret{\strong{class }\code{SamPy.parsing.BeautifulSoup.}\bfcode{BeautifulSoup}}{\emph{*args}, \emph{**kwargs}}{}
Bases: {\hyperref[SamPy.parsing:SamPy.parsing.BeautifulSoup.BeautifulStoneSoup]{\code{SamPy.parsing.BeautifulSoup.BeautifulStoneSoup}}}

This parser knows the following facts about HTML:
\begin{itemize}
\item {} 
Some tags have no closing tag and should be interpreted as being
closed as soon as they are encountered.

\item {} 
The text inside some tags (ie. `script') may contain tags which
are not really part of the document and which should be parsed
as text, not tags. If you want to parse the text as tags, you can
always fetch it and parse it explicitly.

\item {} 
Tag nesting rules:

Most tags can't be nested at all. For instance, the occurance of
a \textless{}p\textgreater{} tag should implicitly close the previous \textless{}p\textgreater{} tag.
\begin{quote}
\begin{description}
\item[{\textless{}p\textgreater{}Para1\textless{}p\textgreater{}Para2}] \leavevmode
should be transformed into:

\end{description}

\textless{}p\textgreater{}Para1\textless{}/p\textgreater{}\textless{}p\textgreater{}Para2
\end{quote}

Some tags can be nested arbitrarily. For instance, the occurance
of a \textless{}blockquote\textgreater{} tag should \_not\_ implicitly close the previous
\textless{}blockquote\textgreater{} tag.
\begin{quote}
\begin{description}
\item[{Alice said: \textless{}blockquote\textgreater{}Bob said: \textless{}blockquote\textgreater{}Blah}] \leavevmode
should NOT be transformed into:

\end{description}

Alice said: \textless{}blockquote\textgreater{}Bob said: \textless{}/blockquote\textgreater{}\textless{}blockquote\textgreater{}Blah
\end{quote}

Some tags can be nested, but the nesting is reset by the
interposition of other tags. For instance, a \textless{}tr\textgreater{} tag should
implicitly close the previous \textless{}tr\textgreater{} tag within the same \textless{}table\textgreater{},
but not close a \textless{}tr\textgreater{} tag in another table.
\begin{quote}
\begin{description}
\item[{\textless{}table\textgreater{}\textless{}tr\textgreater{}Blah\textless{}tr\textgreater{}Blah}] \leavevmode
should be transformed into:

\item[{\textless{}table\textgreater{}\textless{}tr\textgreater{}Blah\textless{}/tr\textgreater{}\textless{}tr\textgreater{}Blah}] \leavevmode
but,

\item[{\textless{}tr\textgreater{}Blah\textless{}table\textgreater{}\textless{}tr\textgreater{}Blah}] \leavevmode
should NOT be transformed into

\end{description}

\textless{}tr\textgreater{}Blah\textless{}table\textgreater{}\textless{}/tr\textgreater{}\textless{}tr\textgreater{}Blah
\end{quote}

\end{itemize}

Differing assumptions about tag nesting rules are a major source
of problems with the BeautifulSoup class. If BeautifulSoup is not
treating as nestable a tag your page author treats as nestable,
try ICantBelieveItsBeautifulSoup, MinimalSoup, or
BeautifulStoneSoup before writing your own subclass.

\index{extractCharsetFromMeta() (SamPy.parsing.BeautifulSoup.BeautifulSoup method)}

\begin{fulllineitems}
\phantomsection\label{SamPy.parsing:SamPy.parsing.BeautifulSoup.BeautifulSoup.extractCharsetFromMeta}\pysiglinewithargsret{\bfcode{extractCharsetFromMeta}}{\emph{attrs}}{}
Beautiful Soup can detect a charset included in a META tag,
try to convert the document to that charset, and re-parse the
document from the beginning.

\end{fulllineitems}


\end{fulllineitems}


\index{BeautifulStoneSoup (class in SamPy.parsing.BeautifulSoup)}

\begin{fulllineitems}
\phantomsection\label{SamPy.parsing:SamPy.parsing.BeautifulSoup.BeautifulStoneSoup}\pysiglinewithargsret{\strong{class }\code{SamPy.parsing.BeautifulSoup.}\bfcode{BeautifulStoneSoup}}{\emph{markup='`}, \emph{parseOnlyThese=None}, \emph{fromEncoding=None}, \emph{markupMassage=True}, \emph{smartQuotesTo='xml'}, \emph{convertEntities=None}, \emph{selfClosingTags=None}, \emph{isHTML=False}, \emph{builder=\textless{}class SamPy.parsing.BeautifulSoup.HTMLParserBuilder at 0x110956738\textgreater{}}}{}
Bases: {\hyperref[SamPy.parsing:SamPy.parsing.BeautifulSoup.Tag]{\code{SamPy.parsing.BeautifulSoup.Tag}}}

This class contains the basic parser and search code. It defines
a parser that knows nothing about tag behavior except for the
following:
\begin{quote}

You can't close a tag without closing all the tags it encloses.
That is, ``\textless{}foo\textgreater{}\textless{}bar\textgreater{}\textless{}/foo\textgreater{}'' actually means
``\textless{}foo\textgreater{}\textless{}bar\textgreater{}\textless{}/bar\textgreater{}\textless{}/foo\textgreater{}''.
\end{quote}

{[}Another possible explanation is ``\textless{}foo\textgreater{}\textless{}bar /\textgreater{}\textless{}/foo\textgreater{}'', but since
this class defines no SELF\_CLOSING\_TAGS, it will never use that
explanation.{]}

This class is useful for parsing XML or made-up markup languages,
or when BeautifulSoup makes an assumption counter to what you were
expecting.

\index{endData() (SamPy.parsing.BeautifulSoup.BeautifulStoneSoup method)}

\begin{fulllineitems}
\phantomsection\label{SamPy.parsing:SamPy.parsing.BeautifulSoup.BeautifulStoneSoup.endData}\pysiglinewithargsret{\bfcode{endData}}{\emph{containerClass=\textless{}class `SamPy.parsing.BeautifulSoup.NavigableString'\textgreater{}}}{}
\end{fulllineitems}


\index{extractCharsetFromMeta() (SamPy.parsing.BeautifulSoup.BeautifulStoneSoup method)}

\begin{fulllineitems}
\phantomsection\label{SamPy.parsing:SamPy.parsing.BeautifulSoup.BeautifulStoneSoup.extractCharsetFromMeta}\pysiglinewithargsret{\bfcode{extractCharsetFromMeta}}{\emph{attrs}}{}
\end{fulllineitems}


\index{handle\_data() (SamPy.parsing.BeautifulSoup.BeautifulStoneSoup method)}

\begin{fulllineitems}
\phantomsection\label{SamPy.parsing:SamPy.parsing.BeautifulSoup.BeautifulStoneSoup.handle_data}\pysiglinewithargsret{\bfcode{handle\_data}}{\emph{data}}{}
\end{fulllineitems}


\index{isSelfClosingTag() (SamPy.parsing.BeautifulSoup.BeautifulStoneSoup method)}

\begin{fulllineitems}
\phantomsection\label{SamPy.parsing:SamPy.parsing.BeautifulSoup.BeautifulStoneSoup.isSelfClosingTag}\pysiglinewithargsret{\bfcode{isSelfClosingTag}}{\emph{name}}{}
Returns true iff the given string is the name of a
self-closing tag according to this parser.

\end{fulllineitems}


\index{popTag() (SamPy.parsing.BeautifulSoup.BeautifulStoneSoup method)}

\begin{fulllineitems}
\phantomsection\label{SamPy.parsing:SamPy.parsing.BeautifulSoup.BeautifulStoneSoup.popTag}\pysiglinewithargsret{\bfcode{popTag}}{}{}
\end{fulllineitems}


\index{pushTag() (SamPy.parsing.BeautifulSoup.BeautifulStoneSoup method)}

\begin{fulllineitems}
\phantomsection\label{SamPy.parsing:SamPy.parsing.BeautifulSoup.BeautifulStoneSoup.pushTag}\pysiglinewithargsret{\bfcode{pushTag}}{\emph{tag}}{}
\end{fulllineitems}


\index{reset() (SamPy.parsing.BeautifulSoup.BeautifulStoneSoup method)}

\begin{fulllineitems}
\phantomsection\label{SamPy.parsing:SamPy.parsing.BeautifulSoup.BeautifulStoneSoup.reset}\pysiglinewithargsret{\bfcode{reset}}{}{}
\end{fulllineitems}


\index{unknown\_endtag() (SamPy.parsing.BeautifulSoup.BeautifulStoneSoup method)}

\begin{fulllineitems}
\phantomsection\label{SamPy.parsing:SamPy.parsing.BeautifulSoup.BeautifulStoneSoup.unknown_endtag}\pysiglinewithargsret{\bfcode{unknown\_endtag}}{\emph{name}}{}
\end{fulllineitems}


\index{unknown\_starttag() (SamPy.parsing.BeautifulSoup.BeautifulStoneSoup method)}

\begin{fulllineitems}
\phantomsection\label{SamPy.parsing:SamPy.parsing.BeautifulSoup.BeautifulStoneSoup.unknown_starttag}\pysiglinewithargsret{\bfcode{unknown\_starttag}}{\emph{name}, \emph{attrs}, \emph{selfClosing=0}}{}
\end{fulllineitems}


\end{fulllineitems}


\index{CData (class in SamPy.parsing.BeautifulSoup)}

\begin{fulllineitems}
\phantomsection\label{SamPy.parsing:SamPy.parsing.BeautifulSoup.CData}\pysigline{\strong{class }\code{SamPy.parsing.BeautifulSoup.}\bfcode{CData}}{}
Bases: {\hyperref[SamPy.parsing:SamPy.parsing.BeautifulSoup.NavigableString]{\code{SamPy.parsing.BeautifulSoup.NavigableString}}}

\index{decodeGivenEventualEncoding() (SamPy.parsing.BeautifulSoup.CData method)}

\begin{fulllineitems}
\phantomsection\label{SamPy.parsing:SamPy.parsing.BeautifulSoup.CData.decodeGivenEventualEncoding}\pysiglinewithargsret{\bfcode{decodeGivenEventualEncoding}}{\emph{eventualEncoding}}{}
\end{fulllineitems}


\end{fulllineitems}


\index{Comment (class in SamPy.parsing.BeautifulSoup)}

\begin{fulllineitems}
\phantomsection\label{SamPy.parsing:SamPy.parsing.BeautifulSoup.Comment}\pysigline{\strong{class }\code{SamPy.parsing.BeautifulSoup.}\bfcode{Comment}}{}
Bases: {\hyperref[SamPy.parsing:SamPy.parsing.BeautifulSoup.NavigableString]{\code{SamPy.parsing.BeautifulSoup.NavigableString}}}

\index{decodeGivenEventualEncoding() (SamPy.parsing.BeautifulSoup.Comment method)}

\begin{fulllineitems}
\phantomsection\label{SamPy.parsing:SamPy.parsing.BeautifulSoup.Comment.decodeGivenEventualEncoding}\pysiglinewithargsret{\bfcode{decodeGivenEventualEncoding}}{\emph{eventualEncoding}}{}
\end{fulllineitems}


\end{fulllineitems}


\index{Declaration (class in SamPy.parsing.BeautifulSoup)}

\begin{fulllineitems}
\phantomsection\label{SamPy.parsing:SamPy.parsing.BeautifulSoup.Declaration}\pysigline{\strong{class }\code{SamPy.parsing.BeautifulSoup.}\bfcode{Declaration}}{}
Bases: {\hyperref[SamPy.parsing:SamPy.parsing.BeautifulSoup.NavigableString]{\code{SamPy.parsing.BeautifulSoup.NavigableString}}}

\index{decodeGivenEventualEncoding() (SamPy.parsing.BeautifulSoup.Declaration method)}

\begin{fulllineitems}
\phantomsection\label{SamPy.parsing:SamPy.parsing.BeautifulSoup.Declaration.decodeGivenEventualEncoding}\pysiglinewithargsret{\bfcode{decodeGivenEventualEncoding}}{\emph{eventualEncoding}}{}
\end{fulllineitems}


\end{fulllineitems}


\index{HTMLParserBuilder (class in SamPy.parsing.BeautifulSoup)}

\begin{fulllineitems}
\phantomsection\label{SamPy.parsing:SamPy.parsing.BeautifulSoup.HTMLParserBuilder}\pysiglinewithargsret{\strong{class }\code{SamPy.parsing.BeautifulSoup.}\bfcode{HTMLParserBuilder}}{\emph{soup}}{}
Bases: \code{HTMLParser.HTMLParser}

\index{handle\_charref() (SamPy.parsing.BeautifulSoup.HTMLParserBuilder method)}

\begin{fulllineitems}
\phantomsection\label{SamPy.parsing:SamPy.parsing.BeautifulSoup.HTMLParserBuilder.handle_charref}\pysiglinewithargsret{\bfcode{handle\_charref}}{\emph{ref}}{}
Handle character references as data.

\end{fulllineitems}


\index{handle\_comment() (SamPy.parsing.BeautifulSoup.HTMLParserBuilder method)}

\begin{fulllineitems}
\phantomsection\label{SamPy.parsing:SamPy.parsing.BeautifulSoup.HTMLParserBuilder.handle_comment}\pysiglinewithargsret{\bfcode{handle\_comment}}{\emph{text}}{}
Handle comments as Comment objects.

\end{fulllineitems}


\index{handle\_data() (SamPy.parsing.BeautifulSoup.HTMLParserBuilder method)}

\begin{fulllineitems}
\phantomsection\label{SamPy.parsing:SamPy.parsing.BeautifulSoup.HTMLParserBuilder.handle_data}\pysiglinewithargsret{\bfcode{handle\_data}}{\emph{content}}{}
\end{fulllineitems}


\index{handle\_decl() (SamPy.parsing.BeautifulSoup.HTMLParserBuilder method)}

\begin{fulllineitems}
\phantomsection\label{SamPy.parsing:SamPy.parsing.BeautifulSoup.HTMLParserBuilder.handle_decl}\pysiglinewithargsret{\bfcode{handle\_decl}}{\emph{data}}{}
Handle DOCTYPEs and the like as Declaration objects.

\end{fulllineitems}


\index{handle\_endtag() (SamPy.parsing.BeautifulSoup.HTMLParserBuilder method)}

\begin{fulllineitems}
\phantomsection\label{SamPy.parsing:SamPy.parsing.BeautifulSoup.HTMLParserBuilder.handle_endtag}\pysiglinewithargsret{\bfcode{handle\_endtag}}{\emph{name}}{}
\end{fulllineitems}


\index{handle\_entityref() (SamPy.parsing.BeautifulSoup.HTMLParserBuilder method)}

\begin{fulllineitems}
\phantomsection\label{SamPy.parsing:SamPy.parsing.BeautifulSoup.HTMLParserBuilder.handle_entityref}\pysiglinewithargsret{\bfcode{handle\_entityref}}{\emph{ref}}{}
Handle entity references as data, possibly converting known
HTML and/or XML entity references to the corresponding Unicode
characters.

\end{fulllineitems}


\index{handle\_pi() (SamPy.parsing.BeautifulSoup.HTMLParserBuilder method)}

\begin{fulllineitems}
\phantomsection\label{SamPy.parsing:SamPy.parsing.BeautifulSoup.HTMLParserBuilder.handle_pi}\pysiglinewithargsret{\bfcode{handle\_pi}}{\emph{text}}{}
Handle a processing instruction as a ProcessingInstruction
object, possibly one with a \%SOUP-ENCODING\% slot into which an
encoding will be plugged later.

\end{fulllineitems}


\index{handle\_starttag() (SamPy.parsing.BeautifulSoup.HTMLParserBuilder method)}

\begin{fulllineitems}
\phantomsection\label{SamPy.parsing:SamPy.parsing.BeautifulSoup.HTMLParserBuilder.handle_starttag}\pysiglinewithargsret{\bfcode{handle\_starttag}}{\emph{name}, \emph{attrs}}{}
\end{fulllineitems}


\index{parse\_declaration() (SamPy.parsing.BeautifulSoup.HTMLParserBuilder method)}

\begin{fulllineitems}
\phantomsection\label{SamPy.parsing:SamPy.parsing.BeautifulSoup.HTMLParserBuilder.parse_declaration}\pysiglinewithargsret{\bfcode{parse\_declaration}}{\emph{i}}{}
Treat a bogus SGML declaration as raw data. Treat a CDATA
declaration as a CData object.

\end{fulllineitems}


\end{fulllineitems}


\index{ICantBelieveItsBeautifulSoup (class in SamPy.parsing.BeautifulSoup)}

\begin{fulllineitems}
\phantomsection\label{SamPy.parsing:SamPy.parsing.BeautifulSoup.ICantBelieveItsBeautifulSoup}\pysiglinewithargsret{\strong{class }\code{SamPy.parsing.BeautifulSoup.}\bfcode{ICantBelieveItsBeautifulSoup}}{\emph{*args}, \emph{**kwargs}}{}
Bases: {\hyperref[SamPy.parsing:SamPy.parsing.BeautifulSoup.BeautifulSoup]{\code{SamPy.parsing.BeautifulSoup.BeautifulSoup}}}

The BeautifulSoup class is oriented towards skipping over
common HTML errors like unclosed tags. However, sometimes it makes
errors of its own. For instance, consider this fragment:
\begin{quote}

\textless{}b\textgreater{}Foo\textless{}b\textgreater{}Bar\textless{}/b\textgreater{}\textless{}/b\textgreater{}
\end{quote}

This is perfectly valid (if bizarre) HTML. However, the
BeautifulSoup class will implicitly close the first b tag when it
encounters the second `b'. It will think the author wrote
``\textless{}b\textgreater{}Foo\textless{}b\textgreater{}Bar'', and didn't close the first `b' tag, because
there's no real-world reason to bold something that's already
bold. When it encounters `\textless{}/b\textgreater{}\textless{}/b\textgreater{}' it will close two more `b'
tags, for a grand total of three tags closed instead of two. This
can throw off the rest of your document structure. The same is
true of a number of other tags, listed below.

It's much more common for someone to forget to close a `b' tag
than to actually use nested `b' tags, and the BeautifulSoup class
handles the common case. This class handles the not-co-common
case: where you can't believe someone wrote what they did, but
it's valid HTML and BeautifulSoup screwed up by assuming it
wouldn't be.

\end{fulllineitems}


\index{MinimalSoup (class in SamPy.parsing.BeautifulSoup)}

\begin{fulllineitems}
\phantomsection\label{SamPy.parsing:SamPy.parsing.BeautifulSoup.MinimalSoup}\pysiglinewithargsret{\strong{class }\code{SamPy.parsing.BeautifulSoup.}\bfcode{MinimalSoup}}{\emph{*args}, \emph{**kwargs}}{}
Bases: {\hyperref[SamPy.parsing:SamPy.parsing.BeautifulSoup.BeautifulSoup]{\code{SamPy.parsing.BeautifulSoup.BeautifulSoup}}}

The MinimalSoup class is for parsing HTML that contains
pathologically bad markup. It makes no assumptions about tag
nesting, but it does know which tags are self-closing, that
\textless{}script\textgreater{} tags contain Javascript and should not be parsed, that
META tags may contain encoding information, and so on.

This also makes it better for subclassing than BeautifulStoneSoup
or BeautifulSoup.

\end{fulllineitems}


\index{NavigableString (class in SamPy.parsing.BeautifulSoup)}

\begin{fulllineitems}
\phantomsection\label{SamPy.parsing:SamPy.parsing.BeautifulSoup.NavigableString}\pysigline{\strong{class }\code{SamPy.parsing.BeautifulSoup.}\bfcode{NavigableString}}{}
Bases: \code{unicode}, {\hyperref[SamPy.parsing:SamPy.parsing.BeautifulSoup.PageElement]{\code{SamPy.parsing.BeautifulSoup.PageElement}}}

\index{decodeGivenEventualEncoding() (SamPy.parsing.BeautifulSoup.NavigableString method)}

\begin{fulllineitems}
\phantomsection\label{SamPy.parsing:SamPy.parsing.BeautifulSoup.NavigableString.decodeGivenEventualEncoding}\pysiglinewithargsret{\bfcode{decodeGivenEventualEncoding}}{\emph{eventualEncoding}}{}
\end{fulllineitems}


\index{encode() (SamPy.parsing.BeautifulSoup.NavigableString method)}

\begin{fulllineitems}
\phantomsection\label{SamPy.parsing:SamPy.parsing.BeautifulSoup.NavigableString.encode}\pysiglinewithargsret{\bfcode{encode}}{\emph{encoding='utf-8'}}{}
\end{fulllineitems}


\end{fulllineitems}


\index{PageElement (class in SamPy.parsing.BeautifulSoup)}

\begin{fulllineitems}
\phantomsection\label{SamPy.parsing:SamPy.parsing.BeautifulSoup.PageElement}\pysigline{\strong{class }\code{SamPy.parsing.BeautifulSoup.}\bfcode{PageElement}}{}
Contains the navigational information for some part of the page
(either a tag or a piece of text)

\index{append() (SamPy.parsing.BeautifulSoup.PageElement method)}

\begin{fulllineitems}
\phantomsection\label{SamPy.parsing:SamPy.parsing.BeautifulSoup.PageElement.append}\pysiglinewithargsret{\bfcode{append}}{\emph{tag}}{}
Appends the given tag to the contents of this tag.

\end{fulllineitems}


\index{extract() (SamPy.parsing.BeautifulSoup.PageElement method)}

\begin{fulllineitems}
\phantomsection\label{SamPy.parsing:SamPy.parsing.BeautifulSoup.PageElement.extract}\pysiglinewithargsret{\bfcode{extract}}{}{}
Destructively rips this element out of the tree.

\end{fulllineitems}


\index{fetchNextSiblings() (SamPy.parsing.BeautifulSoup.PageElement method)}

\begin{fulllineitems}
\phantomsection\label{SamPy.parsing:SamPy.parsing.BeautifulSoup.PageElement.fetchNextSiblings}\pysiglinewithargsret{\bfcode{fetchNextSiblings}}{\emph{name=None}, \emph{attrs=\{\}}, \emph{text=None}, \emph{limit=None}, \emph{**kwargs}}{}
Returns the siblings of this Tag that match the given
criteria and appear after this Tag in the document.

\end{fulllineitems}


\index{fetchParents() (SamPy.parsing.BeautifulSoup.PageElement method)}

\begin{fulllineitems}
\phantomsection\label{SamPy.parsing:SamPy.parsing.BeautifulSoup.PageElement.fetchParents}\pysiglinewithargsret{\bfcode{fetchParents}}{\emph{name=None}, \emph{attrs=\{\}}, \emph{limit=None}, \emph{**kwargs}}{}
Returns the parents of this Tag that match the given
criteria.

\end{fulllineitems}


\index{fetchPrevious() (SamPy.parsing.BeautifulSoup.PageElement method)}

\begin{fulllineitems}
\phantomsection\label{SamPy.parsing:SamPy.parsing.BeautifulSoup.PageElement.fetchPrevious}\pysiglinewithargsret{\bfcode{fetchPrevious}}{\emph{name=None}, \emph{attrs=\{\}}, \emph{text=None}, \emph{limit=None}, \emph{**kwargs}}{}
Returns all items that match the given criteria and appear
before this Tag in the document.

\end{fulllineitems}


\index{fetchPreviousSiblings() (SamPy.parsing.BeautifulSoup.PageElement method)}

\begin{fulllineitems}
\phantomsection\label{SamPy.parsing:SamPy.parsing.BeautifulSoup.PageElement.fetchPreviousSiblings}\pysiglinewithargsret{\bfcode{fetchPreviousSiblings}}{\emph{name=None}, \emph{attrs=\{\}}, \emph{text=None}, \emph{limit=None}, \emph{**kwargs}}{}
Returns the siblings of this Tag that match the given
criteria and appear before this Tag in the document.

\end{fulllineitems}


\index{findAllNext() (SamPy.parsing.BeautifulSoup.PageElement method)}

\begin{fulllineitems}
\phantomsection\label{SamPy.parsing:SamPy.parsing.BeautifulSoup.PageElement.findAllNext}\pysiglinewithargsret{\bfcode{findAllNext}}{\emph{name=None}, \emph{attrs=\{\}}, \emph{text=None}, \emph{limit=None}, \emph{**kwargs}}{}
Returns all items that match the given criteria and appear
after this Tag in the document.

\end{fulllineitems}


\index{findAllPrevious() (SamPy.parsing.BeautifulSoup.PageElement method)}

\begin{fulllineitems}
\phantomsection\label{SamPy.parsing:SamPy.parsing.BeautifulSoup.PageElement.findAllPrevious}\pysiglinewithargsret{\bfcode{findAllPrevious}}{\emph{name=None}, \emph{attrs=\{\}}, \emph{text=None}, \emph{limit=None}, \emph{**kwargs}}{}
Returns all items that match the given criteria and appear
before this Tag in the document.

\end{fulllineitems}


\index{findNext() (SamPy.parsing.BeautifulSoup.PageElement method)}

\begin{fulllineitems}
\phantomsection\label{SamPy.parsing:SamPy.parsing.BeautifulSoup.PageElement.findNext}\pysiglinewithargsret{\bfcode{findNext}}{\emph{name=None}, \emph{attrs=\{\}}, \emph{text=None}, \emph{**kwargs}}{}
Returns the first item that matches the given criteria and
appears after this Tag in the document.

\end{fulllineitems}


\index{findNextSibling() (SamPy.parsing.BeautifulSoup.PageElement method)}

\begin{fulllineitems}
\phantomsection\label{SamPy.parsing:SamPy.parsing.BeautifulSoup.PageElement.findNextSibling}\pysiglinewithargsret{\bfcode{findNextSibling}}{\emph{name=None}, \emph{attrs=\{\}}, \emph{text=None}, \emph{**kwargs}}{}
Returns the closest sibling to this Tag that matches the
given criteria and appears after this Tag in the document.

\end{fulllineitems}


\index{findNextSiblings() (SamPy.parsing.BeautifulSoup.PageElement method)}

\begin{fulllineitems}
\phantomsection\label{SamPy.parsing:SamPy.parsing.BeautifulSoup.PageElement.findNextSiblings}\pysiglinewithargsret{\bfcode{findNextSiblings}}{\emph{name=None}, \emph{attrs=\{\}}, \emph{text=None}, \emph{limit=None}, \emph{**kwargs}}{}
Returns the siblings of this Tag that match the given
criteria and appear after this Tag in the document.

\end{fulllineitems}


\index{findParent() (SamPy.parsing.BeautifulSoup.PageElement method)}

\begin{fulllineitems}
\phantomsection\label{SamPy.parsing:SamPy.parsing.BeautifulSoup.PageElement.findParent}\pysiglinewithargsret{\bfcode{findParent}}{\emph{name=None}, \emph{attrs=\{\}}, \emph{**kwargs}}{}
Returns the closest parent of this Tag that matches the given
criteria.

\end{fulllineitems}


\index{findParents() (SamPy.parsing.BeautifulSoup.PageElement method)}

\begin{fulllineitems}
\phantomsection\label{SamPy.parsing:SamPy.parsing.BeautifulSoup.PageElement.findParents}\pysiglinewithargsret{\bfcode{findParents}}{\emph{name=None}, \emph{attrs=\{\}}, \emph{limit=None}, \emph{**kwargs}}{}
Returns the parents of this Tag that match the given
criteria.

\end{fulllineitems}


\index{findPrevious() (SamPy.parsing.BeautifulSoup.PageElement method)}

\begin{fulllineitems}
\phantomsection\label{SamPy.parsing:SamPy.parsing.BeautifulSoup.PageElement.findPrevious}\pysiglinewithargsret{\bfcode{findPrevious}}{\emph{name=None}, \emph{attrs=\{\}}, \emph{text=None}, \emph{**kwargs}}{}
Returns the first item that matches the given criteria and
appears before this Tag in the document.

\end{fulllineitems}


\index{findPreviousSibling() (SamPy.parsing.BeautifulSoup.PageElement method)}

\begin{fulllineitems}
\phantomsection\label{SamPy.parsing:SamPy.parsing.BeautifulSoup.PageElement.findPreviousSibling}\pysiglinewithargsret{\bfcode{findPreviousSibling}}{\emph{name=None}, \emph{attrs=\{\}}, \emph{text=None}, \emph{**kwargs}}{}
Returns the closest sibling to this Tag that matches the
given criteria and appears before this Tag in the document.

\end{fulllineitems}


\index{findPreviousSiblings() (SamPy.parsing.BeautifulSoup.PageElement method)}

\begin{fulllineitems}
\phantomsection\label{SamPy.parsing:SamPy.parsing.BeautifulSoup.PageElement.findPreviousSiblings}\pysiglinewithargsret{\bfcode{findPreviousSiblings}}{\emph{name=None}, \emph{attrs=\{\}}, \emph{text=None}, \emph{limit=None}, \emph{**kwargs}}{}
Returns the siblings of this Tag that match the given
criteria and appear before this Tag in the document.

\end{fulllineitems}


\index{insert() (SamPy.parsing.BeautifulSoup.PageElement method)}

\begin{fulllineitems}
\phantomsection\label{SamPy.parsing:SamPy.parsing.BeautifulSoup.PageElement.insert}\pysiglinewithargsret{\bfcode{insert}}{\emph{position}, \emph{newChild}}{}
\end{fulllineitems}


\index{nextGenerator() (SamPy.parsing.BeautifulSoup.PageElement method)}

\begin{fulllineitems}
\phantomsection\label{SamPy.parsing:SamPy.parsing.BeautifulSoup.PageElement.nextGenerator}\pysiglinewithargsret{\bfcode{nextGenerator}}{}{}
\end{fulllineitems}


\index{nextSiblingGenerator() (SamPy.parsing.BeautifulSoup.PageElement method)}

\begin{fulllineitems}
\phantomsection\label{SamPy.parsing:SamPy.parsing.BeautifulSoup.PageElement.nextSiblingGenerator}\pysiglinewithargsret{\bfcode{nextSiblingGenerator}}{}{}
\end{fulllineitems}


\index{parentGenerator() (SamPy.parsing.BeautifulSoup.PageElement method)}

\begin{fulllineitems}
\phantomsection\label{SamPy.parsing:SamPy.parsing.BeautifulSoup.PageElement.parentGenerator}\pysiglinewithargsret{\bfcode{parentGenerator}}{}{}
\end{fulllineitems}


\index{previousGenerator() (SamPy.parsing.BeautifulSoup.PageElement method)}

\begin{fulllineitems}
\phantomsection\label{SamPy.parsing:SamPy.parsing.BeautifulSoup.PageElement.previousGenerator}\pysiglinewithargsret{\bfcode{previousGenerator}}{}{}
\end{fulllineitems}


\index{previousSiblingGenerator() (SamPy.parsing.BeautifulSoup.PageElement method)}

\begin{fulllineitems}
\phantomsection\label{SamPy.parsing:SamPy.parsing.BeautifulSoup.PageElement.previousSiblingGenerator}\pysiglinewithargsret{\bfcode{previousSiblingGenerator}}{}{}
\end{fulllineitems}


\index{replaceWith() (SamPy.parsing.BeautifulSoup.PageElement method)}

\begin{fulllineitems}
\phantomsection\label{SamPy.parsing:SamPy.parsing.BeautifulSoup.PageElement.replaceWith}\pysiglinewithargsret{\bfcode{replaceWith}}{\emph{replaceWith}}{}
\end{fulllineitems}


\index{setup() (SamPy.parsing.BeautifulSoup.PageElement method)}

\begin{fulllineitems}
\phantomsection\label{SamPy.parsing:SamPy.parsing.BeautifulSoup.PageElement.setup}\pysiglinewithargsret{\bfcode{setup}}{\emph{parent=None}, \emph{previous=None}}{}
Sets up the initial relations between this element and
other elements.

\end{fulllineitems}


\index{substituteEncoding() (SamPy.parsing.BeautifulSoup.PageElement method)}

\begin{fulllineitems}
\phantomsection\label{SamPy.parsing:SamPy.parsing.BeautifulSoup.PageElement.substituteEncoding}\pysiglinewithargsret{\bfcode{substituteEncoding}}{\emph{str}, \emph{encoding=None}}{}
\end{fulllineitems}


\index{toEncoding() (SamPy.parsing.BeautifulSoup.PageElement method)}

\begin{fulllineitems}
\phantomsection\label{SamPy.parsing:SamPy.parsing.BeautifulSoup.PageElement.toEncoding}\pysiglinewithargsret{\bfcode{toEncoding}}{\emph{s}, \emph{encoding=None}}{}
Encodes an object to a string in some encoding, or to Unicode.
.

\end{fulllineitems}


\end{fulllineitems}


\index{ProcessingInstruction (class in SamPy.parsing.BeautifulSoup)}

\begin{fulllineitems}
\phantomsection\label{SamPy.parsing:SamPy.parsing.BeautifulSoup.ProcessingInstruction}\pysigline{\strong{class }\code{SamPy.parsing.BeautifulSoup.}\bfcode{ProcessingInstruction}}{}
Bases: {\hyperref[SamPy.parsing:SamPy.parsing.BeautifulSoup.NavigableString]{\code{SamPy.parsing.BeautifulSoup.NavigableString}}}

\index{decodeGivenEventualEncoding() (SamPy.parsing.BeautifulSoup.ProcessingInstruction method)}

\begin{fulllineitems}
\phantomsection\label{SamPy.parsing:SamPy.parsing.BeautifulSoup.ProcessingInstruction.decodeGivenEventualEncoding}\pysiglinewithargsret{\bfcode{decodeGivenEventualEncoding}}{\emph{eventualEncoding}}{}
\end{fulllineitems}


\end{fulllineitems}


\index{ResultSet (class in SamPy.parsing.BeautifulSoup)}

\begin{fulllineitems}
\phantomsection\label{SamPy.parsing:SamPy.parsing.BeautifulSoup.ResultSet}\pysiglinewithargsret{\strong{class }\code{SamPy.parsing.BeautifulSoup.}\bfcode{ResultSet}}{\emph{source}}{}
Bases: \code{list}

A ResultSet is just a list that keeps track of the SoupStrainer
that created it.

\end{fulllineitems}


\index{RobustHTMLParser (class in SamPy.parsing.BeautifulSoup)}

\begin{fulllineitems}
\phantomsection\label{SamPy.parsing:SamPy.parsing.BeautifulSoup.RobustHTMLParser}\pysiglinewithargsret{\strong{class }\code{SamPy.parsing.BeautifulSoup.}\bfcode{RobustHTMLParser}}{\emph{*args}, \emph{**kwargs}}{}
Bases: {\hyperref[SamPy.parsing:SamPy.parsing.BeautifulSoup.BeautifulSoup]{\code{SamPy.parsing.BeautifulSoup.BeautifulSoup}}}

\end{fulllineitems}


\index{RobustInsanelyWackAssHTMLParser (class in SamPy.parsing.BeautifulSoup)}

\begin{fulllineitems}
\phantomsection\label{SamPy.parsing:SamPy.parsing.BeautifulSoup.RobustInsanelyWackAssHTMLParser}\pysiglinewithargsret{\strong{class }\code{SamPy.parsing.BeautifulSoup.}\bfcode{RobustInsanelyWackAssHTMLParser}}{\emph{*args}, \emph{**kwargs}}{}
Bases: {\hyperref[SamPy.parsing:SamPy.parsing.BeautifulSoup.MinimalSoup]{\code{SamPy.parsing.BeautifulSoup.MinimalSoup}}}

\end{fulllineitems}


\index{RobustWackAssHTMLParser (class in SamPy.parsing.BeautifulSoup)}

\begin{fulllineitems}
\phantomsection\label{SamPy.parsing:SamPy.parsing.BeautifulSoup.RobustWackAssHTMLParser}\pysiglinewithargsret{\strong{class }\code{SamPy.parsing.BeautifulSoup.}\bfcode{RobustWackAssHTMLParser}}{\emph{*args}, \emph{**kwargs}}{}
Bases: {\hyperref[SamPy.parsing:SamPy.parsing.BeautifulSoup.ICantBelieveItsBeautifulSoup]{\code{SamPy.parsing.BeautifulSoup.ICantBelieveItsBeautifulSoup}}}

\end{fulllineitems}


\index{RobustXMLParser (class in SamPy.parsing.BeautifulSoup)}

\begin{fulllineitems}
\phantomsection\label{SamPy.parsing:SamPy.parsing.BeautifulSoup.RobustXMLParser}\pysiglinewithargsret{\strong{class }\code{SamPy.parsing.BeautifulSoup.}\bfcode{RobustXMLParser}}{\emph{markup='`}, \emph{parseOnlyThese=None}, \emph{fromEncoding=None}, \emph{markupMassage=True}, \emph{smartQuotesTo='xml'}, \emph{convertEntities=None}, \emph{selfClosingTags=None}, \emph{isHTML=False}, \emph{builder=\textless{}class SamPy.parsing.BeautifulSoup.HTMLParserBuilder at 0x110956738\textgreater{}}}{}
Bases: {\hyperref[SamPy.parsing:SamPy.parsing.BeautifulSoup.BeautifulStoneSoup]{\code{SamPy.parsing.BeautifulSoup.BeautifulStoneSoup}}}

\end{fulllineitems}


\index{SimplifyingSOAPParser (class in SamPy.parsing.BeautifulSoup)}

\begin{fulllineitems}
\phantomsection\label{SamPy.parsing:SamPy.parsing.BeautifulSoup.SimplifyingSOAPParser}\pysiglinewithargsret{\strong{class }\code{SamPy.parsing.BeautifulSoup.}\bfcode{SimplifyingSOAPParser}}{\emph{markup='`}, \emph{parseOnlyThese=None}, \emph{fromEncoding=None}, \emph{markupMassage=True}, \emph{smartQuotesTo='xml'}, \emph{convertEntities=None}, \emph{selfClosingTags=None}, \emph{isHTML=False}, \emph{builder=\textless{}class SamPy.parsing.BeautifulSoup.HTMLParserBuilder at 0x110956738\textgreater{}}}{}
Bases: {\hyperref[SamPy.parsing:SamPy.parsing.BeautifulSoup.BeautifulSOAP]{\code{SamPy.parsing.BeautifulSoup.BeautifulSOAP}}}

\end{fulllineitems}


\index{SoupStrainer (class in SamPy.parsing.BeautifulSoup)}

\begin{fulllineitems}
\phantomsection\label{SamPy.parsing:SamPy.parsing.BeautifulSoup.SoupStrainer}\pysiglinewithargsret{\strong{class }\code{SamPy.parsing.BeautifulSoup.}\bfcode{SoupStrainer}}{\emph{name=None}, \emph{attrs=\{\}}, \emph{text=None}, \emph{**kwargs}}{}
Encapsulates a number of ways of matching a markup element (tag or
text).

\index{search() (SamPy.parsing.BeautifulSoup.SoupStrainer method)}

\begin{fulllineitems}
\phantomsection\label{SamPy.parsing:SamPy.parsing.BeautifulSoup.SoupStrainer.search}\pysiglinewithargsret{\bfcode{search}}{\emph{markup}}{}
\end{fulllineitems}


\index{searchTag() (SamPy.parsing.BeautifulSoup.SoupStrainer method)}

\begin{fulllineitems}
\phantomsection\label{SamPy.parsing:SamPy.parsing.BeautifulSoup.SoupStrainer.searchTag}\pysiglinewithargsret{\bfcode{searchTag}}{\emph{markupName=None}, \emph{markupAttrs=\{\}}}{}
\end{fulllineitems}


\end{fulllineitems}


\index{StopParsing}

\begin{fulllineitems}
\phantomsection\label{SamPy.parsing:SamPy.parsing.BeautifulSoup.StopParsing}\pysigline{\strong{exception }\code{SamPy.parsing.BeautifulSoup.}\bfcode{StopParsing}}{}
Bases: \code{exceptions.Exception}

\end{fulllineitems}


\index{Tag (class in SamPy.parsing.BeautifulSoup)}

\begin{fulllineitems}
\phantomsection\label{SamPy.parsing:SamPy.parsing.BeautifulSoup.Tag}\pysiglinewithargsret{\strong{class }\code{SamPy.parsing.BeautifulSoup.}\bfcode{Tag}}{\emph{parser}, \emph{name}, \emph{attrs=None}, \emph{parent=None}, \emph{previous=None}}{}
Bases: {\hyperref[SamPy.parsing:SamPy.parsing.BeautifulSoup.PageElement]{\code{SamPy.parsing.BeautifulSoup.PageElement}}}

Represents a found HTML tag with its attributes and contents.

\index{childGenerator() (SamPy.parsing.BeautifulSoup.Tag method)}

\begin{fulllineitems}
\phantomsection\label{SamPy.parsing:SamPy.parsing.BeautifulSoup.Tag.childGenerator}\pysiglinewithargsret{\bfcode{childGenerator}}{}{}
\end{fulllineitems}


\index{decode() (SamPy.parsing.BeautifulSoup.Tag method)}

\begin{fulllineitems}
\phantomsection\label{SamPy.parsing:SamPy.parsing.BeautifulSoup.Tag.decode}\pysiglinewithargsret{\bfcode{decode}}{\emph{prettyPrint=False}, \emph{indentLevel=0}, \emph{eventualEncoding='utf-8'}}{}
Returns a string or Unicode representation of this tag and
its contents. To get Unicode, pass None for encoding.

\end{fulllineitems}


\index{decodeContents() (SamPy.parsing.BeautifulSoup.Tag method)}

\begin{fulllineitems}
\phantomsection\label{SamPy.parsing:SamPy.parsing.BeautifulSoup.Tag.decodeContents}\pysiglinewithargsret{\bfcode{decodeContents}}{\emph{prettyPrint=False}, \emph{indentLevel=0}, \emph{eventualEncoding='utf-8'}}{}
Renders the contents of this tag as a string in the given
encoding. If encoding is None, returns a Unicode string..

\end{fulllineitems}


\index{decompose() (SamPy.parsing.BeautifulSoup.Tag method)}

\begin{fulllineitems}
\phantomsection\label{SamPy.parsing:SamPy.parsing.BeautifulSoup.Tag.decompose}\pysiglinewithargsret{\bfcode{decompose}}{}{}
Recursively destroys the contents of this tree.

\end{fulllineitems}


\index{encode() (SamPy.parsing.BeautifulSoup.Tag method)}

\begin{fulllineitems}
\phantomsection\label{SamPy.parsing:SamPy.parsing.BeautifulSoup.Tag.encode}\pysiglinewithargsret{\bfcode{encode}}{\emph{encoding='utf-8'}, \emph{prettyPrint=False}, \emph{indentLevel=0}}{}
\end{fulllineitems}


\index{encodeContents() (SamPy.parsing.BeautifulSoup.Tag method)}

\begin{fulllineitems}
\phantomsection\label{SamPy.parsing:SamPy.parsing.BeautifulSoup.Tag.encodeContents}\pysiglinewithargsret{\bfcode{encodeContents}}{\emph{encoding='utf-8'}, \emph{prettyPrint=False}, \emph{indentLevel=0}}{}
\end{fulllineitems}


\index{fetch() (SamPy.parsing.BeautifulSoup.Tag method)}

\begin{fulllineitems}
\phantomsection\label{SamPy.parsing:SamPy.parsing.BeautifulSoup.Tag.fetch}\pysiglinewithargsret{\bfcode{fetch}}{\emph{name=None}, \emph{attrs=\{\}}, \emph{recursive=True}, \emph{text=None}, \emph{limit=None}, \emph{**kwargs}}{}
Extracts a list of Tag objects that match the given
criteria.  You can specify the name of the Tag and any
attributes you want the Tag to have.

The value of a key-value pair in the `attrs' map can be a
string, a list of strings, a regular expression object, or a
callable that takes a string and returns whether or not the
string matches for some custom definition of `matches'. The
same is true of the tag name.

\end{fulllineitems}


\index{fetchText() (SamPy.parsing.BeautifulSoup.Tag method)}

\begin{fulllineitems}
\phantomsection\label{SamPy.parsing:SamPy.parsing.BeautifulSoup.Tag.fetchText}\pysiglinewithargsret{\bfcode{fetchText}}{\emph{text=None}, \emph{recursive=True}, \emph{limit=None}}{}
\end{fulllineitems}


\index{find() (SamPy.parsing.BeautifulSoup.Tag method)}

\begin{fulllineitems}
\phantomsection\label{SamPy.parsing:SamPy.parsing.BeautifulSoup.Tag.find}\pysiglinewithargsret{\bfcode{find}}{\emph{name=None}, \emph{attrs=\{\}}, \emph{recursive=True}, \emph{text=None}, \emph{**kwargs}}{}
Return only the first child of this Tag matching the given
criteria.

\end{fulllineitems}


\index{findAll() (SamPy.parsing.BeautifulSoup.Tag method)}

\begin{fulllineitems}
\phantomsection\label{SamPy.parsing:SamPy.parsing.BeautifulSoup.Tag.findAll}\pysiglinewithargsret{\bfcode{findAll}}{\emph{name=None}, \emph{attrs=\{\}}, \emph{recursive=True}, \emph{text=None}, \emph{limit=None}, \emph{**kwargs}}{}
Extracts a list of Tag objects that match the given
criteria.  You can specify the name of the Tag and any
attributes you want the Tag to have.

The value of a key-value pair in the `attrs' map can be a
string, a list of strings, a regular expression object, or a
callable that takes a string and returns whether or not the
string matches for some custom definition of `matches'. The
same is true of the tag name.

\end{fulllineitems}


\index{findChild() (SamPy.parsing.BeautifulSoup.Tag method)}

\begin{fulllineitems}
\phantomsection\label{SamPy.parsing:SamPy.parsing.BeautifulSoup.Tag.findChild}\pysiglinewithargsret{\bfcode{findChild}}{\emph{name=None}, \emph{attrs=\{\}}, \emph{recursive=True}, \emph{text=None}, \emph{**kwargs}}{}
Return only the first child of this Tag matching the given
criteria.

\end{fulllineitems}


\index{findChildren() (SamPy.parsing.BeautifulSoup.Tag method)}

\begin{fulllineitems}
\phantomsection\label{SamPy.parsing:SamPy.parsing.BeautifulSoup.Tag.findChildren}\pysiglinewithargsret{\bfcode{findChildren}}{\emph{name=None}, \emph{attrs=\{\}}, \emph{recursive=True}, \emph{text=None}, \emph{limit=None}, \emph{**kwargs}}{}
Extracts a list of Tag objects that match the given
criteria.  You can specify the name of the Tag and any
attributes you want the Tag to have.

The value of a key-value pair in the `attrs' map can be a
string, a list of strings, a regular expression object, or a
callable that takes a string and returns whether or not the
string matches for some custom definition of `matches'. The
same is true of the tag name.

\end{fulllineitems}


\index{first() (SamPy.parsing.BeautifulSoup.Tag method)}

\begin{fulllineitems}
\phantomsection\label{SamPy.parsing:SamPy.parsing.BeautifulSoup.Tag.first}\pysiglinewithargsret{\bfcode{first}}{\emph{name=None}, \emph{attrs=\{\}}, \emph{recursive=True}, \emph{text=None}, \emph{**kwargs}}{}
Return only the first child of this Tag matching the given
criteria.

\end{fulllineitems}


\index{firstText() (SamPy.parsing.BeautifulSoup.Tag method)}

\begin{fulllineitems}
\phantomsection\label{SamPy.parsing:SamPy.parsing.BeautifulSoup.Tag.firstText}\pysiglinewithargsret{\bfcode{firstText}}{\emph{text=None}, \emph{recursive=True}}{}
\end{fulllineitems}


\index{get() (SamPy.parsing.BeautifulSoup.Tag method)}

\begin{fulllineitems}
\phantomsection\label{SamPy.parsing:SamPy.parsing.BeautifulSoup.Tag.get}\pysiglinewithargsret{\bfcode{get}}{\emph{key}, \emph{default=None}}{}
Returns the value of the `key' attribute for the tag, or
the value given for `default' if it doesn't have that
attribute.

\end{fulllineitems}


\index{has\_key() (SamPy.parsing.BeautifulSoup.Tag method)}

\begin{fulllineitems}
\phantomsection\label{SamPy.parsing:SamPy.parsing.BeautifulSoup.Tag.has_key}\pysiglinewithargsret{\bfcode{has\_key}}{\emph{key}}{}
\end{fulllineitems}


\index{prettify() (SamPy.parsing.BeautifulSoup.Tag method)}

\begin{fulllineitems}
\phantomsection\label{SamPy.parsing:SamPy.parsing.BeautifulSoup.Tag.prettify}\pysiglinewithargsret{\bfcode{prettify}}{\emph{encoding='utf-8'}}{}
\end{fulllineitems}


\index{recursiveChildGenerator() (SamPy.parsing.BeautifulSoup.Tag method)}

\begin{fulllineitems}
\phantomsection\label{SamPy.parsing:SamPy.parsing.BeautifulSoup.Tag.recursiveChildGenerator}\pysiglinewithargsret{\bfcode{recursiveChildGenerator}}{}{}
\end{fulllineitems}


\index{renderContents() (SamPy.parsing.BeautifulSoup.Tag method)}

\begin{fulllineitems}
\phantomsection\label{SamPy.parsing:SamPy.parsing.BeautifulSoup.Tag.renderContents}\pysiglinewithargsret{\bfcode{renderContents}}{\emph{encoding='utf-8'}, \emph{prettyPrint=False}, \emph{indentLevel=0}}{}
\end{fulllineitems}


\end{fulllineitems}


\index{UnicodeDammit (class in SamPy.parsing.BeautifulSoup)}

\begin{fulllineitems}
\phantomsection\label{SamPy.parsing:SamPy.parsing.BeautifulSoup.UnicodeDammit}\pysiglinewithargsret{\strong{class }\code{SamPy.parsing.BeautifulSoup.}\bfcode{UnicodeDammit}}{\emph{markup}, \emph{overrideEncodings=}\optional{}, \emph{smartQuotesTo='xml'}, \emph{isHTML=False}}{}
A class for detecting the encoding of a {\color{red}\bfseries{}*}ML document and
converting it to a Unicode string. If the source encoding is
windows-1252, can replace MS smart quotes with their HTML or XML
equivalents.

\index{find\_codec() (SamPy.parsing.BeautifulSoup.UnicodeDammit method)}

\begin{fulllineitems}
\phantomsection\label{SamPy.parsing:SamPy.parsing.BeautifulSoup.UnicodeDammit.find_codec}\pysiglinewithargsret{\bfcode{find\_codec}}{\emph{charset}}{}
\end{fulllineitems}


\end{fulllineitems}


\index{buildTagMap() (in module SamPy.parsing.BeautifulSoup)}

\begin{fulllineitems}
\phantomsection\label{SamPy.parsing:SamPy.parsing.BeautifulSoup.buildTagMap}\pysiglinewithargsret{\code{SamPy.parsing.BeautifulSoup.}\bfcode{buildTagMap}}{\emph{default}, \emph{*args}}{}
Turns a list of maps, lists, or scalars into a single map.
Used to build the SELF\_CLOSING\_TAGS, NESTABLE\_TAGS, and
NESTING\_RESET\_TAGS maps out of lists and partial maps.

\end{fulllineitems}


\index{isList() (in module SamPy.parsing.BeautifulSoup)}

\begin{fulllineitems}
\phantomsection\label{SamPy.parsing:SamPy.parsing.BeautifulSoup.isList}\pysiglinewithargsret{\code{SamPy.parsing.BeautifulSoup.}\bfcode{isList}}{\emph{l}}{}
Convenience method that works with all 2.x versions of Python
to determine whether or not something is listlike.

\end{fulllineitems}


\index{isString() (in module SamPy.parsing.BeautifulSoup)}

\begin{fulllineitems}
\phantomsection\label{SamPy.parsing:SamPy.parsing.BeautifulSoup.isString}\pysiglinewithargsret{\code{SamPy.parsing.BeautifulSoup.}\bfcode{isString}}{\emph{s}}{}
Convenience method that works with all 2.x versions of Python
to determine whether or not something is stringlike.

\end{fulllineitems}


\index{sob() (in module SamPy.parsing.BeautifulSoup)}

\begin{fulllineitems}
\phantomsection\label{SamPy.parsing:SamPy.parsing.BeautifulSoup.sob}\pysiglinewithargsret{\code{SamPy.parsing.BeautifulSoup.}\bfcode{sob}}{\emph{unicode}, \emph{encoding}}{}
Returns either the given Unicode string or its encoding.

\end{fulllineitems}



\subsection{pca Package}
\label{SamPy.pca:pca-package}\label{SamPy.pca::doc}

\subsubsection{\texttt{SMNpca} Module}
\label{SamPy.pca:module-SamPy.pca.SMNpca}\label{SamPy.pca:smnpca-module}
\index{SamPy.pca.SMNpca (module)}
\index{SMNpca (class in SamPy.pca.SMNpca)}

\begin{fulllineitems}
\phantomsection\label{SamPy.pca:SamPy.pca.SMNpca.SMNpca}\pysiglinewithargsret{\strong{class }\code{SamPy.pca.SMNpca.}\bfcode{SMNpca}}{\emph{inputMatrix}, \emph{stdnorm=False}}{}
This class calculates the principle component analysis (PCA).

First PC is direction of the maximum variance from origin. Subsequent PCs are orthogonal
to 1st PC and describe maximum residual variance. Eigenvector cells are coefficients of 
the corresponding principal component.

Data input must be a matrix where rows are dimensions/variables and columns
are measurements/observations.

Optional argument stdnorm refers to standard deviation normalisation.
A PCA without dividing by the standard deviation is an eigenanalysis of the covariance matrix, 
and a PCA in which you divide by the standard deviation is an eigenanalysis of the correlation matrix.

Output is...

\index{Cov() (SamPy.pca.SMNpca.SMNpca method)}

\begin{fulllineitems}
\phantomsection\label{SamPy.pca:SamPy.pca.SMNpca.SMNpca.Cov}\pysiglinewithargsret{\bfcode{Cov}}{}{}
\end{fulllineitems}


\index{DisplayAxisVariation() (SamPy.pca.SMNpca.SMNpca method)}

\begin{fulllineitems}
\phantomsection\label{SamPy.pca:SamPy.pca.SMNpca.SMNpca.DisplayAxisVariation}\pysiglinewithargsret{\bfcode{DisplayAxisVariation}}{}{}
\end{fulllineitems}


\index{DisplayCov() (SamPy.pca.SMNpca.SMNpca method)}

\begin{fulllineitems}
\phantomsection\label{SamPy.pca:SamPy.pca.SMNpca.SMNpca.DisplayCov}\pysiglinewithargsret{\bfcode{DisplayCov}}{}{}
\end{fulllineitems}


\index{DisplayEigenvalues() (SamPy.pca.SMNpca.SMNpca method)}

\begin{fulllineitems}
\phantomsection\label{SamPy.pca:SamPy.pca.SMNpca.SMNpca.DisplayEigenvalues}\pysiglinewithargsret{\bfcode{DisplayEigenvalues}}{}{}
\end{fulllineitems}


\index{DisplayEigenvectors() (SamPy.pca.SMNpca.SMNpca method)}

\begin{fulllineitems}
\phantomsection\label{SamPy.pca:SamPy.pca.SMNpca.SMNpca.DisplayEigenvectors}\pysiglinewithargsret{\bfcode{DisplayEigenvectors}}{}{}
\end{fulllineitems}


\index{DisplayFeatureMatrix() (SamPy.pca.SMNpca.SMNpca method)}

\begin{fulllineitems}
\phantomsection\label{SamPy.pca:SamPy.pca.SMNpca.SMNpca.DisplayFeatureMatrix}\pysiglinewithargsret{\bfcode{DisplayFeatureMatrix}}{}{}
\end{fulllineitems}


\index{DisplayMajorComponentData() (SamPy.pca.SMNpca.SMNpca method)}

\begin{fulllineitems}
\phantomsection\label{SamPy.pca:SamPy.pca.SMNpca.SMNpca.DisplayMajorComponentData}\pysiglinewithargsret{\bfcode{DisplayMajorComponentData}}{}{}
\end{fulllineitems}


\index{DisplayNewData() (SamPy.pca.SMNpca.SMNpca method)}

\begin{fulllineitems}
\phantomsection\label{SamPy.pca:SamPy.pca.SMNpca.SMNpca.DisplayNewData}\pysiglinewithargsret{\bfcode{DisplayNewData}}{}{}
\end{fulllineitems}


\index{Eigenvalues() (SamPy.pca.SMNpca.SMNpca method)}

\begin{fulllineitems}
\phantomsection\label{SamPy.pca:SamPy.pca.SMNpca.SMNpca.Eigenvalues}\pysiglinewithargsret{\bfcode{Eigenvalues}}{}{}
\end{fulllineitems}


\index{Eigenvectors() (SamPy.pca.SMNpca.SMNpca method)}

\begin{fulllineitems}
\phantomsection\label{SamPy.pca:SamPy.pca.SMNpca.SMNpca.Eigenvectors}\pysiglinewithargsret{\bfcode{Eigenvectors}}{}{}
\end{fulllineitems}


\index{FeatureMatrix() (SamPy.pca.SMNpca.SMNpca method)}

\begin{fulllineitems}
\phantomsection\label{SamPy.pca:SamPy.pca.SMNpca.SMNpca.FeatureMatrix}\pysiglinewithargsret{\bfcode{FeatureMatrix}}{}{}
\end{fulllineitems}


\index{MajorComponentData() (SamPy.pca.SMNpca.SMNpca method)}

\begin{fulllineitems}
\phantomsection\label{SamPy.pca:SamPy.pca.SMNpca.SMNpca.MajorComponentData}\pysiglinewithargsret{\bfcode{MajorComponentData}}{}{}
\end{fulllineitems}


\index{VisualisePCAtoFileIn2D() (SamPy.pca.SMNpca.SMNpca method)}

\begin{fulllineitems}
\phantomsection\label{SamPy.pca:SamPy.pca.SMNpca.SMNpca.VisualisePCAtoFileIn2D}\pysiglinewithargsret{\bfcode{VisualisePCAtoFileIn2D}}{\emph{data}, \emph{filename}}{}
\end{fulllineitems}


\index{VisualisePCAtoFileIn3D() (SamPy.pca.SMNpca.SMNpca method)}

\begin{fulllineitems}
\phantomsection\label{SamPy.pca:SamPy.pca.SMNpca.SMNpca.VisualisePCAtoFileIn3D}\pysiglinewithargsret{\bfcode{VisualisePCAtoFileIn3D}}{\emph{data}, \emph{filename}}{}
\end{fulllineitems}


\index{doPCA() (SamPy.pca.SMNpca.SMNpca method)}

\begin{fulllineitems}
\phantomsection\label{SamPy.pca:SamPy.pca.SMNpca.SMNpca.doPCA}\pysiglinewithargsret{\bfcode{doPCA}}{}{}
\end{fulllineitems}


\index{evalues() (SamPy.pca.SMNpca.SMNpca method)}

\begin{fulllineitems}
\phantomsection\label{SamPy.pca:SamPy.pca.SMNpca.SMNpca.evalues}\pysiglinewithargsret{\bfcode{evalues}}{}{}
\end{fulllineitems}


\index{evectors() (SamPy.pca.SMNpca.SMNpca method)}

\begin{fulllineitems}
\phantomsection\label{SamPy.pca:SamPy.pca.SMNpca.SMNpca.evectors}\pysiglinewithargsret{\bfcode{evectors}}{}{}
\end{fulllineitems}


\index{featureMatrix() (SamPy.pca.SMNpca.SMNpca method)}

\begin{fulllineitems}
\phantomsection\label{SamPy.pca:SamPy.pca.SMNpca.SMNpca.featureMatrix}\pysiglinewithargsret{\bfcode{featureMatrix}}{}{}
\end{fulllineitems}


\index{principleComponent() (SamPy.pca.SMNpca.SMNpca method)}

\begin{fulllineitems}
\phantomsection\label{SamPy.pca:SamPy.pca.SMNpca.SMNpca.principleComponent}\pysiglinewithargsret{\bfcode{principleComponent}}{}{}
\end{fulllineitems}


\end{fulllineitems}



\subsection{plot Package}
\label{SamPy.plot::doc}\label{SamPy.plot:plot-package}

\subsubsection{\texttt{basic} Module}
\label{SamPy.plot:module-SamPy.plot.basic}\label{SamPy.plot:basic-module}
\index{SamPy.plot.basic (module)}

\paragraph{Basic Plotting Routines}
\label{SamPy.plot:basic-plotting-routines}
\index{scatterHistograms() (in module SamPy.plot.basic)}

\begin{fulllineitems}
\phantomsection\label{SamPy.plot:SamPy.plot.basic.scatterHistograms}\pysiglinewithargsret{\code{SamPy.plot.basic.}\bfcode{scatterHistograms}}{\emph{xdata}, \emph{ydata}, \emph{xlabel}, \emph{ylabel}, \emph{binwidth}, \emph{output}}{}
This functions generates a scatter plot and
projected histograms to both axes.

\end{fulllineitems}



\subsubsection{\texttt{colorbarExample} Module}
\label{SamPy.plot:colorbarexample-module}\label{SamPy.plot:module-SamPy.plot.colorbarExample}
\index{SamPy.plot.colorbarExample (module)}
\index{make\_axes() (in module SamPy.plot.colorbarExample)}

\begin{fulllineitems}
\phantomsection\label{SamPy.plot:SamPy.plot.colorbarExample.make_axes}\pysiglinewithargsret{\code{SamPy.plot.colorbarExample.}\bfcode{make\_axes}}{\emph{parent}, \emph{**kw}}{}
\end{fulllineitems}



\subsubsection{\texttt{colorbarExample2} Module}
\label{SamPy.plot:module-SamPy.plot.colorbarExample2}\label{SamPy.plot:colorbarexample2-module}
\index{SamPy.plot.colorbarExample2 (module)}

\subsubsection{\texttt{interactive\_correlation\_plot} Module}
\label{SamPy.plot:interactive-correlation-plot-module}\label{SamPy.plot:module-SamPy.plot.interactive_correlation_plot}
\index{SamPy.plot.interactive\_correlation\_plot (module)}
\index{check\_if\_click\_is\_on\_an\_existing\_point() (in module SamPy.plot.interactive\_correlation\_plot)}

\begin{fulllineitems}
\phantomsection\label{SamPy.plot:SamPy.plot.interactive_correlation_plot.check_if_click_is_on_an_existing_point}\pysiglinewithargsret{\code{SamPy.plot.interactive\_correlation\_plot.}\bfcode{check\_if\_click\_is\_on\_an\_existing\_point}}{\emph{mouse\_x\_coord}, \emph{mouse\_y\_coord}}{}
\end{fulllineitems}


\index{clear\_the\_figure\_and\_empty\_points\_list() (in module SamPy.plot.interactive\_correlation\_plot)}

\begin{fulllineitems}
\phantomsection\label{SamPy.plot:SamPy.plot.interactive_correlation_plot.clear_the_figure_and_empty_points_list}\pysiglinewithargsret{\code{SamPy.plot.interactive\_correlation\_plot.}\bfcode{clear\_the\_figure\_and\_empty\_points\_list}}{}{}
\end{fulllineitems}


\index{do\_this\_when\_the\_mouse\_is\_clicked() (in module SamPy.plot.interactive\_correlation\_plot)}

\begin{fulllineitems}
\phantomsection\label{SamPy.plot:SamPy.plot.interactive_correlation_plot.do_this_when_the_mouse_is_clicked}\pysiglinewithargsret{\code{SamPy.plot.interactive\_correlation\_plot.}\bfcode{do\_this\_when\_the\_mouse\_is\_clicked}}{\emph{this\_event}}{}
\end{fulllineitems}


\index{plot\_the\_correlation() (in module SamPy.plot.interactive\_correlation\_plot)}

\begin{fulllineitems}
\phantomsection\label{SamPy.plot:SamPy.plot.interactive_correlation_plot.plot_the_correlation}\pysiglinewithargsret{\code{SamPy.plot.interactive\_correlation\_plot.}\bfcode{plot\_the\_correlation}}{}{}
\end{fulllineitems}



\subsubsection{\texttt{interactive\_mean\_std\_normal\_distribution} Module}
\label{SamPy.plot:interactive-mean-std-normal-distribution-module}\label{SamPy.plot:module-SamPy.plot.interactive_mean_std_normal_distribution}
\index{SamPy.plot.interactive\_mean\_std\_normal\_distribution (module)}
\index{check\_if\_click\_is\_on\_an\_existing\_point() (in module SamPy.plot.interactive\_mean\_std\_normal\_distribution)}

\begin{fulllineitems}
\phantomsection\label{SamPy.plot:SamPy.plot.interactive_mean_std_normal_distribution.check_if_click_is_on_an_existing_point}\pysiglinewithargsret{\code{SamPy.plot.interactive\_mean\_std\_normal\_distribution.}\bfcode{check\_if\_click\_is\_on\_an\_existing\_point}}{\emph{mouse\_x\_coord}, \emph{mouse\_y\_coord}}{}
\end{fulllineitems}


\index{clear\_the\_figure\_and\_empty\_points\_list() (in module SamPy.plot.interactive\_mean\_std\_normal\_distribution)}

\begin{fulllineitems}
\phantomsection\label{SamPy.plot:SamPy.plot.interactive_mean_std_normal_distribution.clear_the_figure_and_empty_points_list}\pysiglinewithargsret{\code{SamPy.plot.interactive\_mean\_std\_normal\_distribution.}\bfcode{clear\_the\_figure\_and\_empty\_points\_list}}{}{}
\end{fulllineitems}


\index{do\_this\_when\_the\_mouse\_is\_clicked() (in module SamPy.plot.interactive\_mean\_std\_normal\_distribution)}

\begin{fulllineitems}
\phantomsection\label{SamPy.plot:SamPy.plot.interactive_mean_std_normal_distribution.do_this_when_the_mouse_is_clicked}\pysiglinewithargsret{\code{SamPy.plot.interactive\_mean\_std\_normal\_distribution.}\bfcode{do\_this\_when\_the\_mouse\_is\_clicked}}{\emph{this\_event}}{}
\end{fulllineitems}


\index{plot\_the\_mean\_std\_and\_normal() (in module SamPy.plot.interactive\_mean\_std\_normal\_distribution)}

\begin{fulllineitems}
\phantomsection\label{SamPy.plot:SamPy.plot.interactive_mean_std_normal_distribution.plot_the_mean_std_and_normal}\pysiglinewithargsret{\code{SamPy.plot.interactive\_mean\_std\_normal\_distribution.}\bfcode{plot\_the\_mean\_std\_and\_normal}}{}{}
\end{fulllineitems}



\subsubsection{\texttt{interactive\_two\_sample\_t\_test} Module}
\label{SamPy.plot:interactive-two-sample-t-test-module}\label{SamPy.plot:module-SamPy.plot.interactive_two_sample_t_test}
\index{SamPy.plot.interactive\_two\_sample\_t\_test (module)}
\index{check\_if\_click\_is\_on\_an\_existing\_point() (in module SamPy.plot.interactive\_two\_sample\_t\_test)}

\begin{fulllineitems}
\phantomsection\label{SamPy.plot:SamPy.plot.interactive_two_sample_t_test.check_if_click_is_on_an_existing_point}\pysiglinewithargsret{\code{SamPy.plot.interactive\_two\_sample\_t\_test.}\bfcode{check\_if\_click\_is\_on\_an\_existing\_point}}{\emph{mouse\_x\_coord}, \emph{mouse\_y\_coord}}{}
\end{fulllineitems}


\index{clear\_the\_figure\_and\_empty\_points\_list() (in module SamPy.plot.interactive\_two\_sample\_t\_test)}

\begin{fulllineitems}
\phantomsection\label{SamPy.plot:SamPy.plot.interactive_two_sample_t_test.clear_the_figure_and_empty_points_list}\pysiglinewithargsret{\code{SamPy.plot.interactive\_two\_sample\_t\_test.}\bfcode{clear\_the\_figure\_and\_empty\_points\_list}}{}{}
\end{fulllineitems}


\index{do\_this\_when\_the\_mouse\_is\_clicked() (in module SamPy.plot.interactive\_two\_sample\_t\_test)}

\begin{fulllineitems}
\phantomsection\label{SamPy.plot:SamPy.plot.interactive_two_sample_t_test.do_this_when_the_mouse_is_clicked}\pysiglinewithargsret{\code{SamPy.plot.interactive\_two\_sample\_t\_test.}\bfcode{do\_this\_when\_the\_mouse\_is\_clicked}}{\emph{this\_event}}{}
\end{fulllineitems}


\index{plot\_the\_two\_sample\_t\_test() (in module SamPy.plot.interactive\_two\_sample\_t\_test)}

\begin{fulllineitems}
\phantomsection\label{SamPy.plot:SamPy.plot.interactive_two_sample_t_test.plot_the_two_sample_t_test}\pysiglinewithargsret{\code{SamPy.plot.interactive\_two\_sample\_t\_test.}\bfcode{plot\_the\_two\_sample\_t\_test}}{}{}
\end{fulllineitems}



\subsubsection{\texttt{tools} Module}
\label{SamPy.plot:module-SamPy.plot.tools}\label{SamPy.plot:tools-module}
\index{SamPy.plot.tools (module)}
A collection of functions that may or may not be useful
when plotting data using matplotlib.

@author: Sami-Matias Niemi
@version: 0.1

\index{give\_colours() (in module SamPy.plot.tools)}

\begin{fulllineitems}
\phantomsection\label{SamPy.plot:SamPy.plot.tools.give_colours}\pysiglinewithargsret{\code{SamPy.plot.tools.}\bfcode{give\_colours}}{}{}
Returns a tuple of colours that can be used
with matplotlib when several different colours
are needed.

\end{fulllineitems}



\subsubsection{\texttt{trilogy} Module}
\label{SamPy.plot:module-SamPy.plot.trilogy}\label{SamPy.plot:trilogy-module}
\index{SamPy.plot.trilogy (module)}
\index{RGB2im() (in module SamPy.plot.trilogy)}

\begin{fulllineitems}
\phantomsection\label{SamPy.plot:SamPy.plot.trilogy.RGB2im}\pysiglinewithargsret{\code{SamPy.plot.trilogy.}\bfcode{RGB2im}}{\emph{RGB}}{}
r, g, b = data  (3, ny, nx)
Converts to an Image

\end{fulllineitems}


\index{RGBscale2im() (in module SamPy.plot.trilogy)}

\begin{fulllineitems}
\phantomsection\label{SamPy.plot:SamPy.plot.trilogy.RGBscale2im}\pysiglinewithargsret{\code{SamPy.plot.trilogy.}\bfcode{RGBscale2im}}{\emph{RGB}, \emph{levdict}, \emph{noiselums}, \emph{colorsatfac}, \emph{mode='RGB'}}{}
\end{fulllineitems}


\index{Trilogy (class in SamPy.plot.trilogy)}

\begin{fulllineitems}
\phantomsection\label{SamPy.plot:SamPy.plot.trilogy.Trilogy}\pysiglinewithargsret{\strong{class }\code{SamPy.plot.trilogy.}\bfcode{Trilogy}}{\emph{infile=None}, \emph{images=None}, \emph{imagesorder='BGR'}, \emph{**inparams}}{}~
\index{determinescalings() (SamPy.plot.trilogy.Trilogy method)}

\begin{fulllineitems}
\phantomsection\label{SamPy.plot:SamPy.plot.trilogy.Trilogy.determinescalings}\pysiglinewithargsret{\bfcode{determinescalings}}{}{}
Determine data scalings
will sample a (samplesize x samplesize) region of the (centered) core
make color image of the core as a test if desired

\end{fulllineitems}


\index{loadimagesize() (SamPy.plot.trilogy.Trilogy method)}

\begin{fulllineitems}
\phantomsection\label{SamPy.plot:SamPy.plot.trilogy.Trilogy.loadimagesize}\pysiglinewithargsret{\bfcode{loadimagesize}}{}{}
\end{fulllineitems}


\index{loadinputs() (SamPy.plot.trilogy.Trilogy method)}

\begin{fulllineitems}
\phantomsection\label{SamPy.plot:SamPy.plot.trilogy.Trilogy.loadinputs}\pysiglinewithargsret{\bfcode{loadinputs}}{}{}
Load R,G,B filenames and options

\end{fulllineitems}


\index{loadstamps() (SamPy.plot.trilogy.Trilogy method)}

\begin{fulllineitems}
\phantomsection\label{SamPy.plot:SamPy.plot.trilogy.Trilogy.loadstamps}\pysiglinewithargsret{\bfcode{loadstamps}}{\emph{limits}, \emph{silent=1}}{}
\end{fulllineitems}


\index{makecolorimage() (SamPy.plot.trilogy.Trilogy method)}

\begin{fulllineitems}
\phantomsection\label{SamPy.plot:SamPy.plot.trilogy.Trilogy.makecolorimage}\pysiglinewithargsret{\bfcode{makecolorimage}}{}{}
Make color image (in sections)

\end{fulllineitems}


\index{makethumbnail() (SamPy.plot.trilogy.Trilogy method)}

\begin{fulllineitems}
\phantomsection\label{SamPy.plot:SamPy.plot.trilogy.Trilogy.makethumbnail}\pysiglinewithargsret{\bfcode{makethumbnail}}{}{}
\end{fulllineitems}


\index{makethumbnail1() (SamPy.plot.trilogy.Trilogy method)}

\begin{fulllineitems}
\phantomsection\label{SamPy.plot:SamPy.plot.trilogy.Trilogy.makethumbnail1}\pysiglinewithargsret{\bfcode{makethumbnail1}}{\emph{outroot}, \emph{width}, \emph{fmt='jpg'}}{}
\end{fulllineitems}


\index{run() (SamPy.plot.trilogy.Trilogy method)}

\begin{fulllineitems}
\phantomsection\label{SamPy.plot:SamPy.plot.trilogy.Trilogy.run}\pysiglinewithargsret{\bfcode{run}}{}{}
\end{fulllineitems}


\index{setdefaults() (SamPy.plot.trilogy.Trilogy method)}

\begin{fulllineitems}
\phantomsection\label{SamPy.plot:SamPy.plot.trilogy.Trilogy.setdefaults}\pysiglinewithargsret{\bfcode{setdefaults}}{}{}
\end{fulllineitems}


\index{setimages() (SamPy.plot.trilogy.Trilogy method)}

\begin{fulllineitems}
\phantomsection\label{SamPy.plot:SamPy.plot.trilogy.Trilogy.setimages}\pysiglinewithargsret{\bfcode{setimages}}{\emph{images=None}}{}
\end{fulllineitems}


\index{setinparams() (SamPy.plot.trilogy.Trilogy method)}

\begin{fulllineitems}
\phantomsection\label{SamPy.plot:SamPy.plot.trilogy.Trilogy.setinparams}\pysiglinewithargsret{\bfcode{setinparams}}{}{}
\end{fulllineitems}


\index{setnoiselums() (SamPy.plot.trilogy.Trilogy method)}

\begin{fulllineitems}
\phantomsection\label{SamPy.plot:SamPy.plot.trilogy.Trilogy.setnoiselums}\pysiglinewithargsret{\bfcode{setnoiselums}}{}{}
\end{fulllineitems}


\index{setoutfile() (SamPy.plot.trilogy.Trilogy method)}

\begin{fulllineitems}
\phantomsection\label{SamPy.plot:SamPy.plot.trilogy.Trilogy.setoutfile}\pysiglinewithargsret{\bfcode{setoutfile}}{\emph{outname=None}}{}
\end{fulllineitems}


\index{showsample() (SamPy.plot.trilogy.Trilogy method)}

\begin{fulllineitems}
\phantomsection\label{SamPy.plot:SamPy.plot.trilogy.Trilogy.showsample}\pysiglinewithargsret{\bfcode{showsample}}{\emph{outfile}}{}
\end{fulllineitems}


\end{fulllineitems}


\index{adjsat() (in module SamPy.plot.trilogy)}

\begin{fulllineitems}
\phantomsection\label{SamPy.plot:SamPy.plot.trilogy.adjsat}\pysiglinewithargsret{\code{SamPy.plot.trilogy.}\bfcode{adjsat}}{\emph{RGB}, \emph{K}}{}
Adjust the color saturation of an image.  K \textgreater{} 1 boosts it.

\end{fulllineitems}


\index{clip2() (in module SamPy.plot.trilogy)}

\begin{fulllineitems}
\phantomsection\label{SamPy.plot:SamPy.plot.trilogy.clip2}\pysiglinewithargsret{\code{SamPy.plot.trilogy.}\bfcode{clip2}}{\emph{m}, \emph{m\_min=None}, \emph{m\_max=None}}{}
\end{fulllineitems}


\index{da() (in module SamPy.plot.trilogy)}

\begin{fulllineitems}
\phantomsection\label{SamPy.plot:SamPy.plot.trilogy.da}\pysiglinewithargsret{\code{SamPy.plot.trilogy.}\bfcode{da}}{\emph{k}}{}
\end{fulllineitems}


\index{determinescaling() (in module SamPy.plot.trilogy)}

\begin{fulllineitems}
\phantomsection\label{SamPy.plot:SamPy.plot.trilogy.determinescaling}\pysiglinewithargsret{\code{SamPy.plot.trilogy.}\bfcode{determinescaling}}{\emph{data}, \emph{unsatpercent}}{}
Determines data values (x0,x1,x2) which will be scaled to (0,noiselum,1)

\end{fulllineitems}


\index{grayimage() (in module SamPy.plot.trilogy)}

\begin{fulllineitems}
\phantomsection\label{SamPy.plot:SamPy.plot.trilogy.grayimage}\pysiglinewithargsret{\code{SamPy.plot.trilogy.}\bfcode{grayimage}}{\emph{scaled}}{}
\end{fulllineitems}


\index{grayscaledimage() (in module SamPy.plot.trilogy)}

\begin{fulllineitems}
\phantomsection\label{SamPy.plot:SamPy.plot.trilogy.grayscaledimage}\pysiglinewithargsret{\code{SamPy.plot.trilogy.}\bfcode{grayscaledimage}}{\emph{stamp}, \emph{levels}, \emph{noiselum}}{}
\end{fulllineitems}


\index{imscale1() (in module SamPy.plot.trilogy)}

\begin{fulllineitems}
\phantomsection\label{SamPy.plot:SamPy.plot.trilogy.imscale1}\pysiglinewithargsret{\code{SamPy.plot.trilogy.}\bfcode{imscale1}}{\emph{data}, \emph{levels}}{}
\end{fulllineitems}


\index{imscale2() (in module SamPy.plot.trilogy)}

\begin{fulllineitems}
\phantomsection\label{SamPy.plot:SamPy.plot.trilogy.imscale2}\pysiglinewithargsret{\code{SamPy.plot.trilogy.}\bfcode{imscale2}}{\emph{data}, \emph{levels}, \emph{y1}}{}
\end{fulllineitems}


\index{loaddict() (in module SamPy.plot.trilogy)}

\begin{fulllineitems}
\phantomsection\label{SamPy.plot:SamPy.plot.trilogy.loaddict}\pysiglinewithargsret{\code{SamPy.plot.trilogy.}\bfcode{loaddict}}{\emph{filename}, \emph{dir='`}, \emph{silent=0}}{}
\end{fulllineitems}


\index{loadfile() (in module SamPy.plot.trilogy)}

\begin{fulllineitems}
\phantomsection\label{SamPy.plot:SamPy.plot.trilogy.loadfile}\pysiglinewithargsret{\code{SamPy.plot.trilogy.}\bfcode{loadfile}}{\emph{filename}, \emph{dir='`}, \emph{silent=0}, \emph{keepnewlines=0}}{}
\end{fulllineitems}


\index{loadfitsimagedata() (in module SamPy.plot.trilogy)}

\begin{fulllineitems}
\phantomsection\label{SamPy.plot:SamPy.plot.trilogy.loadfitsimagedata}\pysiglinewithargsret{\code{SamPy.plot.trilogy.}\bfcode{loadfitsimagedata}}{\emph{image}, \emph{indir='`}, \emph{silent=1}}{}
\end{fulllineitems}


\index{meanstd\_robust (class in SamPy.plot.trilogy)}

\begin{fulllineitems}
\phantomsection\label{SamPy.plot:SamPy.plot.trilogy.meanstd_robust}\pysiglinewithargsret{\strong{class }\code{SamPy.plot.trilogy.}\bfcode{meanstd\_robust}}{\emph{x}, \emph{n\_sigma=3}, \emph{n=5}, \emph{sortedalready=False}}{}~
\index{run() (SamPy.plot.trilogy.meanstd\_robust method)}

\begin{fulllineitems}
\phantomsection\label{SamPy.plot:SamPy.plot.trilogy.meanstd_robust.run}\pysiglinewithargsret{\bfcode{run}}{}{}
\end{fulllineitems}


\end{fulllineitems}


\index{offsetarray() (in module SamPy.plot.trilogy)}

\begin{fulllineitems}
\phantomsection\label{SamPy.plot:SamPy.plot.trilogy.offsetarray}\pysiglinewithargsret{\code{SamPy.plot.trilogy.}\bfcode{offsetarray}}{\emph{data}, \emph{offset}}{}
\end{fulllineitems}


\index{params\_cl() (in module SamPy.plot.trilogy)}

\begin{fulllineitems}
\phantomsection\label{SamPy.plot:SamPy.plot.trilogy.params_cl}\pysiglinewithargsret{\code{SamPy.plot.trilogy.}\bfcode{params\_cl}}{\emph{converttonumbers=True}}{}
RETURNS PARAMETERS FROM COMMAND LINE (`cl') AS DICTIONARY:
KEYS ARE OPTIONS BEGINNING WITH `-`
VALUES ARE WHATEVER FOLLOWS KEYS: EITHER NOTHING (`'), A VALUE, OR A LIST OF VALUES
ALL VALUES ARE CONVERTED TO INT / FLOAT WHEN APPROPRIATE

\end{fulllineitems}


\index{processimagename() (in module SamPy.plot.trilogy)}

\begin{fulllineitems}
\phantomsection\label{SamPy.plot:SamPy.plot.trilogy.processimagename}\pysiglinewithargsret{\code{SamPy.plot.trilogy.}\bfcode{processimagename}}{\emph{image}}{}
\end{fulllineitems}


\index{rms() (in module SamPy.plot.trilogy)}

\begin{fulllineitems}
\phantomsection\label{SamPy.plot:SamPy.plot.trilogy.rms}\pysiglinewithargsret{\code{SamPy.plot.trilogy.}\bfcode{rms}}{\emph{x}}{}
\end{fulllineitems}


\index{satK2m() (in module SamPy.plot.trilogy)}

\begin{fulllineitems}
\phantomsection\label{SamPy.plot:SamPy.plot.trilogy.satK2m}\pysiglinewithargsret{\code{SamPy.plot.trilogy.}\bfcode{satK2m}}{\emph{K}}{}
\end{fulllineitems}


\index{savelevels() (in module SamPy.plot.trilogy)}

\begin{fulllineitems}
\phantomsection\label{SamPy.plot:SamPy.plot.trilogy.savelevels}\pysiglinewithargsret{\code{SamPy.plot.trilogy.}\bfcode{savelevels}}{\emph{levdict}, \emph{outfile='levels.txt'}, \emph{outdir='`}}{}
\end{fulllineitems}


\index{setlevels() (in module SamPy.plot.trilogy)}

\begin{fulllineitems}
\phantomsection\label{SamPy.plot:SamPy.plot.trilogy.setlevels}\pysiglinewithargsret{\code{SamPy.plot.trilogy.}\bfcode{setlevels}}{\emph{data}, \emph{pp}, \emph{stripneg=False}, \emph{sortedalready=False}}{}
\end{fulllineitems}


\index{stat\_robust (class in SamPy.plot.trilogy)}

\begin{fulllineitems}
\phantomsection\label{SamPy.plot:SamPy.plot.trilogy.stat_robust}\pysiglinewithargsret{\strong{class }\code{SamPy.plot.trilogy.}\bfcode{stat\_robust}}{\emph{x}, \emph{n\_sigma=3}, \emph{n=5}, \emph{reject\_fraction=None}}{}~
\index{run() (SamPy.plot.trilogy.stat\_robust method)}

\begin{fulllineitems}
\phantomsection\label{SamPy.plot:SamPy.plot.trilogy.stat_robust.run}\pysiglinewithargsret{\bfcode{run}}{}{}
\end{fulllineitems}


\end{fulllineitems}


\index{str2num() (in module SamPy.plot.trilogy)}

\begin{fulllineitems}
\phantomsection\label{SamPy.plot:SamPy.plot.trilogy.str2num}\pysiglinewithargsret{\code{SamPy.plot.trilogy.}\bfcode{str2num}}{\emph{str}, \emph{rf=0}}{}
CONVERTS A STRING TO A NUMBER (INT OR FLOAT) IF POSSIBLE
ALSO RETURNS FORMAT IF rf=1

\end{fulllineitems}


\index{stringsplitatof() (in module SamPy.plot.trilogy)}

\begin{fulllineitems}
\phantomsection\label{SamPy.plot:SamPy.plot.trilogy.stringsplitatof}\pysiglinewithargsret{\code{SamPy.plot.trilogy.}\bfcode{stringsplitatof}}{\emph{str}, \emph{separator='`}}{}
Splits a string into floats

\end{fulllineitems}


\index{striskey() (in module SamPy.plot.trilogy)}

\begin{fulllineitems}
\phantomsection\label{SamPy.plot:SamPy.plot.trilogy.striskey}\pysiglinewithargsret{\code{SamPy.plot.trilogy.}\bfcode{striskey}}{\emph{str}}{}
IS str AN OPTION LIKE -C or -ker
(IT'S NOT IF IT'S -2 or -.9)

\end{fulllineitems}



\subsubsection{Subpackages}
\label{SamPy.plot:subpackages}

\paragraph{COSIHB Package}
\label{SamPy.plot.COSIHB:cosihb-package}\label{SamPy.plot.COSIHB::doc}

\subparagraph{\texttt{IO} Module}
\label{SamPy.plot.COSIHB:module-SamPy.plot.COSIHB.IO}\label{SamPy.plot.COSIHB:io-module}
\index{SamPy.plot.COSIHB.IO (module)}
IO class for COS instrument handbook plotting.
Can be used to read FITS image and tabular data.

Created on Mar 18, 2009

@author: Sami-Matias Niemi (\href{mailto:niemi@stsci.edu}{niemi@stsci.edu}) for STScI

\index{COSHBIO (class in SamPy.plot.COSIHB.IO)}

\begin{fulllineitems}
\phantomsection\label{SamPy.plot.COSIHB:SamPy.plot.COSIHB.IO.COSHBIO}\pysiglinewithargsret{\strong{class }\code{SamPy.plot.COSIHB.IO.}\bfcode{COSHBIO}}{\emph{path}, \emph{output}}{}
IO class for COS instrument handbook plotting.
Can be used to read FITS image and tabular data as well as ASCII data.
As the path of the files is given in the constructor, this is only useful
if the data is in same folder... (should be changed?)

\index{ASCIITable() (SamPy.plot.COSIHB.IO.COSHBIO method)}

\begin{fulllineitems}
\phantomsection\label{SamPy.plot.COSIHB:SamPy.plot.COSIHB.IO.COSHBIO.ASCIITable}\pysiglinewithargsret{\bfcode{ASCIITable}}{\emph{file}, \emph{comment='\#'}}{}
A simple function for reading data from a file into a table.
Comment lines and empty lines are skipped.

\end{fulllineitems}


\index{FITSImage() (SamPy.plot.COSIHB.IO.COSHBIO method)}

\begin{fulllineitems}
\phantomsection\label{SamPy.plot.COSIHB:SamPy.plot.COSIHB.IO.COSHBIO.FITSImage}\pysiglinewithargsret{\bfcode{FITSImage}}{\emph{file}, \emph{extension}}{}
Reads FITS image data to an array.
Returns data as a NumPy array.

\end{fulllineitems}


\index{FITSTable() (SamPy.plot.COSIHB.IO.COSHBIO method)}

\begin{fulllineitems}
\phantomsection\label{SamPy.plot.COSIHB:SamPy.plot.COSIHB.IO.COSHBIO.FITSTable}\pysiglinewithargsret{\bfcode{FITSTable}}{\emph{file}, \emph{extension=1}}{}~\begin{description}
\item[{Reads FITS table to an array...}] \leavevmode
Returns data as a numpy array

\end{description}

\end{fulllineitems}


\index{Header() (SamPy.plot.COSIHB.IO.COSHBIO method)}

\begin{fulllineitems}
\phantomsection\label{SamPy.plot.COSIHB:SamPy.plot.COSIHB.IO.COSHBIO.Header}\pysiglinewithargsret{\bfcode{Header}}{\emph{file}, \emph{extension}}{}
\end{fulllineitems}


\index{HeaderKeyword() (SamPy.plot.COSIHB.IO.COSHBIO method)}

\begin{fulllineitems}
\phantomsection\label{SamPy.plot.COSIHB:SamPy.plot.COSIHB.IO.COSHBIO.HeaderKeyword}\pysiglinewithargsret{\bfcode{HeaderKeyword}}{\emph{file}, \emph{extension}, \emph{keyword}}{}
A simple function to pull out a header keyword value.

\end{fulllineitems}


\index{writeToASCIIFile() (SamPy.plot.COSIHB.IO.COSHBIO method)}

\begin{fulllineitems}
\phantomsection\label{SamPy.plot.COSIHB:SamPy.plot.COSIHB.IO.COSHBIO.writeToASCIIFile}\pysiglinewithargsret{\bfcode{writeToASCIIFile}}{\emph{data}, \emph{outputfile}, \emph{header='`}, \emph{separator=' `}}{}
Writes data to an ASCII file.

\end{fulllineitems}


\end{fulllineitems}



\subparagraph{\texttt{compare\_synphot\_files} Module}
\label{SamPy.plot.COSIHB:compare-synphot-files-module}\label{SamPy.plot.COSIHB:module-SamPy.plot.COSIHB.compare_synphot_files}
\index{SamPy.plot.COSIHB.compare\_synphot\_files (module)}

\subparagraph{\texttt{plotting} Module}
\label{SamPy.plot.COSIHB:plotting-module}\label{SamPy.plot.COSIHB:module-SamPy.plot.COSIHB.plotting}
\index{SamPy.plot.COSIHB.plotting (module)}
A class to plot COS instrument handbook plots. Each figure 
is generated by it's own function.

Note that the same function can be used to generate multiple
figures of similar type.

@author: Sami-Matias Niemi (\href{mailto:niemi@stsci.edu}{niemi@stsci.edu}) for STScI

\index{Fix() (in module SamPy.plot.COSIHB.plotting)}

\begin{fulllineitems}
\phantomsection\label{SamPy.plot.COSIHB:SamPy.plot.COSIHB.plotting.Fix}\pysiglinewithargsret{\code{SamPy.plot.COSIHB.plotting.}\bfcode{Fix}}{\emph{data}, \emph{smooth=True}}{}
Simple function to separate 1st and 2nd-order light from 230L.

\end{fulllineitems}


\index{Plotting (class in SamPy.plot.COSIHB.plotting)}

\begin{fulllineitems}
\phantomsection\label{SamPy.plot.COSIHB:SamPy.plot.COSIHB.plotting.Plotting}\pysiglinewithargsret{\strong{class }\code{SamPy.plot.COSIHB.plotting.}\bfcode{Plotting}}{\emph{filetype='.pdf'}}{}
A class to plot COS instrument handbook plots.
Each figure is generated by a function.

\index{AcqImageExposureTimes() (SamPy.plot.COSIHB.plotting.Plotting method)}

\begin{fulllineitems}
\phantomsection\label{SamPy.plot.COSIHB:SamPy.plot.COSIHB.plotting.Plotting.AcqImageExposureTimes}\pysiglinewithargsret{\bfcode{AcqImageExposureTimes}}{\emph{data}, \emph{output}}{}
Fig. 7.3. in the COS IHB, version 1.0, page 72.

\end{fulllineitems}


\index{BOAMIRRORprofile() (SamPy.plot.COSIHB.plotting.Plotting method)}

\begin{fulllineitems}
\phantomsection\label{SamPy.plot.COSIHB:SamPy.plot.COSIHB.plotting.Plotting.BOAMIRRORprofile}\pysiglinewithargsret{\bfcode{BOAMIRRORprofile}}{\emph{dataxx}, \emph{dataxy}, \emph{datayx}, \emph{datayy}, \emph{output}}{}
Plots the cross-section through an image obtained with the BOA + MIRRORB in both x and y direction.
Corresponds to Figs. 7.6 and 7.7 on page 75 at the first COS instrument handbook.

\end{fulllineitems}


\index{BOATransmission() (SamPy.plot.COSIHB.plotting.Plotting method)}

\begin{fulllineitems}
\phantomsection\label{SamPy.plot.COSIHB:SamPy.plot.COSIHB.plotting.Plotting.BOATransmission}\pysiglinewithargsret{\bfcode{BOATransmission}}{\emph{datax}, \emph{datay}, \emph{every}, \emph{output}}{}
Plots transmission of the COS BOA as a function of wavelength.
Corresponds to Fig. 3.4 on page 14 of the first COS instrument handbook.

\end{fulllineitems}


\index{BOAprofile() (SamPy.plot.COSIHB.plotting.Plotting method)}

\begin{fulllineitems}
\phantomsection\label{SamPy.plot.COSIHB:SamPy.plot.COSIHB.plotting.Plotting.BOAprofile}\pysiglinewithargsret{\bfcode{BOAprofile}}{\emph{datax}, \emph{datay}, \emph{output}}{}
Plots the cross-section through an image obtained with the BOA.
Corresponds to Figure 7.5 on page 74 at the first COS instrument handbook.

\end{fulllineitems}


\index{COSFUVSpectrum() (SamPy.plot.COSIHB.plotting.Plotting method)}

\begin{fulllineitems}
\phantomsection\label{SamPy.plot.COSIHB:SamPy.plot.COSIHB.plotting.Plotting.COSFUVSpectrum}\pysiglinewithargsret{\bfcode{COSFUVSpectrum}}{\emph{imagedata}, \emph{wave}, \emph{counts}, \emph{exptime}, \emph{output}}{}
Plots an example of a COS FUV spectrum.
Corresponds to Figure 4.2 on page 29 at the first COS instrument handbook.

\end{fulllineitems}


\index{COSFUVSpectrumDev2() (SamPy.plot.COSIHB.plotting.Plotting method)}

\begin{fulllineitems}
\phantomsection\label{SamPy.plot.COSIHB:SamPy.plot.COSIHB.plotting.Plotting.COSFUVSpectrumDev2}\pysiglinewithargsret{\bfcode{COSFUVSpectrumDev2}}{\emph{imagedata}, \emph{wave}, \emph{counts}, \emph{exptime}, \emph{output}}{}
Plots an example of a COS FUV spectrum with annotate.
A bug in matplotlib version of STScI installation prevents using this...

\end{fulllineitems}


\index{Chapeter13Plots() (SamPy.plot.COSIHB.plotting.Plotting method)}

\begin{fulllineitems}
\phantomsection\label{SamPy.plot.COSIHB:SamPy.plot.COSIHB.plotting.Plotting.Chapeter13Plots}\pysiglinewithargsret{\bfcode{Chapeter13Plots}}{\emph{datax1}, \emph{datay1}, \emph{boax2}, \emph{boay2}, \emph{title}, \emph{output}, \emph{G140L=False}, \emph{logy=False}}{}
Plots the FUV point-source sensitivities for Chapter 13. Old method, still used for NUV.

\end{fulllineitems}


\index{Chapeter13Sensitivity() (SamPy.plot.COSIHB.plotting.Plotting method)}

\begin{fulllineitems}
\phantomsection\label{SamPy.plot.COSIHB:SamPy.plot.COSIHB.plotting.Plotting.Chapeter13Sensitivity}\pysiglinewithargsret{\bfcode{Chapeter13Sensitivity}}{\emph{psa}, \emph{boa}, \emph{title}, \emph{output}, \emph{G140L=False}, \emph{logy=False}}{}
Plots the FUV point-source sensitivities for Chapter 13.

\end{fulllineitems}


\index{Chapeter13SensitivityNew() (SamPy.plot.COSIHB.plotting.Plotting method)}

\begin{fulllineitems}
\phantomsection\label{SamPy.plot.COSIHB:SamPy.plot.COSIHB.plotting.Plotting.Chapeter13SensitivityNew}\pysiglinewithargsret{\bfcode{Chapeter13SensitivityNew}}{\emph{psa}, \emph{boa}, \emph{disp}, \emph{title}, \emph{output}, \emph{G140L=False}, \emph{logy=False}}{}
Plots the FUV point-source sensitivities for Chapter 13.

\end{fulllineitems}


\index{Chapter13SNPlot() (SamPy.plot.COSIHB.plotting.Plotting method)}

\begin{fulllineitems}
\phantomsection\label{SamPy.plot.COSIHB:SamPy.plot.COSIHB.plotting.Plotting.Chapter13SNPlot}\pysiglinewithargsret{\bfcode{Chapter13SNPlot}}{\emph{data}, \emph{stlim}, \emph{grating}, \emph{wavelength}, \emph{output}, \emph{FUV=True}}{}
Creates signal-to-noise plots for COS instrument HB Chapter 15.
Bright limit has been set for point source, should be changed if
adapted for diffuse source.

\end{fulllineitems}


\index{Chapter13Wavelength() (SamPy.plot.COSIHB.plotting.Plotting method)}

\begin{fulllineitems}
\phantomsection\label{SamPy.plot.COSIHB:SamPy.plot.COSIHB.plotting.Plotting.Chapter13Wavelength}\pysiglinewithargsret{\bfcode{Chapter13Wavelength}}{\emph{grating}, \emph{offset}, \emph{title}, \emph{output}, \emph{G140L=False}}{}
Plots the FUV wavelength ranges for given grating setting.

\end{fulllineitems}


\index{Chapter13WavelengthNUV() (SamPy.plot.COSIHB.plotting.Plotting method)}

\begin{fulllineitems}
\phantomsection\label{SamPy.plot.COSIHB:SamPy.plot.COSIHB.plotting.Plotting.Chapter13WavelengthNUV}\pysiglinewithargsret{\bfcode{Chapter13WavelengthNUV}}{\emph{grating}, \emph{xoffset}, \emph{title}, \emph{output}}{}
Plots the FUV wavelength ranges for given grating setting.

\end{fulllineitems}


\index{Chapter13WavelengthNUVNew() (SamPy.plot.COSIHB.plotting.Plotting method)}

\begin{fulllineitems}
\phantomsection\label{SamPy.plot.COSIHB:SamPy.plot.COSIHB.plotting.Plotting.Chapter13WavelengthNUVNew}\pysiglinewithargsret{\bfcode{Chapter13WavelengthNUVNew}}{\emph{data}, \emph{outputpath}}{}
Plots the FUV wavelength ranges for given grating setting.

\end{fulllineitems}


\index{Chapter13WavelengthOLD() (SamPy.plot.COSIHB.plotting.Plotting method)}

\begin{fulllineitems}
\phantomsection\label{SamPy.plot.COSIHB:SamPy.plot.COSIHB.plotting.Plotting.Chapter13WavelengthOLD}\pysiglinewithargsret{\bfcode{Chapter13WavelengthOLD}}{\emph{grating}, \emph{offset}, \emph{title}, \emph{output}, \emph{G140L=False}}{}
Plots the FUV wavelength ranges for given grating setting.

\end{fulllineitems}


\index{Chapter13Waves() (SamPy.plot.COSIHB.plotting.Plotting method)}

\begin{fulllineitems}
\phantomsection\label{SamPy.plot.COSIHB:SamPy.plot.COSIHB.plotting.Plotting.Chapter13Waves}\pysiglinewithargsret{\bfcode{Chapter13Waves}}{\emph{data}, \emph{grating}, \emph{output}}{}
\end{fulllineitems}


\index{Chapter13WavesOLD() (SamPy.plot.COSIHB.plotting.Plotting method)}

\begin{fulllineitems}
\phantomsection\label{SamPy.plot.COSIHB:SamPy.plot.COSIHB.plotting.Plotting.Chapter13WavesOLD}\pysiglinewithargsret{\bfcode{Chapter13WavesOLD}}{\emph{data}, \emph{grating}, \emph{output}}{}
\end{fulllineitems}


\index{DispersedLightAcquisitionTimes() (SamPy.plot.COSIHB.plotting.Plotting method)}

\begin{fulllineitems}
\phantomsection\label{SamPy.plot.COSIHB:SamPy.plot.COSIHB.plotting.Plotting.DispersedLightAcquisitionTimes}\pysiglinewithargsret{\bfcode{DispersedLightAcquisitionTimes}}{\emph{data}, \emph{gratings}, \emph{output}}{}
Fig. 8.9 in COS IHB, version 2.0, draft 1

\end{fulllineitems}


\index{FUVFlatField() (SamPy.plot.COSIHB.plotting.Plotting method)}

\begin{fulllineitems}
\phantomsection\label{SamPy.plot.COSIHB:SamPy.plot.COSIHB.plotting.Plotting.FUVFlatField}\pysiglinewithargsret{\bfcode{FUVFlatField}}{\emph{datax}, \emph{datay}, \emph{yline}, \emph{output}, \emph{smooth=False}, \emph{smoothing=(50}, \emph{)}}{}
Plots the NUV MAMA flat field.
Corresponds to Figure 5.6 on page 53 at the first COS instrument handbook.

\end{fulllineitems}


\index{FUVsensitivity() (SamPy.plot.COSIHB.plotting.Plotting method)}

\begin{fulllineitems}
\phantomsection\label{SamPy.plot.COSIHB:SamPy.plot.COSIHB.plotting.Plotting.FUVsensitivity}\pysiglinewithargsret{\bfcode{FUVsensitivity}}{\emph{G130}, \emph{G160}, \emph{G140}, \emph{output}}{}
Plots FUV point-source sensitivities in units of effective area
and throughput.

\end{fulllineitems}


\index{FUVsensitivityBOA() (SamPy.plot.COSIHB.plotting.Plotting method)}

\begin{fulllineitems}
\phantomsection\label{SamPy.plot.COSIHB:SamPy.plot.COSIHB.plotting.Plotting.FUVsensitivityBOA}\pysiglinewithargsret{\bfcode{FUVsensitivityBOA}}{\emph{datax1}, \emph{datay1}, \emph{datax2}, \emph{datay2}, \emph{datax3}, \emph{datay3}, \emph{output}}{}
Plots the FUV BOA point-source sensitivities for three gratings.

\end{fulllineitems}


\index{FUVsensitivityOLD() (SamPy.plot.COSIHB.plotting.Plotting method)}

\begin{fulllineitems}
\phantomsection\label{SamPy.plot.COSIHB:SamPy.plot.COSIHB.plotting.Plotting.FUVsensitivityOLD}\pysiglinewithargsret{\bfcode{FUVsensitivityOLD}}{\emph{datax1}, \emph{datay1}, \emph{datax2}, \emph{datay2}, \emph{datax3}, \emph{datay3}, \emph{output}}{}
Plots the FUV point-source sensitivities for three grating.
Corresponds to Figure 5.2 on page 40 at the first COS instrument handbook.

\end{fulllineitems}


\index{LSFcomparison() (SamPy.plot.COSIHB.plotting.Plotting method)}

\begin{fulllineitems}
\phantomsection\label{SamPy.plot.COSIHB:SamPy.plot.COSIHB.plotting.Plotting.LSFcomparison}\pysiglinewithargsret{\bfcode{LSFcomparison}}{\emph{data}, \emph{waves}, \emph{title}, \emph{output}}{}
Line Spread Function Comparison plots for Chapter 13.

\end{fulllineitems}


\index{MIRRORB() (SamPy.plot.COSIHB.plotting.Plotting method)}

\begin{fulllineitems}
\phantomsection\label{SamPy.plot.COSIHB:SamPy.plot.COSIHB.plotting.Plotting.MIRRORB}\pysiglinewithargsret{\bfcode{MIRRORB}}{\emph{datax}, \emph{datay}, \emph{output}}{}
Plots the cross-section of a point-source image with MIRRORB.
Corresponds to Figure 7.4 on page 73 at the first COS instrument handbook.

\end{fulllineitems}


\index{NUVFlatField() (SamPy.plot.COSIHB.plotting.Plotting method)}

\begin{fulllineitems}
\phantomsection\label{SamPy.plot.COSIHB:SamPy.plot.COSIHB.plotting.Plotting.NUVFlatField}\pysiglinewithargsret{\bfcode{NUVFlatField}}{\emph{datax}, \emph{datay}, \emph{yline}, \emph{output}, \emph{smooth=False}, \emph{smoothing=(20}, \emph{)}}{}
Plots the NUV MAMA flat field.
Corresponds to Figure 5.6 on page 53 at the first COS instrument handbook.

\end{fulllineitems}


\index{NUVImageProfile() (SamPy.plot.COSIHB.plotting.Plotting method)}

\begin{fulllineitems}
\phantomsection\label{SamPy.plot.COSIHB:SamPy.plot.COSIHB.plotting.Plotting.NUVImageProfile}\pysiglinewithargsret{\bfcode{NUVImageProfile}}{\emph{xpixels}, \emph{counts}, \emph{output}}{}
Plots NUV imaging profile for COS.
Corresponds to Fig. 6.3 on page 63 at the first COS instrument handbook.

\end{fulllineitems}


\index{NUVPSAImagingSensitivity() (SamPy.plot.COSIHB.plotting.Plotting method)}

\begin{fulllineitems}
\phantomsection\label{SamPy.plot.COSIHB:SamPy.plot.COSIHB.plotting.Plotting.NUVPSAImagingSensitivity}\pysiglinewithargsret{\bfcode{NUVPSAImagingSensitivity}}{\emph{datax}, \emph{datay}, \emph{output}}{}
Plots the sensitivity curve for COS NUV imaging with the PSA.
Corresponds to Fig. 6.1 on page 61 at the first COS instrument handbook.

\end{fulllineitems}


\index{NUVsensitivity() (SamPy.plot.COSIHB.plotting.Plotting method)}

\begin{fulllineitems}
\phantomsection\label{SamPy.plot.COSIHB:SamPy.plot.COSIHB.plotting.Plotting.NUVsensitivity}\pysiglinewithargsret{\bfcode{NUVsensitivity}}{\emph{G185}, \emph{G225}, \emph{G285}, \emph{G230}, \emph{output}}{}
Plots NUV point-source sensitivities in units of effective area
and throughput.

\end{fulllineitems}


\index{NUVsensitivityG230L() (SamPy.plot.COSIHB.plotting.Plotting method)}

\begin{fulllineitems}
\phantomsection\label{SamPy.plot.COSIHB:SamPy.plot.COSIHB.plotting.Plotting.NUVsensitivityG230L}\pysiglinewithargsret{\bfcode{NUVsensitivityG230L}}{\emph{datax}, \emph{datay}, \emph{output}}{}
Plots NUV point-source sensitivities for grating G230L.
Corresponds to Figure 5.4 on page 41 at the first COS instrument handbook.

\end{fulllineitems}


\index{NUVsensitivityM() (SamPy.plot.COSIHB.plotting.Plotting method)}

\begin{fulllineitems}
\phantomsection\label{SamPy.plot.COSIHB:SamPy.plot.COSIHB.plotting.Plotting.NUVsensitivityM}\pysiglinewithargsret{\bfcode{NUVsensitivityM}}{\emph{datax1}, \emph{datay1}, \emph{datax2}, \emph{datay2}, \emph{datax3}, \emph{datay3}, \emph{output}}{}
Plots NUV point-source sensitivities for M gratings.
Corresponds to Figure 5.3 on page 41 at the first COS instrument handbook.

\end{fulllineitems}


\index{PSAMMIRRORANUVImaging() (SamPy.plot.COSIHB.plotting.Plotting method)}

\begin{fulllineitems}
\phantomsection\label{SamPy.plot.COSIHB:SamPy.plot.COSIHB.plotting.Plotting.PSAMMIRRORANUVImaging}\pysiglinewithargsret{\bfcode{PSAMMIRRORANUVImaging}}{\emph{datax}, \emph{datay}, \emph{output}}{}
Test function to plot cross section through and image.

\end{fulllineitems}


\index{PSARelativeTransmission() (SamPy.plot.COSIHB.plotting.Plotting method)}

\begin{fulllineitems}
\phantomsection\label{SamPy.plot.COSIHB:SamPy.plot.COSIHB.plotting.Plotting.PSARelativeTransmission}\pysiglinewithargsret{\bfcode{PSARelativeTransmission}}{\emph{datax}, \emph{datay}, \emph{output}}{}
Plots the relative transmission of the COS PSA at 1450A.
Corresponds to Fig. 7.1 on page 69 at the first COS instrument handbook.

\end{fulllineitems}


\index{PSF() (SamPy.plot.COSIHB.plotting.Plotting method)}

\begin{fulllineitems}
\phantomsection\label{SamPy.plot.COSIHB:SamPy.plot.COSIHB.plotting.Plotting.PSF}\pysiglinewithargsret{\bfcode{PSF}}{\emph{datax}, \emph{datay}, \emph{dataz}, \emph{output}}{}
Does not do very good job with matplotlib 0.99.0.
The labels are somewhat tilted respect to the axis.

\end{fulllineitems}


\index{PSFObsolete() (SamPy.plot.COSIHB.plotting.Plotting method)}

\begin{fulllineitems}
\phantomsection\label{SamPy.plot.COSIHB:SamPy.plot.COSIHB.plotting.Plotting.PSFObsolete}\pysiglinewithargsret{\bfcode{PSFObsolete}}{\emph{data}, \emph{output}}{}
Surface plotting with Gnuplot. Output is horrible... do NOT use!

\end{fulllineitems}


\index{ScatteredLight() (SamPy.plot.COSIHB.plotting.Plotting method)}

\begin{fulllineitems}
\phantomsection\label{SamPy.plot.COSIHB:SamPy.plot.COSIHB.plotting.Plotting.ScatteredLight}\pysiglinewithargsret{\bfcode{ScatteredLight}}{\emph{datax}, \emph{datay}, \emph{output}}{}
Plots the scattered light in the FUV.
Corresponds to Fig. 5.1 on page 39 at the first COS instrument handbook.

\end{fulllineitems}


\index{SpatialProfileFUV() (SamPy.plot.COSIHB.plotting.Plotting method)}

\begin{fulllineitems}
\phantomsection\label{SamPy.plot.COSIHB:SamPy.plot.COSIHB.plotting.Plotting.SpatialProfileFUV}\pysiglinewithargsret{\bfcode{SpatialProfileFUV}}{\emph{corrtag\_data}, \emph{disp}, \emph{xvalues}, \emph{width}, \emph{ymin}, \emph{ymax}, \emph{extr}, \emph{ewidth}, \emph{title}, \emph{output}}{}
\end{fulllineitems}


\index{WavecalSpecPSA() (SamPy.plot.COSIHB.plotting.Plotting method)}

\begin{fulllineitems}
\phantomsection\label{SamPy.plot.COSIHB:SamPy.plot.COSIHB.plotting.Plotting.WavecalSpecPSA}\pysiglinewithargsret{\bfcode{WavecalSpecPSA}}{\emph{wave}, \emph{countss}, \emph{output}}{}
Plots and example wavecal spectrum of PSA FUV.
Corresponds to Figures 13.3 and 13.4 on page 138 of the fist COS instrument handbook.

\end{fulllineitems}


\end{fulllineitems}


\index{Sensitivity2EffectiveArea() (in module SamPy.plot.COSIHB.plotting)}

\begin{fulllineitems}
\phantomsection\label{SamPy.plot.COSIHB:SamPy.plot.COSIHB.plotting.Sensitivity2EffectiveArea}\pysiglinewithargsret{\code{SamPy.plot.COSIHB.plotting.}\bfcode{Sensitivity2EffectiveArea}}{\emph{sensitivity}, \emph{wavelength}, \emph{dispersion}}{}
\end{fulllineitems}


\index{Sensitivity2EffectiveAreaForArrays() (in module SamPy.plot.COSIHB.plotting)}

\begin{fulllineitems}
\phantomsection\label{SamPy.plot.COSIHB:SamPy.plot.COSIHB.plotting.Sensitivity2EffectiveAreaForArrays}\pysiglinewithargsret{\code{SamPy.plot.COSIHB.plotting.}\bfcode{Sensitivity2EffectiveAreaForArrays}}{\emph{wave}, \emph{sens}, \emph{disp}}{}
\end{fulllineitems}


\index{Smooth() (in module SamPy.plot.COSIHB.plotting)}

\begin{fulllineitems}
\phantomsection\label{SamPy.plot.COSIHB:SamPy.plot.COSIHB.plotting.Smooth}\pysiglinewithargsret{\code{SamPy.plot.COSIHB.plotting.}\bfcode{Smooth}}{\emph{wave}, \emph{eff}}{}
\end{fulllineitems}



\subparagraph{\texttt{synphotSensitivity} Module}
\label{SamPy.plot.COSIHB:synphotsensitivity-module}\label{SamPy.plot.COSIHB:module-SamPy.plot.COSIHB.synphotSensitivity}
\index{SamPy.plot.COSIHB.synphotSensitivity (module)}
\index{Chapeter15Sensitivity() (in module SamPy.plot.COSIHB.synphotSensitivity)}

\begin{fulllineitems}
\phantomsection\label{SamPy.plot.COSIHB:SamPy.plot.COSIHB.synphotSensitivity.Chapeter15Sensitivity}\pysiglinewithargsret{\code{SamPy.plot.COSIHB.synphotSensitivity.}\bfcode{Chapeter15Sensitivity}}{\emph{data}, \emph{boafile}, \emph{dispersion}, \emph{grating}, \emph{title}, \emph{output}}{}
Plots the point-source sensitivities for Chapter 15.

\end{fulllineitems}


\index{Chapeter15SensitivityNewG130M() (in module SamPy.plot.COSIHB.synphotSensitivity)}

\begin{fulllineitems}
\phantomsection\label{SamPy.plot.COSIHB:SamPy.plot.COSIHB.synphotSensitivity.Chapeter15SensitivityNewG130M}\pysiglinewithargsret{\code{SamPy.plot.COSIHB.synphotSensitivity.}\bfcode{Chapeter15SensitivityNewG130M}}{\emph{data}, \emph{boafile}, \emph{dispersion}, \emph{grating}, \emph{title}, \emph{output}}{}
Plots the point-source sensitivities for Chapter 15.

\end{fulllineitems}


\index{FUVsensitivity() (in module SamPy.plot.COSIHB.synphotSensitivity)}

\begin{fulllineitems}
\phantomsection\label{SamPy.plot.COSIHB:SamPy.plot.COSIHB.synphotSensitivity.FUVsensitivity}\pysiglinewithargsret{\code{SamPy.plot.COSIHB.synphotSensitivity.}\bfcode{FUVsensitivity}}{\emph{G130}, \emph{G160}, \emph{G140}, \emph{output}}{}
Plots FUV point-source sensitivities in units of 
effective area and throughput. Used in Chapter 6
of COS IHB, version 2.0.

\end{fulllineitems}


\index{NUVsensitivity() (in module SamPy.plot.COSIHB.synphotSensitivity)}

\begin{fulllineitems}
\phantomsection\label{SamPy.plot.COSIHB:SamPy.plot.COSIHB.synphotSensitivity.NUVsensitivity}\pysiglinewithargsret{\code{SamPy.plot.COSIHB.synphotSensitivity.}\bfcode{NUVsensitivity}}{\emph{G185}, \emph{G225}, \emph{G285}, \emph{G230}, \emph{output}}{}
Plots NUV point-source sensitivities in units
of effective area and throughput. Used in Chapter 6
of COS IHB, version 2.0.

\end{fulllineitems}


\index{findLatestFile() (in module SamPy.plot.COSIHB.synphotSensitivity)}

\begin{fulllineitems}
\phantomsection\label{SamPy.plot.COSIHB:SamPy.plot.COSIHB.synphotSensitivity.findLatestFile}\pysiglinewithargsret{\code{SamPy.plot.COSIHB.synphotSensitivity.}\bfcode{findLatestFile}}{\emph{folder}}{}
\end{fulllineitems}


\index{fixThroughputs() (in module SamPy.plot.COSIHB.synphotSensitivity)}

\begin{fulllineitems}
\phantomsection\label{SamPy.plot.COSIHB:SamPy.plot.COSIHB.synphotSensitivity.fixThroughputs}\pysiglinewithargsret{\code{SamPy.plot.COSIHB.synphotSensitivity.}\bfcode{fixThroughputs}}{\emph{data}}{}
\end{fulllineitems}


\index{main() (in module SamPy.plot.COSIHB.synphotSensitivity)}

\begin{fulllineitems}
\phantomsection\label{SamPy.plot.COSIHB:SamPy.plot.COSIHB.synphotSensitivity.main}\pysiglinewithargsret{\code{SamPy.plot.COSIHB.synphotSensitivity.}\bfcode{main}}{}{}
\end{fulllineitems}


\index{multiplyThroughputs() (in module SamPy.plot.COSIHB.synphotSensitivity)}

\begin{fulllineitems}
\phantomsection\label{SamPy.plot.COSIHB:SamPy.plot.COSIHB.synphotSensitivity.multiplyThroughputs}\pysiglinewithargsret{\code{SamPy.plot.COSIHB.synphotSensitivity.}\bfcode{multiplyThroughputs}}{\emph{data}, \emph{ota\_mirror}, \emph{fuv\_mirror='/grp/hst/cdbs/comp/cos/cos\_mirrora\_005\_syn.fits'}, \emph{NUV=False}}{}
\end{fulllineitems}


\index{printSensitivity() (in module SamPy.plot.COSIHB.synphotSensitivity)}

\begin{fulllineitems}
\phantomsection\label{SamPy.plot.COSIHB:SamPy.plot.COSIHB.synphotSensitivity.printSensitivity}\pysiglinewithargsret{\code{SamPy.plot.COSIHB.synphotSensitivity.}\bfcode{printSensitivity}}{\emph{data}, \emph{dispersion}, \emph{header}, \emph{outfile}}{}
\end{fulllineitems}


\index{throughputs() (in module SamPy.plot.COSIHB.synphotSensitivity)}

\begin{fulllineitems}
\phantomsection\label{SamPy.plot.COSIHB:SamPy.plot.COSIHB.synphotSensitivity.throughputs}\pysiglinewithargsret{\code{SamPy.plot.COSIHB.synphotSensitivity.}\bfcode{throughputs}}{\emph{files}}{}
\end{fulllineitems}



\subparagraph{\texttt{wrapperSynphot} Module}
\label{SamPy.plot.COSIHB:wrappersynphot-module}\label{SamPy.plot.COSIHB:module-SamPy.plot.COSIHB.wrapperSynphot}
\index{SamPy.plot.COSIHB.wrapperSynphot (module)}
This script works as a wrapper to create plots for the COS instrument handbook.
Boolean variables in the main program can be set to False to limit the number
of plots.

Calculation of zodiacal and earth shine background requires the use of local
fits files (Zodi.fits and earthshine.fits). If these files cannot be found
the S/N ration related plots of chapter 15 cannot be done.

This wrapper should always be accompanied with plotting.py which contains
a plotting class. This class holds all the plotting functions and is required
for running this wrapper.

@note: This wrapper superseeds wrapper.py

@author: Sami-Matias Niemi (\href{mailto:niemi@stsci.edu}{niemi@stsci.edu}) for STScI

\index{FUVSensitivityDataMod() (in module SamPy.plot.COSIHB.wrapperSynphot)}

\begin{fulllineitems}
\phantomsection\label{SamPy.plot.COSIHB:SamPy.plot.COSIHB.wrapperSynphot.FUVSensitivityDataMod}\pysiglinewithargsret{\code{SamPy.plot.COSIHB.wrapperSynphot.}\bfcode{FUVSensitivityDataMod}}{\emph{data}}{}
Modifies the table containing FUV sensitivity data. Goes through the table and
pulls out wavelength and sensitivity information. Goes through both sides (A and B).
For the latter channel checks that the wavelength is not 50A, which is inside the gap.
Makes sure that the saved wavelengths are not smaller than the first recorded  wavelength
of the side A.
The function could be made faster with array manipulations rather than looping
through the data.

\end{fulllineitems}


\index{NUVSensitivityDataMod() (in module SamPy.plot.COSIHB.wrapperSynphot)}

\begin{fulllineitems}
\phantomsection\label{SamPy.plot.COSIHB:SamPy.plot.COSIHB.wrapperSynphot.NUVSensitivityDataMod}\pysiglinewithargsret{\code{SamPy.plot.COSIHB.wrapperSynphot.}\bfcode{NUVSensitivityDataMod}}{\emph{data}, \emph{smoothing}, \emph{median=True}}{}
Manipulates NUV sensitivity data. Pulls out all central wavelengths for all three
stripes. Arranges the data with ascending wavelength.

\end{fulllineitems}


\index{NUVSensitivityDataMod2() (in module SamPy.plot.COSIHB.wrapperSynphot)}

\begin{fulllineitems}
\phantomsection\label{SamPy.plot.COSIHB:SamPy.plot.COSIHB.wrapperSynphot.NUVSensitivityDataMod2}\pysiglinewithargsret{\code{SamPy.plot.COSIHB.wrapperSynphot.}\bfcode{NUVSensitivityDataMod2}}{\emph{data}, \emph{stepsize}, \emph{Median=True}}{}
Manipulates NUV sensitivity data. Finds either the min flux 
value with given wavelength or takes the median of fluxes
at given wavelength.

\end{fulllineitems}


\index{STMagToFlux() (in module SamPy.plot.COSIHB.wrapperSynphot)}

\begin{fulllineitems}
\phantomsection\label{SamPy.plot.COSIHB:SamPy.plot.COSIHB.wrapperSynphot.STMagToFlux}\pysiglinewithargsret{\code{SamPy.plot.COSIHB.wrapperSynphot.}\bfcode{STMagToFlux}}{\emph{STMag}}{}
Returns:
NumPy array of fluxes.

\end{fulllineitems}


\index{calculateSTMAGlim() (in module SamPy.plot.COSIHB.wrapperSynphot)}

\begin{fulllineitems}
\phantomsection\label{SamPy.plot.COSIHB:SamPy.plot.COSIHB.wrapperSynphot.calculateSTMAGlim}\pysiglinewithargsret{\code{SamPy.plot.COSIHB.wrapperSynphot.}\bfcode{calculateSTMAGlim}}{\emph{global\_limit}, \emph{grating}, \emph{sensitivity}, \emph{gain=1}, \emph{NUV=False}}{}
A fugdge factor of 96\% for encircled energy fraction is being used.
These integration times are so short that we can ignore everything
else but source counts.

\end{fulllineitems}


\index{collapseSpectrumX() (in module SamPy.plot.COSIHB.wrapperSynphot)}

\begin{fulllineitems}
\phantomsection\label{SamPy.plot.COSIHB:SamPy.plot.COSIHB.wrapperSynphot.collapseSpectrumX}\pysiglinewithargsret{\code{SamPy.plot.COSIHB.wrapperSynphot.}\bfcode{collapseSpectrumX}}{\emph{imagedata}, \emph{xline}, \emph{width}}{}
Collapses the spectrum in x (COS dispersion) direction and 
sums pixels over.
@param xline: gives the center point of the spectrum.
@param width: gives the number of (plus and minus) pixels summed 
over in dispersion direction.
@note: This could be done easier with numpy.sum()

\end{fulllineitems}


\index{collapseSpectrumY() (in module SamPy.plot.COSIHB.wrapperSynphot)}

\begin{fulllineitems}
\phantomsection\label{SamPy.plot.COSIHB:SamPy.plot.COSIHB.wrapperSynphot.collapseSpectrumY}\pysiglinewithargsret{\code{SamPy.plot.COSIHB.wrapperSynphot.}\bfcode{collapseSpectrumY}}{\emph{imagedata}, \emph{yline}, \emph{width}}{}
Collapses the spectrum in y (spatial for FUV) direction and 
sums pixels over.
@param yline: gives the center point of the spectrum.
@param width: gives the number of (plus and minus) pixels summed 
over in spatial direction.
@note: This could be done easier with numpy.sum()

\end{fulllineitems}


\index{fluxToSTMag() (in module SamPy.plot.COSIHB.wrapperSynphot)}

\begin{fulllineitems}
\phantomsection\label{SamPy.plot.COSIHB:SamPy.plot.COSIHB.wrapperSynphot.fluxToSTMag}\pysiglinewithargsret{\code{SamPy.plot.COSIHB.wrapperSynphot.}\bfcode{fluxToSTMag}}{\emph{flux}}{}
Returns:
NumPy array of ST magnitudes

\end{fulllineitems}


\index{getDispersion() (in module SamPy.plot.COSIHB.wrapperSynphot)}

\begin{fulllineitems}
\phantomsection\label{SamPy.plot.COSIHB:SamPy.plot.COSIHB.wrapperSynphot.getDispersion}\pysiglinewithargsret{\code{SamPy.plot.COSIHB.wrapperSynphot.}\bfcode{getDispersion}}{\emph{segment}, \emph{grating}, \emph{cenwave}, \emph{fppos}, \emph{fileData}}{}
Return dispersion relation of a given configuration.

\end{fulllineitems}


\index{getEarthShineBackground() (in module SamPy.plot.COSIHB.wrapperSynphot)}

\begin{fulllineitems}
\phantomsection\label{SamPy.plot.COSIHB:SamPy.plot.COSIHB.wrapperSynphot.getEarthShineBackground}\pysiglinewithargsret{\code{SamPy.plot.COSIHB.wrapperSynphot.}\bfcode{getEarthShineBackground}}{\emph{wavelength}, \emph{width}}{}
Returns the Earth Shine background value at given wavelength range.
Note: Uses hard coded file earthshine.fits, which must be present in the folder
this function is called!

\end{fulllineitems}


\index{getZodiacalBackground() (in module SamPy.plot.COSIHB.wrapperSynphot)}

\begin{fulllineitems}
\phantomsection\label{SamPy.plot.COSIHB:SamPy.plot.COSIHB.wrapperSynphot.getZodiacalBackground}\pysiglinewithargsret{\code{SamPy.plot.COSIHB.wrapperSynphot.}\bfcode{getZodiacalBackground}}{\emph{wavelength}, \emph{width}}{}
Returns the Zodiacal light background value at given wavelength range.
Note: Uses hard coded file Zodi.fits, which must be present in the folder
this function is called!

\end{fulllineitems}


\index{printOutSensitivity() (in module SamPy.plot.COSIHB.wrapperSynphot)}

\begin{fulllineitems}
\phantomsection\label{SamPy.plot.COSIHB:SamPy.plot.COSIHB.wrapperSynphot.printOutSensitivity}\pysiglinewithargsret{\code{SamPy.plot.COSIHB.wrapperSynphot.}\bfcode{printOutSensitivity}}{\emph{data}, \emph{dispersion}, \emph{header}, \emph{outfile}, \emph{stepsize=50}}{}
\end{fulllineitems}


\index{selectSensitivity() (in module SamPy.plot.COSIHB.wrapperSynphot)}

\begin{fulllineitems}
\phantomsection\label{SamPy.plot.COSIHB:SamPy.plot.COSIHB.wrapperSynphot.selectSensitivity}\pysiglinewithargsret{\code{SamPy.plot.COSIHB.wrapperSynphot.}\bfcode{selectSensitivity}}{\emph{wavelength}, \emph{sensitivity}, \emph{requiredWave}, \emph{width}}{}
Returns:
Mean of the sensitivity within required wavelength region.

\end{fulllineitems}


\index{setCoordinatesImagingData() (in module SamPy.plot.COSIHB.wrapperSynphot)}

\begin{fulllineitems}
\phantomsection\label{SamPy.plot.COSIHB:SamPy.plot.COSIHB.wrapperSynphot.setCoordinatesImagingData}\pysiglinewithargsret{\code{SamPy.plot.COSIHB.wrapperSynphot.}\bfcode{setCoordinatesImagingData}}{\emph{imagedata}, \emph{x=-1}, \emph{y=-1}}{}
Sets coordinates to imaging data.
@return:  x, y, value found from image data

\end{fulllineitems}


\index{signalToNoise() (in module SamPy.plot.COSIHB.wrapperSynphot)}

\begin{fulllineitems}
\phantomsection\label{SamPy.plot.COSIHB:SamPy.plot.COSIHB.wrapperSynphot.signalToNoise}\pysiglinewithargsret{\code{SamPy.plot.COSIHB.wrapperSynphot.}\bfcode{signalToNoise}}{\emph{sensitivity}, \emph{flux}, \emph{time}, \emph{Npix}, \emph{Bsky}, \emph{Bdet}, \emph{gain=1.0}}{}
Calculates signal to noise ratio when following variables are given:
sensitivity, flux, exposure time, number of pixels, sky backgound, detector background, and gain.
For photon counters gain = 1. Number of pixels is used for the background calculations while the
sensitivity is used for calculating the total number of counts.

\end{fulllineitems}


\index{signalToNoisePerResel() (in module SamPy.plot.COSIHB.wrapperSynphot)}

\begin{fulllineitems}
\phantomsection\label{SamPy.plot.COSIHB:SamPy.plot.COSIHB.wrapperSynphot.signalToNoisePerResel}\pysiglinewithargsret{\code{SamPy.plot.COSIHB.wrapperSynphot.}\bfcode{signalToNoisePerResel}}{\emph{sensitivity}, \emph{flux}, \emph{time}, \emph{resel}, \emph{Bsky}, \emph{Bdet}, \emph{gain=1.0}}{}
Calculates signal to noise ratio using total counts in resolution element.
Input:
sensitivity, flux, time, resel, Bsky, Bdet, gain = 1.
Returns:
Signal-to-noise ratio, source counts, source noise, dark current counts, sky, background, total noise

\end{fulllineitems}



\subparagraph{\texttt{wrapper\_OLD} Module}
\label{SamPy.plot.COSIHB:module-SamPy.plot.COSIHB.wrapper_OLD}\label{SamPy.plot.COSIHB:wrapper-old-module}
\index{SamPy.plot.COSIHB.wrapper\_OLD (module)}
This script works as a wrapper to create plots for the COS instrument handbook.
Boolean variables in the main program can be set to False to limit the number
of plots.

Calculation of zodiacal and earth shine background requires the use of local
fits files (Zodi.fits and earthshine.fits). If these files cannot be found
the S/N ration related plots of chapter 13 cannot be done.

This wrapper should always be accompanied with plotting.py which contains
a plotting class. This class holds all the plotting functions and is required
for running this wrapper.

@author: Sami-Matias Niemi (\href{mailto:niemi@stsci.edu}{niemi@stsci.edu}) for STScI

\index{FUVSensitivityDataMod() (in module SamPy.plot.COSIHB.wrapper\_OLD)}

\begin{fulllineitems}
\phantomsection\label{SamPy.plot.COSIHB:SamPy.plot.COSIHB.wrapper_OLD.FUVSensitivityDataMod}\pysiglinewithargsret{\code{SamPy.plot.COSIHB.wrapper\_OLD.}\bfcode{FUVSensitivityDataMod}}{\emph{data}}{}
Modifies the table containing FUV sensitivity data. Goes through the table and
pulls out wavelength and sensitivity information. Goes through both sides (A and B).
For the latter channel checks that the wavelength is not 50A, which is inside the gap.
Makes sure that the saved wavelengths are not smaller than the first recorded  wavelength
of the side A.
The function could be made faster with array manipulations rather than looping
through the data.

\end{fulllineitems}


\index{NUVSensitivityDataMod() (in module SamPy.plot.COSIHB.wrapper\_OLD)}

\begin{fulllineitems}
\phantomsection\label{SamPy.plot.COSIHB:SamPy.plot.COSIHB.wrapper_OLD.NUVSensitivityDataMod}\pysiglinewithargsret{\code{SamPy.plot.COSIHB.wrapper\_OLD.}\bfcode{NUVSensitivityDataMod}}{\emph{data}, \emph{smoothing}, \emph{median=True}}{}
Manipulates NUV sensitivity data. Pulls out all central wavelengths for all three
stripes. Arranges the data with ascending wavelength.

\end{fulllineitems}


\index{NUVSensitivityDataMod2() (in module SamPy.plot.COSIHB.wrapper\_OLD)}

\begin{fulllineitems}
\phantomsection\label{SamPy.plot.COSIHB:SamPy.plot.COSIHB.wrapper_OLD.NUVSensitivityDataMod2}\pysiglinewithargsret{\code{SamPy.plot.COSIHB.wrapper\_OLD.}\bfcode{NUVSensitivityDataMod2}}{\emph{data}, \emph{stepsize}, \emph{Median=True}}{}
Manipulates NUV sensitivity data. Finds either the min flux value with given wavelength
or takes the median of fluxes at given wavelength.

\end{fulllineitems}


\index{STMagToFlux() (in module SamPy.plot.COSIHB.wrapper\_OLD)}

\begin{fulllineitems}
\phantomsection\label{SamPy.plot.COSIHB:SamPy.plot.COSIHB.wrapper_OLD.STMagToFlux}\pysiglinewithargsret{\code{SamPy.plot.COSIHB.wrapper\_OLD.}\bfcode{STMagToFlux}}{\emph{STMag}}{}
Returns:
NumPy array of fluxes.

\end{fulllineitems}


\index{collapseSpectrumX() (in module SamPy.plot.COSIHB.wrapper\_OLD)}

\begin{fulllineitems}
\phantomsection\label{SamPy.plot.COSIHB:SamPy.plot.COSIHB.wrapper_OLD.collapseSpectrumX}\pysiglinewithargsret{\code{SamPy.plot.COSIHB.wrapper\_OLD.}\bfcode{collapseSpectrumX}}{\emph{imagedata}, \emph{xline}, \emph{width}}{}
Collapses the spectrum in x (COS dispersion) direction and sums pixels over.
xline gives the center point of the spectrum.
width gives the number of (plus and minus) pixels summed over in dispersion direction.

\end{fulllineitems}


\index{collapseSpectrumY() (in module SamPy.plot.COSIHB.wrapper\_OLD)}

\begin{fulllineitems}
\phantomsection\label{SamPy.plot.COSIHB:SamPy.plot.COSIHB.wrapper_OLD.collapseSpectrumY}\pysiglinewithargsret{\code{SamPy.plot.COSIHB.wrapper\_OLD.}\bfcode{collapseSpectrumY}}{\emph{imagedata}, \emph{yline}, \emph{width}}{}
Collapses the spectrum in y (spatial for FUV) direction and sums pixels over.
yline gives the center point of the spectrum.
width gives the number of (plus and minus) pixels summed over in spatial direction.

\end{fulllineitems}


\index{fluxToSTMag() (in module SamPy.plot.COSIHB.wrapper\_OLD)}

\begin{fulllineitems}
\phantomsection\label{SamPy.plot.COSIHB:SamPy.plot.COSIHB.wrapper_OLD.fluxToSTMag}\pysiglinewithargsret{\code{SamPy.plot.COSIHB.wrapper\_OLD.}\bfcode{fluxToSTMag}}{\emph{flux}}{}
Returns:
NumPy array of ST magnitudes

\end{fulllineitems}


\index{getDispersion() (in module SamPy.plot.COSIHB.wrapper\_OLD)}

\begin{fulllineitems}
\phantomsection\label{SamPy.plot.COSIHB:SamPy.plot.COSIHB.wrapper_OLD.getDispersion}\pysiglinewithargsret{\code{SamPy.plot.COSIHB.wrapper\_OLD.}\bfcode{getDispersion}}{\emph{segment}, \emph{grating}, \emph{cenwave}, \emph{fppos}, \emph{fileData}}{}
Return dispersion relation of a given configuration.

\end{fulllineitems}


\index{getEarthShineBackground() (in module SamPy.plot.COSIHB.wrapper\_OLD)}

\begin{fulllineitems}
\phantomsection\label{SamPy.plot.COSIHB:SamPy.plot.COSIHB.wrapper_OLD.getEarthShineBackground}\pysiglinewithargsret{\code{SamPy.plot.COSIHB.wrapper\_OLD.}\bfcode{getEarthShineBackground}}{\emph{wavelength}, \emph{width}}{}
Returns the Earth Shine background value at given wavelength range.
Note: Uses hard coded file earthshine.fits, which must be present in the folder
this function is called!

\end{fulllineitems}


\index{getZodiacalBackground() (in module SamPy.plot.COSIHB.wrapper\_OLD)}

\begin{fulllineitems}
\phantomsection\label{SamPy.plot.COSIHB:SamPy.plot.COSIHB.wrapper_OLD.getZodiacalBackground}\pysiglinewithargsret{\code{SamPy.plot.COSIHB.wrapper\_OLD.}\bfcode{getZodiacalBackground}}{\emph{wavelength}, \emph{width}}{}
Returns the Zodiacal light background value at given wavelength range.
Note: Uses hard coded file Zodi.fits, which must be present in the folder
this function is called!

\end{fulllineitems}


\index{printOutSensitivity() (in module SamPy.plot.COSIHB.wrapper\_OLD)}

\begin{fulllineitems}
\phantomsection\label{SamPy.plot.COSIHB:SamPy.plot.COSIHB.wrapper_OLD.printOutSensitivity}\pysiglinewithargsret{\code{SamPy.plot.COSIHB.wrapper\_OLD.}\bfcode{printOutSensitivity}}{\emph{data}, \emph{dispersion}, \emph{header}, \emph{outfile}, \emph{stepsize=50}}{}
\end{fulllineitems}


\index{selectSensitivity() (in module SamPy.plot.COSIHB.wrapper\_OLD)}

\begin{fulllineitems}
\phantomsection\label{SamPy.plot.COSIHB:SamPy.plot.COSIHB.wrapper_OLD.selectSensitivity}\pysiglinewithargsret{\code{SamPy.plot.COSIHB.wrapper\_OLD.}\bfcode{selectSensitivity}}{\emph{wavelength}, \emph{sensitivity}, \emph{requiredWave}, \emph{width}}{}
Returns:
Mean of the sensitivity within required wavelength region.

\end{fulllineitems}


\index{setCoordinatesImagingData() (in module SamPy.plot.COSIHB.wrapper\_OLD)}

\begin{fulllineitems}
\phantomsection\label{SamPy.plot.COSIHB:SamPy.plot.COSIHB.wrapper_OLD.setCoordinatesImagingData}\pysiglinewithargsret{\code{SamPy.plot.COSIHB.wrapper\_OLD.}\bfcode{setCoordinatesImagingData}}{\emph{imagedata}, \emph{x=-1}, \emph{y=-1}}{}
Sets coordinates to imaging data.
Returns: x, y, value found from image data

\end{fulllineitems}


\index{signalToNoise() (in module SamPy.plot.COSIHB.wrapper\_OLD)}

\begin{fulllineitems}
\phantomsection\label{SamPy.plot.COSIHB:SamPy.plot.COSIHB.wrapper_OLD.signalToNoise}\pysiglinewithargsret{\code{SamPy.plot.COSIHB.wrapper\_OLD.}\bfcode{signalToNoise}}{\emph{sensitivity}, \emph{flux}, \emph{time}, \emph{Npix}, \emph{Bsky}, \emph{Bdet}, \emph{gain=1.0}}{}
Calculates signal to noise ratio when following variables are given:
sensitivity, flux, exposure time, number of pixels, sky backgound, detector background, and gain.
For photon counters gain = 1. Number of pixels is used for the background calculations while the
sensitivity is used for calculating the total number of counts.

\end{fulllineitems}


\index{signalToNoisePerResel() (in module SamPy.plot.COSIHB.wrapper\_OLD)}

\begin{fulllineitems}
\phantomsection\label{SamPy.plot.COSIHB:SamPy.plot.COSIHB.wrapper_OLD.signalToNoisePerResel}\pysiglinewithargsret{\code{SamPy.plot.COSIHB.wrapper\_OLD.}\bfcode{signalToNoisePerResel}}{\emph{sensitivity}, \emph{flux}, \emph{time}, \emph{resel}, \emph{Bsky}, \emph{Bdet}, \emph{gain=1.0}}{}
Calculates signal to noise ratio using total counts in resolution element.
Input:
sensitivity, flux, time, resel, Bsky, Bdet, gain = 1.
Returns:
Signal-to-noise ratio, source counts, source noise, dark current counts, sky, background, total noise

\end{fulllineitems}



\subsection{sandbox Package}
\label{SamPy.sandbox::doc}\label{SamPy.sandbox:sandbox-package}

\subsubsection{\texttt{MyTools} Module}
\label{SamPy.sandbox:module-SamPy.sandbox.MyTools}\label{SamPy.sandbox:mytools-module}
\index{SamPy.sandbox.MyTools (module)}
@requires: NumPy
@requires: SciPy
@requires: Pylab (at least 1.0.0)
@author: Carolin Villforth
@summary: A collection of General Useful Tools of all Trades

\index{AnaKDE (class in SamPy.sandbox.MyTools)}

\begin{fulllineitems}
\phantomsection\label{SamPy.sandbox:SamPy.sandbox.MyTools.AnaKDE}\pysiglinewithargsret{\strong{class }\code{SamPy.sandbox.MyTools.}\bfcode{AnaKDE}}{\emph{input}, \emph{InData=True}, \emph{CFill=True}, \emph{PlotData=False}, \emph{levels=None}}{}
A Tool for Analyzing Kernel Density Estimators (KDE)
Interactive Plotting
@type input: data array/KDE
@type InData: True/False
@type CFill: True/False
@type PlotData: True/False
@type levels: None/List
@param input: Input Data, either a KDE or an array with the data.

If data is input, InData needs to be set to True.

If KDE is input, InData needs to be set to False.
@param InData: Defining Data Input Type (see input)
@param CFill: If True, filled contours will be plotted.

If False, line contours will be plotted
@param PlotData: Overplot raw data points over contour?
@param levels: Define levels for contour plot (if None, pylab will create automatic levels)
@warning: This is not tested for nasty inputs!
@warning: No Exceptions are raised at any point
@todo: Create possibility for log-file output.

\index{Info() (SamPy.sandbox.MyTools.AnaKDE method)}

\begin{fulllineitems}
\phantomsection\label{SamPy.sandbox:SamPy.sandbox.MyTools.AnaKDE.Info}\pysiglinewithargsret{\bfcode{Info}}{}{}
Prints out all current parameters

\end{fulllineitems}


\index{OpenInteractive() (SamPy.sandbox.MyTools.AnaKDE method)}

\begin{fulllineitems}
\phantomsection\label{SamPy.sandbox:SamPy.sandbox.MyTools.AnaKDE.OpenInteractive}\pysiglinewithargsret{\bfcode{OpenInteractive}}{\emph{sampling=100}}{}
Opens an Interactive Window for Evaluating the KDE
@type sampling: int
@param sampling: sampling for vector extraction (OBSOLETE????)
@warning: Open Only after a Contour has been created!

\end{fulllineitems}


\index{RePlot() (SamPy.sandbox.MyTools.AnaKDE method)}

\begin{fulllineitems}
\phantomsection\label{SamPy.sandbox:SamPy.sandbox.MyTools.AnaKDE.RePlot}\pysiglinewithargsret{\bfcode{RePlot}}{}{}
Replots the Contour
Use after changing parameters
@bug: Replot should not re-evaluate the KDE is x\_vec and y\_vec have not been updated!

\end{fulllineitems}


\index{Update() (SamPy.sandbox.MyTools.AnaKDE method)}

\begin{fulllineitems}
\phantomsection\label{SamPy.sandbox:SamPy.sandbox.MyTools.AnaKDE.Update}\pysiglinewithargsret{\bfcode{Update}}{\emph{x\_vec=False}, \emph{y\_vec=False}, \emph{CFill=-999}, \emph{PlotData=-999}, \emph{levels=False}}{}
Update Class Parameters
Use Replot to Create an updated plot, then open an interactive Session.
All parameters as in Main Class

\end{fulllineitems}


\index{contour() (SamPy.sandbox.MyTools.AnaKDE method)}

\begin{fulllineitems}
\phantomsection\label{SamPy.sandbox:SamPy.sandbox.MyTools.AnaKDE.contour}\pysiglinewithargsret{\bfcode{contour}}{\emph{x\_vec}, \emph{y\_vec}, \emph{return\_data=False}}{}
Plotting a contour of the KDE
See Class main for all input parameters
@type x\_vec: array
@type y\_vec: array
@param x\_vec: X Vector along which KDE is evaluated
@param y\_vec: Y Vector along which KDE is evaluated
@note: inherits parameters from main class 
@note: I am not quite sure what contour does on funny vectors (unqeually spaced.....)
@note: self.test is for veryfing that the x,y order is kept intact properly by contour (I believe its now in order)

\end{fulllineitems}


\end{fulllineitems}



\subsubsection{\texttt{simpleMultiprocessingExample} Module}
\label{SamPy.sandbox:simplemultiprocessingexample-module}\label{SamPy.sandbox:module-SamPy.sandbox.simpleMultiprocessingExample}
\index{SamPy.sandbox.simpleMultiprocessingExample (module)}
\index{aFunction() (in module SamPy.sandbox.simpleMultiprocessingExample)}

\begin{fulllineitems}
\phantomsection\label{SamPy.sandbox:SamPy.sandbox.simpleMultiprocessingExample.aFunction}\pysiglinewithargsret{\code{SamPy.sandbox.simpleMultiprocessingExample.}\bfcode{aFunction}}{\emph{x}}{}
\end{fulllineitems}



\subsection{smakced Package}
\label{SamPy.smakced:smakced-package}\label{SamPy.smakced::doc}

\subsubsection{\texttt{getSmackedData} Module}
\label{SamPy.smakced:getsmackeddata-module}\label{SamPy.smakced:module-SamPy.smakced.getSmackedData}
\index{SamPy.smakced.getSmackedData (module)}
Pulls out data from the SMAKCED wiki page
and parses it into a NumPy array, which is
easy to sort and manipulate.
\begin{quote}\begin{description}
\item[{requires}] \leavevmode
NumPy

\item[{requires}] \leavevmode
BeautifulSoup

\item[{date}] \leavevmode
Created on Mar 26, 2010

\item[{version}] \leavevmode
0.1

\item[{author}] \leavevmode
Sami-Matias Niemi

\end{description}\end{quote}

\index{Smakced (class in SamPy.smakced.getSmackedData)}

\begin{fulllineitems}
\phantomsection\label{SamPy.smakced:SamPy.smakced.getSmackedData.Smakced}\pysiglinewithargsret{\strong{class }\code{SamPy.smakced.getSmackedData.}\bfcode{Smakced}}{\emph{url}, \emph{table}}{}
A Class related to SMAKCED collaboration wiki page.

\index{getData() (SamPy.smakced.getSmackedData.Smakced method)}

\begin{fulllineitems}
\phantomsection\label{SamPy.smakced:SamPy.smakced.getSmackedData.Smakced.getData}\pysiglinewithargsret{\bfcode{getData}}{}{}
Retrieve data from the SMAKCED web page.
\begin{quote}\begin{description}
\item[{Returns}] \leavevmode
retrieved raw data

\end{description}\end{quote}

\end{fulllineitems}


\index{parseTable() (SamPy.smakced.getSmackedData.Smakced method)}

\begin{fulllineitems}
\phantomsection\label{SamPy.smakced:SamPy.smakced.getSmackedData.Smakced.parseTable}\pysiglinewithargsret{\bfcode{parseTable}}{\emph{data}}{}
Parses html table data using BeautifulSoup.
Note that table number has been hard coded.
The SMAKCED wiki page returns several ``tables''.
\begin{quote}\begin{description}
\item[{Parameters}] \leavevmode
\textbf{data} -- data that has been retrieved with

\end{description}\end{quote}

the getData method
\begin{quote}\begin{description}
\item[{Returns}] \leavevmode
array containing table entries

\end{description}\end{quote}

\end{fulllineitems}


\index{writeToFile() (SamPy.smakced.getSmackedData.Smakced method)}

\begin{fulllineitems}
\phantomsection\label{SamPy.smakced:SamPy.smakced.getSmackedData.Smakced.writeToFile}\pysiglinewithargsret{\bfcode{writeToFile}}{\emph{data}, \emph{output}}{}
Writes an html page that contains tables to a file
in ascii format.
Uses BeautifulSoup for parsing tables.
\begin{quote}\begin{description}
\item[{Note }] \leavevmode
This method is untested.

\item[{Parameters}] \leavevmode
\textbf{data} -- data that has been retrieved with

\end{description}\end{quote}

the getData method
:param output: name of the output file

\end{fulllineitems}


\end{fulllineitems}



\subsubsection{\texttt{getSmackedDataNoNumpy} Module}
\label{SamPy.smakced:getsmackeddatanonumpy-module}\label{SamPy.smakced:module-SamPy.smakced.getSmackedDataNoNumpy}
\index{SamPy.smakced.getSmackedDataNoNumpy (module)}
Pulls out data from the SMAKCED wiki page
and parses it into an array that is simple
to manipulate and output.
\begin{quote}\begin{description}
\item[{requires}] \leavevmode
BeautifulSoup

\end{description}\end{quote}

HISTORY:
Created on Mar 26, 2010 (version 0.1)
Removed NumPy dependency April 2, 2010 (version 0.2)
\begin{quote}\begin{description}
\item[{version}] \leavevmode
0.2

\item[{author}] \leavevmode
Sami-Matias Niemi

\end{description}\end{quote}

\index{Smakced (class in SamPy.smakced.getSmackedDataNoNumpy)}

\begin{fulllineitems}
\phantomsection\label{SamPy.smakced:SamPy.smakced.getSmackedDataNoNumpy.Smakced}\pysiglinewithargsret{\strong{class }\code{SamPy.smakced.getSmackedDataNoNumpy.}\bfcode{Smakced}}{\emph{url}, \emph{table}}{}
A Class related to SMAKCED collaboration wiki page.

\index{getData() (SamPy.smakced.getSmackedDataNoNumpy.Smakced method)}

\begin{fulllineitems}
\phantomsection\label{SamPy.smakced:SamPy.smakced.getSmackedDataNoNumpy.Smakced.getData}\pysiglinewithargsret{\bfcode{getData}}{}{}
Retrieve data from the SMAKCED web page.
\begin{quote}\begin{description}
\item[{Returns}] \leavevmode
retrieved raw data

\end{description}\end{quote}

\end{fulllineitems}


\index{parseTable() (SamPy.smakced.getSmackedDataNoNumpy.Smakced method)}

\begin{fulllineitems}
\phantomsection\label{SamPy.smakced:SamPy.smakced.getSmackedDataNoNumpy.Smakced.parseTable}\pysiglinewithargsret{\bfcode{parseTable}}{\emph{data}}{}
Parses html table data using BeautifulSoup.
Note that table number has been hard coded.
The SMAKCED wiki page returns several ``tables''.
\begin{quote}\begin{description}
\item[{Parameters}] \leavevmode
\textbf{data} -- data that has been retrieved with

\end{description}\end{quote}

the getData method
\begin{quote}\begin{description}
\item[{Returns}] \leavevmode
array containing table entries

\end{description}\end{quote}

\end{fulllineitems}


\index{writeToFile() (SamPy.smakced.getSmackedDataNoNumpy.Smakced method)}

\begin{fulllineitems}
\phantomsection\label{SamPy.smakced:SamPy.smakced.getSmackedDataNoNumpy.Smakced.writeToFile}\pysiglinewithargsret{\bfcode{writeToFile}}{\emph{data}, \emph{output}}{}
Writes an html page that contains tables to a file
in ascii format.
Uses BeautifulSoup for parsing tables.
\begin{quote}\begin{description}
\item[{Note }] \leavevmode
This method is untested.

\item[{Parameters}] \leavevmode
\textbf{data} -- data that has been retrieved with

\end{description}\end{quote}

the getData method
:param output: name of the output file

\end{fulllineitems}


\end{fulllineitems}



\subsection{smnIO Package}
\label{SamPy.smnIO:smnio-package}\label{SamPy.smnIO::doc}

\subsubsection{\texttt{IO} Module}
\label{SamPy.smnIO:module-SamPy.smnIO.IO}\label{SamPy.smnIO:io-module}
\index{SamPy.smnIO.IO (module)}
IO class for COS instrument handbook plotting.
Can be used to read FITS image and tabular data.

Created on Mar 18, 2009

@author: Sami-Matias Niemi (\href{mailto:niemi@stsci.edu}{niemi@stsci.edu}) for STScI

\index{COSHBIO (class in SamPy.smnIO.IO)}

\begin{fulllineitems}
\phantomsection\label{SamPy.smnIO:SamPy.smnIO.IO.COSHBIO}\pysiglinewithargsret{\strong{class }\code{SamPy.smnIO.IO.}\bfcode{COSHBIO}}{\emph{path}, \emph{output}}{}
IO class for COS instrument handbook plotting.
Can be used to read FITS image and tabular data as well as ASCII data.
As the path of the files is given in the constructor, this is only useful
if the data is in same folder... (should be changed?)

\index{ASCIITable() (SamPy.smnIO.IO.COSHBIO method)}

\begin{fulllineitems}
\phantomsection\label{SamPy.smnIO:SamPy.smnIO.IO.COSHBIO.ASCIITable}\pysiglinewithargsret{\bfcode{ASCIITable}}{\emph{file}, \emph{comment='\#'}}{}
A simple function for reading data from a file into a table.
Comment lines and empty lines are skipped.

\end{fulllineitems}


\index{FITSImage() (SamPy.smnIO.IO.COSHBIO method)}

\begin{fulllineitems}
\phantomsection\label{SamPy.smnIO:SamPy.smnIO.IO.COSHBIO.FITSImage}\pysiglinewithargsret{\bfcode{FITSImage}}{\emph{file}, \emph{extension}}{}
Reads FITS image data to an array.
Returns data as a NumPy array.

\end{fulllineitems}


\index{FITSTable() (SamPy.smnIO.IO.COSHBIO method)}

\begin{fulllineitems}
\phantomsection\label{SamPy.smnIO:SamPy.smnIO.IO.COSHBIO.FITSTable}\pysiglinewithargsret{\bfcode{FITSTable}}{\emph{file}, \emph{extension=1}}{}~\begin{description}
\item[{Reads FITS table to an array...}] \leavevmode
Returns data as a numpy array

\end{description}

\end{fulllineitems}


\index{Header() (SamPy.smnIO.IO.COSHBIO method)}

\begin{fulllineitems}
\phantomsection\label{SamPy.smnIO:SamPy.smnIO.IO.COSHBIO.Header}\pysiglinewithargsret{\bfcode{Header}}{\emph{file}, \emph{extension}}{}
\end{fulllineitems}


\index{HeaderKeyword() (SamPy.smnIO.IO.COSHBIO method)}

\begin{fulllineitems}
\phantomsection\label{SamPy.smnIO:SamPy.smnIO.IO.COSHBIO.HeaderKeyword}\pysiglinewithargsret{\bfcode{HeaderKeyword}}{\emph{file}, \emph{extension}, \emph{keyword}}{}
A simple function to pull out a header keyword value.

\end{fulllineitems}


\index{writeToASCIIFile() (SamPy.smnIO.IO.COSHBIO method)}

\begin{fulllineitems}
\phantomsection\label{SamPy.smnIO:SamPy.smnIO.IO.COSHBIO.writeToASCIIFile}\pysiglinewithargsret{\bfcode{writeToASCIIFile}}{\emph{data}, \emph{outputfile}, \emph{header='`}, \emph{separator=' `}}{}
Writes data to an ASCII file.

\end{fulllineitems}


\end{fulllineitems}



\subsubsection{\texttt{findFiles} Module}
\label{SamPy.smnIO:findfiles-module}\label{SamPy.smnIO:module-SamPy.smnIO.findFiles}
\index{SamPy.smnIO.findFiles (module)}
\index{findFitsFiles() (in module SamPy.smnIO.findFiles)}

\begin{fulllineitems}
\phantomsection\label{SamPy.smnIO:SamPy.smnIO.findFiles.findFitsFiles}\pysiglinewithargsret{\code{SamPy.smnIO.findFiles.}\bfcode{findFitsFiles}}{}{}
Returns a list of FITS files
in the current working directory

\end{fulllineitems}



\subsubsection{\texttt{read} Module}
\label{SamPy.smnIO:module-SamPy.smnIO.read}\label{SamPy.smnIO:read-module}
\index{SamPy.smnIO.read (module)}
This file contains some helper functions to
parse different data files.

:author : Sami-Matias Niemi
:contact : \href{mailto:niemi@stsci.edu}{niemi@stsci.edu}
:version : 0.1
\begin{quote}\begin{description}
\item[{requires}] \leavevmode
NumPy

\end{description}\end{quote}

\index{GFBasicData() (in module SamPy.smnIO.read)}

\begin{fulllineitems}
\phantomsection\label{SamPy.smnIO:SamPy.smnIO.read.GFBasicData}\pysiglinewithargsret{\code{SamPy.smnIO.read.}\bfcode{GFBasicData}}{\emph{path}, \emph{AB=True}}{}
Reads Rachel's SAMs output data.
If AB = True then the function converts
Johnson U, B, V, and K bands to Vega system.
:param path (string): path in which the files are located
:param AB (boolean): whether Johnson U, B, V, and K bands
are converted to Vega system or not.

\end{fulllineitems}


\index{readBolshoiDMfile() (in module SamPy.smnIO.read)}

\begin{fulllineitems}
\phantomsection\label{SamPy.smnIO:SamPy.smnIO.read.readBolshoiDMfile}\pysiglinewithargsret{\code{SamPy.smnIO.read.}\bfcode{readBolshoiDMfile}}{\emph{filename}, \emph{column}, \emph{no\_phantoms}}{}
This little helper function can be used to read
dark matter halo masses from files produced from
the merger trees of the Bolshoi simulation.
:param filename (string): name of the file
:param column (int): which column to grep
:param no\_phantoms (boolean): wchich dark matter mass to read
:return: NumPy array of dark matter halo masses

\end{fulllineitems}



\subsubsection{\texttt{sextutils} Module}
\label{SamPy.smnIO:sextutils-module}\label{SamPy.smnIO:module-SamPy.smnIO.sextutils}
\index{SamPy.smnIO.sextutils (module)}
\index{getcol() (in module SamPy.smnIO.sextutils)}

\begin{fulllineitems}
\phantomsection\label{SamPy.smnIO:SamPy.smnIO.sextutils.getcol}\pysiglinewithargsret{\code{SamPy.smnIO.sextutils.}\bfcode{getcol}}{\emph{col}, \emph{lines}}{}
Get a column from a SExtractor catalog. Determine the type
(integer, float, string) and return either an array of that
type (Int32, Float64) or a list of strings

\end{fulllineitems}


\index{getcols() (in module SamPy.smnIO.sextutils)}

\begin{fulllineitems}
\phantomsection\label{SamPy.smnIO:SamPy.smnIO.sextutils.getcols}\pysiglinewithargsret{\code{SamPy.smnIO.sextutils.}\bfcode{getcols}}{\emph{d}, \emph{l}, \emph{*args}}{}
Get multiple columns from SExtractor list using getcol()

\end{fulllineitems}


\index{getcolvalues() (in module SamPy.smnIO.sextutils)}

\begin{fulllineitems}
\phantomsection\label{SamPy.smnIO:SamPy.smnIO.sextutils.getcolvalues}\pysiglinewithargsret{\code{SamPy.smnIO.sextutils.}\bfcode{getcolvalues}}{\emph{col}, \emph{coltype}, \emph{colentries}, \emph{colzero=False}}{}
Get a column from a SExtractor catalog. Determine the type
(integer, float, string) and return either an array of that
type (Int32, Float64) or a list of strings

\end{fulllineitems}


\index{getfloats() (in module SamPy.smnIO.sextutils)}

\begin{fulllineitems}
\phantomsection\label{SamPy.smnIO:SamPy.smnIO.sextutils.getfloats}\pysiglinewithargsret{\code{SamPy.smnIO.sextutils.}\bfcode{getfloats}}{\emph{col}, \emph{lines}, \emph{values}}{}
\end{fulllineitems}


\index{getints() (in module SamPy.smnIO.sextutils)}

\begin{fulllineitems}
\phantomsection\label{SamPy.smnIO:SamPy.smnIO.sextutils.getints}\pysiglinewithargsret{\code{SamPy.smnIO.sextutils.}\bfcode{getints}}{\emph{col}, \emph{lines}, \emph{values}}{}
\end{fulllineitems}


\index{getstrings() (in module SamPy.smnIO.sextutils)}

\begin{fulllineitems}
\phantomsection\label{SamPy.smnIO:SamPy.smnIO.sextutils.getstrings}\pysiglinewithargsret{\code{SamPy.smnIO.sextutils.}\bfcode{getstrings}}{\emph{col}, \emph{lines}, \emph{values}}{}
\end{fulllineitems}


\index{initcat() (in module SamPy.smnIO.sextutils)}

\begin{fulllineitems}
\phantomsection\label{SamPy.smnIO:SamPy.smnIO.sextutils.initcat}\pysiglinewithargsret{\code{SamPy.smnIO.sextutils.}\bfcode{initcat}}{\emph{catfile}, \emph{preserve\_case=False}}{}
parseheader -- reads the header of a SExtractor catalog file and
returns a dictionary of parameter names and column numbers.
Also returns a list of lines containing the data.

\end{fulllineitems}


\index{invert\_dict() (in module SamPy.smnIO.sextutils)}

\begin{fulllineitems}
\phantomsection\label{SamPy.smnIO:SamPy.smnIO.sextutils.invert_dict}\pysiglinewithargsret{\code{SamPy.smnIO.sextutils.}\bfcode{invert\_dict}}{\emph{d}}{}
Generate a new dictionary with the key/value relationship inverted

\end{fulllineitems}


\index{parseconfig\_se() (in module SamPy.smnIO.sextutils)}

\begin{fulllineitems}
\phantomsection\label{SamPy.smnIO:SamPy.smnIO.sextutils.parseconfig_se}\pysiglinewithargsret{\code{SamPy.smnIO.sextutils.}\bfcode{parseconfig\_se}}{\emph{cfile}}{}
parseconfig -- read a SExtractor .sex file and return a dictionary
of options \& values. Comments are ignored.

\end{fulllineitems}


\index{rw\_catalog (class in SamPy.smnIO.sextutils)}

\begin{fulllineitems}
\phantomsection\label{SamPy.smnIO:SamPy.smnIO.sextutils.rw_catalog}\pysiglinewithargsret{\strong{class }\code{SamPy.smnIO.sextutils.}\bfcode{rw\_catalog}}{\emph{fname}}{}
Bases: {\hyperref[SamPy.smnIO:SamPy.smnIO.sextutils.se_catalog]{\code{SamPy.smnIO.sextutils.se\_catalog}}}

Extend the se\_catalog class to support adding new columns,
and writing out the new version.

\index{addcolumn() (SamPy.smnIO.sextutils.rw\_catalog method)}

\begin{fulllineitems}
\phantomsection\label{SamPy.smnIO:SamPy.smnIO.sextutils.rw_catalog.addcolumn}\pysiglinewithargsret{\bfcode{addcolumn}}{\emph{input\_colname}, \emph{coldata}}{}
coldata must be a 1d numarray of the correct length

\end{fulllineitems}


\index{addemptycolumn() (SamPy.smnIO.sextutils.rw\_catalog method)}

\begin{fulllineitems}
\phantomsection\label{SamPy.smnIO:SamPy.smnIO.sextutils.rw_catalog.addemptycolumn}\pysiglinewithargsret{\bfcode{addemptycolumn}}{\emph{input\_colname}, \emph{coltype}}{}
Defines a new column \& updates all the bookkeeping, but
does not actually fill in the data.

\end{fulllineitems}


\index{line() (SamPy.smnIO.sextutils.rw\_catalog method)}

\begin{fulllineitems}
\phantomsection\label{SamPy.smnIO:SamPy.smnIO.sextutils.rw_catalog.line}\pysiglinewithargsret{\bfcode{line}}{\emph{rownum}}{}
Construct a new line as to be printed out

\end{fulllineitems}


\index{printme() (SamPy.smnIO.sextutils.rw\_catalog method)}

\begin{fulllineitems}
\phantomsection\label{SamPy.smnIO:SamPy.smnIO.sextutils.rw_catalog.printme}\pysiglinewithargsret{\bfcode{printme}}{}{}
Like writeto, but for sys.stdout

\end{fulllineitems}


\index{writeto() (SamPy.smnIO.sextutils.rw\_catalog method)}

\begin{fulllineitems}
\phantomsection\label{SamPy.smnIO:SamPy.smnIO.sextutils.rw_catalog.writeto}\pysiglinewithargsret{\bfcode{writeto}}{\emph{outname}, \emph{clobber=False}}{}
\end{fulllineitems}


\end{fulllineitems}


\index{se\_catalog (class in SamPy.smnIO.sextutils)}

\begin{fulllineitems}
\phantomsection\label{SamPy.smnIO:SamPy.smnIO.sextutils.se_catalog}\pysiglinewithargsret{\strong{class }\code{SamPy.smnIO.sextutils.}\bfcode{se\_catalog}}{\emph{cfile}, \emph{readfile=True}, \emph{preserve\_case=False}}{}
Bases: \code{object}

Read a SExtractor-style catalog. 
Usage: c=se\_catalog(catalog,readfile=True,preserve\_case=False)
Will read the catalog and return an object c, whose attributes are 
arrays containing the data. For example, c.mag\_auto contains the 
mag\_auto values.
Arguments:
catalog -- The input SExtractor catalog. 
readfile -- True means read the data. False means return the
\begin{quote}

object without reading the data. The lines from the catalog
are returned as a list of ascii strings c.l. Useful if you want
to do some special parsing of some sort.
\end{quote}

preserve\_case -- default (False) converts column names to lower case
\begin{description}
\item[{The input catalog MUST have a header with the SExtractor format:}] \leavevmode
\# 1 ID comment
\# 2 ALPHA\_J200 another comment

\end{description}

That is, first column is the comment symbol \#, second column is
the column number, third column is the column name, and the rest
of the line is a comment. SExtractor allows ``vectors'' to be identified
only by the first column...e.g.
\begin{quote}

\# 12 FLUX\_APER
\# 20 FLUXERR\_APER
\end{quote}

the missing columns are all aperture fluxes through different
apertures. These will be read into attributes:
\begin{quote}

c.flux\_aper   \# The first one
c.flux\_aper\_1 \# the second one, and so on
\end{quote}

The case of aperture radii is a bit nasty, since these only
appear in the SExtractor configuration file. Use parseconfig()
to read that file.

\index{buildheader() (SamPy.smnIO.sextutils.se\_catalog method)}

\begin{fulllineitems}
\phantomsection\label{SamPy.smnIO:SamPy.smnIO.sextutils.se_catalog.buildheader}\pysiglinewithargsret{\bfcode{buildheader}}{}{}
Reconstruct the header from the header dictionary.
This might be useful if only a few columns were selected
from the file; otherwise just use the `header' attribute.

\end{fulllineitems}


\index{getcol() (SamPy.smnIO.sextutils.se\_catalog method)}

\begin{fulllineitems}
\phantomsection\label{SamPy.smnIO:SamPy.smnIO.sextutils.se_catalog.getcol}\pysiglinewithargsret{\bfcode{getcol}}{\emph{col}, \emph{offset=0}}{}
\end{fulllineitems}


\index{getcols() (SamPy.smnIO.sextutils.se\_catalog method)}

\begin{fulllineitems}
\phantomsection\label{SamPy.smnIO:SamPy.smnIO.sextutils.se_catalog.getcols}\pysiglinewithargsret{\bfcode{getcols}}{\emph{*args}}{}
\end{fulllineitems}


\index{gettypes() (SamPy.smnIO.sextutils.se\_catalog method)}

\begin{fulllineitems}
\phantomsection\label{SamPy.smnIO:SamPy.smnIO.sextutils.se_catalog.gettypes}\pysiglinewithargsret{\bfcode{gettypes}}{}{}
\end{fulllineitems}


\index{line() (SamPy.smnIO.sextutils.se\_catalog method)}

\begin{fulllineitems}
\phantomsection\label{SamPy.smnIO:SamPy.smnIO.sextutils.se_catalog.line}\pysiglinewithargsret{\bfcode{line}}{\emph{i}}{}
Returns an assembled line of this catalog suitable for writing.
Except it doesn't really, if we modified the individual columns...

\end{fulllineitems}


\end{fulllineitems}


\index{sextractor (class in SamPy.smnIO.sextutils)}

\begin{fulllineitems}
\phantomsection\label{SamPy.smnIO:SamPy.smnIO.sextutils.sextractor}\pysiglinewithargsret{\strong{class }\code{SamPy.smnIO.sextutils.}\bfcode{sextractor}}{\emph{cfile}, \emph{readfile=True}, \emph{preserve\_case=False}}{}
Bases: {\hyperref[SamPy.smnIO:SamPy.smnIO.sextutils.se_catalog]{\code{SamPy.smnIO.sextutils.se\_catalog}}}

Read SExtractor catalog...just an alias for se\_catalog

\end{fulllineitems}


\index{writeheader() (in module SamPy.smnIO.sextutils)}

\begin{fulllineitems}
\phantomsection\label{SamPy.smnIO:SamPy.smnIO.sextutils.writeheader}\pysiglinewithargsret{\code{SamPy.smnIO.sextutils.}\bfcode{writeheader}}{\emph{fh}, \emph{colnames}}{}
Write an SExtractor-style header to an open file handle.

@param fh: file handle
@type fh: file

@param colnames: list of column names
@type colnames: list

@todo: add space checking to colnames
@todo: permit passing a filename?
@todo: handle comments

\end{fulllineitems}



\subsubsection{\texttt{write} Module}
\label{SamPy.smnIO:module-SamPy.smnIO.write}\label{SamPy.smnIO:write-module}
\index{SamPy.smnIO.write (module)}
\index{combineFiles() (in module SamPy.smnIO.write)}

\begin{fulllineitems}
\phantomsection\label{SamPy.smnIO:SamPy.smnIO.write.combineFiles}\pysiglinewithargsret{\code{SamPy.smnIO.write.}\bfcode{combineFiles}}{\emph{files}, \emph{outputfile}}{}
Combines the content of all files that are listed
in the files list to a single file named outputfile.
Iterates over the input files line-by-line to save
memory.

\end{fulllineitems}



\subsection{statistics Package}
\label{SamPy.statistics:statistics-package}\label{SamPy.statistics::doc}

\subsubsection{\texttt{StatTool} Module}
\label{SamPy.statistics:module-SamPy.statistics.StatTool}\label{SamPy.statistics:stattool-module}
\index{SamPy.statistics.StatTool (module)}
\index{process\_args() (in module SamPy.statistics.StatTool)}

\begin{fulllineitems}
\phantomsection\label{SamPy.statistics:SamPy.statistics.StatTool.process_args}\pysiglinewithargsret{\code{SamPy.statistics.StatTool.}\bfcode{process\_args}}{}{}
\end{fulllineitems}



\subsubsection{\texttt{StatTool2files} Module}
\label{SamPy.statistics:module-SamPy.statistics.StatTool2files}\label{SamPy.statistics:stattool2files-module}
\index{SamPy.statistics.StatTool2files (module)}
\index{process\_args() (in module SamPy.statistics.StatTool2files)}

\begin{fulllineitems}
\phantomsection\label{SamPy.statistics:SamPy.statistics.StatTool2files.process_args}\pysiglinewithargsret{\code{SamPy.statistics.StatTool2files.}\bfcode{process\_args}}{}{}
\end{fulllineitems}



\subsubsection{\texttt{plot\_ica\_vs\_pca} Module}
\label{SamPy.statistics:module-SamPy.statistics.plot_ica_vs_pca}\label{SamPy.statistics:plot-ica-vs-pca-module}
\index{SamPy.statistics.plot\_ica\_vs\_pca (module)}

\paragraph{FastICA on 2D point clouds}
\label{SamPy.statistics:fastica-on-2d-point-clouds}
Illustrate visually the results of \emph{ICA} vs \emph{PCA} in the
feature space.

Representing ICA in the feature space gives the view of `geometric ICA':
ICA is an algorithm that finds directions in the feature space
corresponding to projections with high non-Gaussianity. These directions
need not be orthogonal in the original feature space, but they are
orthogonal in the whitened feature space, in which all directions
correspond to the same variance.

PCA, on the other hand, finds orthogonal directions in the raw feature
space that correspond to directions accoutning for maximum variance.

Here we simulate independent sources using a highly non-Gaussian
process, 2 student T with a low number of degrees of freedom (top left
figure). We mix them to create observations (top right figure).
In this raw observation space, directions identified by PCA are
represented by green vectors. We represent the signal in the PCA space,
after whitening by the variance corresponding to the PCA vectors (lower
left). Running ICA corresponds to finding a rotation in this space to
identify the directions of largest non-Gaussianity (lower right).


\subsection{stis Package}
\label{SamPy.stis:stis-package}\label{SamPy.stis::doc}

\subsubsection{\texttt{STISCCDFlatWavelengthDependency} Module}
\label{SamPy.stis:module-SamPy.stis.STISCCDFlatWavelengthDependency}\label{SamPy.stis:stisccdflatwavelengthdependency-module}
\index{SamPy.stis.STISCCDFlatWavelengthDependency (module)}
Created on Dec 8, 2009

@author: niemi

\index{CombineFlat() (in module SamPy.stis.STISCCDFlatWavelengthDependency)}

\begin{fulllineitems}
\phantomsection\label{SamPy.stis:SamPy.stis.STISCCDFlatWavelengthDependency.CombineFlat}\pysiglinewithargsret{\code{SamPy.stis.STISCCDFlatWavelengthDependency.}\bfcode{CombineFlat}}{\emph{self}, \emph{filelist}}{}
Combines created images to form a final flats.

\end{fulllineitems}


\index{RMS() (in module SamPy.stis.STISCCDFlatWavelengthDependency)}

\begin{fulllineitems}
\phantomsection\label{SamPy.stis:SamPy.stis.STISCCDFlatWavelengthDependency.RMS}\pysiglinewithargsret{\code{SamPy.stis.STISCCDFlatWavelengthDependency.}\bfcode{RMS}}{\emph{data}}{}
\end{fulllineitems}



\subsubsection{\texttt{STISCCDImagingFlats} Module}
\label{SamPy.stis:stisccdimagingflats-module}\label{SamPy.stis:module-SamPy.stis.STISCCDImagingFlats}
\index{SamPy.stis.STISCCDImagingFlats (module)}
ABOUT:
Plots ratios between STIS CCD imaging flats.

DEPENDS:
Python 2.5 or later (no 3.x compatible)
NumPy
PyFITS
matplotlib

TESTED:
Python 2.5.1
NumPy: 1.4.0.dev7576
PyFITS: 2.2.2
matplotlib 1.0.svn

HISTORY:
Created on November 18, 2009

VERSION:
0.15: test release (SMN)

@author: Sami-Matias Niemi (\href{mailto:niemi@stsci.edu}{niemi@stsci.edu})

\index{Idetifiers() (in module SamPy.stis.STISCCDImagingFlats)}

\begin{fulllineitems}
\phantomsection\label{SamPy.stis:SamPy.stis.STISCCDImagingFlats.Idetifiers}\pysiglinewithargsret{\code{SamPy.stis.STISCCDImagingFlats.}\bfcode{Idetifiers}}{}{}
0th extension header keywords.
Used to select and pair FITS files.

\end{fulllineitems}


\index{ImStats() (in module SamPy.stis.STISCCDImagingFlats)}

\begin{fulllineitems}
\phantomsection\label{SamPy.stis:SamPy.stis.STISCCDImagingFlats.ImStats}\pysiglinewithargsret{\code{SamPy.stis.STISCCDImagingFlats.}\bfcode{ImStats}}{\emph{im1}, \emph{im2}, \emph{scale}}{}
Calculates some basic image statistics from 
two images given.
@return: ratio, image1 mean, image1 std, ratio mean

\end{fulllineitems}



\subsubsection{\texttt{STISCCDSpectroscopyFlat} Module}
\label{SamPy.stis:stisccdspectroscopyflat-module}

\subsubsection{\texttt{STISmonthly} Module}
\label{SamPy.stis:stismonthly-module}\label{SamPy.stis:module-SamPy.stis.STISmonthly}
\index{SamPy.stis.STISmonthly (module)}
A script to pull out information from ZEPPO for STIS MSM montly monitoring.

Created on Mar 23, 2009

@author: Sami-Matias Niemi

\index{getMonth() (in module SamPy.stis.STISmonthly)}

\begin{fulllineitems}
\phantomsection\label{SamPy.stis:SamPy.stis.STISmonthly.getMonth}\pysiglinewithargsret{\code{SamPy.stis.STISmonthly.}\bfcode{getMonth}}{}{}
Returns a date 4 weeks ago from today.

\end{fulllineitems}


\index{saveDataToFile() (in module SamPy.stis.STISmonthly)}

\begin{fulllineitems}
\phantomsection\label{SamPy.stis:SamPy.stis.STISmonthly.saveDataToFile}\pysiglinewithargsret{\code{SamPy.stis.STISmonthly.}\bfcode{saveDataToFile}}{\emph{filename}, \emph{data}}{}
Prints data to a file.

\end{fulllineitems}



\subsection{wrapper Package}
\label{SamPy.wrapper:wrapper-package}\label{SamPy.wrapper::doc}

\subsubsection{\texttt{gf} Module}
\label{SamPy.wrapper:module-SamPy.wrapper.gf}\label{SamPy.wrapper:gf-module}
\index{SamPy.wrapper.gf (module)}
@summary: This little wrapper can be used to
run GF (R. Somerville's SAMs program) using
threading.

@author: Sami-Matias Niemi
@contact: \href{mailto:niemi@stsci.edu}{niemi@stsci.edu}

\index{Run\_GF\_Threaded (class in SamPy.wrapper.gf)}

\begin{fulllineitems}
\phantomsection\label{SamPy.wrapper:SamPy.wrapper.gf.Run_GF_Threaded}\pysiglinewithargsret{\strong{class }\code{SamPy.wrapper.gf.}\bfcode{Run\_GF\_Threaded}}{\emph{queue}, \emph{out\_path='/Users/niemi/Desktop/Research/run/'}, \emph{param\_template='/Users/niemi/Desktop/Research/orig\_param\_file'}, \emph{gf\_binary='/Users/niemi/Desktop/Research/gf\_bolshoi/gf'}}{}
Bases: \code{threading.Thread}

Threaded way of running GF.

\index{run() (SamPy.wrapper.gf.Run\_GF\_Threaded method)}

\begin{fulllineitems}
\phantomsection\label{SamPy.wrapper:SamPy.wrapper.gf.Run_GF_Threaded.run}\pysiglinewithargsret{\bfcode{run}}{}{}
Method threading will call.

\end{fulllineitems}


\end{fulllineitems}


\index{main() (in module SamPy.wrapper.gf)}

\begin{fulllineitems}
\phantomsection\label{SamPy.wrapper:SamPy.wrapper.gf.main}\pysiglinewithargsret{\code{SamPy.wrapper.gf.}\bfcode{main}}{\emph{input\_files}, \emph{cores=6}}{}
Main driver function of the wrapper.

\end{fulllineitems}



\chapter{Indices and tables}
\label{index:indices-and-tables}\begin{itemize}
\item {} 
\emph{genindex}

\item {} 
\emph{modindex}

\item {} 
\emph{search}

\end{itemize}


\renewcommand{\indexname}{Python Module Index}
\begin{theindex}
\def\bigletter#1{{\Large\sffamily#1}\nopagebreak\vspace{1mm}}
\bigletter{s}
\item {\texttt{SamPy.astronomy.baryons}}, \pageref{SamPy.astronomy:module-SamPy.astronomy.baryons}
\item {\texttt{SamPy.astronomy.basics}}, \pageref{SamPy.astronomy:module-SamPy.astronomy.basics}
\item {\texttt{SamPy.astronomy.blackholes}}, \pageref{SamPy.astronomy:module-SamPy.astronomy.blackholes}
\item {\texttt{SamPy.astronomy.conversions}}, \pageref{SamPy.astronomy:module-SamPy.astronomy.conversions}
\item {\texttt{SamPy.astronomy.datamanipulation}}, \pageref{SamPy.astronomy:module-SamPy.astronomy.datamanipulation}
\item {\texttt{SamPy.astronomy.differentialfunctions}}, \pageref{SamPy.astronomy:module-SamPy.astronomy.differentialfunctions}
\item {\texttt{SamPy.astronomy.footprintfinder}}, \pageref{SamPy.astronomy:module-SamPy.astronomy.footprintfinder}
\item {\texttt{SamPy.astronomy.gasmasses}}, \pageref{SamPy.astronomy:module-SamPy.astronomy.gasmasses}
\item {\texttt{SamPy.astronomy.hess\_plot}}, \pageref{SamPy.astronomy:module-SamPy.astronomy.hess_plot}
\item {\texttt{SamPy.astronomy.luminosityFunctionss}}, \pageref{SamPy.astronomy:module-SamPy.astronomy.luminosityFunctionss}
\item {\texttt{SamPy.astronomy.metals}}, \pageref{SamPy.astronomy:module-SamPy.astronomy.metals}
\item {\texttt{SamPy.astronomy.randomizers}}, \pageref{SamPy.astronomy:module-SamPy.astronomy.randomizers}
\item {\texttt{SamPy.astronomy.stellarMFs}}, \pageref{SamPy.astronomy:module-SamPy.astronomy.stellarMFs}
\item {\texttt{SamPy.bolshoi.collecthalomasses}}, \pageref{SamPy.bolshoi:module-SamPy.bolshoi.collecthalomasses}
\item {\texttt{SamPy.bolshoi.collecthalomassesThreading}}, \pageref{SamPy.bolshoi:module-SamPy.bolshoi.collecthalomassesThreading}
\item {\texttt{SamPy.bolshoi.findRedshiftsBolshoiTrees}}, \pageref{SamPy.bolshoi:module-SamPy.bolshoi.findRedshiftsBolshoiTrees}
\item {\texttt{SamPy.bolshoi.generateTestPlots}}, \pageref{SamPy.bolshoi:module-SamPy.bolshoi.generateTestPlots}
\item {\texttt{SamPy.bolshoi.plotdarkmattermf}}, \pageref{SamPy.bolshoi:module-SamPy.bolshoi.plotdarkmattermf}
\item {\texttt{SamPy.bolshoi.plotMassRatios}}, \pageref{SamPy.bolshoi:module-SamPy.bolshoi.plotMassRatios}
\item {\texttt{SamPy.bolshoi.plotstellarmf}}, \pageref{SamPy.bolshoi:module-SamPy.bolshoi.plotstellarmf}
\item {\texttt{SamPy.bolshoi.randomizeSubhaloPosition}}, \pageref{SamPy.bolshoi:module-SamPy.bolshoi.randomizeSubhaloPosition}
\item {\texttt{SamPy.candels.plotFilters}}, \pageref{SamPy.candels:module-SamPy.candels.plotFilters}
\item {\texttt{SamPy.cos.plot\_all\_GET}}, \pageref{SamPy.cos:module-SamPy.cos.plot_all_GET}
\item {\texttt{SamPy.cos.plot\_GET}}, \pageref{SamPy.cos:module-SamPy.cos.plot_GET}
\item {\texttt{SamPy.cosmology.cc}}, \pageref{SamPy.cosmology:module-SamPy.cosmology.cc}
\item {\texttt{SamPy.cosmology.distances}}, \pageref{SamPy.cosmology:module-SamPy.cosmology.distances}
\item {\texttt{SamPy.dates.julians}}, \pageref{SamPy.dates:module-SamPy.dates.julians}
\item {\texttt{SamPy.dates.NumberDate}}, \pageref{SamPy.dates:module-SamPy.dates.NumberDate}
\item {\texttt{SamPy.db.DB}}, \pageref{SamPy.db:module-SamPy.db.DB}
\item {\texttt{SamPy.db.insertSAMTablesToMySQL}}, \pageref{SamPy.db:module-SamPy.db.insertSAMTablesToMySQL}
\item {\texttt{SamPy.db.insertSAMTablesToSQLite}}, \pageref{SamPy.db:module-SamPy.db.insertSAMTablesToSQLite}
\item {\texttt{SamPy.db.MSdata}}, \pageref{SamPy.db:module-SamPy.db.MSdata}
\item {\texttt{SamPy.db.MySQLdbSMN}}, \pageref{SamPy.db:module-SamPy.db.MySQLdbSMN}
\item {\texttt{SamPy.db.sqlite}}, \pageref{SamPy.db:module-SamPy.db.sqlite}
\item {\texttt{SamPy.finance.finance}}, \pageref{SamPy.finance:module-SamPy.finance.finance}
\item {\texttt{SamPy.finance.simple\_example}}, \pageref{SamPy.finance:module-SamPy.finance.simple_example}
\item {\texttt{SamPy.fits.combine}}, \pageref{SamPy.fits:module-SamPy.fits.combine}
\item {\texttt{SamPy.fits.ShowHeader}}, \pageref{SamPy.fits:module-SamPy.fits.ShowHeader}
\item {\texttt{SamPy.fitting.fits}}, \pageref{SamPy.fitting:module-SamPy.fitting.fits}
\item {\texttt{SamPy.fitting.SplineArtist}}, \pageref{SamPy.fitting:module-SamPy.fitting.SplineArtist}
\item {\texttt{SamPy.fitting.SplineFitting}}, \pageref{SamPy.fitting:module-SamPy.fitting.SplineFitting}
\item {\texttt{SamPy.fitting.SplineFitting2D}}, \pageref{SamPy.fitting:module-SamPy.fitting.SplineFitting2D}
\item {\texttt{SamPy.fitting.weigtedFittingExample}}, \pageref{SamPy.fitting:module-SamPy.fitting.weigtedFittingExample}
\item {\texttt{SamPy.focus.ACSImageExtensionerTinyTim}}, \pageref{SamPy.focus:module-SamPy.focus.ACSImageExtensionerTinyTim}
\item {\texttt{SamPy.focus.collect\_focus}}, \pageref{SamPy.focus:module-SamPy.focus.collect_focus}
\item {\texttt{SamPy.focus.FocusModel}}, \pageref{SamPy.focus:module-SamPy.focus.FocusModel}
\item {\texttt{SamPy.focus.FocusPlots}}, \pageref{SamPy.focus:module-SamPy.focus.FocusPlots}
\item {\texttt{SamPy.focus.HSTfocus}}, \pageref{SamPy.focus:module-SamPy.focus.HSTfocus}
\item {\texttt{SamPy.focus.phaseretrievalresults}}, \pageref{SamPy.focus:module-SamPy.focus.phaseretrievalresults}
\item {\texttt{SamPy.focus.PhaseretrievalresultsTinyTim}}, \pageref{SamPy.focus:module-SamPy.focus.PhaseretrievalresultsTinyTim}
\item {\texttt{SamPy.focus.PhaseretrievalresultsTinyTimACS}}, \pageref{SamPy.focus:module-SamPy.focus.PhaseretrievalresultsTinyTimACS}
\item {\texttt{SamPy.focus.plotextractionboxdata}}, \pageref{SamPy.focus:module-SamPy.focus.plotextractionboxdata}
\item {\texttt{SamPy.focus.WFC3ImageExtensioner}}, \pageref{SamPy.focus:module-SamPy.focus.WFC3ImageExtensioner}
\item {\texttt{SamPy.focus.WFC3ImageExtensionerTinyTim}}, \pageref{SamPy.focus:module-SamPy.focus.WFC3ImageExtensionerTinyTim}
\item {\texttt{SamPy.grisms.generateAssociationDB}}, \pageref{SamPy.grisms:module-SamPy.grisms.generateAssociationDB}
\item {\texttt{SamPy.grisms.overplot\_sextractor}}, \pageref{SamPy.grisms:module-SamPy.grisms.overplot_sextractor}
\item {\texttt{SamPy.herschel.calculateMergers}}, \pageref{SamPy.herschel:module-SamPy.herschel.calculateMergers}
\item {\texttt{SamPy.herschel.plotage}}, \pageref{SamPy.herschel:module-SamPy.herschel.plotage}
\item {\texttt{SamPy.herschel.plotcolourproperties}}, \pageref{SamPy.herschel:module-SamPy.herschel.plotcolourproperties}
\item {\texttt{SamPy.herschel.plotcorrelations}}, \pageref{SamPy.herschel:module-SamPy.herschel.plotcorrelations}
\item {\texttt{SamPy.herschel.plotFluxDistribution}}, \pageref{SamPy.herschel:module-SamPy.herschel.plotFluxDistribution}
\item {\texttt{SamPy.herschel.plotFluxRedshiftDistribution}}, \pageref{SamPy.herschel:module-SamPy.herschel.plotFluxRedshiftDistribution}
\item {\texttt{SamPy.herschel.plotluminosityfunctions}}, \pageref{SamPy.herschel:module-SamPy.herschel.plotluminosityfunctions}
\item {\texttt{SamPy.herschel.plotmergers}}, \pageref{SamPy.herschel:module-SamPy.herschel.plotmergers}
\item {\texttt{SamPy.herschel.plotmergers2}}, \pageref{SamPy.herschel:module-SamPy.herschel.plotmergers2}
\item {\texttt{SamPy.herschel.plotmergers3}}, \pageref{SamPy.herschel:module-SamPy.herschel.plotmergers3}
\item {\texttt{SamPy.herschel.plotmergers4}}, \pageref{SamPy.herschel:module-SamPy.herschel.plotmergers4}
\item {\texttt{SamPy.herschel.plotmergersPaper}}, \pageref{SamPy.herschel:module-SamPy.herschel.plotmergersPaper}
\item {\texttt{SamPy.herschel.plotnumbercounts}}, \pageref{SamPy.herschel:module-SamPy.herschel.plotnumbercounts}
\item {\texttt{SamPy.herschel.plotpredictions}}, \pageref{SamPy.herschel:module-SamPy.herschel.plotpredictions}
\item {\texttt{SamPy.herschel.plotproperties}}, \pageref{SamPy.herschel:module-SamPy.herschel.plotproperties}
\item {\texttt{SamPy.herschel.plotsfr}}, \pageref{SamPy.herschel:module-SamPy.herschel.plotsfr}
\item {\texttt{SamPy.herschel.plotsize}}, \pageref{SamPy.herschel:module-SamPy.herschel.plotsize}
\item {\texttt{SamPy.herschel.plotssfr}}, \pageref{SamPy.herschel:module-SamPy.herschel.plotssfr}
\item {\texttt{SamPy.herschel.plotssfrredshift}}, \pageref{SamPy.herschel:module-SamPy.herschel.plotssfrredshift}
\item {\texttt{SamPy.herschel.plotstellarmassfunctions}}, \pageref{SamPy.herschel:module-SamPy.herschel.plotstellarmassfunctions}
\item {\texttt{SamPy.image.ImageConvolution}}, \pageref{SamPy.image:module-SamPy.image.ImageConvolution}
\item {\texttt{SamPy.log.Logger}}, \pageref{SamPy.log:module-SamPy.log.Logger}
\item {\texttt{SamPy.parsing.BeautifulSoup}}, \pageref{SamPy.parsing:module-SamPy.parsing.BeautifulSoup}
\item {\texttt{SamPy.pca.SMNpca}}, \pageref{SamPy.pca:module-SamPy.pca.SMNpca}
\item {\texttt{SamPy.plot.basic}}, \pageref{SamPy.plot:module-SamPy.plot.basic}
\item {\texttt{SamPy.plot.colorbarExample}}, \pageref{SamPy.plot:module-SamPy.plot.colorbarExample}
\item {\texttt{SamPy.plot.colorbarExample2}}, \pageref{SamPy.plot:module-SamPy.plot.colorbarExample2}
\item {\texttt{SamPy.plot.COSIHB.compare\_synphot\_files}}, \pageref{SamPy.plot.COSIHB:module-SamPy.plot.COSIHB.compare_synphot_files}
\item {\texttt{SamPy.plot.COSIHB.IO}}, \pageref{SamPy.plot.COSIHB:module-SamPy.plot.COSIHB.IO}
\item {\texttt{SamPy.plot.COSIHB.plotting}}, \pageref{SamPy.plot.COSIHB:module-SamPy.plot.COSIHB.plotting}
\item {\texttt{SamPy.plot.COSIHB.synphotSensitivity}}, \pageref{SamPy.plot.COSIHB:module-SamPy.plot.COSIHB.synphotSensitivity}
\item {\texttt{SamPy.plot.COSIHB.wrapper\_OLD}}, \pageref{SamPy.plot.COSIHB:module-SamPy.plot.COSIHB.wrapper_OLD}
\item {\texttt{SamPy.plot.COSIHB.wrapperSynphot}}, \pageref{SamPy.plot.COSIHB:module-SamPy.plot.COSIHB.wrapperSynphot}
\item {\texttt{SamPy.plot.interactive\_correlation\_plot}}, \pageref{SamPy.plot:module-SamPy.plot.interactive_correlation_plot}
\item {\texttt{SamPy.plot.interactive\_mean\_std\_normal\_distribution}}, \pageref{SamPy.plot:module-SamPy.plot.interactive_mean_std_normal_distribution}
\item {\texttt{SamPy.plot.interactive\_two\_sample\_t\_test}}, \pageref{SamPy.plot:module-SamPy.plot.interactive_two_sample_t_test}
\item {\texttt{SamPy.plot.tools}}, \pageref{SamPy.plot:module-SamPy.plot.tools}
\item {\texttt{SamPy.plot.trilogy}}, \pageref{SamPy.plot:module-SamPy.plot.trilogy}
\item {\texttt{SamPy.sandbox.MyTools}}, \pageref{SamPy.sandbox:module-SamPy.sandbox.MyTools}
\item {\texttt{SamPy.sandbox.simpleMultiprocessingExample}}, \pageref{SamPy.sandbox:module-SamPy.sandbox.simpleMultiprocessingExample}
\item {\texttt{SamPy.smakced.getSmackedData}}, \pageref{SamPy.smakced:module-SamPy.smakced.getSmackedData}
\item {\texttt{SamPy.smakced.getSmackedDataNoNumpy}}, \pageref{SamPy.smakced:module-SamPy.smakced.getSmackedDataNoNumpy}
\item {\texttt{SamPy.smnIO.findFiles}}, \pageref{SamPy.smnIO:module-SamPy.smnIO.findFiles}
\item {\texttt{SamPy.smnIO.IO}}, \pageref{SamPy.smnIO:module-SamPy.smnIO.IO}
\item {\texttt{SamPy.smnIO.read}}, \pageref{SamPy.smnIO:module-SamPy.smnIO.read}
\item {\texttt{SamPy.smnIO.sextutils}}, \pageref{SamPy.smnIO:module-SamPy.smnIO.sextutils}
\item {\texttt{SamPy.smnIO.write}}, \pageref{SamPy.smnIO:module-SamPy.smnIO.write}
\item {\texttt{SamPy.statistics.plot\_ica\_vs\_pca}}, \pageref{SamPy.statistics:module-SamPy.statistics.plot_ica_vs_pca}
\item {\texttt{SamPy.statistics.StatTool}}, \pageref{SamPy.statistics:module-SamPy.statistics.StatTool}
\item {\texttt{SamPy.statistics.StatTool2files}}, \pageref{SamPy.statistics:module-SamPy.statistics.StatTool2files}
\item {\texttt{SamPy.stis.STISCCDFlatWavelengthDependency}}, \pageref{SamPy.stis:module-SamPy.stis.STISCCDFlatWavelengthDependency}
\item {\texttt{SamPy.stis.STISCCDImagingFlats}}, \pageref{SamPy.stis:module-SamPy.stis.STISCCDImagingFlats}
\item {\texttt{SamPy.stis.STISmonthly}}, \pageref{SamPy.stis:module-SamPy.stis.STISmonthly}
\item {\texttt{SamPy.wrapper.gf}}, \pageref{SamPy.wrapper:module-SamPy.wrapper.gf}
\end{theindex}

\renewcommand{\indexname}{Index}
\printindex
\end{document}
